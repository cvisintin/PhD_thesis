\documentclass[titlesmallcaps,copyrightpage,examinerscopy]{uomthesis}
%----------------------------------------------------------
% OPTIONS
% Options you can use in the documentclass above:
%
% phd/mres/hons = set the degree text [default=phd]
% copyrightpage = print a copyright page on the back 
%                 of the title page [default=false]
% examinerscopy = print "Examiner's Copy" of title page and 
%                 change linespacing to 1.5 [default=false]
% titlesmallcaps = Use smallcaps for the title [default=false]
%---------------------------------------------------------- 

% this shows what labels you are using for cross references
% \usepackage{showkeys} 
%---------------------------------------------------------- 
% STRUCTURE
% this document is a skeleton which pulls in the meat of the thesis
% from other files. Comment out and add lines as appropriate.
%---------------------------------------------------------- 
\setcounter{tocdepth}{3}
\setcounter{secnumdepth}{3}

%%% PACKAGES
\usepackage{tabularx}
\usepackage{booktabs}
\usepackage{multirow}
\usepackage{array}
\usepackage{cleveref}

\usepackage[caption=false]{subfig}
%\useackage{tikz}
\usepackage{pdfpages}
\usepackage{multirow}
\usepackage{wrapfig}
\usepackage{lipsum}

%%% DEFINE NEW COMMANDS
\newcommand{\B}{\textbf}
\newcommand{\TL}{\textless}
\newcommand{\PRZ}{\text{Pr}\bigl( >\mid Z \mid \bigr)}

%remove title and space above bibliography
\makeatletter
\renewenvironment{thebibliography}[1]
     {\list{\@biblabel{\@arabic\c@enumiv}}
           {\settowidth\labelwidth{\@biblabel{#1}}
            \leftmargin\labelwidth
            \setlength{\topsep}{0pt} 
            \setlength{\partopsep}{0pt}
            \advance\leftmargin\labelsep
            \@openbib@code
            \usecounter{enumiv}%
            \let\p@enumiv\@empty
            \renewcommand\theenumiv{\@arabic\c@enumiv}}
      \sloppy
      \clubpenalty4000
      \@clubpenalty \clubpenalty
      \widowpenalty4000
      \sfcode`\.\@m}
     {\def\@noitemerr
       {\@latex@warning{Empty `thebibliography' environment}}
      \endlist}
\makeatother   

%define abstract environment for chapters
\newenvironment{localsize}[1]
{%
  \clearpage
  \let\orignewcommand\newcommand
  \let\newcommand\renewcommand
  \makeatletter
  \input{bk#1.clo}%
  \leftskip4em
  \rightskip\leftskip
  \par
  \makeatother
  \let\newcommand\orignewcommand
}

%redefine tabular environment and commands for chapters
\let\oldtabular\tabularx
\renewcommand{\tabularx}{\scriptsize\oldtabular}
\newcolumntype{P}[1]{>{\linespread{.9}\raggedright\arraybackslash}p{#1}}
\newcolumntype{Y}{>{\linespread{.7}\raggedright\arraybackslash}X}

%Add prefix 'Appendix'
\makeatletter
\newcommand\appendix@chapter[1]{%
  \refstepcounter{chapter}%
  \orig@chapter*{Appendix \@Alph\c@chapter: #1}%
  \addcontentsline{toc}{chapter}{Appendix \@Alph\c@chapter: #1}%
}
\let\orig@chapter\chapter
\g@addto@macro\appendix{\let\chapter\appendix@chapter}
\makeatother

\begin{document}

\frontmatter
\title{Modelling and predicting collision risks between wildlife and moving vehicles in space and time}
\author{Casey C. Visintin}
\department{BioSciences}

\titlepage

\chapter{Abstract}
This PhD thesis aims to identify and further investigate collision risks where wildlife and vehicles co-occur in space and time.  The research develops methods to quantify and analyse collision risk with a simple model framework, explore temporal effects, test across multiple species and geographic areas, assess model calibration, and utilise existing and accessible data.  This framework will be set in the larger context of risk analysis and targeted at informing management decisions and communicating risk to the general public; this may include remotely-hosted, self-updating model and feedback mechanisms (vehicle navigation devices and smart-phone technologies).

\chapter{Declaration}
This is to certify that:
{\renewcommand{\theenumi}{\roman{enumi}}%
\begin{enumerate}
 \item the thesis comprises only my original work towards the PhD except where indicated in the Preface,
 \item due acknowledgement has been made in the text to all other material used,
 \item the thesis is fewer than 100 000 words in length, exclusive of tables, maps, bibliographies and appendices.
\end{enumerate}
}
\hrulefill \\

\begin{flushright} 
\includegraphics[scale=.5,trim={0 0 1.5cm 0},clip]{signature.png}

Casey Visintin

March, 2017
\end{flushright}

\chapter{Preface}
This thesis utilises several datasets from various sources to train and test statistical models.  Wildlife Victoria Injury Database (WVD), Victorian Biodiversity Atlas (VBA), Global Biodiversity Information Facility (GBIF), VicRoads, V/Line, California Roadkill Observation System (CROS), and Insurance Australia Group Animal-strikes Database (IAG)\ldots

\newpage
\textbf{First authored publications}
\begin{enumerate}
\item Casey Visintin, Rodney van der Ree, Michael A. McCarthy (2016) A simple framework for a complex problem?  Predicting wildlife-vehicle collisions. \textit{Ecology and Evolution}, 6, 6409-6421

\item Casey Visintin, Rodney van der Ree, Michael A. McCarthy (in press) Consistent patterns of vehicle collision risk for six mammal species. \textit{Journal of Environmental Management}

\item Casey Visintin, Fraser Shilling, Rodney van der Ree, Michael A. McCarthy (submitted) A generalised predictive model for wildlife-vehicle collision risk with large mammals. \textit{Journal of Wildlife Management}

\item Casey Visintin, Rodney van der Ree, Michael A. McCarthy (in review) Predicting wildlife-train collisions across space and time to inform railway operations. \textit{Journal of Applied Ecology}
\end{enumerate}

\chapter{Acknowledgements}

My supervisors have been instrumental in both my academic and professional development and the synthesis of this thesis.  Michael McCarthy provided invaluable mathematical and modelling advice and Rodney van der Ree advised on many aspects of road ecology.  Both were consistently enthusiastic and optimistic, despite my occasional bouts with ``research fatigue" \ldots

\tableofcontents
\listoffigures
\listoftables

\mainmatter

%\chapter{General Introduction}\label{sec:intro}
\newpage

\section{The wildlife and road conundrum}

\subsection{Human-wildlife interactions}

We are now in what some refer to as the \textit{Anthropocene}, where \textit{Homo sapiens} exert strong influences on climatic, geologic and biotic systems \citep{crut06}. Studies estimate that humans have transformed more than seventy-five percent of the earth's ice-free terrestrial surface \citep[e.g.][]{elli08}. These modifications result from large-scale activities; perhaps the most influential being intensified agricultural production, provision of linear infrastructure (e.g. roads, rail, and utility easements), and urban development \citep{vito97,sand02,fole05}. Human population is forecast to grow in the future, and this will increase pressure on unmodified landscapes.

As permanent human settlements, industry, and related infrastructure spread over the Earth’s surface, bringing people closer to animals, the dominant anthopocentricism of western culture emphasises a conceptual distance between humans and the natural world.  Emblematic of this distance, and disjunction, is the field of research referred to as `human-wildlife interactions'; which has a dedicated journal (Human-Wildlife Interactions: ISSN 2155-3858) and monographs \citep[e.g.][]{manf08}. In this field, all manners of interactions are considered; both positive \citep{dall12} or negative \citep{jone99} -- see Figure XX for possible human-wildlife interactions. 

\begin{center}
--Insert diagram of human-wildlife interactions here--
\end{center}

For stationary configurations of human settlement, increased exposure is due to tolerant species remaining or re-establishing in human-dominated landscapes \citep{soul16}. Although the suite of organisms that humans have contact with is quite varied -- the most frequent contact possibly being with invertebrates -- medium to large species (i.e., $>$10 kg) tend to get more attention, especially when issues arise \citep{seor16}. Human beings generally appreciate the existence of wildlife within and around their places of settlement \citep{dick86,kell93,raad07}. Bird watching is a popular recreational activity practised worldwide and many communities rely on wildlife tourism as an economic resource \citep{news05}. In urban settings, humans may also consider species that surround them throughout their everyday lives as fellow-creatures \citep{zinn08}.

Certain kinds of interactions, usually where wildlife and humans compete for resources, or intersect in a way that causes nuisance or harm to human activities are referred to as `human-wildlife conflicts'. This, however, may be misleading as most studies using the term `wildlife-human conflicts' actually reference conflicts amongst humans in regards to wildlife \citep{pete10}. Nevertheless, studies have demonstrated notable impacts on society (e.g. human injuries, illness, and economic losses) from wildlife \citep{cono95}. Larger species are more likely to cause significant harm to person or damage to property \citep{cono01} and are less easily managed by individuals that lack proper training. Therefore, the public expectation is of government intervention to manage human-wildlife conflicts \citep{reit99}. Further, the general public often views the management of human-wildlife conflicts to reside in the scope of wildlife-focussed disciplines, however, issues arising from human-wildlife interactions are complex and may require knowledge from several non-ecological fields \citep{deck97,madd04}. For example, specialist knowledge on both animal behaviour (zoology) and human behaviour (sociology) may be combined to arrive at effective solutions \citep{dicka10}.

%From the other perspective, large charismatic species are often persecuted when residing in close proximity to human settlements and therefore targets of conservation programs \citep{trev03}.

%\subsection{Effects of wildlife on humans}
%
%\begin{figure*}[htp]
%  \centering
%  \includegraphics[width=0.8\textwidth]{microbats.jpg}
%  \caption[Public involvement with wildlife]{Public involvement in scientific research to learn about microbats in the Royal Botanic Gardens, Victoria, Australia. Photograph by Meg Bauer.}
%  \label{microbats}
%\end{figure*}
%
%Human beings generally appreciate the existence of wildlife within and around their places of settlement \citep{dick86,kell93,news05,raad07}. Bird watching is a popular recreational activity practised worldwide and many communities rely of wildlife tourism as an economic resource. Further, citizens are excited to learn about some of the more cryptic species that inhabit their immediate environment (Visintin, personal observation -- see \Cref{microbats}). In urban settings, human-wildlife interactions may be considered benign due to human's consideration of species that surround them throughout their everyday lives as fellow-creatures \citep{zinn08}.

%In many cases, wildlife aggression towards humans is due to protection of offspring; for example, swooping Australian magpies \citep{jone99} as shown in \Cref{magpie}.

%\begin{figure*}[htp]
%  \centering
%  \includegraphics[width=0.8\textwidth]{magpie.jpg}
%  \caption[Wildlife aggression towards humans]{Australian magpie swooping humans during breeding season. Photograph by Adam McLean.}
%  \label{magpie}
%\end{figure*}

%\subsection{Effects of the built environment on wildlife}
%
%In contrast to the natural environment, the built environment refers to human-manufactured structures, infrastructure, and recreational landscapes (e.g. urban parks) where people reside and conduct daily activities.  The built environment has mixed effects on wildlife persistence. Some species thrive in urban landscapes (termed `urban adapters') whilst others perish \citep{shoc06}. `Urban adapter' species benefit from increased availability of resources. For example, gulls and other urban birds rely on overflowing rubbish bins for food \citep{ditc06} and bats utilise small cavities in buildings and overpasses for shelter \citep{kunz82}. Many of the urban-adapted species, however, are invasive and displace native wildlife \citep{mcki06}. Built environments can also negatively effect sensitive species through direct displacement from habitat \citep{czec97}, or disturbance due to light and noise pollution \citep{parr16}. Further, wildlife mortality may result from static (e.g. bird strikes to glass) and non-static (e.g. car strikes to wildlife) elements of the built environment. 

\subsection{Impacts of linear infrastructure on wildlife}

Linear infrastructure can traverse vast areas between human settlements, and penetrate remaining natural and semi-natural areas. In some cases, these easements may be the only remaining semi-natural refuges for wildlife \citep{benn91}. Linear infrastructure is most commonly thought of as roads, however, can also include railways, utility easements, and other continuous human-engineered resources in the landscape. Impacts of linear infrastructure on wildlife can be positive, neutral, or negative. For example, disused road reserves lead to retention of habitat that may benefit wildlife as they often contain old and diverse remnant vegetation \citep{lent11}. Further, linear infrastructure without moving vehicles may provide connection corridors and assist migration for some species \citep{rvdr15}. Effects may be neutral when suitable habitat is provided on road reserves with very little traffic volume or occupied by species with low mobility. Nonetheless, negative effects of linear infrastructure may certainly outweigh the benefits, especially with respect to roads and wildlife persistence \citep{fahr09}. There are few places left in the terrestrial environment that are not accessible by some form of road (\Cref{roads}) and, consequently, the vast sphere of ecological influence makes roads a global concern \citep{laur14}. The field of `road ecology' was thus developed to study problems and solutions associated with the construction and use of roads across the natural landscape \citep{form03,rvdr15}. 

\begin{figure*}[!t]
  \centering
  \includegraphics[width=\textwidth]{world_roads.png}
  \caption[Global distribution of known roads]{Global distribution of known roads. Actual distribution may be under-represented in some developing countries due to data deficiencies. Data used to create the map was sourced from the Center for International Earth Science Information Network (CIESIN).}
  \label{roads}
\end{figure*}

\subsection{Wildlife mortality on roads}

Human-wildlife conflicts occur where moving vehicles and animals co-occur in space and time, and this is facilitated by transportation networks. Private transport is increasingly common; people travel further and faster all the time, particularly in developing nations. This will continue to increase as developing nations become more affluent \citep{rvdr15}. What will also increase is wildlife mortality due to collisions with moving vehicles; commonly referred to as `roadkill'. It is estimated that billions of fauna expire annually due to collisions with vehicles on transportation networks \citep{seil06}. Animals that are struck and die on roadsides or railway corridors attract predators and may result in secondary collisions which further increase mortality rates \citep{spel98}. From a conservation perspective, these mortalities can result in population effects and disruption to food-chain networks \citep{ramp06b,cham10,pola14}.  Although collisions with wildlife bring about matters of animal welfare, biodiversity, and wildlife persistence, perhaps the largest concern is human safety. 

From an anthropocentric perspective, animal-vehicle collisions with larger species cause injuries, property damage, and occasional loss of human life which compels road and wildlife managers to find solutions. The visibility of the problem results in negative public perception and has caused some road authorities to regularly patrol problematic roads and collect carcasses. In some cases the data is recorded \citep{huij07a}. This has also initiated partnerships between academics and road authorities to conduct scientific research on animal-vehicle collisions.  These endeavours have created a wealth of research and literature on larger common species \citep{biss08b,clev02,romi96,sudh09}, but proportionally lower amounts on smaller (and perhaps more threatened) fauna \citep{clev03}.

\subsection{Factors influencing collision risk}

The factors influencing collision risk are quite broad and \cite{form03} groups these into four major categories: animal behaviour, human behaviour, the road environment, and the landscape environment. For example, increases in collisions may be due to the following: animal behaviours, such as migration, flight response, or feeding patterns, thereby increasing species presence on roads; the operation of motor vehicles by humans at high frequencies, high speeds, and at dangerous times; landscape features attracting animals to, and influencing movement across, roads and; road features hindering safe vehicle operation. In seventy-one studies that analysed wildlife-vehicle collision risk, the ten most common variables selected for use in predictive models were classified in three of the four categories proposed by road ecology theory (\Cref{wvc_studies}). See \Cref{apx:D} for all variables and studies used in the analysis. Traffic related variables (volume and speed) -- which may be considered human behaviour -- featured in the top four most commonly used predictor variables.

\begin{figure*}[!t]
  \centering
  \includegraphics[width=0.7\textwidth]{wvc_studies.png}
  \caption[Variables used in wildlife-vehicle collision studies]{Common variables used in wildlife-vehicle collision studies. The dots show the frequency each modelling variable was used in a total of seventy-one studies on wildlife-vehicle collisions.}
  \label{wvc_studies}
\end{figure*}

\subsection{Management of wildlife-vehicle collisions}

%Worldwide, many studies have been undertaken to better understand wildlife-vehicle collisions in an effort to supplement management.  Some studies related collisions to animal biology, behaviour and characteristics. In Australia, \cite{coul97} showed a male bias across several species of macropod roadkills, whilst \cite{kloc06} determined a female bias in red kangaroos.  \cite{lee04} concluded that kangaroo density near roads increased during periods of drought due to relative greenness of vegetation at road verges. \cite{ramp06b} used collision events in a population viability analysis model to evaluate effects on swamp wallabies.  In North America, animal travel speed was related to collisions \citep{litv08}. Research has also demonstrated that collision occurrence is related to many other combinations of environmental and anthropogenic variables \citep{barn07}.

Research has demonstrated that collision occurrence is related to many combinations of environmental and anthropogenic variables \citep{barn07}. Some of these factors are more easily managed than others and that presents an optimisation problem for managers; where to have the largest effect depends on knowing the relative influence of each factor on collision risk. In some cases the conflicts can be mitigated with engineering solutions (e.g. tunnels and land bridges \citep{bond08}, or rope ladders \citep{soan13}), but these are more suited to known species movements or behaviours. Where mobile wildlife are in contact diffusely over a road network throughout the year, at all hours, it is a more difficult problem to manage.  Mitigation strategies to reduce wildlife-vehicle collisions have varying costs and effectiveness \citep{huij09} and costs can be compounded by additional collisions occurring from ineffective mitigation. Therefore, an important objective in managing wildlife-vehicle collisions is knowing where, when, and how to intervene.

%may include (in order of increasing costs), road signage \citep{bond13}, discouragement and/or detection devices \citep{huij03,huij06}, fencing installations \citep{clev02,jaeg04}, road crossing structures (overpasses and underpasses) , or a combination of treatments \citep{dani98,pola14,romi96}. 

%One difficulty is the implementation and/or translation of scientific research and problem-solving into management actions \citep{roux06}.  The disconnects between environmental research, conservation policy and management/implementation are extensively referred to in the ecological literature.  Technology, most notably electronic media and computers, has been used to overcome some of these problems and functions as a tool to bridge the information divide.  Software applications to inform management decisions have been developed in areas of broad landscape and seascape conservation \citep{ball09,watt09}, species distributions \citep{phil06}, climate change \citep{east09}, water resources \citep{liu08}, and many others.  Moreover, the translation of scientific research into accessible public knowledge and education is also a formidable problem \citep{lubc98}.  This is an area that deserves more attention as human behaviour change can be a powerful upstream solution to complement collision management efforts.

\section{Improving management and decision-making}

\subsection{Incorporating risk theory}

Wildlife-vehicle collisions are a risk analysis problem. Risks are chances of adverse events, that have defined consequences, happening in specified time frames \citep{burg05}. Managing risks involves assessing potential hazards, exposure to hazards, and probabilities of hazards occurring, simulating outcomes under different scenarios, and making decisions based on acceptable thresholds. To help make decisions about risk, it is useful to employ quantitative methods either independently or in combination with qualitative methods. Quantitative analysis is used to understand, communicate, and manage uncertainty about risk in many fields such as engineering \citep{apos04}, biological sciences \citep{sute16}, and finance \citep{mcne15}.  All of these fields define risk a bit differently and the parameterisations used in their analyses vary accordingly. Among these disciplines, there are some important ideas about general risk theory that can be useful to help frame wildlife-vehicle collision risk at a conceptual level.

\subsection{Spatial statistics and modelling}

Many studies have utilised environmental modelling to analyse and predict risks imposed on species (fauna and flora), biotic systems, and human beings resulting from anthropocentric activity. The outputs of these models have many benefits including informing management operations, supplementing additional models, facilitating scientific research, and optimising mitigation strategies. The use of modelling in conservation efforts is well established \citep{star86}, however, predicting risks to wildlife by vehicles is a relatively new practice. The most common use of modelling has been to identify collision hotspots in geographic space in an effort to direct management efforts. For example, \cite{malo04} predicted collision risk of multiple species at two spatial scales using logistic regression. Similar methods have been used by \cite{roge09} to predict collision risk with wombats, and \cite{sudh09} for deer. Whilst the predictions from such models may be useful to managers, a comparative framework that describes simple relationships between variables, with tunable parts, and allows simulation of outcomes is more useful. \cite{clev15} alludes to this when calling for a general framework in road ecology to combine the disparate factors relating to wildlife-vehicle collisions.

When undertaking statistical modelling of events, such as collisions, it is common to analyse either binary data (occurred/did not occur) or count data (how many occurred).  As it is not plausible to have negative or partial events, these data are always constrained to be zero or positive integers. Bernoulli and Poisson distributions both satisfy these requirements and, thus, are often chosen for modelling collision events. A common statistical model used for collisions is based on generalised linear regression \citep{mccu89}; a robust and widely-used statistical method in which non-continuous (discrete) responses are regressed on a linear predictor using a link function. Spatial statistics is an extension of these methods and studies geographic properties of objects or events; in two-dimensional space, the choice of link function must be deliberate \citep{badd10}. Related to this concept, patterns of events can be analysed using point process models which consider events occurring across space with a single rate parameter. Modelling events on linear networks is a special case of one-dimensional point process analysis that assumes collisions occurring outside the network are impossible \citep{okab09} -- collisions will always occur coincident with the movement vectors of vehicles. One advantage of using Poisson point processes is that they assume collision events occur over a domain bound by time and are not influenced by the choice of sampling area or spatial scale. Yet, these sorts of models are not regularly used in road ecology studies.

\subsection{Important considerations}

The contemporary modeller has considerable access to publicly-available data. Anthropogenic data, such as demography, exist for most developed parts of the world. Environmental data, particularly satellite-based and remotely-sensed, have global coverage. Collecting data can be quite expensive, and this information provides low-cost sources of information and has the potential to significantly expand the scope of modelling wildlife-vehicle collisions. However, data has inherent uncertainties and biases \citep{glym97}; uncertainty may result from measurement error, under-reporting, and general environmental stochasticity whilst bias is often due to selective sampling (either intentional or unintentional). Considering these shortcomings, data should always include metadata that reports these deficiencies so modellers may adequately address them in analyses. Many online datasets, derived by scientific organisations and often post-processed, include metadata about uncertainties.

Citizen-science data can supplement data-deficient ecological studies, and whilst shown to be effective in road ecology \citep{paul14}, is rarely used in the discipline. The use of such data, however, should be subject to rigorous quality control and verification which may, or may not, be supplemented by adequate metadata. Records may not be useful due to incomplete/incorrect data or geographical ambiguity. Although under-reporting of data has limited effects on model robustness, spatial biases can adversely affect model performance \citep{snow15}. In particular, survey efforts are seldom reported in collision data and vary considerably amongst and within jurisdictions \citep{huij07a}.

It is important to identify and address spatial biases when developing predictive models. Models trained on observations that are spatially biased (e.g. collected at particular places or dependent on characteristics of reporting individuals) may cause model predictions to reflect patterns of collection rather than actual distributions of organisms or events. Methods have been developed to address bias in models fit to both presence-absence \citep{hijm12} and presence-only data \citep{kram13}. For both types, implicit biases in data may be incorporated into modelling if they are adequately described in metadata \citep{wart13}. For example, many of the records in fauna atlases are subject to geographical bias (e.g. close to roads and towns), and this is also prevalent in collision or carcass data.
%\subsection{Temporal variation in data}

Temporal variation occurs in both natural and anthropogenic systems. Wildlife species have different patterns of activities (e.g. diurnal, nocturnal, crepuscular) that make them more or less susceptible to wildlife-vehicle collisions during the day. Seasonal patterns are particularly important for migratory species \citep{hick85,bern92}, where seasonal variation has a large influence on collision risk \citep{shep08}. These patterns are more easily incorporated into modelling as there is less uncertainty about observations -- the temporal resolution is large and thus discrepancies in time between a collision event and carcass discovery are negligible. In contrast, daily patterns in collision data are subject to measurement errors and require more careful scrutiny, however, are highly informative in determining when collision risks are high. To address some of this uncertainty, studies may represent temporal data as a categorical variable \citep{mizu14} with discrete ranges \citep{rowd08}. As a continuous variable, hour may be represented by cyclic functions in predictive models \citep{neum12,thur15}. When predictions of collision risk are made using temporal information, they are referred to as `hot moments'.

\subsection{Reducing wildlife-vehicle collisions}

It is clear that wildlife-vehicle collisions are a complex problem that have global implications. This problem will continue into the future as more developing nations expand their linear transportation networks. Effective management will require knowing where, when, and how to mitigate collision risks, but this will require a clear understanding of the factors, and their interactions, relating to wildlife-vehicle collisions. Previous work on modelling and predicting collision risk have been contextually useful but their generality or relationship to management objectives were not explored. There is a critical need for research that develops a conceptual risk framework for modelling and predicting wildlife-vehicle collisions. This framework should be flexible to different parameterisations, informative for management objectives, generalisable to different spatial scales and transportation networks, and responsive to temporal information. These features will support global efforts to reduce wildlife-vehicle collisions.

\section{Thesis aims and scope}

The broad aim of this thesis is to analyse and predict where and when wildlife will be at high risk of collisions with moving vehicles, across large transportation networks. As I intend the work to supplement managers working in the wildlife and transportation fields, I have developed analyses that are generalisable, transferable, reproducible, and based on open-source data and methods. Therefore, each chapter tests different model applications and demonstrates management value. \Cref{sec:egk} introduces a general conceptual framework to model wildlife-vehicle collision risk using kangaroos in south-east Australia as a case study. Using the same geographic area, \Cref{sec:6sp} extends the work of \Cref{sec:egk} by applying the model framework to six mammal species commonly involved in wildlife-vehicle collisions. In \Cref{sec:cal}, the conceptual framework is applied to a different geographic area (California, USA) and a different species (Mule deer). To analyse risk on a different transportation network, I extend the conceptual model to examine wildlife-train collisions in both space and time in \Cref{sec:train}. To assess the performance of the model introduced in \Cref{sec:egk}, and demonstrate the importance of unified and standardised data repositories of wildlife-vehicle collisions, model predictions are compared to additional datasets collected by various organisations in \Cref{sec:val}. \Cref{sec:conc} summarises a selection of key findings within the chapters whilst suggesting novel areas for further research and potential applications of the work.
\chapter{General Introduction}\label{sec:intro}

%\chapter{A conceptual risk framework for modelling and predicting wildlife-vehicle collisions}\label{sec:egk}
\newpage

\begin{localsize}{10}
\section*{\centering Abstract}

Collisions of vehicles with wildlife kill and injure animals, and are also a risk to vehicle occupants, but preventing these collisions is challenging. Surveys to identify problem areas are expensive and logistically difficult. Computer modelling has identified correlates of collisions yet these can be difficult for managers to interpret in a way that will help them reduce collision risk.

We introduce a novel method to predict collision risk by modelling hazard (presence and movement of vehicles) and exposure (animal presence) across geographic space. To estimate the hazard, we predict relative traffic volume and speed along road segments across south-eastern Australia using regression models based on human demographic variables. We model exposure by predicting suitable habitat for our case study species (Eastern Grey Kangaroo Macropus giganteus) based on existing fauna survey records and geographic and climatic variables. Records of reported kangaroo-vehicle collisions are used to investigate how these factors collectively contribute to collision risk.

The species occurrence (exposure) model generated plausible predictions across the study area, reducing the null deviance by 30.4\%. The vehicle (hazard) models explained 54.7\% variance in the traffic volume data and 58.7\% in the traffic speed data. Using these as predictors of collision risk explained 23.7\% of the deviance in incidence of collisions. Discrimination ability of the model was good when predicting to an independent dataset.

The research demonstrates that collision risks can be modelled across geographic space with a conceptual analytical framework using existing sources of data, reducing the need for expensive or time-consuming field data collection. The framework is novel because it disentangles natural and anthropogenic effects on the likelihood of wildlife-vehicle collisions by representing hazard and exposure with separate, tunable sub-models.

\end{localsize}

\newpage
\section{Introduction}

Roads have well-documented negative ecological impacts \citep{form98,spel98,rvdr15} including effects on terrestrial fauna. Road construction and use fragments and destroys habitat, causes pollution (e.g. noise, light and chemical run-off), and kills and injures animals. Perhaps the most visible impact is direct mortality through wildlife-vehicle collisions (WVC) - billions of fauna are killed annually around the world \citep{seil06}. Such an issue has prompted many road management authorities to routinely collect animal carcasses struck and killed by moving vehicles to reduce visual impacts for road travellers \citep{huij07b} and avoid secondary collisions with scavenging wildlife species. In addition, many governments around the world incur significant costs installing wildlife-proof fencing and under- and over-passes to reduce the rate of wildlife-vehicle collisions and improve landscape connectivity \citep{rvdr15}.

Worldwide, the frequency, magnitude and distribution of WVC have been widely studied. Many such studies relate rate of collisions to environmental conditions, anthropogenic variables, and animal biology, behaviour and characteristics. While many studies have used statistical modelling to determine hotspots for WVC, most are limited to a single stretch of road or small collections of roads \citep{gund98,clev01,clev03,ramp05,ramp06b,gome08,lang09,hurl09,roge09,hoth12,mark12,sant13,seo15}. These models perform relatively well at local scales, and some have been employed to communicate areas of high-risk to road managers, but many cannot extrapolate to other sections of road or entire networks. We extend this work by developing and testing a framework that may be applied at much larger scales; consistent with the boundaries of road authority jurisdictions (i.e. state/provincial). \cite{clev15} asserts that the variability of significant predictors among regions and geographic scales highlights a critical need for a useful broad-scale conceptual framework to analyse/predict WVC.

Managers often have limited time and budgets to survey WVC across large areas or road networks. Many existing studies utilise data collected at fine spatial scales to model WVC, which is only feasible for limited areas because it is prohibitively expensive to collect data at the required regional scale. Methods to incorporate publicly-accessible, existing sources of data or model data-deficient parameters are useful to reduce costs. For example, remotely-sensed data is useful to determine environmental influences, while GIS-based census data may be used to characterise road conditions. Our framework is a desktop-based exercise for managers to determine risk across a large network with the ability to fine-tune based on available data. Moreover, it can be adapted with additional or modified data without changing the underlying methodology.

When modelling is used to determine hotspots for WVC, predictions are made using single models that combine both environmental and anthropogenic variables \citep{malo04,ramp05,gome08,roge09,hoth12,bart14,meis14,snow14}. These studies are valuable for managers to identify locations  for mitigation or further study, however, they are often limited in their ability to extrapolate beyond the study area and do not clearly indicate potential confounding effects or suggest what mitigation to use (e.g. on the road environment or on the species). For example, vegetation on the road verge is commonly used to model collisions, and has been shown to be an effective predictor in previous work.  But as a single covariate among others in a regression model, it is difficult to determine if vegetation is related to collisions based on its affect on visibility for drivers, or its attraction for animals. We extend the utility of this research by disentangling the effects of human activity and wildlife behaviour, which has not been achieved before. We aim to improve the accessibility of collision analysis and predictions by relating risk to two components, exposure and hazard. This hierarchical classification and structure of the predictors enables a more straight-forward calibration and interpretation of the models - a useful feature for managers - without compromising predictive performance.

\section{Materials and Methods}

\subsection{Conceptual Model Framework}
Risk (as a rate) of animal collisions can be expressed as a function of exposure and hazard: 


\begin{equation} \label{eq:21}
R_i=aE_iH_i
\end{equation}

where $R_i$ is the risk, $E_i$ is the exposure, $H_i$ is the hazard, $a$ is a constant of proportionality and represents a modelling unit (e.g. site, road, road segment). This equation is multiplicative and to enable linear analysis, we logarithmically transform the variables: 

\begin{equation} \label{eq:22}
\text{log}(R_i)=\text{log}(a)+\text{log}(E_i)+\text{log}(H_i)
\end{equation}

This suggests risk is perfectly related to both exposure and hazard and we allow modified responses by expressing the constant of proportionality as an intercept and introducing regression coefficients such that risk is modelled as: 

\begin{equation} \label{eq:23}
\text{log}(R_i)=\beta_0+\beta_1\text{log}(E_i)+\beta_2\text{log}(H_i)
\end{equation}

\begin{center}
-or equivalently as-
\end{center}

\begin{equation} \label{eq:24}
R_i=e^{\beta_0+\beta_1\text{log}(E_i)+\beta_2\text{log}(H_i)}
\end{equation}

Here, $\beta_1$ and $\beta_2$ indicate the relative influence of each predictor on risk. This makes inferences on the model fit more tractable and clearly identifies areas for management focus (e.g. the exposure or the hazard). If risk is exactly related to exposure and hazard as in equation \ref{eq:22}, then both regression coefficients $\beta_1$ and $\beta_2$ will equal one.

For this study, we represent exposure with animal presence, and hazard with both traffic volume and speed as neither one in isolation would be a realistic threat. We envisage risk as being measured by the rate of collisions. Using this configuration, the rate of collisions ($C_i$) is modelled as a function of three predictors of (1) species occurrence, (2) traffic volume and (3) traffic speed: 
 
\begin{equation} \label{eq:25}
C_i=e^{\beta_0+\beta_1\text{log}(O_i)+\beta_2\text{log}(V_i)+\beta_3\text{log}(S_i)}
\end{equation}

where $O_i$ is species occurrence, $V_i$ is traffic volume, $S_i$ is traffic speed, in a given place $i$.

While we consider risk as being measured by the rate of collisions (e.g. collisions per month), our data consist of binary events (observations of individual collisions and no-collisions, coded as ones or zeroes, respectively). If collisions are treated as events in a Poisson encounter model, then the probability of no collisions occurring equals $e^{-C_i}$. Let $Y_i=1$ indicate a collision occurred at site $i$, and $Y_i=0$ indicate no collision, with $p_i=\text{Pr}(Y_i=1)$. Since the probability of no-collision is also equal to one minus the probability of collision:

\begin{equation} \label{eq:26}
\text{log}(1-p_i)=-e^{\beta_0+\beta_1\text{log}(O_i)+\beta_2\text{log}(V_i)+\beta_3\text{log}(S_i)}
\end{equation}

\begin{center}
-or equivalently as- 
\end{center}

\begin{equation} \label{eq:27}
\text{log}(-\text{log}(1-p_i))=\beta_0+\beta_1\text{log}(O_i)+\beta_2\text{log}(V_i)+\beta_3\text{log}(S_i)
\end{equation}

The left-hand side of equation \ref{eq:27} is the complementary log-log link function, which has similar properties to the more common logit function of logistic regression. However, we use the complementary log-log function here, because it relates more clearly to the rate of collisions (\ref{eq:25}), and to our conceptual risk model (\ref{eq:21}). Thus: 

\begin{equation} \label{eq:28}
\text{cloglog}(p_i)=\beta_0+\beta_1\text{log}(O_i)+\beta_2\text{log}(V_i)+\beta_3\text{log}(S_i)
\end{equation}

Information on species occurrence, traffic volume and traffic speed are unlikely to be available for every place $i$ and we propose that this can be modelled. Several transportation modelling methods exist for estimating traffic patterns: generalised linear modelling \citep{seav00,zhao01}, neural network analysis \citep{dudd13}, empirical Bayes estimation \citep{yang02}, universal kriging \citep{eom06,selb13}, and support vector machines \citep{cast09}. Modelling approaches to estimate species occurrence include correlative \citep{guis05,elit09} and mechanistic \citep{kear09} species distribution modelling (SDM) and extensions of population viability analysis (PVA) \citep{gilp86,shaf90}.

We apply our framework to study vehicle collisions with kangaroos in Australia using species distribution modelling to estimate kangaroo occurrence and linear regression to estimate traffic volume and speed (\Cref{egk_framework}).

\begin{figure*}[htp]
  \centering
  \includegraphics[scale=1]{egk_fig01.png}
  \caption[]{Diagram of modelling framework. Three sub-models are used to generate covariates used in the collision model per the ``risk equals exposure multiplied by hazard" analytical framework.}
  \label{egk_framework}
\end{figure*}
 
\subsection{Study Area \& Species}

We selected the State of Victoria in south-east Australia as a study area as its diversity across its area of 227,819 square kilometres (Australian Bureau of Statistics, 2011) provides a good platform to illustrate the framework. Our study combines all sealed roads within the state (approx. 150,000 kilometres) and predicts collision risk across six motorway class types. To organise our spatial data, we overlaid a spatial grid of one square kilometre resolution on the study area. To produce modelling units for the collision model, we further segmented the roads by intersecting all roads and the spatial grid. The open-source software package ’R’ \citep{rdct16} was used to perform all spatial and statistical analyses.

We used the native species Eastern Grey Kangaroo (Macropus giganteus, Shaw, hereafter referred to as “grey kangaroo”) as the case study species. In Victoria, the grey kangaroo is the most abundant of the macropod family and frequently involved in wildlife-vehicle collisions. Between 2005 and 2013, over 600 incidents were reported to the Victorian Police and documented in the VicRoads crashstats database \citep{vicr09}. Actual incident rates, however, are much higher as the largest Victorian wildlife organisation, Wildlife Victoria, received over 5000 reports of grey kangaroo-vehicle collisions over the same period (Wildlife Victoria, 2014). Grey kangaroos are the second largest native terrestrial mammal in Australia and share many similar characteristics and management issues with ungulates found in North America and Europe \citep{crof04,coul10}. Moreover, there is a research gap involving collision modelling of single macropod species in Australia \citep{bond14}.

\subsection{Species Occurrence Sub-Model}

We downloaded survey records of grey kangaroos from the Victorian Biodiversity Atlas (VBA) occurring in the period 2000-2014, and spatial accuracy within 500 metres \citep{depi16}. The presences were both systematic targeted surveys and incidental sightings from data maintained by the Arthur Rylah Institute, a division of the Victorian Department of Environment, Land, Water and Planning.  Kangaroos are generalists and widely distributed; it was therefore assumed that although abundance may fluctuate based on drought conditions, distribution would not change significantly from the year 2000.  We used data from this period because it improved the sample size for occurrence modelling.  We derived background data by randomly sampling 10,000 points across the entire study area.  After sampling values from the covariate rasters and eliminating null values, the final dataset included 901 species presence and 9957 background observations.

\begin{figure*}[htp]
  \centering
  \includegraphics[scale=1]{egk_fig02.png}
  \caption[]{Predicted relative likelihood of grey kangaroo presence in study area. Darker shades indicate higher relative probabilities of occurrence (mean: 0.057; range: 0.002-0.986).}
  \label{egk_occmap}
\end{figure*}

To estimate occurrence of grey kangaroos, we used Boosted Regression Trees (BRT) \citep{frie02}. BRT modelling offers advantages of handling different types of predictor variables, accommodating missing data and outliers, fitting complex non-linear relationships, and incorporating interaction effects between predictors \citep{elit08}. We selected a tree complexity of five (limit on number of terminal nodes per tree used to include potential interactions) and a learning rate of .005 (contribution of each tree to the model). Classification methods have an established use in studies of species distributions \citep{walk90,skid96}, however, our framework is not limited to any particular modelling method. We selected seven predictors (\Cref{egk_variables}) based on the biology and behaviour of grey kangaroos. All of the species occurrence sub-model predictor variables were below a pairwise correlation threshold of 0.75 to reduce potential effects of multicollinearity. We predicted relative likelihood of grey kangaroo occurrence from the model fit at a one square kilometre resolution across Victoria (\Cref{egk_occmap}).

\begin{table}[htp]
\caption{Variables used in statistical models.}
%\renewcommand{\arraystretch}{0.8}
\begin{tabularx}{\textwidth}{llY} \toprule
Model &Variable    &Definition\\ \midrule 
\multirow{8}{*}{\rotatebox[origin=c]{90}{Species Occurrence}}   &KANG        &presences and psuedo-absences of Eastern Grey Kangaroos\\
           		 &ELEV        &elevation of terrain in meters above sea level\\
           		 &GREEN       &remote-sensed mean seasonal change in greenness (2003-2013) in vegetation\\
           		 &LIGHT       &remote-sensed relative artificial light intensity\\
           		 &MNTEMPWQ    &mean temperature of wettest quarter in $^{\circ}C$\\
           		 &PRECDM      &precipitation of driest month in millimeters\\
           		 &SLOPE       &slope of terrain in decimal percent rise\\
           		 &TREEDENS    &tree canopy coverage within 1 square kilometre in decimal percentage\\
&&\\
\multirow{6}{*}{\rotatebox[origin=c]{90}{Traffic Volume}}   &AADT        &average annual daily traffic counts per road segment\\
           		 &KMTODEV     &distance in kilometres to commercial planning zones\\
           		 &KMTOHWY     &distance in kilometres to major road segments (freeways and highways)\\
           		 &POPDENS     &2011 population divided by area in square kilometres\\
           		 &RDCLASS     &road class category (``freeway'', ``highway'', ``arterial'', ``sub-arterial'', ``collector'' or ``local'')\\
           		 &RDDENS      &total lengths in kilometres of road segments within 1 square kilometre\\
&&\\
\multirow{3}{*}{\rotatebox[origin=c]{90}{Traffic Speed}} &&\\
    			 &SPEEDLMT    &posted speed limit per road segment	\\
           		 &RDCLASS     &road class category (see above)\\
           		 &RDDENS      &total lengths in kilometres of road segments within 1 square kilometre\\
&&\\
&&\\
\multirow{4}{*}{\rotatebox[origin=c]{90}{Collision}}  		 &COLL        &presences and psuedo-absences of grey kangaroo-vehicle collisions\\
           		 &EGK         &predicted relative likelihood of kangaroo presence\\
           		 &TVOL        &predicted traffic volume (number of vehicles per day) per road segment\\
           		 &TSPD        &predicted posted traffic speed (kilometres per hour) per road segment\\
\bottomrule
\end{tabularx}
\label{egk_variables}
\end{table}

\subsection{Traffic Volume \& Speed Sub-Models}

\begin{figure*}[htp]
  \centering
  \includegraphics[scale=1]{egk_fig03.png}
  \caption[]{Predicted relative traffic volume in number of vehicles day-1 per road segment in study area. Darker shades indicate higher predicted traffic volumes (mean: 4481; range: 274-60850).}
  \label{egk_tvolmap}
\end{figure*}

Average Annual Daily Traffic (AADT) represents the sum of traffic travelling in both directions which pass a roadside observation point during a full year divided by 365 days for a given road segment. AADT volume is usually only available for major road segments and we did not have data for most local, collector and sub-arterial roads under municipal district control. We predicted volume estimates for all road segments in the study area with Random Forest (RF) regression \citep{brei01}. The dependent variable was 2013 AADT recorded by VicRoads on 3174 road segments. We included seven predictor variables (\Cref{egk_variables}) that related to processes in traditional four-step traffic demand modelling (trip generation, trip distribution, mode choice and route assignment). All of the traffic model predictor variables were below a pairwise correlation threshold of 0.7 to reduce potential effects of multicollinearity. The traffic volume sub-model used the log-link function on the dependent variable (\Cref{egk_models}) due to the approximate log-normal distribution of AADT. We predicted AADT to all road segments using the model fit (\Cref{egk_tvolmap}).

\begin{figure*}[htp]
  \centering
  \includegraphics[scale=1]{egk_fig04.png}
  \caption[]{Predicted relative traffic speed in kilometres hour-1 per road segment in study area. Darker shades indicate higher predicted traffic speeds (mean: 62; range: 42-106).}
  \label{egk_tspdmap}
\end{figure*}

Traffic speed was modelled and predicted using a similar methodology to the traffic volume model. As with the traffic volume data, we did not have access to municipal records, however, we required speed values for all road segments across Victoria. We obtained 2014 posted speed limit data for all major road segments (n=42,439) and used road density and road class predictors (\Cref{egk_variables}) in a linear regression model (\Cref{egk_models}) to predict speed for all road segments (\Cref{egk_tspdmap}). 

\subsection{Collision Modelling}

To produce a kangaroo-vehicle collision dataset, we obtained records from the Wildlife Victoria database for a three year period between 1 January, 2010 and 1 January, 2013. The data for collisions were selected from a period where consistent techniques were used to collect and record the data.  Pre-2010 records exist, however, are sparse and more prone to error.  We first verified all automatically geocoded coordinates for accuracy using an online latitude/longitude mapping system \citep{gpsv13} and excluded all records with a geographical accuracy exceeding 300 metres. We selected road segments that were closest to the collision records in space and coded them with ones. To produce background points, we randomly selected approximately twice the number of collision-coded road segments and coded them with zeros. We combined the collision and background segments to produce a final dataset of 2264 collision and 4489 background segments. Using the same methodology, we developed an additional dataset of road segments for a period of one year between 1 January, 2013 and 1 January, 2014 for model validation (2125 collision and 4212 background records). Each segment contained predicted values for both traffic speed and volume from the previously described sub-models. As species occurrence predictions were expressed across a one square kilometre raster grid, we used the mid-point of each road segment for sampling the species occurrence sub-model predictions.

Our collision model fitted predicted values from the sub-models to collision or background occurrences at each road segment (\Cref{egk_models}). We used the fitted collision model to predict relative probabilities of collision on all road segments in the study area and, by using symbol classification (colour and line thickness) in Quantum GIS \citep{qgis09}, produced a map (\Cref{egk_collmap}).

\begin{figure*}[htp]
  \centering
  \includegraphics[scale=1]{egk_fig05.png}
  \caption[]{Map of collision risk per road segment. Darker shades indicate higher relative risk of collisions with kangaroos (mean: 0.24; range: 0.01-0.99).}
  \label{egk_collmap}
\end{figure*}

To determine the effectiveness of partitioning the variables into logical sub-models, we also modelled collision risk with a combined set of all variables (\Cref{egk_models}). That is, we analysed a logistic regression model that related collisions directly to the original variables used to model kangaroo occurrence, traffic volume and traffic speed. This emulated past work for purposes of comparison. We also compiled a list of variables used in 74 collision modelling papers and verified that nearly all of our predictors were used in at least five or more published studies. The remaining highly-used variables were either irrelevant to our scope or difficult to obtain.

\section{Results}

The species occurrence model predicted the relative likelihood of kangaroo presence to vary (.002 to .986) across the study area using 5400 regression trees. The deviance explained by the model was approximately 30.4\% (null deviance = 0.572, estimated cross-validation (CV) deviance = 0.398 +/-0.008, CV AUC = 0.878 +/-0.006; see \Cref{egk_models}). The three most influential variables were artificial light (19.6\% contribution to model), elevation (18.4\%), and precipitation of the driest month (14.8\%). Partial dependence plots demonstrated plausible relationships between the predictors and grey kangaroo presence (see Figure S1 in Supporting Information). We extended our model predictions to continental Australia (including areas well beyond Victoria) which aligned well with known grey kangaroo range (see Figure S2). Approximately fifty percent of the grey kangaroo collision records were within areas where the predicted probability of grey kangaroo occurrence was above 0.2. Collision records in areas of lower predicted occurrence areas may be due to misclassification errors (i.e. mis-identification of the species of kangaroo), reporting bias/spatial error in the collision data, or sampling bias in the data used to train the kangaroo occurrence model.

\begin{table}[htp]
\caption{Statistical models used in framework.}
\begin{tabularx}{\textwidth}{lYll} \toprule
Model Type		    &Model &Reduction in Error	&ROC\\ \midrule 
Species Occurrence  &\scriptsize{$\Pr(\text{KANG}=1) \approx \mathrm{logit}^{-1}(\beta_0 + \beta_1\text{ELEV} + \beta_2\text{GREEN} + \beta_3\text{LIGHT} + \beta_4\text{MNTEMPWQ} + \beta_5\text{PRECDM} + \beta_6\text{SLOPE} + \beta_7\text{TREEDENS})$}        &30.4\%		&0.88\\
           &                                                                                &&\\
Traffic Volume    &\scriptsize{$\log(\text{AADT}) \approx \beta_0 + \beta_1\text{KMTODEV} + \beta_2\text{KMTOHWY} + \beta_3\text{POPDENS} + \beta_4\text{RDDENS} + \beta_5\text{RDCLASS}$} &54.4\% 	&--\\
           &                                                                                &&\\
Traffic Speed    &\scriptsize{$\text{SPEEDLMT} \approx \beta_0 + \beta_1\text{RDCLASS} + \beta_2\text{RDDENS}$}          &58.7\%	&--\\
           &                                                                                &&\\
Collision  &\scriptsize{$\mathrm{cloglog}(\Pr(\text{COLL}=1)) \approx \beta_0 + \beta_1\ln (\text{EGK}) + \beta_2\ln (\text{TVOL}) + \beta_3\ln (\text{TSPD})$}        &23.7\%	&0.81\\
           &                                                                                &&\\
Alternative Collision  &\scriptsize{$\mathrm{cloglog}(\Pr(\text{COLL}=1)) \approx \beta_0 + \beta_1\text{ELEV} + \beta_2\text{GREEN} + \beta_3\text{KMTODEV} + \beta_4\text{KMTOHWY} + \beta_5\text{LIGHT} + \beta_6\text{MNTEMPWQ} + \beta_7\text{POPDENS} + \beta_{8}\text{PRECDM} + \beta_{9}\text{RDCLASS} + \beta_{10}\text{RDDENS} + \beta_{11}\text{SLOPE} + \beta_{12}\text{TREEDENS}$}        &24.9\%	&0.84\\
\bottomrule
\end{tabularx}
\label{egk_models}
\end{table}

The traffic model volume explained 54.4\% of the variation in the AADT data and the traffic speed model explained 58.7\% of the variation in the posted speed limit data (\Cref{egk_models}).  Each model used the default value of 500 trees to fit the data.  All predictor variables used in the traffic models demonstrated plausible relationships to both AADT and speed (see Figure S3); traffic volume decreased with distance to activity centres and major thoroughfares and increased with affluence. Signs of the coefficients, indicating direction of relationships between the dependent variable and predictors, were as expected (\Cref{egk_sum_coll}). The most influential variable for traffic volume was road class (33.8\% relative contribution to reduction of variance), followed by distance to urban development (19.7\%) and distance to freeways and highways (16.4\%). Both traffic speed variables were significant.

\begin{table}[htp]
\caption{Summary of collision model fit. Coefficients and significance of variables are shown with relative contribution to model fit. Highly significant variables are marked with an asterisk. Anova contribution of variables are expressed as decimal percent reduction in deviance.}
\begin{tabularx}{\textwidth}{lllllll} \toprule
Model Type &Variable         &Coefficient &Std. Error			&Wald-{\large$\chi$}$^2$ &{\normalsize$P$} &ANOVA\\ \midrule 
Collision  				&\emph{Intercept}	& -12.82			& 0.6635   	& -19.33	&\TL 2e-16*	& -- \\
           				& EGK				& 0.6583			& 0.0206   	& 32.01		&\TL 2e-16*	& 0.7268 \\
           				& TVOL				& 0.2715			& 0.0252   	& 10.77		&\TL 2e-16*	& 0.0005 \\
           				& TSPD				& 2.694				& 0.1308   	& 20.59		&\TL 2e-16*	& 0.2726 \\
           				&                  	&                 	&         	&         	&       	& \\
Alternative Collision	& \emph{Intercept} 	& -2.567 			& 0.2642	& -9.716 	&\TL 2e-16*	& -- \\ 
   						& ELEV				& 0.003177			& 0.0002233	& 14.23 	&\TL 2e-16*	& 0.1729 \\ 
   						& GREEN				& 1.405				& 0.344 	& 4.085		& 4.42e-05*	& 0.0011 \\ 
   						& KMTODEV			& -0.02784			& 0.002097 	& -13.28 	&\TL 2e-16*	& 0.2079 \\ 
   						& KMTOHWY			& 0.006001			& 0.004206 	& 1.427 	& 0.1537 	& 0.0004 \\ 
   						& LIGHT				& 0.004142			& 0.002046 	& 2.025 	& 0.043 	& 0.0119 \\ 
   						& MNTEMPWQ			& 0.1519			& 0.01673 	& 9.077 	&\TL 2e-16*	& 0.0398 \\ 
   						& POPDENS			& -0.0006986		& 0.0000535	& -13.06 	&\TL 2e-16*	& 0.0922 \\ 
   						& PRECDM			& 0.02573			& 0.003404 	& 7.559 	& 4.05e-14*	& 0.0483 \\ 
   						& RDCLASS			& -0.4166			& 0.01467 	& -28.4 	&\TL 2e-16*	& 0.4205 \\ 
   						& RDDENS			& 0.02353			& 0.008762 	& 2.686 	& 0.0072 	& 0.0038 \\ 
   						& SLOPE				& 0.008252			& 0.00739 	& 1.117 	& 0.2641	& 0.0004 \\ 
   						& TREEDENS			& -0.1761			& 0.1387 	& -1.27 	& 0.2041	& 0.0008 \\ 
\bottomrule
\end{tabularx}
\label{egk_sum_coll}
\end{table}

The collision model explained 23.7\% of the deviance (\Cref{egk_models}). Using the independent dataset to verify the predictive accuracy of the model resulted in an ROC score of 0.81.  All three variables were highly significant (p \textless 2e-16) with traffic speed and grey kangaroo occurrence contributing the most to overall reduction in deviance; 27.3\% and 72.7\%, respectively.  The Akaike Information Criterion (AIC) score – measuring the quality of the model given the data and the parameters - was 6579.  All predictor variables demonstrated logical relationships to collision likelihood in the partial dependency plots (\Cref{egk_effects}). The rapid ascent and gradual levelling off of collision risk to increasing traffic volume suggests a threshold of around 2000 vehicles day$^{-1}$.

\begin{figure*}[htp]
  \centering
  \includegraphics[scale=.7]{egk_fig06a.png}
  \includegraphics[scale=.7]{egk_fig06b.png}
  \includegraphics[scale=.7]{egk_fig06c.png}
  \caption[]{Effects of predictor variables on relative likelihood of collision. EGK is the relative likelihood of kangaroo occurrence. TVOL is the predicted daily traffic volume in vehicles day$^{-1}$. TSPD is the predicted traffic speeds in kilometres hour$^{-1}$.}
  \label{egk_effects}
\end{figure*}

Our alternative collision model (all sub-model predictor variables combined in a single model) explained 24.9\% of the deviance; less than 2\% more than the four-model framework.  The model AIC was 6496.2. The most influential variables were road class (42.1\% relative contribution to reduction of deviance), distance to urban development (20.8\%), elevation (17.3\%) and population density (9.2\%). We used the same independent dataset to verify the predictive accuracy of the model, resulting in an ROC score of 0.84.

\section{Discussion}

This research introduces a new conceptual framework for predicting risk of wildlife-vehicle collisions. It is distinct from other collision models in its treatment of the analytical modelling framework. Where other research has related multiple environmental and anthropogenic variables to collision occurrence in a single statistical model (e.g. \cite{lee04}; \cite{ramp05}; \cite{kloc06}; \cite{litv08}; \cite{guns09}; \cite{roge09}) and predicted risk from model fits (e.g. \cite{malo04}; \cite{sudh09}; \cite{guns12}), we separate and develop the predictors in sub-models that would most suitably inform management. Each sub-model may be independently scrutinised for bias, uncertainty, and spatial autocorrelation and tuned accordingly. In this case study, our approach identifies relationships between, and effects of, species presence and traffic volume/speed on collision risk to grey kangaroos. In a similar manner, \cite{baud13} developed an index of co-occurrence to assess collision risk between manatees and recreational watercraft, however, the management implications were not extensively discussed. Particular to our study, road managers and environmental managers may be interested in whether to reduce collisions by focusing on the road environment (e.g. \cite{clev01}; \cite{jaeg04}; \cite{bond08,bond13}), traffic conditions, or species (e.g. \cite{huij03}; \cite{huij06}). Both the collision and alternative collision models had similar fits and made similar predictions. However, we argue that cause and effect is easier to interpret when using our proposed framework. For example, the variable of population density (human) is shown to contribute to collisions, however, it is unclear to what degree it is correlated with grey kangaroo occurrence, traffic conditions, or both, in the alternative collision model.

The models used in the study were all correlative and assumed static equilibrium in the environment. Temporal patterns of wildlife-vehicle collisions exist and would be useful to incorporate into the collision model, but the variable of time was not easily integrated into this study. The original collision dataset indicated that the lowest number of incidents occurred in summer (December-February) and the highest were reported in August. The times of highest density of reports were consistent with the crepuscular (active between dusk and dawn) nature of kangaroo movements \citep{mccu00} and peaked at approximately 7:00 and 17:00 hours. Comparing the model performance on data accounting for time of day and of year is an area for future research. Other modelling methods that explicitly address interactions between space and time exist, such as multi-dimensional Poisson process models, and would be useful to incorporate into the framework.

This study demonstrates the usefulness of existing sources of data for scientific research and environmental management. Data provided by non-government organisations (NGOs) such as Wildlife Victoria, are not only inexpensive to collate, but are also valuable ecological indicators and sources of information, in particular, on species distributions. However, it is cautioned that use of such data should be subject to rigorous quality control and verification. A large amount of entered records were not useful for the scale of this study due to incomplete data or geographical ambiguity. Fauna atlases have the same potential issues. Many of the records in the Victorian Biodiversity Atlas (VBA) were subject to geographical bias (e.g. close to roads and towns) and some potential inaccuracies (e.g. mis-identification of species or approximate location). Graham et al. (2004) elaborates on the use of such data in scientific studies. Moreover, testing the model with alternative sources of collision reports (e.g. insurance records) would help account for biases present in the data. Although under-reporting of collision data has limited effects on model robustness, spatial biases can adversely affect model performance \citep{snow15}.

Our statistical modelling methods were chosen to match the data and purpose (as suggested by \cite{wint05} and \cite{guil15}), and are based on well-established analyses in the modelling literature. However, our framework is not restricted to the particular models that we built; any appropriate statistical methods may be used for each of the sub-models. For example, boosted regression trees are a well developed method for species distribution modelling \citep{elit08} but may also be suited to predict traffic volume and speed. Further exploration of modelling methods, including those based on machine learning, may result in better calibrated models with less uncertainty and more robust inferences. Moreover, the use of mechanistic models to explain population dynamics (requiring collection of additional information such as age, sex, and size of species involved in wildlife-vehicle collisions) may be beneficial as the integration of Population Viability Analysis (PVA) into species distribution and collision models can be informative \citep{tyre01,elit10,pola14}.

All of the models with binary dependent variables were subject to spatial autocorrelation producing potentially biased standard errors and predictions of grey kangaroo presence and collision risk. Techniques to incorporate autocovariates in the models similar to \cite{cras12} might improve the predictive performance but may only be logical for particular species (e.g. ranging or nomadic species). Further, we chose to test the model framework using a binomial dependent variable, however, it can be adapted to analyse other data types such as non-negative integers (e.g. number of collisions) or categorical information (e.g. groupings of number of collisions). These alternative methods might usefully identify hotspots (i.e. sites with multiple collisions) at smaller scales such as road segments or simulating effects of planned road expansion projects. For example, population effects on target species can be quantified and estimated based on predicting the magnitudes of collision hotspots.
 
\subsection{Applied Management Implications}
To realise the full potential use by managers, a model framework must be conceptually simple, flexible and adaptable. All of the input data must be accessible and the framework must allow inferences which are relevant and draw conclusions which are tractable. Our study uses vehicle collisions with grey kangaroos in Australia to demonstrate this analytic framework, however, we envision extending the model to (i) explore and predict risk to other species (e.g. wombats or deer) arising from vehicle collisions by identifying and quantifying probabilities of occurrence using alternative species distribution models and (ii) quantify risks to wildlife arising from other anthropogenic threats such as linear infrastructure (e.g. electrocutions), pollution (e.g. entanglements) or introduction of domestic predators (e.g. dog and cat attacks). Managers can use our framework across several disciplines, and it would benefit from additional testing across several spatial and temporal scales to determine its full potential flexibility and generality.
\chapter{A conceptual risk framework for modelling and predicting wildlife-vehicle collisions}\label{sec:egk}


%\chapter{Predicting road collision risk for six native Australian mammals in Victoria}\label{sec:6sp}
\newpage

\begin{localsize}{10}
\section*{\centering Abstract}

The occurrence and rate of wildlife-vehicle collisions is related to both anthropocentric and environmental variables, however, few studies compare collision risks for multiple species within a model framework that is adaptable and transferable.  Our research compares collision risk for multiple species across a large geographic area using a conceptually simple risk framework.

We used six species of native terrestrial mammal often involved with wildlife-vehicle collisions in south-east Australia.  We related collisions reported to a wildlife organisation to the co-occurrence of each species and a threatening process (presence and movement of road vehicles). For each species, we constructed statistical models from wildlife atlas data to predict occurrence across geographic space. Traffic volume and speed on road segments (also modelled) characterised the magnitude of threatening processes.

The species occurrence models made plausible spatial predictions. Each model reduced the unexplained variation in patterns and distributions of species between 29.5\% (black swamp wallaby) and 34.3\% (koala). The collision models reduced the unexplained variation in collision event data between 9.1\% (black swamp wallaby) and 19.4\% (common ringtail possum), predictor variables correlating similarly with collision risk across species.

Road authorities and environmental managers need simple and flexible tools to inform projects. Our model framework is useful for directing mitigation efforts (e.g. on road effects or species presence), predicting risk across differing spatial and temporal scales and target species, inferring patterns of threat, and identifying areas warranting additional data collection, analysis, and study.

\end{localsize}

\newpage
\section{Introduction}

Roads have considerable negative ecological impacts, prompting numerous scientific studies and mitigation projects over the past five decades \citep{form98,spel98,rvdr15}. The mortality of wildlife from collisions with vehicles is a significant problem throughout most of the developed world, and will likely become more so in the next few decades in the developing world \citep{rvdr15}. Wildlife-vehicle collisions (WVC) are estimated to kill billions of fauna annually on transportation networks \citep{seil06}, fostering an interdisciplinary management problem. To address this problem, studies seek to understand the frequency, magnitude and distribution of WVC.  Through this understanding, managers can identify and apply appropriate mitigation strategies.

WVC are financially costly \citep{biss08b,huij09,rowd08} and thus knowing where and how to mitigate is important.  If mitigation measures are not appropriately specified or located in the landscape, costs arise from both wasted installation labour, materials and time, and on-going collisions resulting from the omission.  Moreover, some forms of mitigation, such as fencing, also create "barrier-effects" - a direct impediment to the movement of many species which affects movement and gene flow \citep{epps05}.  Thus, strategic, informed use of mitigation is important for both conservation and road safety objectives.

The literature contains many studies on WVC with several papers utilising statistical modelling methods.  Quantitative models are a useful way to support decision-making for managers by helping to clearly organise problems, test inputs, and make inferences \citep{ande15}.  By determining the factors that contribute to collisions with quantitative analysis, managers can begin to understand relationships and optimise mitigation strategies.

We view two challenges in current WVC modelling and management practices.  First, methods that generalise to multiple species and across taxonomic groups are under-represented in scientific studies (see Farmer and Brooks, 2012).  This is likely due to the complexity in resource requirements for different taxa.  Another challenge is identifying relationships between predictors and collisions that are able to be mitigated.  Although management implications are highlighted in many studies, it is not always clear how the results of the analyses could be easily applied due to variable interactions and confounding effects in the model predictors \citep{guns11}.  Some studies explicitly suggest mitigation that relates directly to environmental variables based on model results (see \cite{gril09}).  Although these recommendations are useful in the specific contexts of the studies, they may not generalise to varying spatial scales - specifically large areas. We argue that generality in methods is useful and it is important to create analytical tools that help managers identify risk to both individual species and taxonomic groups across large areas with predictors that are manageable or suggest areas for more specific examination.

In this study we have two objectives: testing a conceptual risk model framework and analysing patterns of WVC for multiple species.  We first model wildlife-vehicle collisions by expanding upon the conceptual analytical framework in \Cref{sec:egk} that relates risk to exposure (species occurrence) and hazard (traffic volume and speed).  As this model framework is able to quantify risk over large spatial scales (demonstrated on a large region of south-east Australia as a case study), we apply it to six Australian terrestrial mammal species commonly involved in WVC to test its flexibility and potential to support management decision-making.  We analyse model outputs to determine the relationships of common factors contributing to wildlife-vehicle collisions and make predictions of risk to identify/prioritise areas requiring mitigation.  As the need for and location of WVC mitigation is often derived using species characteristics and movements (e.g. \cite{clev02}), we use the model framework to determine the importance of species occurrence in collision risks on road segments.  Ignoring variables that influence habitat preferences of species may lead to less robust inference by managers \citep{roge09}.  Where occurrence is influential for multiple species, managers may decide to control animal presence on or near the roads and this may involve a mixture of mitigation strategies (see \cite{beck10}).  Likewise, if traffic speed or volume is a significant driver of collisions for many species, managers may consider control mechanisms such as speed enforcement or alternative transportation planning.

\section{Materials \& Methods}

\subsection{Study Area}

We used the 227,819 square kilometres state of Victoria in south-east Australia as a study area (\Cref{6sp_study_area}). The Victorian road authority (VicRoads) manages 25,256 kilometres of major roadway within the state. The remaining sealed roads (approx. 125,000 kilometres) are controlled by seventy-nine municipal districts. Our study analysed collision risk across 147,970 kilometres of sealed roads. To organise our spatial data and modelling, we overlaid a spatial grid of 1km$^2$ resolution (extents: -58000,5661000 x 764000,6224000, projection: GDA94 MGA zone 55, number of cells: 462,786) on the study area. Each grid cell was the modelling unit for species occurrence. All roads in the study area were bisected by the grid resulting in road segments that were approximately one kilometre or less in length. We used these 612,791 segments as our modelling units for the collision model.

\begin{figure*}[htp]
  \centering
  \includegraphics[scale=.5]{6sp_study_area.png}
  \caption[Wildlife-vehicle collisions in Victoria]{Study area (state of Victoria, in south-east Australia) showing location of wildlife vehicle collisions. Light gray lines are the sealed road network and black crosses are locations of reported wildlife-vehicle collisions. Major urban centres are starred.}
  \label{6sp_study_area}
\end{figure*}

\subsection{Study Species}

We obtained records of WVC from the Wildlife Victoria database (see 'Collision Model' section below), and selected the species of mammal most frequently recorded as roadkill (\Cref{6sp_species_data}). Only mammals with at least 80 reported collision events were selected for use in the study, resulting in six study species.  Eastern grey kangaroos (\textit{Macropus giganteus}, Shaw) are the second largest mammal (up to 85 kilograms for males) in Australia and share many management issues regarding abundance with ungulates found in North America and Europe \citep{crof04,coul10}. They occur in groups and have home range sizes between 25 and 125 hectares \citep{daws12}.  Black swamp wallabies (\textit{Wallabia bicolor}, Desmarest) are medium-sized (13 to 17 kilograms), solitary mammals more often found at higher elevations and in areas of denser foliage \citep{vand08}. Common wombats (\textit{Vombatus ursinus}, Shaw) are medium-sized (22 to 39 kilograms), burrowing mammals and one of three wombat species in Australia \citep{vand08}. Although wombats occasionally share burrows, they are typically solitary animals and occupy home range sizes of about 20 hectares. Common possums, brushtail (\textit{Trichosurus vulpecula}, Kerr) and ringtail (\textit{Pseudocheirus peregrinus}, Boddaert), are small arboreal mammals (one to four kilograms) that are often more abundant throughout, Victorian urban and suburban environments than other arboreal mammals.  Koalas (\textit{Phascolarctos cinereus}, Goldfuss) are medium-sized (eight to 15 kilograms), arboreal mammals.  They are mainly sedentary due to the exclusive diet of Eucalyptus leaves which have low caloric content and nutritional value and are toxic to many other species \citep{vand08}.

\subsection{Conceptual Model Framework}

We employed the quantitative risk framework in \Cref{sec:egk} to examine how collision risk of each species relates to occurrence of the species, traffic volume, and traffic speed on all road segments. Using the open-source software package `R' version 3.3.0 \citep{rdct16} to perform all statistical analyses, we developed species distribution models (SDM) to predict occurrence across the study area for each of the six species, and linear regression models to predict traffic volume and speed. Predicted traffic volume and speed values for all road segments were modelled by regressing annual average daily traffic (AADT) counts and posted speed limit data on anthropogenic variables using random forests; detailed methods are provided in \Cref{sec:egk}. 

\begin{table}[htp]
\caption[Six mammal species most frequently reported in wildlife-vehicle collisions]{Top six species most frequently reported in wildlife-vehicle collisions in the Wildlife Victoria database (WVD) between the years of 2010 and 2014. Species observations from the Victorian Biodiversity Atlas were used for the species occurrence models and recorded collisions from the WVD for the collision risk models. The datasets used in the models are comprised of both presence (P) and background (B) observations. Note, some species occurrence modelling datasets incorporate less than 10,000 background observations due to reductions in spatial duplicates - i.e., presence and background grid cells sharing the same coordinates were reduced to single presence observations.}
\centering
\begin{tabularx}{0.9\textwidth}{llll} \toprule
Species Name                     &Species Code     &Species Occurrence Model Data     &Collision Model Data \\ \midrule 
Eastern Grey Kangaroo 	& EGK	& 10508 : 703(P) / 9805(B) 	& 612791 : 1344(P) / 611447(B) \\ 
Common Brushtail Possum & BTP	& 10808 : 1021(P) / 9787(B)	& 612791 : 257(P) / 612534(B) \\ 
Common Ringtail Possum 	& RTP	& 10590 : 791(P) / 9799(B)	& 612791 : 212(P) / 612579(B) \\ 
Black Swamp Wallaby 	& BSW	& 10548 : 755(P) / 9793(B) 	& 612791 : 108(P) / 612683(B) \\ 
Common Wombat 			& WOM	& 10330 : 515(P) / 9815(B) 	& 612791 : 156(P) / 612635(B) \\ 
Koala 					& KOA 	& 10304 : 489(P) / 9815(B) 	& 612791 : 81(P) / 612710(B) \\ 
\bottomrule
\end{tabularx}
\label{6sp_species_data}
\end{table}

\subsection{Species Occurrence Models}

We obtained observation records of the six study species from the Victorian Biodiversity Atlas (VBA) that satisfied the following criteria: survey date between 1 January 2000 and 31 December 2013 and spatial coordinate certainty of $\leq$500 metres \citep{depi16}.  As our occurrence models are correlative, we grouped the records based on identical spatial coordinates regardless of observation dates; thus multiple observations were aggregated to single presence observations in space. For each of the six study species, we selected all 1km$^2$ grid cells that contained at least one occurrence record to represent presences across the study area. This ensured that we maintained at least 1000 meters between all observations. As we did not have access to recorded absence data, we randomly selected 10,000 1km$^2$ grid cells (without replacement) within study area as background data and combined them with presence data (\Cref{6sp_species_data}). Note, any spatially duplicated presence and background grid cells were reduced to single presence observations; therefore, some datasets may incorporate less than 10,000 background observations.

\begin{table}[htp]
\caption[Predictor variables used in species occurrence models and collision risk models]{Predictor variables used in species occurrence (SO) models and collision risk (CR) models. The spatial coordinates of centroids for grids with species presences and 10,000 randomly selected background grids were used to sample from 1km$^2$ resolution predictor variable grids for occurrence models. The mid-points of road segments were used to sample from 1km$^2$ resolution occurrence model predictions. Note, reported means and ranges are for entire study area.}
\centering
\begin{tabularx}{0.9\textwidth}{lYll} \toprule
Variable       &Description                                               &Units          &Range : Mean\\
\midrule 
ELEV (SO)       &Elevation of Terrain Above Sea Level                     &$m$			  &-0.39-1890.66 : 252.83\\
GRASS (SO)      &Mean Annual Grass Growth \citep{cart03}	              &$kg ha^{-1}$   &0-15.74 : 5.45\\
GREEN (SO)      &Mean Seasonal Change in Vegetation Greenness             &--             &-.62-.68 : 0.21\\
HYDRODIST (SO)  &Distance from Major Waterway                             &$km$           &0-95 : 10.09\\
HYDROFLOW (SO)  &Hydrological Flow Accumulation                           &--             &1-32767 : 197.95\\
ISOTHERM (SO)   &Isothermality                                            &$^{\circ}C*10$ &0-53 : 48.40\\
LIGHT (SO)      &Remote-sensed Artificial Light Intensity                 &--             &0-63 : 1.23\\
MNTEMPWQ (SO)   &Mean Temperature of Wettest Quarter                      &$^{\circ}C*10$ &-8-162 : 102.18\\
PRECDM (SO)     &Precipitation of Driest Month                            &$mm$           &0-81 : 34.70\\
PRECSEAS (SO)   &Precipitation Seasonality                                &$mm$           &0-43 : 23.99\\
ROADDIST (SO)   &Distance from Roadway                                    &$km$           &0-36.55 : 2.02\\
SLOPE (SO)      &Slope of Terrain                                         &\%             &0-41.61 : 2.69\\
SOILBULK (SO)   &Mean Soil Bulk Density at 2.5cm Depth                    &$kg m^{-3}$       &0-1365 : 1129.13\\
SOILEXCH (SO)   &Mean Soil Cation Exchange Capacity at 2.5cm Depth        &$cmolc kg^{-1}$     &10-102 : 24.84\\
SOILSAND (SO)   &Mean Soil Texture Fraction Sand at 2.5cm Depth           &\%             &23-100 : 55.11\\
TEMPANRANGE (SO)&Temperature Annual Range                                 &$^{\circ}C*10$ &0-289 : 245.10\\
TOWNDIST (SO)   &Distance from Town Center                                &$km$           &0-46.10 : 5.76\\
TREEDENS (SO)   &Tree Cover                                               &decimal \%     &0-1 : 0.28\\
&&&\\
BSW (CR)        &Relative Probability of Wallaby Occ. per $km^2$          &--             &0.00-0.96 : 0.18\\
BTP (CR)        &Relative Probability of Brushtail Possum Occ. per $km^2$ &--             &0.00-0.99 : 0.20\\
EGK (CR)        &Relative Probability of Kangaroo Occ. per $km^2$         &--             &0.00-0.99 : 0.13\\
KOA (CR)        &Relative Probability of Koala Occ. per $km^2$            &--             &0.00-0.97 : 0.12\\
RTP (CR)        &Relative Probability of Ringtail Possum Occ. per $km^2$  &--             &0.00-0.98 : 0.16\\
TSPD (CR)       &Predicted Vehicle Speed per Road Segment                 &$km h^{-1}$         &20.36-107.55 : 61.41\\
TVOL (CR)       &Predicted Vehicle Volume per Road Segment                &$vehicles day^{-1}$ &0.29-47639.76 : 4309.27\\
WOM (CR)        &Relative Probability of Wombat Occ. per $km^2$           &--             &0.00-0.97 : 0.12\\
\bottomrule
\end{tabularx}
\label{6sp_variables}
\end{table}

We selected predictor variables which influenced the biology, behaviour, and characteristics of all species based on literature review, consultation with species experts, and ecological principles. We considered a range of variables that describe three strata that our six species occupy; the subsurface zone for burrowing animals, the ground plane for ranging species, and the canopy for tree-dwelling wildlife. Each predictor was represented as a 1km$^2$ raster grid and values were sampled using the respective grid centroid coordinates in the presence/background datasets for each species. To reduce effects of collinearity, we used a predictor set that exhibited Pearson correlation coefficients less than 0.70 for all pairs of explanatory variables. There were two exceptions; elevation and slope (0.71), which were expected to exhibit some collinearity, and tree density and precipitation of the driest month (0.72).

For each species, we fit Boosted Regression Trees (BRT) \citep{frie02} to the occurrence data using tools and methods by \cite{elit08}. These methods included the use of internal cross-validation to reduce over-fitting. BRT fit complex non-linear relationships and automatically incorporate interaction effects between predictors \citep{elit09}. We selected a tree complexity of five (limit on number of terminal nodes per tree used to regulate interactions), a learning rate of .005 (contribution of each tree to the model), and bag fraction of 0.5 (decimal percent of data values used to cross validate the model predictions). For each model, we employed a selection mechanism - based on \cite{elit08} - to determine the most useful and parsimonious combination of predictors (\Cref{6sp_final_var}) from a full set of 18 variables (\Cref{6sp_variables}).

\begin{table}[htp]
\caption[Predictors selected for six mammal species occurrence models]{Final predictors selected for each species in species occurrence models and relative importance in model expressed as percentage contribution to reduction in unexplained variation in the data. “---” indicates variable was not selected for use in the model and bold text signifies most influential variable per species.}
\centering
\begin{tabularx}{0.9\textwidth}{llllllll} \toprule
Predictor & EGK & BTP & RTP & BSW & WOM & KOA \\ 
\midrule
  ELEV & 9.73 & 7.67 & 4.72 & 7.52 & 7.25 & 6.86 \\ 
  GRASS & 8.33 & 6.64 & 5.44 & 6.6 & 6.32 & 10.34 \\ 
  GREEN & 6.71 & 7.27 & 5.68 & 8.54 & 7.05 & 7.98 \\ 
  HYDRODIST & 2.31 & 2.88 & 2.07 & --- & --- & --- \\ 
  HYDROFLOW & --- & 2.56 & --- & 2.6 & 3.22 & --- \\ 
  ISOTHERM & 2.49 & 2.82 & 3.09 & 2.7 & 2.29 & --- \\ 
  LIGHT & \B{14.85} & 10.19 & \B{13.44} & 7.58 & 8.18 & 5.37 \\ 
  MNTEMPWQ & 3.57 & 5.44 & 6.32 & 4.42 & 4.61 & 4.54 \\ 
  PRECDM & 7.13 & 7.49 & 7.98 & 4.41 & \B{14.73} & 7.06 \\ 
  PRECSEAS & 3.41 & 5.27 & 4.23 & 4.45 & 4.75 & 8.48 \\ 
  ROADDIST & 4.66 & 7.03 & 8.08 & 9.07 & 6.44 & 8.9 \\ 
  SLOPE & 7.41 & 5.04 & 5.55 & 6.33 & 6.9 & 6.37 \\ 
  SOILBULK & 7.94 & 5.18 & 4.98 & 4.49 & 6.87 & 5.53 \\ 
  SOILEXCH & 2.28 & 2.93 & 3.43 & 3.24 & 3.95 & --- \\ 
  SOILSAND & 3.73 & 2.91 & 2.94 & 3.4 & 3.15 & 4.46 \\ 
  TEMPANRANGE & 5.22 & \B{12.15} & 10.63 & \B{11.5} & 6.79 & \B{12.94} \\ 
  TOWNDIST & 2.73 & --- & 2.54 & 2.83 & 2.9 & --- \\ 
  TREEDENS & 7.51 & 6.53 & 8.88 & 10.35 & 4.58 & 11.18 \\  
\bottomrule
\end{tabularx}
\label{6sp_final_var}
\end{table}

We refit the models using the best combination of predictors and predicted relative likelihood of occurrence to 1km$^2$ spatial grids for six species (\Cref{6sp_occ_preds}). We examined the rasters for unrealistic predictions of distributions by plotting atlas records of occurrences on prediction grids and also cross-referencing with expert-derived species range maps. Spatial autocorrelation was also assessed for each species by calculating Moran’s I from model residuals and spatial coordinates and plotting against distance in one kilometre bins.

\begin{figure*}[htp]
  \centering
  \subfloat[Eastern Grey Kangaroo]{\label{6sp_occ_preds:a}\includegraphics[width=0.45\textwidth]{6sp_EGK_preds_brt.png}}
  \subfloat[Common Brushtail Possum]{\label{6sp_occ_preds:b}\includegraphics[width=0.45\textwidth]{6sp_BTP_preds_brt.png}}\\
  \subfloat[Common Ringtail Possum]{\label{6sp_occ_preds:c}\includegraphics[width=0.45\textwidth]{6sp_RTP_preds_brt.png}}
  \subfloat[Black Swamp Wallaby]{\label{6sp_occ_preds:d}\includegraphics[width=0.45\textwidth]{6sp_BSW_preds_brt.png}}\\  
  \subfloat[Common Wombat]{\label{6sp_occ_preds:e}\includegraphics[width=0.45\textwidth]{6sp_WOM_preds_brt.png}}
  \subfloat[Koala]{\label{6sp_occ_preds:f}\includegraphics[width=0.45\textwidth]{6sp_KOA_preds_brt.png}}  
  \caption[Predicted relative likelihood of six mammal species occurrence in Victoria]{Predicted relative likelihood of occurrence of each species across the State of Victoria. Darker shading indicates higher relative likelihood of occurrence.}
  \label{6sp_occ_preds}
\end{figure*}

\subsection{Collision Model}

We obtained spatially unique collision record coordinates from the Wildlife Victoria database for each species spanning a four year period between 1 January, 2010 and 31 December, 2013. Wildlife Victoria is a not-for profit organisation that maintains a professional database of state-wide reported wildlife injuries and includes spatial and temporal data for more than 80,000 records.  For this study, we extracted records with cause of injury reported as 'hit by vehicle' and only included records with a spatial accuracy $<$300 metres.

For each target species, we selected the road segments that intersected with the reported location of each collision and re-coded them with ones.  We selected the remaining road segments in the study area, coded them with zeros to represent background data, and combined them with the 'collision' records (\Cref{6sp_species_data}).  We sampled the relative species occurrence predictions at mid-points of the road segments in the collision dataset by intersecting the road features with the respective occurrence spatial grids.  Traffic modelling in \Cref{sec:egk} had already predicted traffic speed and volumes for each road segment.  This resulted in six modelling datasets (one for each species), each with a binary dependent variable of collision (1) or background (0) and three continuous predictors of species occurrence, traffic volume, and traffic speed.

We regressed collisions on the predictor variables using a complementary log-log (cloglog) link.  This link considers the collision data as resulting from a truncated count process which agrees with both our underlying theory about how collisions occur (as a rate following a Poisson distribution) and our data construction methodology. Further, this link can be used to specify a model that is invariant to the spatial scale of the observations allowing greater flexibility as a modelling tool. The cloglog link contrasts with the popular logit link, commonly used for binomial data, which does not have the same interpretation. In this study, we included two variations on the original model presented in \Cref{sec:egk}. First, we added an offset term to account for variation in road lengths and the temporal span of the data. We intend our model outputs to be suitable for management and the offset allows us to represent risk across the network using annual collisions per kilometre. Second, we included a quadratic term for traffic volume to allow a unimodal response shape. Former studies indicate that some species are repelled by roads when traffic volume increases beyond a threshold value \citep{seil05,gagn07}. Other possible determinants of this response shape include changes in local wildlife abundance resulting from previous wildlife-vehicle collisions, absence of wildlife in highly urbanised areas (i.e., heavy traffic areas), and reporting bias due to stopping difficulty on roads with high traffic volumes. We tested quadratic terms for the other predictors, however, they were not informative in this work.

Our final model is expressed as: 

\begin{equation}
\text{cloglog}(p_i) = \beta_0 + \beta_1 \text{ln}(O_i) + \beta_2 \text{ln}(V_i) + \beta_3 (\text{ln}(V_i))^2 + \beta_4 \text{ln}(S_i) + \text{ln}(L_i \times T)
\end{equation}

\noindent where $p_i=\text{Pr}(Y_i=1)$ is the relative likelihood of a collision occurring, $O_i$ is species occurrence, $V_i$ is traffic volume, $S_i$ is traffic speed, and $L_i$ is the length of a given road segment $i$. $T$ is a constant representing the total time span of the data in years and is used in the road length offset to scale the model to an annual baseline.

Using the model fits, we predicted relative collision risk to all roads segments for each of the six study species.  To validate the predictions of the model, we created independent datasets for each species using the same methods as the modelling data but for collisions recorded in the year 2014.  We calculated receiver operator characteristic (ROC) scores \citep{metz78}, also known as area under the curve (AUC), to assess the ability of the models to correctly predict collision events. Although other metrics are used to measure model discrimination (e.g. true skill statistic (TSS), see \cite{allo06}), they are often only suited to presence-absence data where prevalence is known. ROC uses non-specific threshold values and is an appropriate measure for assessing the discriminative ability of presence-only models \citep{laws14}. 

To determine collective risk on each road segment, we summed the total relative collision rates for all species on each segment.  These results were mapped to highlight road segments with generally high risk regardless of target species. For each species, we also identified road segments that ranked in the top 0.1\% for predicted collision rates within each respective range; collision rates varied by species due to differences in reported collisions. We recorded the frequency that individual road segments appeared in this category, the values ranging from zero (no species) to six (all species).

All of the modelling in our study was correlative, however, we explored the influence of seasonal phenomena by partitioning the collision data into four seasons (summer, autumn, winter, spring) and running the models for each subset. Note, due to the small sample bias through partitioning, we fit the data using penalised likelihood methods \citep{firt93}.  We plotted the marginal effects of the predictors on collisions to determine if any season deviated from expected patterns.

\section{Results}

The species occurrence models produced ecologically-plausible predictions about the relative likelihood of species presence across Victoria. The deviance (unexplained variation in the data) reduced by the models varied between 29.5\% and 34.3\% and the mean cross-validated ROC scores varied between 0.87 and 0.92 (\Cref{6sp_models}). The model that best fit the data was for the koala. 

\begin{table}[htp]
\caption[Statistical model performance for six mammal species]{Statistical model performance as percent reduction of deviance on the null model and receiver operator characteristic (ROC) scores for species occurrence (SO) and collision risk (CR) models by species. Note, due to the nature of gradient boosting, errors on both deviance and ROC scores are reported as $\pm$ values for species occurrence models.}
\centering
\begin{tabularx}{0.9\textwidth}{lllll} \toprule
Species Name			&\% Deviance (SO)	&ROC (SO)			&\% Deviance (CR)	&ROC (CR)\\
\midrule 
Eastern Grey Kangaroo	& 30.7$\pm$0.7 		& 0.89$\pm$0.004	& 11.8 				& 0.80 \\ 
Common Brushtail Possum & 32.6$\pm$1.2 		& 0.89$\pm$0.005 	& 16.1 				& 0.83 \\ 
Common Ringtail Possum 	& 33.7$\pm$1.1 		& 0.90$\pm$0.005 	& 19.4 				& 0.87 \\ 
Black Swamp Wallaby 	& 29.5$\pm$1.5 		& 0.87$\pm$0.007 	& 9.1 				& 0.80 \\ 
Common Wombat 			& 33.7$\pm$1.9 		& 0.91$\pm$0.006 	& 15.7 				& 0.89 \\ 
Koala 					& 34.3$\pm$1.9 		& 0.92$\pm$0.007 	& 7.4 				& 0.82 \\ 
\bottomrule
\end{tabularx}
\label{6sp_models}
\end{table}

Five variables were not selected in every species model and the least selected variables were distance to major waterway and hydrological flow accumulation (\Cref{6sp_final_var}). The three most influential predictors were annual range of temperature, artificial light, and precipitation in the driest month (\Cref{6sp_term_occ}). The lowest variation in relative likelihood of occurrence for all species was for annual range of temperature (0.00 to 0.13). Brushtail possums and black swamp wallabies both had the largest spikes in relative occurrence at values of approximately 18 degrees Celsius. Relative likelihood of occurrence increased at higher values of artificial light for all species except koalas and wombats. At high values of precipitation in the driest month, occurrence of the two possums deviated considerably from the trends observed in all other species. 

\begin{figure*}[htp]
  \centering
  \subfloat[]{\label{6sp_term_occ:a}\includegraphics[width=0.45\textwidth]{6sp_tempanrange.png}}
  \includegraphics[width=0.45\textwidth]{6sp_legend.png}\\
  \subfloat[]{\label{6sp_term_occ:b}\includegraphics[width=0.45\textwidth]{6sp_light.png}}
  \subfloat[]{\label{6sp_term_occ:c}\includegraphics[width=0.45\textwidth]{6sp_precdm.png}}
  \caption[Most significant predictor variables on relative likelihood of occurrence for six mammal species]{Effects of three most significant predictor variables on relative likelihood of occurrence per species.}
  \label{6sp_term_occ}
\end{figure*}

Spatial autocorrelation patterning was evident in the species occurrence model residuals between two and approximately twelve kilometres (\Cref{6sp_sac}). Correlation values (Moran's I) exceeded 0.4 at distances of two kilometres or less for all species except the ringtail possum. The highest overall spatial autocorrelation was in the koala model residuals, however, the wombat, koala, and black swamp wallaby models all demonstrated high levels of spatial autocorrelation at one kilometre distances.

\begin{figure*}[htp]
  \centering
  \subfloat[Eastern Grey Kangaroo]{\label{6sp_sac:a}\includegraphics[width=0.45\textwidth]{6sp_sac_a.png}}
  \subfloat[Common Brushtail Possum]{\label{6sp_sac:b}\includegraphics[width=0.45\textwidth]{6sp_sac_b.png}}\\
  \subfloat[Common Ringtail Possum]{\label{6sp_sac:c}\includegraphics[width=0.45\textwidth]{6sp_sac_c.png}}
  \subfloat[Black Swamp Wallaby]{\label{6sp_sac:d}\includegraphics[width=0.45\textwidth]{6sp_sac_d.png}}\\
  \subfloat[Common Wombat]{\label{6sp_sac:e}\includegraphics[width=0.45\textwidth]{6sp_sac_e.png}}
  \subfloat[Koala]{\label{6sp_sac:f}\includegraphics[width=0.45\textwidth]{6sp_sac_f.png}}
  \caption[Spatial autocorrelation in occupancy models residuals for six mammal species]{Spatial autocorrelation in occupancy models residuals for each species grouped by distance between observations. Trend lines use numbers to indicate species (see legend).}
  \label{6sp_sac}
\end{figure*}

All of the collision models fit the data as expected.  Reduction on the null deviance was between 7.4\% and 19.4\% and all of the collision models attained high ROC scores when predictions were compared to the independent validation data (\Cref{6sp_models}).  The black swamp wallaby model had the lowest ability to discriminate between false positives and false negatives (0.80) and the koala model had the highest (0.89). All of the predictors were highly significant (p$<$.001) in all of the models except traffic speed which was not a highly significant variable for the three arboreal mammal species

\begin{table}[htp]
\caption[Summary of collision models for six mammal species]{Summary of collision models fits. Coefficients and significance of variables are shown with relative percent contribution (ANOVA) to model fits for each species.  Highly significant variables (p$<$.001) are marked with asterisks.}
\centering
\begin{tabularx}{0.9\textwidth}{lllllll} \toprule
Species & Variable & Coefficient & Std. Error & $Z\text{-value}$ & $\PRZ$ & ANOVA \\ 
\midrule
Eastern Grey Kangaroo & Intercept & -46.55 & 1.961 & -23.74 & $<$.001* & --- \\ 
   & EGK & 0.666 & 0.02413 & 27.6 & $<$.001* & 69.53 \\ 
   & TVOL & 7.981 & 0.4631 & 17.23 & $<$.001* & 7.179 \\ 
   & TVOL$^2$ & -0.4688 & 0.02899 & -16.17 & $<$.001* & 11.63 \\ 
   & TSPD & 2.187 & 0.1261 & 17.35 & $<$.001* & 11.66 \\ 
   &  &  &  &  &  &  \\ 
Common Brushtail Possum & Intercept & -57.19 & 9.55 & -5.989 & $<$.001* & --- \\ 
   & BTP & 0.8599 & 0.08795 & 9.777 & $<$.001* & 73.29 \\ 
   & TVOL & 11 & 2.006 & 5.483 & $<$.001* & 19.16 \\ 
   & TVOL$^2$ & -0.589 & 0.115 & -5.121 & $<$.001* & 7.518 \\ 
   & TSPD & 0.1785 & 0.3644 & 0.4898 & 0.620 & 0.031 \\ 
   &  &  &  &  &  &  \\ 
Common Ringtail Possum & Intercept & -146.7 & 25.95 & -5.651 & $<$.001* & --- \\ 
   & RTP & 1.079 & 0.1027 & 10.51 & $<$.001* & 74.44 \\ 
   & TVOL & 30.47 & 5.675 & 5.37 & $<$.001* & 13.93 \\ 
   & TVOL$^2$ & -1.681 & 0.319 & -5.271 & $<$.001* & 11.04 \\ 
   & TSPD & 0.9182 & 0.4093 & 2.244 & 0.025 & 0.590 \\ 
   &  &  &  &  &  &  \\ 
Black Swamp Wallaby & Intercept & -64.59 & 8.024 & -8.049 & $<$.001* & --- \\ 
   & BSW & 0.729 & 0.08703 & 8.377 & $<$.001* & 60.03 \\ 
   & TVOL & 12.46 & 2.013 & 6.193 & $<$.001* & 6.234 \\ 
   & TVOL$^2$ & -0.7598 & 0.128 & -5.938 & $<$.001* & 27.77 \\ 
   & TSPD & 1.877 & 0.5203 & 3.607 & $<$.001* & 5.966 \\ 
   &  &  &  &  &  &  \\ 
Common Wombat & Intercept & -64.38 & 7.182 & -8.965 & $<$.001* & --- \\ 
   & WOM & 0.8487 & 0.06144 & 13.81 & $<$.001* & 75.19 \\ 
   & TVOL & 12.01 & 1.763 & 6.815 & $<$.001* & 5.406 \\ 
   & TVOL$^2$ & -0.7186 & 0.1107 & -6.491 & $<$.001* & 13.50 \\ 
   & TSPD & 2.302 & 0.4135 & 5.567 & $<$.001* & 5.903 \\ 
   &  &  &  &  &  &  \\ 
Koala & Intercept & -60.98 & 9.01 & -6.768 & $<$.001* & --- \\ 
   & KOA & 0.5093 & 0.07744 & 6.576 & $<$.001* & 53.63 \\ 
   & TVOL & 11.75 & 2.311 & 5.085 & $<$.001* & 10.28 \\ 
   & TVOL$^2$ & -0.7203 & 0.149 & -4.834 & $<$.001* & 30.73 \\ 
   & TSPD & 1.663 & 0.6261 & 2.656 & 0.008 & 5.366 \\  
\bottomrule
\end{tabularx}
\label{6sp_sum_coll}
\end{table}

All of the predictor variables demonstrated plausible relationships to collision likelihood in the partial dependency plots (\Cref{6sp_effects}) and the signs of the coefficients were as expected (\Cref{6sp_sum_coll}). Further, these response shapes were similar across all seasons for each species except the two possums (\Cref{6sp_occ_seas,6sp_tvol_seas,6sp_tspd_seas}). Based on analysis of variation (ANOVA), species occurrence had the largest contribution to reduction in deviance for all of the models.

%\begin{figure*}[htp]
%  \centering
%  	\begin{minipage}[t][][b]{.13\textwidth}
%    	\centering
%    	\includegraphics[width=\textwidth]{MODELS_connectivity.png}\\
%    	\includegraphics[width=\textwidth]{MODELS_population.png}   	
%    \end{minipage}
%    \hspace{.05\textwidth}
%  	\begin{minipage}{.38\textwidth}
%    	\centering
%    	\includegraphics[width=\textwidth]{MODELS.png}
%    \end{minipage}
%    \hspace{.05\textwidth}
%  	\begin{minipage}[t][][b]{.13\textwidth}
%    	\centering
%    	\includegraphics[width=\textwidth]{MODELS_predators.png}\\
%    	\includegraphics[width=\textwidth]{MODELS_infrastructure.png}
%    \end{minipage}    	   	   
%  \caption[Conceptual risk model framework]{Conceptual risk model framework.}
%  \label{gen_framework}
%\end{figure*}

\begin{figure*}[htp]
  \centering
  \subfloat[]{\label{6sp_effects:a}
%  	\setlength\tabcolsep{2pt}
%  	\begin{tabularx}{0.9\textwidth}{*{3}{c}}
%  		\multirow{3}{*}[2cm]{\includegraphics[width=0.4\textwidth]{6sp_occ_a.png}}	& \includegraphics[width=.12\textwidth]{6sp_occ_a_egk.png}	& \includegraphics[width=.12\textwidth]{6sp_occ_a_btp.png}\\
%  		& \includegraphics[width=.12\textwidth]{6sp_occ_a_rtp.png}	& \includegraphics[width=.12\textwidth]{6sp_occ_a_bsw.png}\\
%  		& \includegraphics[width=.12\textwidth]{6sp_occ_a_wom.png}	& \includegraphics[width=.12\textwidth]{6sp_occ_a_koa.png}
%   	\end{tabularx}
  	\begin{minipage}[t][][]{.45\textwidth}
    	\centering
    	\includegraphics[width=\textwidth]{6sp_tvol_b.png}   	
    \end{minipage}
    %\hspace{.05\textwidth}
  	\begin{minipage}[t][][]{.25\textwidth}
    	\centering
    	\includegraphics[width=.5\textwidth]{6sp_tvol_b_egk.png}
    	\includegraphics[width=.5\textwidth]{6sp_tvol_b_btp.png}
    	
    	\vfill
    	\includegraphics[width=.5\textwidth]{6sp_tvol_b_rtp.png}
    	\includegraphics[width=.5\textwidth]{6sp_tvol_b_bsw.png}
    	
    	\vfill
    	\includegraphics[width=.5\textwidth]{6sp_tvol_b_wom.png}
    	\includegraphics[width=.5\textwidth]{6sp_tvol_b_koa.png}
    \end{minipage}
  }%\\
%  \subfloat[]{\label{6sp_effects:b}
%  	\setlength\tabcolsep{2pt}
%  	\begin{tabularx}{0.9\textwidth}{*{3}{c}}
%  		\multirow{3}{*}[2cm]{\includegraphics[width=0.4\textwidth]{6sp_tvol_b.png}}	& \includegraphics[width=.12\textwidth]{6sp_tvol_b_egk.png}	& \includegraphics[width=.12\textwidth]{6sp_tvol_b_btp.png}\\
%  		& \includegraphics[width=.12\textwidth]{6sp_tvol_b_rtp.png}	& \includegraphics[width=.12\textwidth]{6sp_tvol_b_bsw.png}\\
%  		& \includegraphics[width=.12\textwidth]{6sp_tvol_b_wom.png}	& \includegraphics[width=.12\textwidth]{6sp_tvol_b_koa.png}
%   	\end{tabularx}
%  }\\
%  \subfloat[]{\label{6sp_effects:c}
%  	\setlength\tabcolsep{2pt}
%  	\begin{tabularx}{0.9\textwidth}{*{3}{c}}
%  		\multirow{3}{*}[2cm]{\includegraphics[width=0.4\textwidth]{6sp_tspd_c.png}}	& \includegraphics[width=.12\textwidth]{6sp_tspd_c_egk.png}	& \includegraphics[width=.12\textwidth]{6sp_tspd_c_btp.png}\\
%  		& \includegraphics[width=.12\textwidth]{6sp_tspd_c_rtp.png}	& \includegraphics[width=.12\textwidth]{6sp_tspd_c_bsw.png}\\
%  		& \includegraphics[width=.12\textwidth]{6sp_tspd_c_wom.png}	& \includegraphics[width=.12\textwidth]{6sp_tspd_c_koa.png}
%   	\end{tabularx}
%  }
  \caption[Marginal effects of predictor variables on relative likelihood of collision for six mammal species]{Marginal effects of predictor variables on relative likelihood of collision per species. Each predictor is represented by an 1) overall plot showing relative collision rates for all species and 2) six sub-panels depicting response shapes with rescaled collision risk (minimum to maximum) for each species.}
  \label{6sp_effects}
\end{figure*}

Among species, the response of collision risk to occurrence likelihood demonstrated the least amount of variation in shape (\Cref{6sp_effects_a}) compared to traffic volume (\Cref{6sp_effects_b}) and traffic speed (\Cref{6sp_effects_c}). Koalas had the highest increases in collision risk at lower values of occurrence likelihood and gradually reduced risk at higher values. This response was consistent for all species except ringtail possums which had an exponential relationship, albeit minor (coefficient of 1.08), between risk and occurrence.

\begin{figure*}[htp]
  \centering
  \subfloat[Eastern Grey Kangaroo]{\label{6sp_occ_seas:a}\includegraphics[width=0.4\textwidth]{6sp_occ_seasons_a.png}}
  \subfloat[Common Brushtail Possum]{\label{6sp_occ_seas:b}\includegraphics[width=0.4\textwidth]{6sp_occ_seasons_b.png}}\\
  \subfloat[Common Ringtail Possum]{\label{6sp_occ_seas:c}\includegraphics[width=0.4\textwidth]{6sp_occ_seasons_c.png}}
  \subfloat[Black Swamp Wallaby]{\label{6sp_occ_seas:d}\includegraphics[width=0.4\textwidth]{6sp_occ_seasons_d.png}}\\
  \subfloat[Common Wombat]{\label{6sp_occ_seas:e}\includegraphics[width=0.4\textwidth]{6sp_occ_seasons_e.png}}
  \subfloat[Koala]{\label{6sp_occ_seas:f}\includegraphics[width=0.4\textwidth]{6sp_occ_seasons_f.png}}
  \caption[Marginal effects of predictor variables on relative likelihood of collision for six mammal species by season]{Marginal effects of species occurrence on relative likelihood of collision per species.}
  \label{6sp_occ_seas}
\end{figure*}

\begin{figure*}[htp]
  \centering
  \subfloat[Eastern Grey Kangaroo]{\label{6sp_tvol_seas:a}\includegraphics[width=0.4\textwidth]{6sp_tvol_seasons_a.png}}
  \subfloat[Common Brushtail Possum]{\label{6sp_tvol_seas:b}\includegraphics[width=0.4\textwidth]{6sp_tvol_seasons_b.png}}\\
  \subfloat[Common Ringtail Possum]{\label{6sp_tvol_seas:c}\includegraphics[width=0.4\textwidth]{6sp_tvol_seasons_c.png}}
  \subfloat[Black Swamp Wallaby]{\label{6sp_tvol_seas:d}\includegraphics[width=0.4\textwidth]{6sp_tvol_seasons_d.png}}\\
  \subfloat[Common Wombat]{\label{6sp_tvol_seas:e}\includegraphics[width=0.4\textwidth]{6sp_tvol_seasons_e.png}}
  \subfloat[Koala]{\label{6sp_tvol_seas:f}\includegraphics[width=0.4\textwidth]{6sp_tvol_seasons_f.png}}
  \caption[Marginal effects of predictor variables on relative likelihood of collision for six mammal species by season]{Marginal effects of traffic volume on relative likelihood of collision per species.}
  \label{6sp_tvol_seas}
\end{figure*}

\begin{figure*}[htp]
  \centering
  \subfloat[Eastern Grey Kangaroo]{\label{6sp_tspd_seas:a}\includegraphics[width=0.4\textwidth]{6sp_tspd_seasons_a.png}}
  \subfloat[Common Brushtail Possum]{\label{6sp_tspd_seas:b}\includegraphics[width=0.4\textwidth]{6sp_tspd_seasons_b.png}}\\
  \subfloat[Common Ringtail Possum]{\label{6sp_tspd_seas:c}\includegraphics[width=0.4\textwidth]{6sp_tspd_seasons_c.png}}
  \subfloat[Black Swamp Wallaby]{\label{6sp_tspd_seas:d}\includegraphics[width=0.4\textwidth]{6sp_tspd_seasons_d.png}}\\
  \subfloat[Common Wombat]{\label{6sp_tspd_seas:e}\includegraphics[width=0.4\textwidth]{6sp_tspd_seasons_e.png}}
  \subfloat[Koala]{\label{6sp_tspd_seas:f}\includegraphics[width=0.4\textwidth]{6sp_tspd_seasons_f.png}}
  \caption[Marginal effects of predictor variables on relative likelihood of collision for six mammal species by season]{Marginal effects of traffic speed on relative likelihood of collision per species.}
  \label{6sp_tspd_seas}
\end{figure*}

All of the models had highly significant coefficients for both linear and quadratic terms of traffic volume.  All of the model fits demonstrated curvilinear responses to AADT with consistent peaks around 3,000-5,000 vehicles day$^{−1}$ for all species except the two possums which peaked between 9,000-12,000 vehicles day$^{−1}$. Between the two possums, collision risk peaked at higher values of AADT for the brushtail possum and did not increase or decrease as rapidly across all values of AADT.  Koalas and black swamp wallabies had very similar response shapes and also showed the quickest rise and fall of collision risk between 0-10,000 vehicles day$^{−1}$ compared to all other species.

Collision risk had an exponential relationship to traffic speed for all species except the two possums. For brushtail possums, collision risk increased rapidly for low values of traffic speed and increased slowly for values above 30 km hr$^{−1}$. It should be noted, however, that the predictor of traffic speed was not significant for brustail possums and had less than 0.1\% relative influence on the model fit. As with species occurrence, ringtail possums demonstrated a nearly linear relationship between collision risk and traffic speed. Wombats demonstrated the highest rate of increase in collision risk for increasing collision speed compared to all other species.

\begin{figure*}[htp]
  \centering
  \includegraphics[width=\textwidth]{6sp_coll_risk.png}
  \caption[Road segments with predictions of high collision likelihood for six mammal species]{Road segments with summed collision rates for all six species. The entire road network is shown as faint gray lines for context.}
  \label{6sp_collrisk}
\end{figure*}

The aggregated predictions of collision likelihood were highest on several sections of the Hume Highway and portions of the Calder Freeway just north of the urban centre of Melbourne (\Cref{6sp_collrisk}). There were 60 road segments that were identified with the highest 0.1\% of collision risk for all six species and 3,678 road segments for at least one species.

\section{Discussion}

Our study successfully applied a collision risk model framework to six different mammal species in south-east Australia and identified individual species' and cumulative collision risk for all road segments. We acknowledge that our study is not the first to model collision risk for multiple taxa; others have studied mammals \citep{clev02,cser13,jaar06}, reptiles \citep{guns12,lang12} and mixed taxonomic groups \citep{clev02,garr15,lang09,litv08}.  And like ours, all these studies used mixtures of specific environmental and anthropogenic variables in multivariate models. What our research provides is a useful analytical framework for managers that generalises well and allows ongoing scrutiny and tuning of the working parts.  For example, if species occurrence is a strong predictor of collisions but the performance of the SDMs are not satisfactory, managers may choose to further improve the sub-models or re-consider how much emphasis is placed on installing mitigation measures that influence animal occurrence.  Similarly, spatial autocorrelation, detectability, bias, and other factors affecting model calibration and robustness may be assessed accordingly. Moreover, our methodology can be applied to single road segments or across an entire linear network, and on a single species, taxonomic class or mixed group.

The conceptual framework utilised predictors that connect to clear management objectives - species occurrence on road, traffic volume, and driver behaviour (i.e. traffic speed) - and are also consistent with theories of road ecology (see \cite{form03}). The models identified common trends in the relationships between the predictors and collision risk. Collision risk increased consistently with higher likelihoods of species occurrence suggesting the usefulness of separating out the exposure parameter via SDMs. \cite{roge09} demonstrated the importance of habitat variables (a surrogate for species occurrence) for modelling collisions. The work was extended to use an SDM for wombats \citep{roge12}, indicating higher predicted likelihoods of species occurrence resulted in higher likelihoods of collisions.

Collision risks for ringtail possums increased more slowly with occurrence likelihood than for the other mammal species.  This makes ecological sense considering the species spends most of the time elevated (mostly in highly connected urban development via power lines and street trees) and would come into contact with moving vehicles less frequently than ground-dwelling species.  However, it is difficult to make direct comparisons between species due to uncertainties in the data.  Potential reporting biases (e.g. not reporting smaller species) and errors (e.g. possum was assumed hit by car due to discovery on or near road verge) may result in artificially increased or decreased sample sizes. Of the 6546 collision records from the Wildlife Victoria database used in this study, ringtail possums were the third most reported species (\Cref{6sp_species_data}).  Thus, possums could be much more abundant than wombats or wallabies. \cite{gril14} explore collision risk among species based on differences in abundance and other traits that might pre-dispose species to collisions. Studies on reporting bias \citep{snow15} and spatial error \citep{guns09} in WVC also examine some of these issues in more detail.

Collision likelihoods also increased with higher traffic speeds for all species.  With the exception of the brushtail and ringtail possums, the magnitudes of increase were significant, suggesting high vehicle speed is a hazard for wildlife.  One plausible explanation is that higher traffic speeds reduce reaction times for both wildlife and drivers. Previous work has used simulation analysis to model the relationship of speed and collisions by using species traits to determine crossing success \citep{jaar06}. Several other studies have shown similar results on the effects of traffic speed on road mortality of wildlife (e.g. \cite{farm12};\cite{gkri13};\cite{lao11};\cite{ramp06a};\cite{seil05};\cite{seil06};\cite{sudh09};\cite{vanl09}).

The consistent quadratic shape observed in the responses to traffic volume for all species (\Cref{6sp_effects}) indicates a possible threshold effect.  Varying levels of traffic have been shown to affect animal presence on or near roads \citep{jaeg05,rhod14}. Aside from species behavioural responses, traffic volumes above a certain level may affect species abundance due to mortality caused by wildlife-vehicle collisions. Lower abundances result in less animals on the road and, therefore, less collisions. Under this hypothesis, traffic volume is especially problematic for species that are attracted to roads \citep{form03}. Nonetheless, identifying traffic volumes that have high relative collision risks for multiple species may help reduce roadkill.  For example, high likelihood of collisions at particular traffic speed and volume levels may warrant further examination to determine potential mitigation such as redistribution or re-routing of vehicles and traffic-calming infrastructure.

As mentioned previously, our model framework is correlative and disregards temporal patterns of collisions.  A qualitative comparison of the effects plots revealed similar trends between seasons for all species except ringtail and brushtail possums, however, the small sample sizes used in modelling warrant a cautionary interpretation. None of our species demonstrate migratory patterns and have relatively stable patterns of activity throughout the year. For species that exhibit seasonal patterns of movement, it would be useful to incorporate this information into the collision model (see \cite{neum12}). Partitioning the temporal component at the hourly scale may be more useful as systematic patterns of collisions exist throughout the day (e.g. \cite{litv08} or \cite{rhod14}). This was beyond the scope of this work as a robust temporal analysis would require the expansion of all explanatory variables to a complementary temporal scale, significantly increasing the computational requirements.  Temporal differences in species movements and lags in collision reporting must also be considered.

The spatial autocorrelation in the occurrence models' residuals was high at small distances and similar for all species.  Any spatial correlation or patterning in the residuals violates statistical independence assumptions and several methods exist in the SDM literature to deal with this issue \citep{augu96,dorm07,dorm13}.  Patterning and higher values of spatial autocorrelation are plausible in certain ecological data and for some species.  For example, correlations in model residuals based on the locations of koalas may be expected as they are selective in their choice of tree species, which are often found in homogeneous patches.  However, high values of spatial autocorrelation or patterning should be carefully reviewed and appropriately addressed when used in statistical models \citep{wint06}.  We included covariates to represent both sampling bias and the spatial autocorrelation arising from such in our models.  Distance to road was selected by all species models and distance to town was selected by all models except brushtail possum and koala.  Moreover, boosted regression trees are equipped to partially address spatial autocorrelation by reducing bias and variance \citep{elit08}.

We did not find significant spatial autocorrelation or patterning in the collision model residuals and, thus, we did not choose to address it as part of the scope of this project.  As mentioned previously, our model framework easily allows for this correction by the inclusion of auto-covariates in the model specification (e.g. \cite{dwye16};\cite{farm12};\cite{gome08}), or by selecting alternative statistical modelling methods \citep{zhan05}.

\subsection{Conclusions \& Implications for Management}

Our study analysed reported vehicle collisions for six terrestrial mammal species in Australia to test the generality of a model framework and to help infer cause and effect. Species occurrence modelling is useful to determine wildlife exposure to risk (on a relative scale). Species occurrence, traffic volume and traffic speed, as single modelled predictors, allow managers to improve models as new data becomes available. Managers may also predict effects by varying predictor values that reside in their authoritative control such as species occurrence or speed limit; here, the model framework functions as a decision planning tool. For example, the relationship between species occurrence and collision risk was similar for the six species in our study. Thus, managers may simulate changes in expected collisions resulting from fencing parts of the road network by reducing values of occurrence on road segments and predicting with the model. Further, the model predictions may also be spatially aggregated to suggest areas on the road network that require general mitigation or further study.  In the case of Victoria, high-speed freeways and highways north of Melbourne had the highest overall collision risk. Our framework may also be extended to explore additional threats and predict risk to other taxonomic groups and vulnerable species (i.e., focus of conservation efforts) arising from the use of linear infrastructure.


\chapter{Predicting road collision risk for six native Australian mammals in Victoria}\label{sec:6sp}

%\chapter{Application of wildlife-vehicle collision model framework on different continents: Victoria and California}\label{sec:cal}
\newpage

\begin{localsize}{10}
\section*{\centering Abstract}

Our study is the first to apply a generalised model framework to analyse and predict WVC in two continents. Despite being different species occurring in different geographic areas, we examine whether collision risk correlates similarly with animal behaviour (occurrence of the species) and human behaviour (traffic volume and speed).

We applied a conceptual risk framework in the State of Victoria in south-east Australia and a section of central California in the USA. Our study species were eastern grey kangaroos in Victoria and mule deer in California. We trained our collision models with citizen-science collected, collision/carcass records for each species and predicted collision risk for all road segments in each study area.  We used a very different set of data, police records of road accidents, to evaluate the predictions.

All predictor variables were highly significant in the collision models and the shape of the response to each predictor were similar for both species.  Relative collision risk increased with increased species occurrence, traffic volume, and traffic speed. Speed had the highest relative importance followed by species occurrence. Predicted relative collision risk on all road segments agreed with the distribution of known collision hotspots throughout most of the two study areas. For both species, collision model predictions using the validation datasets were highly correlated with independent observations, indicating well calibrated models.

Our study demonstrates that a conceptual risk framework generalises well to species in different geographical areas. The framework enables managers to conduct sensitivity analyses and calculate overall reductions in expected collisions based on applying mitigation strategies on different road segments. Our analysis also suggests appropriate mitigation for both species may include reduced speeds or fencing on road segments where species occurrence is predicted to be high.

\end{localsize}

\newpage
\section{Introduction}

Wildlife-vehicle collisions (WVC) result from human development and activity. Approximately one-million vertebrates are killed per day in the United States (Forman and Alexander, 1998) and up to 27-millions birds are killed annually in select European countries (Erritzoe et al., 2003).  Costs of WVC, including vehicle repair, human medical costs and the value of a human life, are estimated to exceed eight billion dollars annually in the United States (Huijser et al., 2007) with deer collisions comprising the largest share (see Bissonette et al., 2008).  These costs and impacts will continue to worsen globally as new roads are constructed and existing roads are upgraded in developing countries; with worldwide traffic volume doubling by 2050 and increasing fivefold in developing countries (van der Ree et al., 2015).

Mitigation strategies aim to reduce the rate and severity of WVC by influencing either human or animal behaviour (see Huijser and McGowen, 2010).  Traffic planning (e.g. the magnitude and routing of vehicles), speed control (e.g. signage, traffic calming mechanisms and penalty enforcement), and education/training can influence the operation of vehicles on roads.  Exclusion (e.g. fencing) and re-routing, either over or under the road, can influence the presence and movement of wildlife on the roads.  Other strategies involve visual or auditory discouragement such as flashing lights and ultra-sonic whistles, however, these are largely ineffective (Reeve and Anderson, 1993; Bender, 2003; Scheifele et al., 2003; Ramp and Croft, 2006). Choices to implement mitigation strategies are influenced by an understanding of the factors contributing to WVC in a particular area.

Decisions about where and when to mitigate are important because applying ineffective strategies incurs costs.  To address this issue, many studies examine patterns of collisions and related variables to inform management, often involving spatial modelling and prediction (see Gunson et al., 2011).  Most research is specific to an area (i.e. section of road) or problem and is not easily transferable due to variation in scale, geographic location and/or species traits, and confounding effects of explanatory variables.  Although these issues cannot be totally eliminated - partly due to the uncertainty present in ecological systems - it is possible to develop analytical frameworks to assist managers to deal with uncertainty and make inferences regarding wildlife-vehicle collision risk (see Visintin et al., 2016).

We tested the general applicability of a previously developed conceptual model framework for predicting collision risk in California and Victoria.  We had two aims; first we were interested in the applicability of our methods to planning for WVC mitigation irrespective of geographical location or scale (see van der Ree et al., 2011).  We propose that our analytical methods will supplement management efforts by allowing clearer inferences about the factors related to collision risk, identifying potential biases and uncertainties in the analysis (thereby suggesting further research or data collection), and suggesting appropriate mitigation for WVC.  Second, collisions of vehicles occur at a similar frequency for both kangaroos and deer, have similar impacts (Langley and Mathison, 2008), and are subject to similar management practices (McShea and Underwood, 1997; Croft, 2004).  So while models of collision risk should differ between species because of the different contexts, we expect that our previous framework should reveal similar patterns of collision risk for the two species as a function of traffic attributes and species occurrence.  To our knowledge, this work represents the first study to test a broadly applicable model for two analogous species on different continents for the purpose of satisfying contrasting needs of management.

\section{Materials \& Methods}

\subsection{Overview of Workflow}

Our analysis is comprised of the following overall workflow for each of the two study areas:
1)  Define the study area and divide it into a regular spatial grid.
2)  Split all roads in the study area into segments using the spatial grid.
3)  Develop traffic models and predict relative traffic volume and speed for all road segments in the study area.
4)  Develop a species distribution model and predict relative occurrence for each road segment based on the grid cell in which it occurs.
5)  Classify road segments based on occurrence of collision ('1' for one or more collisions; '0' otherwise) for the training data.
6)  Develop a regression model of the collisions (training data) with species occurrence, traffic volume and traffic speed as explanatory variables.
7)  Use the regression model to predict relative rate of collisions for each road segment.
8)  Compare the predicted rate of collision to a new set of collision data (validation data) to evaluate the regression model.
Each of these steps is explained in more detail in the following sections.

\subsection{Study Areas}

We use two unique geographic locations to test our framework; the 227,819 km2 State of Victoria in south-east Australia and a 146,478 km2 section of central California in North America (Figure 1). The case study regions were chosen because of availability of data on animal occurrence, collision records, and the road networks. Each locale has a single agency responsible for the major roads (e.g. high-speed, high-capacity, or both) while all other roads are managed by municipal districts. We predict collision risk for all sealed roads within the study areas. To organise our spatial data and modelling, we overlay spatial grids of one km2 resolution on the study area. Each grid cell is used as the modelling unit for species occurrence. All roads in the study area are bisected by the grid resulting in road segments that are approximately one kilometre or less in length. We use 655,348 road segments for Victoria and 646,705 road segments for California as our modelling units for each respective collision model.

\subsection{Data}

We selected one species from each study area that are frequently involved in WVC. Eastern grey kangaroos (Macropus giganteus, Shaw) are the second largest mammal in Australia - up to 85 kilograms for males (Van Dyck and Strahan, 2008) - and share management issues with ungulates found in North America and Europe (Croft, 2004; Coulson and Eldridge, 2010).  Mule deer (Odocoileus hemionus, Rafinesque) are common across western North America and adults range in size up to 120 kilograms (Kays and Wilson, 2009).

To train our collision models, we obtained spatially unique collision/carcass records for each species from two citizen-science databases; the Wildlife Victoria Database, or WVD, (Wildlife Victoria, 2016) for kangaroos, and the California Roadkill Observation System, or CROS, (see Shilling and Waetjen, 2015) for deer.  The collision/carcass records spanned a six year period between 1 January, 2010 and 1 January, 2016 for kangaroos and a ten year period between 1 June, 2006 and 1 June, 2016 for deer. We limited our datasets to records with associated global positioning system (GPS) coordinates. Kangaroo records were initially inspected for spatial accuracy and we only retained observations with less than three-hundred metre error. Deer records from CROS have greater than ninety-nine percent species identification accuracy and less than one-hundred metre spatial error (Shilling and Waetjen, unpublished observations).

For each target species, we selected road segments that intersected with reported species’ collision records and coded them with ones. We coded all other road segments in each study area with zeros, to represent background data, and combined them with the collision record segments.  After removing spatial duplicates, there were 4,245 presence and 640,470 background points in the kangaroo collision modelling dataset and 933 presence and 644,296 background points in the deer collision modelling dataset.

Traffic volume and speed values for all road segments in both Victoria and California were predicted following methods of Visintin et al. (2016).  We regressed annual average daily traffic (AADT) and speed on explanatory variables of distance to anthropogenic development (derived from remotely-sensed land use), distance to highway/freeway, road class, road density within one km of each road segment, and population density (from the Australian Bureau of Statistics and the United States Census Bureau, respectively) in random forest models (Brieman, 2001). We developed species distribution models (SDMs) to predict occurrence for each species across their respective geographic areas (see Appendix S1 in Supporting Information) and used the mid-points of the road segments to sample each species occurrence prediction grids. This resulted in two modelling datasets, each with a binary dependent variable of collision (1) or background (0) and three continuous predictors of species occurrence, traffic volume, and traffic speed.  

We used police records of road accidents to evaluate our model fits.  For each species, we extracted spatially-unique records where the incidents were collisions with our target species.  For kangaroos, we used data reported between 1 January, 2010 and 1 October, 2016. For deer, the reporting period was 1 February 2015 to 1 December 2016.  Again, we used all road segments as background and coded the segments with collisions as ones and the remainder with zeros. The final datasets had 481 presence and 644,234 background points for kangaroo collision model validation and 1,795 presence and 643,434 background points for deer. As with the training data, each road segment had values for species occurrence, traffic volume and traffic speed.

\subsection{Collision Modelling \& Validation}

We used a quantitative risk model (Visintin et al., 2016) to fit and compare the relationship of species presence and road threat to collision likelihood and the open-source software package ’R’ version 3.3.0 (R Development Core Team, 2004) to perform all statistical analyses. We regressed collisions on the predictor variables with an added quadratic term for traffic volume. Our model is expressed as:

cloglog(pi) = β0 + β1 log(Oi) + β2 log(Vi) + β3 (log(Vi))2 + β4 log(Si)			(1)

where (pi = Pr(Yi=1)) is the relative likelihood of a collision occurring on a road segment i, Oi is species occurrence, Vi is traffic volume, Si is traffic speed. We chose the complementary log-log link on the linear predictor due to the mathematical theory underpinning our model - risk being measured by the rate of collisions (see Visintin et al., 2016).

To test for spatial autocorrelation, we calculated normally-distributed, randomised quantile residuals (Dunn and Smyth, 1996) using 5000 simulations from each model fit. We then calculated Moran's I on the residuals at 20 spatial lags at intervals of both one kilometre and 250 metres.  For each spatial resolution, we repeated the Moran's I calculation twenty times for 5000 randomly sampled road segments (due to computational limitations) and plotted the twenty trend lines for visual inspection at both spatial scales (Figure 3).

The two models were used to predict relative collision risk of all road segments in each respective study area (Figure 4).  We tested the calibration strength and discrimination ability of our models with their respective independent validation datasets. To assess calibration strength, we regressed the collision observations in the validation sets on predictions made using the validation data in the trained models. An intercept coefficient of 'zero' and slope coefficient of 'one' indicates a perfectly calibrated model (see Miller et al., 1991). To measure ability to discriminate between true positive and false positives we used a receiver operating characteristic (ROC) score - a score of one indicating perfect discrimination ability while 0.5 suggesting a performance no better than random (see Metz, 1978).

\section{Results}

All of the predictor variables demonstrated plausible relationships to collision likelihood in the partial dependency plots. The probability of collision for both species changed similarly with occurrence of the species (Figure 2a and b), traffic volume (Figure 2c and d), and traffic speed (Figure 2e and f).

Traffic speed and species occurrence were highly significant variables for kangaroos and deer (Table 2). Speed had the highest relative importance for both kangaroos and deer based on an analysis of variance (ANOVA) for each predictor. Increasing speed from 80 to 100 km hr-1 approximately doubled the relative collision risk for both species.  Collision risk with deer was less than kangaroos at lower speeds and increased at a faster rate at higher speeds. Species occurrence influenced collision risk the second most for both species.  The response shape of collision risk against likelihood of occurrence was similar for both species, however, deer had larger confidence intervals around the marginal response (Figure 2a and b).

Relative collision risk  for kangaroos increased with traffic volume until approximately 5,000 vehicles day-1 and then declined (Figure 2c); the relationship was similar for deer although the peak was closer to 10,000 vehicles day-1 (Figure 2d). Collision likelihood of kangaroos decreased more quickly than deer at higher traffic volumes, but deer had larger confidence intervals around the marginal response curve indicating the possibility of a shape more consistent with kangaroos (Figure 2d). The actual magnitude of the relative risk is not comparable between species because sizes of the collision and background datasets were not standardised across the two species.

Spatial correlation in the randomised quantile residuals was low and showed no patterns across distances at the two spatial scales analysed (Figure 3), suggesting assumptions of statistical independence were reasonable. All of the predictor variables were highly significant and the signs of the coefficients indicated predictor-response relationships that were plausible (Table 2). The deer model reduced deviance compared to the null model by nearly twice that of the kangaroo model and also had higher discrimination ability.

For each species, the models had similarly high calibration statistics when independent data (police reports) were used for validation. The kangaroo model had a slope coefficient of .99 and the deer model had a slope coefficient of .94 (Table 1). Each model had a different intercept coefficient due to varying numbers of collisions used for model fitting.

The predictions for relative collision risk across all road segments were plausible in most regions of the two study areas.  In Victoria, predicted collision risk was highest in the northern suburbs of Melbourne where collisions frequently occur. In contrast, predictions in California were highest on road segments in both the San Francisco Bay Area (SFBA), where deer collisions are common, and along the Interstate 5 and 99 corridors through the Central Valley region, where deer collisions are less commonly observed/reported.

\section{Discussion}

Our results suggest that the proposed conceptual framework may have utility irrespective of locality, spatial scale, or species.  Species occurrence, traffic volume and speed, and  predictor variable data are publicly available for many jurisdictions, therefore, an analyst only requires data on vehicle collisions with wildlife species to train the models in the conceptual framework, make inferences and predict risk.  Moreover, the collection of data on collisions with vehicles by road authorities (via carcass collection operations) or citizen scientists such as wildlife groups is increasing due to human safety or animal welfare concerns and technological innovations (see Olson 2014; Shilling et al., 2015).  Analysis from our framework may guide mitigation actions in several problematic areas to reduce collisions.

Forman (2003) identifies both animal and human behaviour as major drivers for WVC.  Our framework operates on this hypothesised relationship and indicates maximum relative risk where animal presence and vehicles moving at speed co-occur in space.  Road authorities mitigate collision risk by managing human activity (traffic volume and speed) or animal behaviour (occurrence on or movement across the road) and these have different costs.  Our results demonstrated that speed is an important predictor of collision risk for both kangaroos and deer and therefore mitigation that either addresses animal presence near high speed road segments (e.g. fencing and crossing structures) or traffic speeds (e.g. controls and enforcement) in collision hotspot areas should be considered.  This pattern is consistent with other studies on deer collisions (Gkritza, 2013; Meisingset, 2014; Sudharsan, 2009), kangaroo collisions (Rowden, 2008), and other taxa (Gunson, 2011).  Moreover, the traffic models demonstrated plausible fits to the posted speed training data (Appendix S2: Table S2) therefore suggesting greater reliability in their predictions.  However, when assessing where to place mitigation for speed, the analyst and managers should also consider the uncertainty around predictions at each road segment for both speed and animal presence. If there is higher confidence around speed predictions, for example, more emphasis (and funds) may be placed on controlling vehicle speeds, rather than excluding animals.

The conceptual framework with sub-models allows us to more clearly identify bias and uncertainty in the analysis.  For example, the two occurrence models are produced using presence-only data that explicitly assumes perfect detection and includes potential sampling bias.  If these assumptions are deemed unsatisfactory, an analyst may choose to use alternative statistical methods that allow incorporation of uncertainty into the model (see Dorazio, 2014) or improve the training data (e.g. increase accuracy standards or collect additional records that include true absences).  The deer occurrence predictions may also be less reliable than the kangaroo predictions due to a smaller number of training observations.  Likewise, the lower statistical significance and wide confidence intervals in the marginal effects of traffic volume on collision likelihood for deer may warrant further analysis. The wide confidence intervals mean that reducing vehicles on problematic road segments may not have the desired effect.  Moreover, reported or observed collision data are subject to the same limitations as species occurrence data (e.g. reporting biases or data deficiencies such as under-reporting of actual presences) and should be taken into account.

As expected, collision risk for kangaroos and deer were similar in response to species occurrence, traffic volume and traffic speed.  Collision risk increased monotonically with increasing species occurrence for both kangaroos and deer, however, there was an order of magnitude greater change for kangaroos due to differences in ratios of collision to background observations.  Assuming species occurrence is of equal importance as the other collision model predictors, the smaller increase in collision risk for deer may be attributed to lower predictions of species occurrence arising from a mis-specified species distribution model (Guillera-Arroita, 2015).  Collision risk response to traffic volume was uni-modal for both species as expected, however, the change in collision risk was an order of magnitude greater per unit of traffic volume for kangaroos.  This response shape can arise from different mechanisms.  First, there may be a potential observer bias meaning people are less likely to stop or report collisions on high-volume roads out of concern for their own safety or simply not seeing the dead animal.  Second, there may be road avoidance effects as some species are repelled by roads when traffic volume increases beyond a threshold value due to high noise loads or small crossing intervals (see van Langevelde and Jaarsma, 2004; Seiler, 2005; Gagnon et al., 2007).  Our results are consistent with other studies on behavioural responses to roads that include kangaroos and deer (see Jacobson et al., 2016).  At higher values of traffic speed, collision risk increased monotonically with increasing speed with a similar magnitude for both species (Figures 2e and 2f).  Kangaroos were exposed to higher collision risk at lower traffic speeds than deer.  This may be due to Melbourne’s suburbs expanding into natural kangaroo habitat, thereby increasing wildlife-vehicle collisions.

We did not explore temporal effects in this study, however, this may be incorporated into our framework.  One such method is to add a function-based term to the model that allows collision risk to vary as a function of time and season (Visintin et al., manuscript in prep). The functional form may be expressed in relation to the known activity periods of the target species. Kangaroos are crepuscular, with peaks of activity at dawn and dusk, and deer are either crepuscular or nocturnal depending on the season.  Hour of day and time of year are useful predictors of collisions with ungulates (Meisingset et al., 2014; Mountrakis and Gunson, 2009), however, not yet fully tested for kangaroos.

The inferences made from the conceptual framework will be more robust as more observations are added to train the models.  We currently use a mixture of professional and citizen-science collected data - each with unique advantages, disadvantages, and implications for analysis.  Citizen-science data can supplement data-deficient ecological studies, and are effective in road ecology (Paul et al., 2014).  Moreover, citizen-science data is relatively low-cost and can cover large spatial scales.  For example, the collision records in both the Wildlife Victoria database and the California Roadkill Observation System are systematically collected and stored in databases and employ mechanisms to elicit data from unrestrained spatial distances.  As innovative and standardised data collection techniques are implemented (see Aanensen et al., 2009; Donaldson and Lafon, 2010; Shilling et al., 2015), inferences and predictions from the model framework will improve.

Our analysis suggests appropriate mitigation for both species may include reduced speeds or fencing on road segments where species occurrence is predicted to be high.  Although omitted for this work, our methods enable managers to conduct sensitivity analyses and calculate overall reductions in expected collisions based on mixed applications of mitigation on different road segments (assuming the mitigation is 100\% effective, but see Huijser et al., 2009).  As costs are often known for specific mitigation measures, simulations can determine optimal uses of resources to maximise reductions of collisions and minimise costs. We are excited to apply these methods to other species and geographical areas.
\chapter{Application of wildlife-vehicle collision model framework on different continents: Victoria and California}\label{sec:cal}

%\chapter{Spatial methods for modelling collision risk by other linear transport: kangaroo-train collisions in Victoria}\label{sec:train}
\newpage

\begin{localsize}{10}
\section*{\centering Abstract}

Despite considerable research on collisions between and wildlife and vehicles, animal mortality from strikes by trains is rarely studied.  In addition to animal welfare and conservation concerns, costs from train strikes may be considerable, and railway authorities have a vested interest to manage the problem. As railways expand spatially and operationally throughout the world, predicting and mitigating wildlife-train collisions will become increasingly important.

To assess the risk of wildlife-train collisions, we quantified regional train movements in space and time, and determined the likelihood of occurrence of eastern grey kangaroos in the State of Victoria, Australia. We then fitted a model to data on collisions between trains and kangaroos, accounting for time of day, train occurrence and speed, and kangaroo occurrence.  We predicted collision rates on a passenger railway network based on three management scenarios that influenced train speed and occurrence of kangaroos near the railway lines.

The model fit and predictions were plausible.  Train speed was the most influential variable followed by presence of kangaroos.  Reducing speeds in areas of high predicted kangaroo occurrence, during periods of peak animal activity within a twenty-four hour cycle, reduced collision rate the most.

Predictions from the model can help managers decide where, when and how best to mitigate collisions between animals and trains.  It can also be used to predict high-risk locations or times for (a) timetable/schedule changes (b) proposals for new railway lines or (c) disused lines considered for re-opening. The model framework is easily adaptable to other species and railway operations, and it allows managers to assess model performance and calibrate/update analyses accordingly.

\end{localsize}

\newpage
\section{Introduction}

Roads and railways support human civilisations by facilitating economic and recreational activities. However, transportation networks may directly or indirectly disrupt ecological systems \citep{seil06,rvdr15} and their environmental impacts must be managed \citep{spel98}. Animal mortality due to collision with vehicles is one of the most visible impacts \citep{form03} and a primary cause of reductions in population size \citep{fahr09}.
 
Wildlife-vehicle collisions pose a serious global problem \citep{litv08}, spawning a new discipline (road ecology) and inspiring research to develop solutions. For example, deer-vehicle collisions on roads are well-studied in North America \citep{huij07a,romi96} and Europe \citep{saen15,seil04}. Moreover, management of wildlife-vehicle collisions will become increasingly important as new transportation networks are constructed and existing networks are expanded, perhaps most significantly in developing countries.

In addition to concerns about animal welfare \citep{sain95} and conservation status of threatened species \citep{dwye16,jone00}, collisions with large animals pose direct risks to the life of humans \citep{lang06,rowd08}.  For example, moose are one of the largest animals struck by vehicles in North America and Europe, causing significant damage and injuries \citep{huij07a,hurl09}.  Vehicle collisions with deer, although smaller than moose, frequently kill humans in North America \citep{will05}.

Information about the spatial and temporal distribution and magnitude of wildlife-vehicle collisions is useful to managers because it may help more effectively mitigate impacts \citep{moun09}. For example, knowing a collision hotspot along a transportation network for a particular species will inform the most appropriate form of mitigation (e.g. animal exclusion or change in vehicle speed). Data can also inform statistical modelling which helps to predict the probability of wildlife-vehicle collisions \citep{guns11}.

The majority of wildlife-vehicle collision modelling deals with roads and traffic \citep{rvdr15}, yet, the problem extends to other forms of vehicular transportation such as air \citep{vanb07}, railway \citep{well99} and shipping \citep{lais01} operations. Regardless of the mode of transport, the modelling of collisions share some common attributes \citep{form03}. The movements or presence of animals are often considered in the models and may include behavioural traits \citep{roge09}, and vehicle presence or movements can also be considered and may be grouped into a larger category of human behaviour as humans ultimately control speeds and trajectories of vehicles \citep{ramp08}.

Extensive railway networks with considerable train movements exist on every inhabited continent in the world, and although broader ecological effects have been discussed \citep{desa93,givo06} and analysed \citep{wall05}, very few studies analyse wildlife-train collisions (but see \cite{bela95,onoy98}). Many anecdotal records/observations illustrate the severity of wildlife-train collisions such as group fatalities (e.g. herds of ungulates or flocks of birds) due to browsing on railways or train derailments from very large species \citep{dors15}. These serious events suggest a growing need for more research into predicting wildlife-train collisions; however, we are only aware of one published study (see \cite{gund98}). Here, we develop a modelling framework to predict the rate of kangaroo collisions on the regional passenger railway network in southeast Australia. In addition to informing railway operators in south-east Australia of potential kangaroo collision risks, our approach can be generalised to other species (e.g. deer) and railway operations (e.g. freight transport) elsewhere in the world.

\section{Materials \& Methods}

\subsection{Study Area}

\begin{figure*}[htp]
  \centering
  \includegraphics[scale=.5]{05_study_area.png}
  \caption[]{Regional passenger train network in the state of Victoria. Inset shows location of Victoria in Australia. The railway network is shown as thin lines through major towns (stars). Wildlife-train collisions (reported between 2009-2014) are shown as crosses.}
  \label{trains_study_area}
\end{figure*}

We used a 1712-kilometre passenger railway network from regional Victoria, Australia (operated by V/line, a government-owned corporation) in south-east Australia to conduct our study (\Cref{trains_study_area}).  Trains operate on all sections of the network between 4 a.m. and 2 a.m. (the following day), with the largest volume occurring Monday through Friday between 7 a.m. and 9 a.m., and 4 p.m. and 6 p.m. Most trains operate at speeds of 100 km h-1 or less, however, on some sections of track commuter trains operate at maximum speeds of 160 km h-1. Due to limited data available, we did not include freight operations in our study.

\subsection{Data Preparation}

To organise our data and modelling, we overlaid a 1-km2 cell grid on the railway network (\Cref{trains_grid}).  In each grid cell we modelled species occurrences and quantified train movements and speeds.

\begin{figure*}[htp]
  \centering
  \includegraphics[scale=.5]{05_gridwork.png}
  \caption[]{One km$^2$ grid framework used to organise modelling data: 2,015 total cells; extent coordinates 104000E,5741000N x 556000E,6084000N; GDA94 MGA zone 55 projection. The railway network is shown as a heavy dashed line and wildlife-train collisions are shown as crosses.}
  \label{trains_grid}
\end{figure*}

Eastern grey kangaroos (\textit{Macropus giganteus}, Shaw, 1790; "kangaroos" hereafter) are frequently struck in regional Victoria and large enough to cause damage and subsequent maintenance (e.g. cleaning) of trains.  V/line provided records of all driver-reported collisions with kangaroos spanning a six-year period between 1 January 2009 and 31 December 2015, a total of 439 collisions. Collisions with large mammals (e.g. domestic livestock or kangaroos) must be reported to allow trains to be inspected and maintenance performed as required. Each record included incident date and time, name of service line (unique route between two towns), and nearest fraction of a kilometre post (physical sign markers indicating distance along a railway line).  Using geographic information system (GIS) data on the regional railway network, we determined spatial coordinates (GDA94 MGA zone 55 projection) for all collisions from the reported kilometre post and service line. Although, uncertainty in estimates of location was not explicitly reported, we assumed a maximum possible error of $pm$500 metres.

\subsection{Species Occurrence}

Kangaroos are widespread \citep{daws12} and abundant in many parts of Victoria, yet comprehensive distribution records are lacking in many areas. To represent risk of collision by exposure to threat, we required distributional data across the entire study area and used species distribution modelling to predict relative likelihood of kangaroo occurrence.  We emulated methods by \cite{elit08}) to model and predict occurrence in each grid cell for the whole State of Victoria. The model was trained on data from the online Victorian Biodiversity Atlas \citep{depi16} and included several environmental variables relating to the biology and behaviour of kangaroos (see \Cref{sec:egk}).  To reduce the effects of sampling bias, we also included four additional predictors:  1) distance to urban areas, 2) distance to roads, and the 3) easting and 4) northing spatial coordinate of the grid cell centroid.

\subsection{Characteristics of Railway Network}

To determine train movements across space and time, we accessed publicly available locations of stops and times along train routes from V/Line general transit feed specification (GTFS) data (Public Transport Victoria, accessed online 3 March, 2016).  GTFS is a standard publishing format developed and maintained by a community of public transport agencies for scheduling and spatial data.  Since it is publicly available, it also allows software developers to write applications for mobile devices that track and report the locations of public transportation (see \cite{hill11}).  We used a spatial database (Postgres version 9.6; PostGIS version 2.3.0) to process this information and report the average scheduled number of trains, the total length of track, and average train speed in each grid cell for each hour of the day where trains occurred.

\subsection{Statistical Modelling}

We adapted the single-species quantitative risk model in \Cref{sec:egk} to fit and compare the relationship of kangaroo presence and characteristics of the railway network to collision likelihood. To account for temporal variation in collision risk throughout the day, we considered periods of train movements and animal activity in relation to time of day. By adding a function that allowed a bimodal response of collision rate to hour of day across all seasons, the crepuscular lifestyle of kangaroos (most active at dawn and dusk; see \cite{daws12}) is incorporated. The likelihood that a collision occurs in a given grid cell $i$ at hour $j$ in month $k$ ($p_{ijk} = \text{Pr}(Y_{ijk}=1)$) depends on species occurrence $O$, average number of trains $V$, average train speed $S$, length of track $T$ (offset term), and a term $C_{ijk}$ that accounts for the crepuscular behaviour of kangaroos:

\begin{equation} \label{eq:51}
\text{cloglog}(p_{ijk})=\beta_0+\beta_1\text{log}(O_i)+\beta_2\text{log}(V_{ij})+\beta_3\text{log}(S_{ij}) + C_{ijk} + \text{log}(T_i)
\end{equation}

\noindent The crepuscular term $C_{ijk}$ has three components:

\begin{equation} \label{eq:52}
C_{ijk} = \gamma_1\text{sin}\left(\frac{\pi(j-6)}{12}\right) + \gamma_2\text{sin}\left(\frac{\pi(j-6)}{12}\right)^2 + \gamma_3 A_{ijk}
\end{equation}

The first two terms are the linear and quadratic terms of a sine function that varies relative to 6 a.m. The third term $A_{ijk}$ models the influence of time since dawn and dusk. This function takes one of three forms depending on whether the time is before dawn, after dusk, or between dawn and dusk:

\begin{equation} \label{eq:53}
A_{ijk}=
 \begin{cases}
 
e^{ - \left( \frac{j-U_{ik}}{\frac{(24-D_{ik})+U_{ik}}{2}}\right)^2} - e^{ - \left( \frac{j-(D_{ik}-24)}{ \frac{(24-D_{ik})+U_{ik}}{2} } \right)^2} & , \text{if}\ j < U_{ik}\\[.2cm]

e^{ - \left( \frac{j-U_{ik}}{\frac{D_{ik}-U_{ik}}{2}}\right)^2} - e^{ - \left( \frac{j-D_{ik}}{ \frac{D_{ik}-U_{ik}}{2} } \right)^2} & , \text{if}\ U_{ik} \leq j < D_{ik}\\[.2cm]

e^{ - \left( \frac{j-(U_{ik}+24)}{\frac{(24-D_{ik})+U_{ik}}{2}}\right)^2} - e^{ - \left( \frac{j-D_{ik}}{ \frac{(24-D_{ik})+U_{ik}}{2} } \right)^2} & , \text{if}\ j \geq D_{ik}

 \end{cases}
\end{equation}

Dawn $U$ and dusk $D$ times for civil twilight (six degrees below horizon) were determined for each observation in the dataset using a National Oceanic and Atmospheric Administration (NOAA) astronomical algorithm in the `R' package \textit{maptools}. We selected the 15th day of each month $k$ at the centroid of each respective grid cell $i$ to calculate dawn and dusk times.

The coefficients $\gamma_1$, $\gamma_2$, $\gamma_3$ are estimated from the data and measure the relative influence of the three components (Equation 2) on the shape of the curve. One advantage of our parameterisation of the crepuscular term is that it is cyclical over a twenty-four hour period (\Cref{train_crep}) regardless of the parameter estimates of $\gamma_1$, $\gamma_2$, $\gamma_3$.

\begin{figure*}[htp]
  \captionsetup[subfloat]{farskip=-2pt,nearskip=-2pt}
  \centering
  \subfloat[]{\label{train_crep:a}\includegraphics[width=0.4\textwidth]{05_crep01.png}}
  \subfloat[]{\label{train_crep:b}\includegraphics[width=0.4\textwidth]{05_crep29.png}}\\
  \subfloat[]{\label{train_crep:c}\includegraphics[width=0.4\textwidth]{05_crep24.png}}
  \subfloat[]{\label{train_crep:d}\includegraphics[width=0.4\textwidth]{05_crep26.png}}
  \caption[]{a) Components used to build crepuscular term in \Cref{eq:52}; the solid curve controls activity relative to day or night, the dashed line controls uni-modal (e.g. diurnal) or bi-modal (e.g. crepuscular) activity, the dotted line controls activity closer to dawn or dusk. The model estimates a coefficient ($\gamma$ in \Cref{eq:52}) for each component that controls the final curve shape. The resulting curves may reflect b) crepuscular, c) nocturnal, or d) diurnal activity patterns. Example coefficient values (g1, g2, g3) used to produce curves are shown above each respective graph.}
  \label{train_crep}
\end{figure*}

Prior to modelling, we centred all explanatory variables by subtracting their means (after log-transforming the variables indicated in Equation 1). Using pairwise analysis, all predictors exhibited Pearson's product moment correlation coefficients of less than 0.4 indicating low potential effects of multi-collinearity.  \Cref{train_predictors} further describes the variables used in the model.

\begin{table}[htp]
\caption{Predictor variables used in collision model; means and ranges are expressed on untransformed scale.  C1, C2 and C3 determine a curve shape representing relative species activity (see \Cref{eq:52} and \Cref{train_crep}).}
\begin{tabularx}{\textwidth}{llll} \toprule
Variable	&Description												&Units				&Mean; Range \\ \midrule 
EGK			&Relative likelihood of kangaroo occurrence in grid cell	&-					&0.13; 0.01:0.87 \\ 
TRAINS		&Train frequency in grid cell								&trains h$^{-1}$	&6; 1:80 \\ 
SPEED		&Mean train speed in grid cell								&km h$^{-1}$		&86.75; 3.92:147.73 \\ 
C1			&Crepuscular function linear component						&- 					&0.25; -1.00:1.00 \\
C2			&Crepuscular function quadratic component					&-					&0.47; 0.00:1.00 \\
C3			&Crepuscular function astronomical component				&-					&-0.05; -0.98:0.98 \\
\bottomrule
\end{tabularx}
\label{train_predictors}
\end{table}

We fit the data (n=291,120 one-kilometre grid cells along the train network) to a generalised linear model \citep{mccu89} using maximum likelihood estimation with a binomial distribution and a complementary log-log link on the linear predictor.  The complementary log-log link was selected over the more common logit link due to the mathematical theory underpinning our model - risk being measured by the rate of collisions (see \Cref{sec:egk}).  The model is similar to a proportional hazards model (discrete censored time) often used in survival analysis and epidemiological studies \citep{cox84}.

To examine model fit, we generated randomised quantile residuals \citep{dunn96} and plotted them against collision probability, kangaroo occurrence, number of trains per hour, average train speed, and hour. To assess performance, we cross-validated the model by randomly splitting the data into K=10 partitions. We used nine of these subsets for model fitting and one for assessing model accuracy.  For each assessment we obtained two performance metrics; area under the receiver operator characteristic (ROC) curve \citep{metz78} and regression of observations on predictions \citep{cox89,mill91}. We repeated this procedure for 100 iterations producing a total of 1000 sets of performance metrics and compared them with those from the model fitted to all data.

Using the model fitted on all data, we predicted the number of expected kangaroo-train collisions in the study area under different management scenarios: 
A) no change to operations, 
B) moderated train speeds in high kangaroo occurrence areas, and 
C) reduced kangaroo occurrence in areas with highest average speed of trains.

Scenario B involved capping train speeds at 80 km h-1 in grid cells with kangaroo relative occurrence likelihoods of 0.5 or above during the hours of 5 a.m. to 9 a.m. and 4 p.m. to 8 p.m.  This change affected 42 cells and 275 out of 3,024 unique weekly routes.  In scenario C, relative kangaroo occurrence was reduced by approximately half in all grid cells with average train speeds of more than 120 km h-1 (n = 154 cells, total track length = 121 km). This simulates reducing access of kangaroos to the railway network (e.g., by fencing).  The values for reducing kangaroo abundance were used for all hours of the day as this management strategy would most likely involve exclusion or reduction in animal populations which operate irrespective of temporal variation.

\section{Results}

Our model fitted the data and the signs of model coefficients (positive or negative) indicated plausible relationships between collision likelihood and each predictor variable.  All variables except train frequency demonstrated high statistical significance in the model fit (\Cref{train_coefs}). The model residuals demonstrated no strong patterns with respect to predicted collision probabilities and model predictors (\Cref{train_brq}).

\begin{table}[htp]
\caption{Summary of model fit using all data ($n$=291,120 grid cells).  Highly significant variables are marked with an asterisk.}
\begin{tabularx}{\textwidth}{lllll} \toprule
Variable			&Beta Coefficient Estimate	&Standard Error of Coefficient Estimate	&z-value &$\PRZ$ \\ \midrule 
Intercept			&-7,2						&0.09	&-79.03 	&$<$1.0x10$^{-14}$* \\ 
EGK	($\beta_1$)		&0.61						&0.06	&10.46		&$<$1.0x10$^{-14}$* \\ 
TRAINS ($\beta_2$)	&0.02						&0.09	&0.17		&8.7x10$^{-2}$ \\ 
SPEED ($\beta_3$)	&3.62						&0.31	&11.53		&$<$1.0x10$^{-14}$* \\ 
C1 ($\gamma_1$)		&-0.66						&0.11 	&-5.83		&5.7x10$^{-7}$* \\
C2 ($\gamma_2$)		&-1.87						&0.17	&-10.84		&$<$1.0x10$^{-14}$* \\
C3 ($\gamma_3$)		&0.25						&0.07	&3.77		&1.7x10$^{-5}$* \\
\bottomrule
\end{tabularx}
\label{train_coefs}
\end{table}

The relative risk of collisions increased with higher average train speeds, kangaroo occurrence, train frequency and during hours of high kangaroo activity in grid cells. The strongest and most significant predictor was train speed; collision risk increased exponentially with considerable increases at speeds above 85 km hr-1.  Increasing train speed from 110 to 130 km hr-1 doubled collision risk, however, the effect of speed also demonstrated large uncertainty, with wide confidence intervals (\Cref{train_effects:a}). Kangaroo occurrence was the second most influential predictor of risk of collision.  Collision risk increased rapidly at low values of occurrence and more slowly at higher values.  Collision risk increased approximately 10-fold across the range of values for kangaroo occurrence (\Cref{train_effects:b}). Train frequency had very little influence on collisions, compared to the other predictors (\Cref{train_coefs}; \Cref{train_effects:c}).

\begin{figure*}[htp]
  \captionsetup[subfloat]{farskip=-2pt,nearskip=-2pt}
  \centering
  \subfloat[]{\label{train_effects:a}\includegraphics[width=0.4\textwidth]{05_speed.png}}
  \subfloat[]{\label{train_effects:b}\includegraphics[width=0.4\textwidth]{05_egk.png}}\\
  \subfloat[]{\label{train_effects:c}\includegraphics[width=0.4\textwidth]{05_trains.png}}
  \subfloat[]{\label{train_effects:d}\includegraphics[width=0.4\textwidth]{05_hour.png}}
  \caption[]{Marginal effects of model predictors on collision risk.  For the effects of species occurrence, number of trains, and train speed on collision risk, non-target variables were held constant at mean values.  For the effect of hour on collision risk, species occurrence, number of trains, and train speed were held at mean values. Shading indicates 95\% confidence intervals around coefficient estimates.}
  \label{train_effects}
\end{figure*}

The bi-modal functional form for the effect of kangaroo activity on collision risk demonstrated a plausible shape (\Cref{train_effects:d}).  Collision risk peaked at approximately 5:45 a.m. and 6:15 p.m. with a higher risk occurring in the morning.  The response of collision risk over time was most uncertain in the evening peak.  Both peaks showed similar shape - the density and spread of collision risk around the maxima - and the lowest collision risk occurred around noon.

\begin{figure*}[htp]
  \captionsetup[subfloat]{farskip=-2pt,nearskip=-2pt}
  \centering
  \subfloat[]{\label{train_calib:a}\includegraphics[width=0.5\textwidth]{05_calib.png}}\\
  \subfloat[]{\label{train_calib:b}\includegraphics[width=0.4\textwidth]{05_roc.png}}
  \subfloat[]{\label{train_calib:c}\includegraphics[width=0.4\textwidth]{05_validate.png}}
  \caption[]{a) Calibration plot showing rate of observed collisions against predicted rate of collisions.  Dots represent the observed rate with 95\% confidence intervals at the medians of each bin of predictions (10 total). Labels indicate the total observations in each bin. A regression line is shown between the dependent variable and the predicted values (response-scale) of the model; perfect calibration has an intercept of 0 and slope of 1; b) ROC (receiver operating characteristic) curve measuring discrimination ability of model at all threshold values. Values shown are averages of all cross-validations; c) Comparison between the collision model fit on full data and on cross-validated subsets. "Intercept" and "Slope" result from regressing the dependent variable on the predicted values and measure calibration (see plot a); ROC measures discrimination between collisions and no-collisions (see plot b). For each metric, open circles represent the full data model and solid dots represent mean values (95\% confidence intervals shown as bars) for the 1000 cross-validated subsets.  Dashed lines indicate the expected values for a perfectly calibrated and discriminatory mode}
  \label{train_calib}
\end{figure*}

The models fitted on all data and fitted on the subsets of the data during cross-validation performed similarly.  The ROC value was 0.82 for both the full data model (\Cref{train_calib:b}) and mean of the cross-validated models (\Cref{train_calib:c}).  Likewise, the calibration statistics (intercept and slope of regression line between observations and predictions) were similar for both the full data model (\Cref{train_calib:a}) and mean of the cross-validated models (\Cref{train_calib:c}).  The uncertainty in the calibration metrics was higher than that of the ROC values as shown by the 95\% confidence intervals.  The overall calibration of the full data model was good for low collision rates where the uncertainty around the observed rates was also low. Observed rates with higher uncertainty were evident at higher predicted values (\Cref{train_calib:a}).

\begin{table}[htp]
\caption{Summary of model fit using all data (n=291,120 grid cells).  Highly significant variables are marked with an asterisk.}
\begin{tabularx}{\textwidth}{lll} \toprule
Scenario	&Description	&Expected Total Collisions \\ \midrule 
A			&no change to current operations or infrastructure			&404 \\ 
B			&moderated train speeds in high kangaroo occurrence areas during peak travel times			&400 \\ 
C			&controlled kangaroo occurrence in areas with highest average speed of trains			&392 \\ 
\bottomrule
\end{tabularx}
\label{train_manage}
\end{table}

Both of the simulated management scenarios reduced the predicted number of collisions from the baseline estimated with no management (Scenario A).  Scenario C reduced expected collisions by approximately 3.0\% whilst scenario B reduced collisions by 1.0\% (\Cref{train_manage}).

\begin{figure*}[htp]
  \captionsetup[subfloat]{farskip=-2pt,nearskip=-2pt}
  \centering
  \subfloat[]{\label{train_brq:a}\includegraphics[width=0.45\textwidth]{05_brq_prob.png}}\\
  \subfloat[]{\label{train_brq:b}\includegraphics[width=0.35\textwidth]{05_brq_egk.png}}
  \subfloat[]{\label{train_brq:c}\includegraphics[width=0.35\textwidth]{05_brq_trains.png}}\\
  \subfloat[]{\label{train_brq:d}\includegraphics[width=0.35\textwidth]{05_brq_speed.png}}
  \subfloat[]{\label{train_brq:e}\includegraphics[width=0.35\textwidth]{05_brq_hour.png}}
  \caption[]{Binned randomised quantile (RQ) residuals plots - average residual versus the average predictor value for each bin. For all continuous predictors, bins are equally spaced and determined by taking the square root of the total number of unique values for each predictor; for the discrete predictor of hour, there are 23 total bins. The gray lines indicate ±2 standard-error bounds which should contain 95\% of the residuals if the model were actually true. Note some variables are log-transformed to assist with visual inspection of the residuals.}
  \label{train_brq}
\end{figure*}

\section{Discussion}

Our model demonstrates that kangaroo-train collisions are related to train speed, kangaroo exposure to moving trains, and the coincidence of periods of high train and kangaroo activity.  All of these relationships are consistent with our initial expectations and also shown in related studies on railway \citep{gund98} and road \citep{lao11,roge12} collisions. Our work extends further by presenting a conceptual modelling tool to assist managers create safer and more cost effective railways.

Train speed was an important predictor for collision risk.  As trains increased speed, the risk of collisions increased rapidly, up to our typical maximum speed of 160 km h-1.  Collision risk relating vehicle speed to the size and velocity of animals has been demonstrated through simulation \citep{jaar06} concluding that smaller and slower-moving species are more vulnerable (also see \cite{fahr09}). Moreover, as these relationships are often exponential, high-speed vehicles may become significantly problematic regardless of species trait. It should be noted that our study utilised published schedule data to interpolate train movements in space and time. Therefore, there is uncertainty in both the location and trajectories of actual trains. Further study using global positioning system (GPS) waypoints of train movements would reduce some of this uncertainty.

Kangaroo occurrence is also a useful predictor for collision risk. Collision risk consistently increased with predicted relative occurrence, as found for roads \citep{lao11,roge09}.  One feature of the model framework is the flexibility of choice in how to represent species occurrence.  We employed published methods to determine kangaroo occurrence, however, our framework is not limited to data derived using only this type of model.  The species distribution modelling literature is vast and covers topics relating to model choice \citep{guil15}, calibration and bias \citep{phil10}, sources of data \citep{vans13}, and validation \citep{chiv14}.  Our framework can also incorporate data from population viability analyses to test the effects of population dynamics on collision risk. For example, collisions can correlate with expected counts of single species in a given area \citep{skor13}.  Kangaroos in Victoria are not subject to hunting pressure to the same extent as ungulates in parts of North America and Europe, which has been shown to affect the rate of train-wildlife collision \citep{seil05}.  However, population control of kangaroos occurs periodically in certain areas and our model framework can accommodate changes in occurrence driven by these factors by including relevant predictor variables. 

Temporal patterns, such as the crepuscular activity of wildlife, have implications for collision risk. Hourly and seasonal patterns have been applied differently in wildlife collision research.  Some studies treat temporal predictors as a categorical variable \citep{duss06}, whilst others have explicitly defined cyclic functions \citep{thur15}.  Our model uses a term with three components to define a functional form that is flexible to the crepuscular (bi-modal) nature of kangaroo activity; which also happens to coincide with peak train activity in particular seasons. The use of control parameters estimated from the data allows this temporal function to suit the behaviour of any target species (e.g. nocturnal, diurnal, crepuscular and random) - see \Cref{train_crep}.  Moreover, seasonal movements such as migration (see \cite{neum12}) may also be included in the model specification, however, kangaroos do not migrate \citep{daws12}.

The collision data used for this study has unique properties with respect to reporting bias and errors.  Train drivers are obligated to report the time and location of large animal strikes as they usually result in damage to or required cleaning of trains.  Therefore, these data are less subject to reporting bias compared to many road collision studies (but see \cite{snow15}).  Moreover, spatial and temporal errors in reported collisions are assumed to be less as standardised mechanisms such as collision report forms, GPS devices, and distance signage are implemented in railway operations. Similar practices are used by some road authorities to collect and archive carcass data \citep{huij07a}, however, the coverage is often sparse due to the spatial extent of road networks and temporal uncertainty of collision events.  Technology has been shown to assist with data collection \citep{olso14,shil15b} and similar approaches may also be applied to railway networks.

We used existing data to create predictors for the model framework. The data are publicly-available online and, in some cases, maintained and updated regularly. This reduces potential costs involved in the collection of data. Moreover, the model framework may be easily updated as new information becomes available.  As some of the data in the framework result from modelling (species occurrence) or interpolation (train movements), potential inherent uncertainty should be considered when drawing inferences. For example, the relative effect of each predictor may be weighted according to associated uncertainty or experts may be used to assess sub-model predictions (e.g. species occurrence - see \cite{clev02} and \cite{wint05}).

Our framework supports management decisions in two distinct areas: reduction of animal presence (e.g. deterrents or exclusions) or reduction of train threat (e.g. adjusted schedules or speeds).  The choice of mitigation may be influenced by the effects of each predictor on collision likelihood (e.g. if speed is more correlated or has a stronger influence).  This is determined by examining the model fit or predicting responses based on changes in parameter values (e.g. increasing likelihood of kangaroos).  Mitigation choice may also be limited by operational objectives.  Fencing may be chosen to exclude animals on railways when changes to train speed and frequencies are not possible or desirable, regardless of the effect in the model. Most high-speed railways are fenced \citep{camp09} as the speed of trains is the dominant technological characteristic and reducing it may be contrary to its public service objective. The scenario we did not model - installing 100\% effective fencing along the entire network - would have the effect of reducing train collisions with ground-traversing species to zero, but is not realistic and thus omitted in this study.

Each of our management scenarios reduces collisions which is a positive outcome.  Although many negative outcomes - not easily quantified - are related to collisions (e.g. trauma, loss of life), monetary costs are a useful metric for comparing management options. From a transportation authority perspective, collisions with kangaroos in Australia incur costs through removal of trains from service for cleaning and repair.  These costs will vary by railway operator and are useful to compare with costs of mitigation.  For example, let us assume that a Victorian regional passenger train must be taken out of service following a collision with a kangaroo and the cost of this activity is \$20,000.  By reducing the annual number of collisions by 10, or approximately 2.5\% of reported collisions in this study, we have an estimated savings of \$200,000 per year.  If the costs of different management strategies are also calculated, a cost-benefit analysis may be performed.  This has been applied to collisions with road vehicles \citep{huij09} and the concepts are similar for railway transport. Further, the benefit from reductions in annual collisions will have varying magnitudes depending on the costs incurred from wildlife-train collisions. In India, for example, collisions with elephants may de-rail trains \citep{dors15} resulting in significant costs. In these situations, avoiding a single collision event has remarkable benefit.

Herein, we have demonstrated a model framework that functions as an effective management support tool. It utilises existing sources of data, is logically organised, and is transferable/scalable to other networks and species. Other potential uses of the framework may include an ongoing implementation where the model is updated based on new information and reports risk to operators in real-time.
\chapter{Spatial methods for modelling collision risk by other linear transport: kangaroo-train collisions in Victoria}\label{sec:train}

%\chapter{Validating collision model predictions with disparate datasets: a case for a unified system}\label{sec:val}
\newpage

\begin{localsize}{10}
\section*{\centering Abstract}

Wildlife-vehicle collisions are often costly due to property damage and injury and have been estimated to kill billions of fauna each year. Many studies use predictive modelling to determine where and when WVC will occur. These models are often trained on collision data that is sourced from non-government organisations, private-sector entities, academic institutions, government agencies, community groups, and individual citizen scientists. These data are not easily attainable and often have implicit biases and errors.

We analysed change in model performance by combining four disparate datasets of collisions in the State of Victoria in south-east Australia. For each combination of training datasets, we fitted a model, made predictions, and validated the predictions with an independent dataset not used to train the model. We also used a collision dataset of total insurance claims per town to validate the predictions of models using different combinations of data. 

Among the four datasets, the discrimination ability of the model did not increase as data were added to models. Using the town-level independent data to validate models resulted in an increase in model calibration as more combinations of training data were used. Likewise, the variation in the data explained by the model also increased as more combinations of data were used.

We demonstrate that increasing the amount of collision data used for modelling, by combining disparate datasets, has the potential to improve model performance and predictive strength. These results highlight the importance of uniting the currently disparate sources of collision data with a comprehensive and systematic archive. We also suggest qualities that are integral to the success of such a system.
\end{localsize}

\newpage
\section{Introduction}

Roads are critical infrastructure, but their development and use has ecological impacts \citep{form03}. Perhaps one of the most directly visible and confronting issues is vehicle strikes to animals, especially large terrestrial vertebrates. These collision events are often costly due to property damage, injury, and death. Globally, wildlife-vehicle collisions (WVC) on transportation networks have been estimated to kill billions of fauna each year \citep{seil06}. In North America, millions of animals are estimated to be killed by vehicle strikes each day \citep{form98} and annual economic costs due to collisions exceed eight billion dollars \citep{huij07b}. However, these adverse effects are not unique to North America and occur at all locations where roads dissect habitat for wildlife. Due to such large impacts, governments, scientists, private-sector companies and citizens throughout many nations have undertaken data collection and analysis of WVC (see \cite{rvdr15}.

A large amount of research seeks to determine the frequencies and magnitudes of wildlife-vehicle collisions. This often involves examining several environmental (e.g. landscape, climate, road characteristics), anthropogenic (e.g. human behaviour, vehicle movements) and biotic (e.g. animal traits) factors at different spatial and temporal scales \citep{litv08}. More recently, computer modelling has been employed to predict where and when collisions are most likely to occur based on a large range of predictor variables \citep{guns11}. These predictions aim to help environmental managers and road authorities to identify and mitigate problematic areas and better construct new roads. Effective decisions made from these predictions rely on robust, calibrated models as incorrect inferences may result in costly outcomes such as ineffective mitigation (see \cite{huij09}) or continuing effects of unmitigated collisions (see \cite{biss08b}). Data deficiencies generate uncertainty in models and result from data collected with specific procedures or spatial/temporal trends (e.g. bias), measurement error or inappropriate sample size \citep{bean12}. Although other factors, such as model misspecification, may also result in biased and uncertain predictions, we argue that the quality and quantity of data used to train WVC predictive models are of paramount importance.

Many studies that use predictive modelling to determine where and when WVC will occur are limited to single sources of data - either field collected based on predetermined project requirements (e.g. \cite{lang09}; \cite{roge09}) or pre-existing from independent surveys or routine collections (e.g. \cite{hoth12}; \cite{malo04}). Bias and error may arise from the data sources and collection methods, however, these are not often explicitly reported or accounted for in the modelling methods. Further, few studies utilise independently-obtained data to validate model performance; often this is due to limitations in availability. In response to the importance of consistent data collection, technology has been proposed to augment field activities. For example, mobile phones have been proposed to assist with data collection of wildlife-vehicle collisions \citep{aane09,olso14}. Tablets or personal digital assistants (PDA) have also been proposed to assist road authorities to record collisions \citep{amen07}. Cataloguing and reporting systems are used to store data on WVC, however, few are open access or incorporate data from other institutions or previous surveying events. Perhaps one of the largest systems in current use is the California Roadkill Observation System (CROS), maintained by the Road Ecology Center at the University of California, Davis \citep{shil15b}. This system offers many features such as public-accessibility, an intuitive user interface, and collection of important metadata on reliability and precision of reporting, but has yet to achieve significant uptake by agencies outside the academic and citizen-science networks it supports.

%\begin{table}[htp]
%\caption[Potential biases present in collision data]{Potential biases present in collision data collected under different example scenarios. Levels of bias are relative and coded as: high (H), medium (M), low (L)}
%\centering
%\begin{tabularx}{0.9\textwidth}{P{7cm}ccc} \toprule
%Scenario						&Spatial Bias		&Temporal Bias		&Species Bias \\ \midrule
%Contract for carcass collection along a specific road section	&H	&H	&L \\
%	&					&					& \\
%Well-designed survey to identify factors influencing WVC	&L	&L	&M \\
%	&					&					& \\
%Citizen recording carcasses of large animal along daily commute	&H	&	&H \\
%	&					&					& \\
%Retired citizen recording carcasses of all animals in regional area	&	&	& \\
%	&					&					& \\
%Citizen only recording carcasses of interesting animals	&	&	& \\
%	&					&					& \\
%Member reporting collision with animal to insurance company	&	&	& \\
%	&					&					& \\
%Road authority collecting carcasses of large animals	&	&	& \\
%	&					&					& \\
%City council collecting carcasses of reported animals	&	&	& \\
%	&					&					& \\
%Citizen reporting collision with animal to wildlife organisation	&	&	& \\
%	&					&					& \\
%Police responding to collision with animal	&					&					& \\
%\bottomrule
%\end{tabularx}
%\label{biases}
%\end{table}

Worldwide, collision data is collected by non-government organisations, private-sector entities (e.g. insurance companies), academic institutions, government agencies (e.g. road authorities and environmental managers), community groups, and individual citizen scientists but these data are not always easily attainable due to privacy restrictions or data formats. We assert that modelling and prediction of WVC will benefit from extensive, open-access data repositories that catalogue collision records from these entities. These mechanisms exist in other areas of research. For example, contemporary species distribution modelling draws from worldwide atlases of species occurrence; the Global Biodiversity Information Facility (www.gbif.org) contains more than 700 million records from over 800 data publishers and the Atlas of Living Australia (http://www.ala.org.au) provides over 50 million records from a wide range of data providers.

In our study, we analyse the predictions of a wildlife-vehicle collision risk model using a citizen-science dataset joined with four external independent datasets in different combinations. As the sample size and variation in collection techniques (location, expertise, redundancy) of our training data increases, we analyse the performance of our models. Our overall aim is to demonstrate the value of a unified, publicly-accessible data storage system to support analysis and prediction of WVC and determine characteristics of data that are important to ensure robust analyses.

\section{Materials \& Methods}

Our study analyses risk of WVC over approximately 150,000 kilometres of sealed roadway (i.e. only bitumen surfaces) in the State of Victoria in south-east Australia (\Cref{val_study_area}). The Victorian road authority (VicRoads) manages 25,256 kilometres of major roadway and seventy-nine municipal districts manage the remaining roads. We split the sealed road network into 644,715 road segments by intersecting the roads with a 1km$^2$ spatial grid. We predicted traffic volume and speed to each road segment using linear regression and predicted species occurrence to the spatial grid using species distribution models. Species occurrence was sampled using the midpoints of the road segments resulting in each having predicted values of kangaroo occurrence, traffic speed and traffic volume in the final dataset - more details are provided in \Cref{sec:egk}. 

\begin{figure*}[htp]
  \centering
  \includegraphics[scale=.5]{06_study_area.png}
  \caption[Study area for model validation]{Study area (state of Victoria, south-east Australia) showing all sealed road segments as light gray lines. The inset shows the geographic location of Victoria in Australia. The darker shaded region is the City of Bendigo and the lighter shaded region is the VicRoads Western District. Major towns ($>$100,000 residents) are starred and labelled accordingly.}
  \label{val_study_area}
\end{figure*}

Eastern Grey kangaroos (\textit{Macropus giganteus}, Shaw) are the second largest native mammal (up to 85 kilograms for males) in Australia \citep{coul10}, and the most commonly reported species struck by motor vehicles in Victoria \citep{rowd08}, prompting several organisations to record carcass collection and collision locations. We obtained four datasets of spatially-explicit collisions with Eastern Grey kangaroos in the state of Victoria in south-east Australia for our study (\Cref{val_data}). The data were sourced from: the City of Bendigo, a large regional center located in north-east Victoria, the Western District, a large VicRoads administrative boundary located in the center of the western portion of Victoria, Crashstats, an online repository of police-reported WVC, and Insurance Australia Group, an insurance company that records WVC reported by their members. Data from the City of Bendigo was based on carcasses reported to the 3,001 km$^2$ municipality by members of the public. The Western District data was comprised of reported carcasses, along 1,417 km of major roads within an area of 24,373 km$^2$, made by a VicRoads-authorised litter-abatement contractor. Crashstats data covered the entire state and was based on reports made by the state police regarding injuries from WVC. Similarly, the Insurance Australia Group data was also based on statewide reports of WVC, however, originating from insurance claims made by motorists. Each of the road segment-based datasets (City of Bendigo, Western District, and Crashstats) were used to augment a dataset obtained from Wildlife Victoria for modelling and to validate predictions resulting from these models. To develop each modelling dataset for our study, we selected road segments that intersected with reported kangaroo collision records to represent presences. All other road segments without collisions represented background data. When combining datasets, multiple collisions on the same road segments were reduced to single presences. Thus, each modelling dataset was comprised of 644,715 road segments having either a collision or no collision. Eight permutations were developed by combining independent data with Wildlife Victoria (baseline) data (codes shown in brackets):

\begin{enumerate}
	\item Wildlife Victoria (o)
	\item Wildlife Victoria + City of Bendigo (ob)
	\item Wildlife Victoria + Western District (ow)
	\item Wildlife Victoria + Crashstats (oc)
	\item Wildlife Victoria + City of Bendigo + Western District (obw)
	\item Wildlife Victoria + Western District + Crashstats (owc)
	\item Wildlife Victoria + Crashstats + City of Bendigo (ocb)
	\item Wildlife Victoria + City of Bendigo + Western District + Crashstats (obwc)
\end{enumerate}

\begin{table}[htp]
\caption[Datasets used to fit collision models and validate predictions]{Datasets used to fit models and validate predictions.}
\centering
\begin{tabularx}{0.9\textwidth}{YYYYYYYY} \toprule
Source						&Data Code	&Type	&Records	&Reporting Method	&Spatial Unit	&Spatial Coverage	&Temporal Coverage \\ \midrule

City of Bendigo		&b		&Public municipality			&395		&Public reported carcasses			&Road segment		&City (3,001 km$^2$)	&8 years : 2007-2014 \\

&&&&&&\\

VicRoads Western District	&w		&State government organisation	&815		&Independent contractor recorded  carcasses			&Road segment		&District (24,373 km$^2$)	&2 years : 2014-2015 \\

&&&&&&\\

Crashstats		&c			&State government organisation	&487		&Police reported wildlife strikes			&Road segment		&State of Victoria (227,819 km$^2$)	&7 years : 2010-2016 \\

&&&&&&\\

Wildlife Victoria	&o		&Non-government organisation	&4245		&Public reported observations of wildlife strikes			&Road segment		&State of Victoria (227,819 km$^2$)	&6 years : 2010-2015 \\

&&&&&&\\

Insurance Australia Group	&iag	&Private corporation			&1344		&Public reported claims of wildlife strikes			&Town		&State of Victoria (227,819 km$^2$)	&3 years : 2011-2013 \\

\bottomrule
\end{tabularx}
\label{val_data}
\end{table}

To test the effects of combining independent datasets, we fitted a collision model (\Cref{eq:28}) to all permutations of data.  We predicted collision risk from a model that regressed reported collisions on species occurrence, traffic volume and traffic speed:

\begin{equation} \label{eq:61}
cloglog(p_i) = \beta_0 + \beta_1*\text{log}(O_i) + \beta_2*\text{log}(V_i) + \beta_3*\text{log}(S_i)
\end{equation}

\noindent where $p_i=\text{Pr}(Y_i=1)$ is the relative likelihood of a collision occurring on a road segment $i$, $O_i$ is species occurrence, $V_i$ is traffic volume, $S_i$ is traffic speed.

To test model sensitivity to different combinations of data, we validated the predictions from our models using held-out datasets. For each combination of training datasets, we fitted a model, made predictions, and validated the predictions with an independent dataset not used to train the model. To assess model calibration, we regressed the observed collisions in each independent dataset on values predicted from models using the independent data (see \cite{mill91}). A perfectly calibrated model would report an intercept coefficient of zero and a slope coefficient of one. To measure the ability of each model to discriminate between false positive and false negative values, we also calculated receiver operator characteristic scores (see \cite{metz78}. A score of one indicates perfect discrimination while 0.5 indicates no better than random selection; 0.75 or above is often considered acceptable performance.

In addition to the four datasets with data on individual collisions, we also had access to dataset in which the number of collisions that involved insurance claims was recorded for towns. The Insurance Australia Group (``iag" hereafter) collision data detailed counts of collisions within each town boundary in Victoria (2917 total). From the predictions made to all road segments by the models using different combinations of data, we calculated the mean collision risk within each town boundary. Using each town as a datapoint, we regressed the mean collision risk prediction (log-transformed) on the total collisions reported. Due to the response data being positive integers, we used the Poisson link:

\begin{equation} \label{eq:62}
\text{log}(C_j) = \beta_0 + \beta_1\text{log}(P_j)
\end{equation}

\noindent where $C_j$ is the count of reported collisions and $P_j$ is the mean predicted collision risk, in a town $j$. As with the road segment-based data analysis, calibration was assessed using the intercept and slope of the model fits. We also calculated the reduction in deviance (unexplained variation in the data) for each model.

To further investigate how each independent dataset contributed to the models utilising combined data, we fitted the model described in \Cref{eq:61} to each road segment-based independent dataset (City of Bendigo, Western District, and Crashstats). We added an offset term of road length multiplied by the data period in years to standardise the model outputs for comparison. For each dataset, we restricted road segments to a domain that corresponded to the respective survey or reporting area. The City of Bendigo data were restricted to road segments that fell within the city boundary as carcass reports outside of the jurisdiction were not catalogued. Similarly, only road segments that were surveyed in the Western District were included for analysis. Crashstats covers the entire state and so all 644,715 road segments were used. 

\section{Results}

\begin{table}[htp]
\caption[Summary of model fits using original and validation data]{Summary of model fits using original and validation data in solidarity (i.e. no combinations of data are used). Highly significant variables (p$<$.0001) are marked with asterisks.}
\centering
\begin{tabularx}{0.9\textwidth}{lllllll} \toprule
Dataset & Variable & Coefficient & Standard Error & $Z\text{-value}$ & $\PRZ$ & Deviance Explained \\ 
  \midrule
Wildlife Victoria & Intercept & -42.30 & 1.07 & -39.66 & $<$.0001* & 11.83 \\ 
   & EGK & 0.70 & 0.01 & 47.77 & $<$.0001* &  \\ 
   & TVOL & 5.55 & 0.24 & 22.68 & $<$.0001* &  \\ 
   & TVOL2 & -0.33 & 0.02 & -21.76 & $<$.0001* &  \\ 
   & TSPD & 3.93 & 0.07 & 54.49 & $<$.0001* &  \\ 
  City of Bendigo & Intercept & -53.81 & 4.87 & -11.05 & $<$.0001* & 5.16 \\ 
   & EGK & 0.49 & 0.09 & 5.28 & $<$.0001* &  \\ 
   & TVOL & 11.71 & 1.20 & 9.74 & $<$.0001* &  \\ 
   & TVOL2 & -0.74 & 0.08 & -9.60 & $<$.0001* &  \\ 
   & TSPD & 1.19 & 0.39 & 3.08 & 0.0021 &  \\ 
  Western District & Intercept & -10.19 & 3.47 & -2.93 & 0.0034 & 5.15 \\ 
   & EGK & 0.02 & 0.05 & 0.36 & 0.7207 &  \\ 
   & TVOL & 0.37 & 0.80 & 0.47 & 0.6408 &  \\ 
   & TVOL2 & -0.07 & 0.05 & -1.32 & 0.1885 &  \\ 
   & TSPD & 2.03 & 0.45 & 4.48 & $<$.0001* &  \\ 
  Crashstats & Intercept & -51.74 & 2.89 & -17.88 & $<$.0001* & 11.75 \\ 
   & EGK & 0.55 & 0.04 & 12.89 & $<$.0001* &  \\ 
   & TVOL & 4.89 & 0.69 & 7.13 & $<$.0001* &  \\ 
   & TVOL2 & -0.32 & 0.04 & -7.19 & $<$.0001* &  \\ 
   & TSPD & 6.52 & 0.27 & 24.43 & $<$.0001* &  \\ 
\bottomrule
\end{tabularx}
\label{val_glm_perf}
\end{table}

All three models fitted to independent road segment-based datasets (City of Bendigo, Western District, and Crashstats) had mixed results when compared to the model fitted to the original dataset (\Cref{val_glm_perf}). Both of the models using statewide data (Wildlife Victoria and Crashstats) performed better than the two using localised data (City of Bendigo and Western District). The worst performing model was fit to data from the Western District of Victoria - only traffic speed was a significant predictor. In contrast, the only non-significant predictor was traffic speed in the model fit to data from the City of Bendigo. All other predictors in every model were significant. The reduction of unexplained variation in the data were remarkably similar for the two statewide data models ($\approx$11.8\%) and the localised data models ($\approx$5.2\%).

\begin{figure*}[htp]
  \centering
  \includegraphics[scale=.7]{06_occ.png}
  \includegraphics[scale=.7]{06_tvol.png}
  \includegraphics[scale=.7]{06_tspd.png}
  \caption[Marginal effects of predictor variables on relative likelihood of collision using independent datasets to train models]{Marginal effects of each predictor on relative likelihood of collisions. Codes for data combinations are: 'o' - Original (Wildlife Victoria); 'b' - City of Bendigo; 'w' - Western District; 'c' - Crashstats. Note likelihoods of collision have been rescaled for comparison as all datasets have different numbers of data points and thus variations in the range of predicted values.}
  \label{val_effects}
\end{figure*}

Most of the marginal effects of the predictors on collision risk had similar shapes for all of the model fits. Collision risk rose with increasing species occurrence for all models, however, the Western District data model reached nearly 75\% of the maximum collision likelihood at low values of species occurrence ($\approx$0.2). Collision risk had a quadratic relationship to traffic volume, and peaked between 2,000 and 7,000 vehicles day$^{-1}$ for all models except the Western District data model. The Western District data model showed a monotonically decreasing collision risk with rising traffic volume beginning at a value of $\approx$0.95 at zero vehicles day$^{-1}$. Collision risk increased with traffic speed for all models. Collision risk increased nearly linearly between 40 and 110 km hr$^{-1}$ in the model fit to the City of Bendigo data whilst collision risk rose slowly with the maximum increase of risk per unit speed occurring above 95 km hr$^{-1}$ in the model fit to the Crashstats data.

\begin{table}[htp]
\caption[Discrimination ability of models using all combinations of independent data]{Discrimination ability of models expressed as receiver operator characteristic scores. Data combinations used to model and make predictions are shown as column headings. Data used to validate model predictions are shown as row headings.}
\centering
\begin{tabularx}{0.9\textwidth}{YYYYYYYY} \toprule
	& Wildlife Victoria
(original) & original + City of Bendigo & original + Western District & original + Crashstats & original + City of Bendigo + Western District & original + Western District + Crashstats & original + Crashstats + City of Bendigo \\ 
  \midrule
City of Bendigo & 0.87 & - & 0.85 & 0.87 & - & 0.85 & - \\ 
Western District & 0.76 & 0.75 & - & 0.78 & - & - & 0.77 \\ 
Crashstats & 0.8 & 0.8 & 0.83 & - & 0.82 & - & - \\
\bottomrule
\end{tabularx}
\label{val_glm_roc}
\end{table}

The ability of models to discriminate between false-positive and true-positive rates of collisions were stable using different training data and validation data (\Cref{val_glm_roc}); discrimination ability did not increase as data were added to models. Among the different validation datasets, variation in the receiver operator characteristic (ROC) scores were less than 4\% between different combinations of data used to fit models. All of the models discriminated best when validated with the City of Bendigo dataset (ROC scores of 0.85-0.87) and worst with the Western District dataset (ROC scores of 0.75-0.78). 

\begin{figure*}[htp]
  \centering
  \includegraphics[scale=1]{06_calib.png}
  \caption[Collision model calibration for all combinations of original and independent data]{Model performance for all combinations of data. Codes for data combinations are: 'o' - Original (Wildlife Victoria); 'b' - City of Bendigo; 'w' - Western District; 'c' - Crashstats. Characters before the hyphen represent the datasets used for training the model and making predictions; characters after the hyphen indicate the data used for validation. Estimated calibration coefficients are shown as dots with bars representing standard errors.}
  \label{val_calib}
\end{figure*}

There was no observable trend in calibration when validating model predictions made with increasing amounts of data (\Cref{val_calib}). The lowest calibration scores were for models validated with the Western District (w) collision data. The model fit on the Wildlife Victoria data (o) and validated with data from Crashstats (c) had the highest calibration score which was close to one. Moreover, there was no apparent improvement in the relationship between observed and predicted values when increasing the amounts of data used for modelling. All of the models demonstrated similar trajectories for observed-versus-predicted rates at low values and then diverged at high values (\Cref{val_calib2}).

\begin{figure*}[htp]
  \centering
  \includegraphics[scale=1]{06_calib2.png}
  \caption[Comparisons of observations versus model predictions for all combinations of original and independent data]{Comparisons of observations versus model predictions for all combinations of data. Codes for data combinations are: 'o' - Original (Wildlife Victoria); 'b' - City of Bendigo; 'w' - Western District; 'c' - Crashstats. Characters before the hyphen represent the datasets used for training the model and making predictions; characters after the hyphen indicate the data used for validation.}
  \label{val_calib2}
\end{figure*}

Using the aggregated independent data (iag) to validate models resulted in a slight increase in model calibration as more combinations of training data were used (\Cref{val_calib_iag}).  The lowest calibration coefficient (0.943) occurred using only the original (o) data to fit the model and the highest (0.945) occurred using a combination of the original and all independent (obwc) data. Higher reductions of unexplained variation in the data also occurred with more combinations of training data (\Cref{val_calib_dev}). Deviance explained by the model fit on the original (o) data was 33.2\% whilst the model fit on a combination of the original and all independent (obwc) data explained 33.3\% of the deviance.

\begin{figure*}[htp]
  \centering
  \includegraphics[scale=1]{06_calib_iag.png}
  \caption[Model calibration for all combinations of training data using the aggregated independent data for validation]{Model performance for all combinations of data using the aggregated independent data (iag) for validation. Codes for data combinations are: 'o' - Original (Wildlife Victoria); 'b' - City of Bendigo; 'w' - Western District; 'c' - Crashstats. Characters before the hyphen represent the datasets used for training the model and making predictions; the same data aggregated data (iag) were used for all validation. Estimated calibration coefficients are shown as dots with bars representing standard errors.}
  \label{val_calib_iag}
\end{figure*}

\begin{figure*}[htp]
  \centering
  \includegraphics[scale=1]{06_dev_iag.png}
  \caption[Model discrimination ability for all combinations of training data using the aggregated independent data for validation]{Model performance for all combinations of data using the aggregated independent data (iag) for validation. Codes for data combinations are: 'o' - Original (Wildlife Victoria); 'b' - City of Bendigo; 'w' - Western District; 'c' - Crashstats. The percent of variation in the training data explained by the model (deviance) are shown as dots.}
  \label{val_calib_dev}
\end{figure*}

\section{Discussion}

This study demonstrated that increasing the amount of collision data used for modelling, by combining disparate datasets, improved model performance and predictive strength. The improvements were marginal, most likely due to the limited amount and significant variation of the data used in this study. However, even small, incremental improvements can be quite valuable when comparing to the costs WVC. For example, if the model trained on more data can predict and prevent a single additional WVC from occurring, it has tremendous value when considering the high costs of building roads, a human life, or other unquantifiable effects such as species extinctions. From this perspective, the results of this study suggest that the development of a unified data system has high utility given its relatively inexpensive costs to implement. Further, such a system would be able systematically catalogue disparate sources of data and enable more robust analyses of wildlife-vehicle collisions, however, there are several important factors to consider. Large quantities of data may help to overcome issues of underreporting \citep{snow15}, and limited temporal/spatial coverage that affect generality \citep{clev15}, but equally important is the quality of the data.

It is important to identify and address spatial biases when developing predictive models using presence-only data \citep{kram13}. Models trained on observations that are spatially biased (e.g. collected at particular places/times or dependent on characteristics of reporting individuals) may cause model predictions to reflect patterns of collection rather than distributions of collision risk. Each of our datasets had varying levels of spatial bias but these were difficult to identify. For example, the original Wildlife Victoria dataset appears to have a reporting bias that is spatially correlated with proximity to the urban centre of Melbourne. Other datasets can be expected to exhibit less spatial bias due to the nature of reporting - Crashstats is comprised of statewide police records of kangaroo-vehicle collisions. However, Crashstats data is subject to under-reporting because it only includes records involving human injuries or deaths resulting from WVC.  Regardless of differences, implicit biases in data may be considered in modelling efforts if they are adequately described in metadata \citep{wart13} but generally, collision data does not include this information.

Metadata should also include the survey effort involved in collecting collision data as this will have modelling implications and an effect on predictions of road risk. Despite its importance, this information is seldom reported and varies considerably amongst and within jurisdictions \citep{huij07a}. For example, the Western District data was collected using a combination of systematic surveys and opportunistic methods. Road cleaning contractors regularly patrolled major segments of roads in the district (Freeways, Highways, Major Arterials) and recorded carcasses collected but were also irregularly dispatched based on carcass observations reported by the public. It should be noted that contractors only collected carcasses that were in close proximity to the road, and deemed a risk to motorists, which leads to under-reporting. It is reasonable to assume our four datasets had different surveying efforts, however, because these were not explicitly defined, we did not incorporate them into the modelling. Information on survey effort can improve models by generating parameters that account for detectability \citep{dora14}.

Citizen-science collected data has the potential to contribute large amounts of observational data on wildlife-vehicle collisions \citep{cose14,dwye16,paul14}, however, controlling the quality of the data is critical to ensure sound statistical analyses. To augment collision or carcass observation data, some systems store information on reliability and accuracy (e.g. \cite{shil15a}). This is often reported by the user and requires additional review by a system administrator to determine its merit but is a good foundation for assessing and comparing the effects of the data on model outputs. In the Victorian data, we did not have access to such metadata and assumed all datasets had the same reliability and accuracy. Yet, we suspect there is some variation between datasets as some data, such as insurance or police reports, may be more reliable than observations reported by citizens. Moreover, our data only included records on collisions with kangaroos, which are quite identifiable; smaller species may be harder to identify and introduce additional uncertainty. Nonetheless, reliability may be incorporated into modelling if properly reported \citep{guns09}.

Changes in relative risk over different temporal scales is important to reflect in collision model predictions (see \Cref{sec:train}). Many species exhibit different patterns of activity, as do humans, and thus collision risk will ultimately vary hourly \citep{joyc01}, seasonally \citep{alve12,beau10,gril09}, or both \citep{mizu14,more13}. Most data collected on carcasses or injured animals incorporates this information but does not report the uncertainty around reported times. This is problematic due to potential temporal lags between actual collision events and subsequent observations. Similar to issues resulting from spatial bias, models trained on data with inaccurate times will result in predictions that correspond to collection times rather than collision times. Smaller species are subject to more reporting error, as compared to larger species, as they often become unrecognisable after a few hours or days. Although kangaroos are large species, we did not incorporate hourly effects due to a lack of information about the reported times and associated uncertainty. Seasonal effects are less prone to this shortcoming due to larger temporal scales, however, we excluded seasonal effects from this analysis as kangaroos do not exhibit seasonal patterns. 

Our road segment-based datasets were represented by truncated counts of collisions to `zeros' or `ones' based on the mathematical framework used for modelling. This was appropriate considering that more than one collision on a road segment rarely occurred in the data. More importantly, we were concerned about potential duplicate records arising from different data sources (i.e. the same collision reported twice). But it is beneficial to maximise the use of available collision data and some studies take advantage of count data to model and predict wildlife-vehicle collision risk \citep{cser13}; and explicitly reconcile disparate data sources within the model \citep{lao11a}. However, duplication of recorded collision events or carcass observations may affect model predictions and should be scrutinised. If duplicates could be identified reliably, then those records might help to examine biases in the reported collisions in a way that is similar to using multiple observers to estimate detection probabilities \citep{nich00}.

At time of writing, we are not aware of a unified data storage system that is accessible to and used by governments, organisations, companies and individual citizens to catalogue wildlife-vehicle collisions anywhere in the world. Although this study is limited to the State of Victoria in south-east Australia, the results and general discussion suggest qualities that a centralised storage system should have and we recommend the development and implementation of a WVC database to support future modelling and prediction efforts. Although systems maintained at the global scale, such as the Global Biodiversity Information Facility (GBIF), may reduce potential redundancies and ambiguities, national systems are also useful to achieve this objective. Of paramount importance is uniting the currently disparate sources of data with an archive that is both comprehensive and systematic. Such a repository will enable informed analyses of WVC, with improved model calibration, to ultimately make more reliable predictions of wildlife-vehicle collision risk. 

\chapter{Validating collision model predictions with citizen-science, private-sector and government collected datasets: a case for a unified data system}\label{sec:val}

%\chapter{General Discussion}\label{sec:conc}
\newpage

\section{Synthesis of work}

In this thesis, I present a modelling framework for determining wildlife-vehicle collision risk that may be used without taxonomic or geographic restriction by wildlife, road, and environmental managers. Drawing on well-established concepts of road ecology and risk theory, this conceptual framework is the first of its kind to address the disparate factors that contribute to wildlife mortality on roads. \cite{clev15} has recently identified this critical need in the discipline of road ecology. In many respects, my work is quite similar to predictive modelling studies that have come before it. I use regression modelling to relate collisions to environmental and anthropogenic variables and make predictions of risk to segments on linear transportation networks. However, I extend this previous work in several notable ways. Perhaps the most useful deviation from other studies is the grouping of predictors that influence wildlife collision risk into two categories that more directly relate to management activities. Among the suite of mitigation used by transport managers, most of it corresponds to controlling either animal presence on a transportation network (e.g. fencing) or the operation of vehicles (e.g. speed) \citep{glis09}.  Using species occurrence to represent exposure and traffic volume/speed to represent hazard creates a more direct interpretation by managers. Further, all of the other predictors used in past wildlife-vehicle collision modelling may be incorporated into the sub-models in each respective group thereby reducing uncertainty about potentially confounding effects. From this simplified conceptual framework, managers are able to compare the relative influences of exposure and hazard, as well as the uncertainties of each one. If costs of mitigation are known for each category, the conceptual framework is useful for performing cost-benefit analyses -- this is explored in \Cref{sec:train}.

Another useful departure from previous work that uses generalised linear regression to model wildlife-vehicle collisions is the choice of link function. This thesis is the first study to employ the use of a complementary log-log (cloglog) link function for modelling wildlife-vehicle collision events. This is important because the choice of a link function has theoretical underpinnings which are normally based on assumptions about how events arise in a system. I use the cloglog link as it assumes that collisions arise from a Poisson process which is reasonable for collision events occurring over a domain bound by time. Another attractive feature of the cloglog is that it is not influenced by the choice of sampling area. This makes it possible to compare parameter estimates for models using different spatial scales.  In contrast to several other studies that use a logit link (i.e. logistic regression) which imply strong assumptions about spatial scale and data generating processes \citep[e.g.][]{seo15,gund98}, this thesis demonstrates a robust and flexible modelling method that will enhance new and existing road ecology work.

Beyond demonstrating the broad utility of the conceptual framework, my thesis also highlights three key results. First, the speed of moving vehicles is a considerable hazard for wildlife having access to transportation networks. \Cref{sec:6sp,sec:cal} demonstrated similar patterns of collision risk with respect to traffic speed for seven mammals. Nearly all species had exponential increases in risk with increasing speed. Collision risk also increased exponentially with increasing train speed (\Cref{sec:train}), suggesting this phenomenon is not isolated to road networks. These results are consistent with the broader literature on wildlife-vehicle collision analysis, and also other collision analyses involving pedestrians and vehicles \citep{rosen09} and only vehicles \citep{liu97}. As mitigation to reduce animal proximity to, or exclude entirely from, transportation networks is costly \citep{huij09}, these results suggest that speed reductions may be a more viable option for reducing collision risk. The conceptual framework I have developed in this thesis is able to assist managers to target specific areas for speed reductions and also simulate the effects from these proposed mitigations. Further, the predictions of risk from the conceptual framework may be used to influence motorist behaviour (i.e. speed) in areas, or at times, of high risk. Both of these potential uses are discussed further in subsequent sections.

Second, temporal variation is important to incorporate into models that predict collision risk. Both animal and human activities have distinct patterns that correspond to time. In \Cref{sec:train}, I introduced a function into the model framework that accounted for cyclical patterns of animal activity at the hourly scale. This was a significant variable and useful for extending predictions or risk from the spatial domain to the spatio-temporal domain. Temporal variation is rarely accounted for in predictive modelling of wildlife-vehicle collisions \citep[but see][]{beau10,gund98,neum12}. The function used in \Cref{sec:train} generalises to any species, contrasting with existing work that only characterises expected activity patterns for single species. Further, the function also uses sunrise and sunset times (relative to geographic location) to consider time of day with respect to ambient light -- a factor known to influence animal activity. Both of these are useful contributions to improve future collision analysis. Uncertainty is one reason that collision modellers may be reluctant to utilise fine resolution temporal data (e.g. hourly scale). Indeed, all of the collision data used in \Cref{sec:egk,sec:6sp,sec:cal,sec:val} had temporal information at an hourly scale, however, differences between actual collision times and reported collision (or carcass collection) times were difficult to determine. Thus, I excluded the use of this data for modelling and prediction on the road networks. Patterns are evident from the histograms of reported collisions for all six species (\Cref{temporal_all}) and suggest temporal components may be useful for modelling and predicting collisions if the uncertainty can be reduced or accounted for. The data used in \Cref{sec:train} had a high degree of certainty around the collision times reported by the train drivers. This was likely due to the systematic and procedural nature of train operations. Road collision events are not subject to the same stringencies thereby making it more necessary to document uncertainty in the data and represent it in the modelling. Uncertainty is elaborated upon in subsequent discussion. 

\begin{figure*}[htp]
  \centering
	\begin{minipage}[t]{.9\textwidth}
    	\centering
    	\includegraphics[width=.45\textwidth]{egk_coll_hour.png}\hspace{.05\textwidth}
    	\includegraphics[width=.45\textwidth]{btp_coll_hour.png}\\ 
    	\includegraphics[width=.45\textwidth]{rtp_coll_hour.png}\hspace{.05\textwidth}
    	\includegraphics[width=.45\textwidth]{bsw_coll_hour.png}\\
    	\includegraphics[width=.45\textwidth]{wom_coll_hour.png}\hspace{.05\textwidth}
    	\includegraphics[width=.45\textwidth]{koa_coll_hour.png}
    \end{minipage}
  \caption[Total collisions by hour for six mammal species]{Histograms showing the distributions of total collisions by hour for six mammal species. Note, Records indicate the time that wildlife-vehicle collision events were reported and may not accurately reflect actual times due to reporting lags.}
  \label{temporal_all}
\end{figure*}

Lastly, collision models and predictions may benefit from increases in data quantity and quality. \Cref{sec:val} introduced the first known study in the discipline of road ecology to compare the effects of combining data from different data sources on collision risk modelling and predictions. This compliments work that evaluates the use of citizen science to collect collision data \citep{paul14, dwye16}, and proposes technological advances for collecting data \citep{olso14}. Moreover, these studies call for the development of a centralised data storage system to improve road ecologists' understanding of where and when collisions occur. As demonstrated in \Cref{sec:val}, as more disparate collision datasets were combined together, model performance and calibration improved. Prospects for centralising collision data are discussed in a subsequent section. 

\section{Benefits of spatial modelling}

Most, if not all, problems resulting from the distribution of humans across the landscape have inherent spatial properties. Following on from this, research involving ecological phenomenon and issues often involves spatial analyses and outputs; such as species distribution modelling \citep[e.g.][]{elit09}. The use of spatial modelling in this thesis was fundamental to its intended purpose -- to represent collision risk across large transportation networks. Further, the use of spatial modelling in this work also supports its future application as a decision support tool for management or a public outreach mechanism.

\subsection{Access to spatial data}

Currently, there is a remarkable amount of publicly-accessible spatial data available on the world-wide web \citep{ma15} and it is reasonable to expect increases in future years. Data derived by scientific or government organisations often include additional metadata that detail reliability and data-generating procedures. In this thesis, nearly all of the predictor data used in modelling were derived from information that existing in the public domain. This enables the methods to be reproducible and transferable; for example, I developed several of the predictors in \Cref{sec:cal} using the same source of satellite-derived data (see \Cref{apx:C}) but for different geographic locations.

\subsection{Spatial analysis in management}

The use of geographic information systems (GIS) to catalogue and view spatial features is widespread in management. Contemporary versions of these software applications incorporate functionality to internally or externally perform spatial analyses, including modelling. As most managers are familiar with spatial data, I employed spatial methods in this thesis that would easily integrate into GIS software. For example, all computer code was written in \textit{R} \citep{rdct16}, statistical software that can be called to perform operations on spatial data from within programs such \textit{QGIS} \citep{qgis09}. Further, I developed computer code to convey collision risk as outputs that managers would be able to most effectively utilise (e.g. maps).

\subsection{Communication and public outreach}

Conveying information created with spatial methods is an effective communication tool \citep{king00}. Most of the general public is familiar with geographic information through daily exposure to maps and other spatial data. Spatial data and analyses are natural candidates for data visualisation \citep{osul14}, a technique used to communicate information through the use of visual objects, and may increase public engagement through interactive content. Although the spatial outputs included in this thesis are static, the methods used to create them are easily adaptable to interfaces that enable users to perform analysis on collision risk by varying model parameters.

\section{Limitations of models and caveats}

Models are simplified abstractions of reality and therefore subject to uncertainty \citep{burg05}. Models are useful as conceptual tools for organising complex problems and supplementing decision-making, however, must be used cautiously. Further, \cite{levi66} asserts that models can only be built to maximise two of three characteristics, generality, realism, and precision. The primary intent of the conceptual model framework introduced in this thesis was to maximise generality, or applicability to any ecological issue involving threats to wildlife. The model also exhibits realism in its theoretical treatment of collisions resulting from the coincidence of animals and moving vehicles in space and time. Much of the literature in road ecology that focuses on collision modelling operates on this assumption \citep{form03,guns11}, but does not make this explicit in the modelling methods. Nevertheless, assumptions about data-generating processes (how collisions occur across the landscape) and inadvertent omissions of pertinent data both add uncertainty to the model framework. Further, the conceptual framework in this thesis uses data derived from sub-models, and these are also subject to the same uncertainties discussed previously.  

One strategy to explore model uncertainty is to perform collision modelling analyses using Bayesian inference. Bayesian methods allow prior information to be incorporated into a model \citep{mcca07}. Prior information (or prior probabilities) weight the influence of data in a model (likelihood) to produce posterior probabilities. With respect to the work presented in the thesis, this may involve using expert opinion to provide judgement on the reliability of the predictions from the sub-models. For example, I had experts in macropod ecology examine the distribution of kangaroos predicted by the species occurrence model. Although there were some minor discrepancies, the overall patterns were realistic. However, since managers may base decisions about mitigation on the relative strength of exposure (animal presence) or hazard (vehicle movement), it is important to represent the relative confidence about each predictor in the model. Bayesian analysis also estimates all model parameters (coefficients and predicted values) as probability distributions thereby allowing further scrutiny of uncertainty. Credible intervals around predictions allow managers to make decisions from either a 'risk averse' or 'risk inclined' standpoint by using higher or lower values, respectively. 

The data used to train the collision model in this thesis were what is referred to as 'presence-only' data which require special assumptions and treatments \citep{hast13,wart13}. True prevalence is not able to be inferred from presence-only data without a systematic and exhaustive search of the entire spatial domain. Thus, the conceptual model in this thesis makes assumptions about prevalence based on the selection of background data (used to represent road segments where no collisions occurred). In this respect, predictions from the models are relative to reporting rates, as opposed to absolute.  In \Cref{sec:egk}, I randomly selected background data that was equivalent to twice that of reported collisions. This has implications for relative predictions as it makes strong assumptions about reporting rates for collision occurrence. Fortunately, as discussed previously, the theoretical underpinnings and choice of link function in the statistical model make it indifferent to this variation. In contrast, \Cref{sec:6sp,sec:cal,sec:train,sec:val} all use the entire set of linear segments without reported collisions as background sets. This is a more realistic interpretation of the system, yet still does not preclude it from assumptions of collision prevalence. A particular special case may be in \Cref{sec:train} where the reporting rates are quite good and, therefore, the modelling dataset may be more reflective of true prevalence. Another potential issue of presence-only data is sample selection bias which refers to the manner in which data is collected and reported \citep{phil09}. In a spatial context, this may bias model predictions by over-representing and under-representing collision risk due to patterns of reporting, such as proximity to urban areas. Recent work in species distribution modelling has suggested the utility of spatial and temporal cross-validation for improving model performance \citep{weng12,robe17}. These methods have important implications for reducing bias and ensuring the generality of model predictions. The work in this thesis does not incorporate these methods, however, doing so would improve the analysis and resulting outputs.

\section{Novel future research}

Species traits, such as body size, are useful to incorporate into the analysis of this thesis. Although risk rates did not perfectly correspond to mammal sizes in \Cref{sec:6sp}, body size may influence reporting rates as smaller species are less likely to be observed due to reduced detectability and faster rates of decay or predation. Body size may also play a role in susceptibility to collisions based on movement speed \citep{jaar06} or visibility to motorists. Body size may be included in the collision model as an additional exposure term, however, care must be taken when interpreting results due to potential confounding effects (visibility and speed). Species behaviour may also influence exposure to wildlife vehicle collisions, such as flight response to vehicles \citep{deva14,lee10} or feeding patterns (e.g. near roads). As with species traits, this information may be used as an additional exposure term. In road ecology, there are very few studies that make comparisons of predicted risk between species based on traits or behaviour \citep[but see][]{litv08}. As such, this would be a novel extension of the work in this thesis. 

In a similar manner, incorporating additional characteristics of perceived threats (hazards) into the risk model is also useful. In this thesis, I represented hazard with the frequency and speed of vehicles across transportation networks. Only passenger vehicles were considered in the analysis, however, freight transport comprises approximately 25\% of road usage \citep{abs11}. Large trucks have considerable momentum and reduced stopping distances and are, therefore, potentially more threatening. One novel extension of the model would be to represent hazard with kinetic energy calculated by combining all masses and velocities of objects on each road segment. General collision risk literature has made use of this concept \citep{aart06}, but road ecology has yet to apply it to wildlife-vehicle collision modelling.

The work in this thesis models risk exclusively for common species and thus begs the question of whether the methods are also applicable to rare species. Indeed, this would be an important feature for managers, and also more directly support conservation efforts. The theory underpinning the model framework enables analysis irrespective of species commonality, however, a primary difficulty is obtaining adequate amounts of data with which to train the models. Additional research that uses the framework to compare performance between rare species and more common surrogate species, collected using a small targeted study, would be useful. If the results are deemed comparable, extrapolating to other areas where data on the surrogate species are available but data on rare species are more deficient would be interesting. This may involve the use of population viability analysis which helps to understand changes in species persistence due to threatening processes \citep{rhod14}. Although a few wildlife-vehicle collision modelling studies have addressed conservation of rare or endangered species \citep{dwye16}, none have explored the use of surrogate species to suggest risk to rare species.

There are several concepts in wildlife-vehicle collision modelling that parallel those used in pedestrian-vehicle collisions, a vastly more broad literature. One common feature is that activity patterns of humans and animals is quite important in determining when and where collision risk may be high \citep{mira11,lao11}. For the reasons discussed previously, the analysis in \Cref{sec:egk,sec:6sp,sec:cal,sec:val} only included spatial patterns and disregarded temporal variation altogether. However, \Cref{sec:train} demonstrated the utility of accounting for both spatial and temporal patterns of animal activity. Future collision modelling work would benefit from additional development of sub-models that simulate the motion of animals \citep[see][]{jaar07}, similar to that of pedestrians \citep[see][]{lohn10}, and also other objects such as vehicles.

\section{Generalisation of methods to other wildlife threats}

The conceptual framework developed in this thesis may be used to analyse risk for other ecological problems, thereby increasing its potential contribution to other environmental research. Its modular nature allows analysis of other adverse risks to wildlife (\Cref{gen_framework}) by replacing the hazard component with an alternative threatening process of interest, such as attacks from domesticated predators or human poaching. Further, the exposure of species to hazards may be represented by data generated by alternative models; for example, movement patterns from connectivity analyses \citep{mcra08} or local abundances from population viability analyses \citep{beis02}. \cite{baud13} used a similar approach to determine collision risk between seafaring vessels and manatees, although did not emphasise the generality in their approach. The modularity in my conceptual framework allows unlimited possibilities for future research projects.

\begin{figure*}[htp]
  \centering
  	\begin{minipage}[t][][b]{.13\textwidth}
    	\centering
    	\includegraphics[width=\textwidth]{MODELS_connectivity.png}\\
    	\includegraphics[width=\textwidth]{MODELS_population.png}   	
    \end{minipage}
    \hspace{.05\textwidth}
  	\begin{minipage}{.38\textwidth}
    	\centering
    	\includegraphics[width=\textwidth]{MODELS.png}
    \end{minipage}
    \hspace{.05\textwidth}
  	\begin{minipage}[t][][b]{.13\textwidth}
    	\centering
    	\includegraphics[width=\textwidth]{MODELS_predators.png}\\
    	\includegraphics[width=\textwidth]{MODELS_infrastructure.png}
    \end{minipage}    	   	   
  \caption[Conceptual risk model framework]{Conceptual risk model framework.}
  \label{gen_framework}
\end{figure*}

\section{Potential applications of research}

In an effort to maximise potential applications of this work to reducing wildlife-vehicle collisions, I have utilised publicly-accessible data and open-source software tools, and developed reproducible computer code (\Cref{apx:A}). This allows the methods to be used for developing additional resources to supplement management and engage the public. Given these qualities, there are several foreseeable future extensions of this work. 

\subsection{Centralised data collection and reporting system}

Beyond offering incredible global interconnectedness, the information age has been marked by a massive amount of data generation that is publicly-accessible via world-wide web portals. One example is atlases of species occurrence data, such as the global biodiversity information facility (GBIF), which holds nearly three-quarter billion records worldwide. Modelling and predicting collision risk would benefit tremendously from the creation of an open-access, single-source repository of wildlife-vehicle collision records. Worldwide, collision data is collected by governments, private-sector corporations, non-profit organisations, community groups, and individual citizen scientists. This information is stored in several different formats (e.g. electronic or hard copy) and rarely made publicly-accessible. Collision data that is made available through world-wide web portals is often read-only and does not provide access to detailed spatial information. I propose a centralised data system that not only catalogues collision records and metadata, but also provide other important functions (\Cref{wvc_server}). Most importantly, the system should be able to communicate with other data systems and external applications. Such systems exist for data storage and sharing (e.g. Center for Open Science), however, require additional development to employ automated interfacing with external modelling tools.

\begin{figure*}[htp]
  \centering
  \includegraphics[width=0.8\textwidth]{interfacing.pdf}
  \caption[Centralised data collection and reporting system]{Schematic diagram of centralised data collection and reporting system for wildlife-vehicle collisions. Arrows indicate directions of information flow. Additional collection of collisions data (in blue) is by both citizen scientists (top) and professionals (bottom).}
  \label{wvc_server}
\end{figure*}

\subsection{Integration of work with existing and new technology}

In developed nations, technology pervades nearly every aspect of human lives. Every office desk is adorned with a computer and nearly all persons carry a mobile device (mostly smartphones). We use these tools to elicit information and to assist us with decision-making. Unbeknownst to a significant portion of the public, models are continuously running and recalibrating on servers throughout the world to deliver information to these devices. Some help to optimise our search results on a web page whilst others determine efficient transportation routes. This approach to disseminating real-time, model-generated information can be applied to road safety, specifically wildlife-vehicle collisions, to provide decision support for managers or warnings to motorists. I envision the work in this thesis to be easily extended to either application.

\subsubsection{Desktop-based decision support tool for managers}

Decision support tools are used by managers in a wide range of fields \citep{shim02}. \Cref{wvc_tool} shows a web-based interface to the model framework presented in this thesis. Developed as a proof of concept, I wrapped the modelling methods with a graphical user interface (GUI) that allows managers to change values of species occurrence, traffic volume, and traffic speed on road segments and simulate expected collision risk across a network. Although this is quite rudimentary, it illustrates the potential features that a decision support tool may provide to direct efforts on reducing wildlife-vehicle collisions.

\begin{figure*}[htp]
  \centering
  \includegraphics[width=0.8\textwidth]{shinyapp.png}
  \caption[Decision support tool for wildlife-vehicle collisions]{Screen capture of web-based interface for a wildlife-vehicle collision risk modelling tool. The tool was originally developed for Bendigo, a medium-sized town in south-east Australia that experiences high numbers of kangaroo-vehicle collisions.}
  \label{wvc_tool}
\end{figure*}

\subsubsection{Real-time warning 'app' for mobile devices}

Changing human behaviour is a difficult task and a large amount of sociological and psychological literature is devoted to it. Studies suggest that one important factor to precipitate human behaviour change is the convenience of attaining outcomes (least effort) \citep{zipf49}. Further, technology has also been shown to be persuasive \citep{fogg02,lath13}, and therefore, mobile devices may be an effective media for influencing human behaviour. As such, mobile phones and global-positioning units are ideal candidates for receiving relative collision risk (based on geographic location and time of day) from model predictions and disseminating the information to motorists. The dynamic nature of a warning application may exceed the utility of traditional warning devices, such as signage, which are considered largely ineffective \citep{bond13}.  Smartphone applications that report locations of collisions (and also accept observations) exist and are in use today \citep{aane09}, however, do not utilise modelling mechanisms. In this respect, these applications are simply reporting mechanisms and, thus, may only be slightly more effective than road signs.  

\section{Concluding remarks}

Human-wildlife interactions, and thus wildlife-vehicle collisions, are a global phenomenon. Issues such as animal mortality on roads will be exasperated by future human development as we reduce our proximity to wildlife. To address these issues, we need to be strategic about where and when we mitigate. This thesis provides analytical tools for making better decisions about environmental management. It is my hope that the contents of this work will assist others in solving contemporary ecological issues.
\chapter{General Discussion}\label{sec:conc}

%\chapter{References}

\bibliography{/home/casey/Research/Know_Repo/Literature/Bib_Master}

{
\backmatter
\chapter{References}
%\newpage
\bibliography{/home/casey/Research/Know_Repo/Literature/Bib_Master}
}

%---------------------------------------------------------- 
% APPENDICES
% include chapters as needed (will be numbered differently)
%---------------------------------------------------------- 
\clearpage
\appendix
\fancyhead[R]{\sc Appendix \thechapter}

%\chapter{Computer code}\label{apx:A}
\newpage

\section{R code}

All \textit{R} code used to perform the analyses in this thesis are archived and publicly available at GitHub. Note, most of the analyses utilised a spatially enabled-database (PostGIS, version 2.3) that is behind a university firewall and therefore cannot be accessed without appropriate credentials. For access to this data, I encourage the reader to contact me. The work is organised by chapter and objective and includes links to the respective online code.

\vspace{.5cm}
\noindent Chapter 1: Prepare species occurrence model data\\
\textit{https://github.com/cvisintin/coll\_framework\_egk/blob/master/R/sdm\_model\_data.R}
%\lstinputlisting{/home/casey/Research/Github/coll_framework_egk/R/sdm_model_data.R}

\vspace{.5cm}
\noindent Chapter 1: Species occurrence modelling\\
\textit{https://github.com/cvisintin/coll\_framework\_egk/blob/master/R/sdm.R}
%\lstinputlisting{/home/casey/Research/Github/coll_framework_egk/R/sdm.R}

\vspace{.5cm}
\noindent Chapter 1: Species occurrence plots\\
\textit{https://github.com/cvisintin/coll\_framework\_egk/blob/master/R/sdm\_plots.R}
%\lstinputlisting{/home/casey/Research/Github/coll_framework_egk/R/sdm_plots.R}

\vspace{.5cm}
\noindent Chapter 1: Prepare traffic model data\\
\textit{https://github.com/cvisintin/coll\_framework\_egk/blob/master/R/traffic\_model\_data.R}
%\lstinputlisting{/home/casey/Research/Github/coll_framework_egk/R/traffic_model_data.R}

\vspace{.5cm}
\noindent Chapter 1: Traffic modelling\\
\textit{https://github.com/cvisintin/coll\_framework\_egk/blob/master/R/traffic.R}
%\lstinputlisting{/home/casey/Research/Github/coll_framework_egk/R/traffic.R}

\vspace{.5cm}
\noindent Chapter 1: Traffic plots\\
\textit{https://github.com/cvisintin/coll\_framework\_egk/blob/master/R/traffic\_plots.R}
%\lstinputlisting{/home/casey/Research/Github/coll_framework_egk/R/traffic_plots.R}

\vspace{.5cm}
\noindent Chapter 1: Collision modelling\\
\textit{https://github.com/cvisintin/coll\_framework\_egk/blob/master/R/collision.R}
%\lstinputlisting{/home/casey/Research/Github/coll_framework_egk/R/collision.R}

\vspace{.5cm}
\noindent Chapter 1: Alternative collision modelling\\
\textit{https://github.com/cvisintin/coll\_framework\_egk/blob/master/R/collision-alt.R}
%\lstinputlisting{/home/casey/Research/Github/coll_framework_egk/R/collision-alt.R}

\vspace{.5cm}
\noindent Chapter 1: Collision plots\\
\textit{https://github.com/cvisintin/coll\_framework\_egk/blob/master/R/collision\_plots.R}
%\lstinputlisting{/home/casey/Research/Github/coll_framework_egk/R/collision_plots.R}

%%%%%%%%%%%%%%%%%%%%%%%%%%%

\vspace{.5cm}
\noindent Chapter 2: Prepare species occurrence model data

%\lstinputlisting{/home/casey/Research/Github/coll_framework_6sp/R/sdm_data.R}

\vspace{.5cm}
\noindent Chapter 2: Species occurrence modelling

%\lstinputlisting{/home/casey/Research/Github/coll_framework_6sp/R/sdm_model.R}

\vspace{.5cm}
\noindent Chapter 2: Species occurrence plots

%\lstinputlisting{/home/casey/Research/Github/coll_framework_6sp/R/sdm_plots.R}

\vspace{.5cm}
\noindent Chapter 2: Prepare collision modelling data

%\lstinputlisting{/home/casey/Research/Github/coll_framework_6sp/R/collision_risk_data.R}

\vspace{.5cm}
\noindent Chapter 2: Collision modelling

%\lstinputlisting{/home/casey/Research/Github/coll_framework_6sp/R/collision_risk_model.R}

\vspace{.5cm}
\noindent Chapter 2: Collision plots

%\lstinputlisting{/home/casey/Research/Github/coll_framework_6sp/R/collision_risk_plot.R}

%%%%%%%%%%%%%%%%%%%%%%%%%%%

\vspace{.5cm}
\noindent Chapter 3: Prepare deer occurrence model data

%\lstinputlisting{/home/casey/Research/Github/coll_framework_cal/R/deer_data.R}

\vspace{.5cm}
\noindent Chapter 3: Deer occurrence modelling

%\lstinputlisting{/home/casey/Research/Github/coll_framework_cal/R/deer_model.R}

\vspace{.5cm}
\noindent Chapter 3: Prepare traffic model data for California

%\lstinputlisting{/home/casey/Research/Github/coll_framework_cal/R/cal_traffic_model_data.R}

\vspace{.5cm}
\noindent Chapter 3: Traffic modelling for California

%\lstinputlisting{/home/casey/Research/Github/coll_framework_cal/R/cal_traffic_model.R}

\vspace{.5cm}
\noindent Chapter 3: Data preparation and collision modelling for California

%\lstinputlisting{/home/casey/Research/Github/coll_framework_cal/R/cal_collision.R}

%%%%%%%%%%%%%%%%%%%%%%%%%%%

\vspace{.5cm}
\noindent Chapter 4: Obtain train movement data

%\lstinputlisting{/home/casey/Research/Github/coll_model_trains/R/gtfs_train_data.R}

\vspace{.5cm}
\noindent Chapter 4: Process train movement data

%\lstinputlisting{/home/casey/Research/Github/coll_model_trains/R/train_times_c.R}

\vspace{.5cm}
\noindent Chapter 4: Train collision modelling

%\lstinputlisting{/home/casey/Research/Github/coll_model_trains/R/train_model_c.R}

\vspace{.5cm}
\noindent Chapter 4: Train collision plots

%\lstinputlisting{/home/casey/Research/Github/coll_model_trains/R/train_model_plot.R}

\vspace{.5cm}
\noindent Chapter 4: Train model predictions

%\lstinputlisting{/home/casey/Research/Github/coll_model_trains/R/train_model_predictions.R}

%%%%%%%%%%%%%%%%%%%%%%%%%%%

\vspace{.5cm}
\noindent Chapter 5: Prepare validation model data

%\lstinputlisting{/home/casey/Research/Github/coll_framework_val/R/independent_data.R}

\vspace{.5cm}
\noindent Chapter 5: Validate modelling

%\lstinputlisting{/home/casey/Research/Github/coll_framework_val/R/model_validation.R}

\vspace{.5cm}
\noindent Chapter 5: Validation plots

%\lstinputlisting{/home/casey/Research/Github/coll_framework_val/R/validation_plots.R}
\chapter{Computer code}\label{apx:A}

%\chapter{Species used in research}\label{apx:B}
\newpage

The following sections provide information on the species used in this research. Unless otherwise cited, all of the descriptions provided were synthesised from information found in \cite{vand08}. All of the images are used under the Creative Commons License, version 4.0.

\section{Mule deer}\label{deer}
\vspace{-0.3cm}
\setlength\intextsep{0pt}
\begin{wrapfigure}{l}{0.29\textwidth}
\centering
\includegraphics[width=0.27\textwidth]{deer_list_sc.jpg}
\end{wrapfigure}
Mule deer (\textit{Odocoileus hemionus}) are placental mammals that belong to a family of hoofed species that annually shed their antlers. They are of considerable size, males averaging between seventy-five and one-hundred kilograms (females are approximately seventy-five percent the weight). The fur of mule deer changes colour seasonally, from blue-gray in winter to reddish-brown in summer. Mule deer travel in herds during migration, however, males are generally solitary. Their home ranges is influenced by habitat quality and season and may cover between 100--5000 hectares. Due to limited ability to digest fibrous roughage, Mule deer require easily digestible forage that includes leaves, needles, stems, fruits/nuts, shrubs and grasses -- both living and dead. Occurring across most of western North America, except in tundra, sub-tropic, and extreme desert regions, Mule deer are quite adaptable to most habitats, however, migrate seasonally. Mule deer are subject to population management through harvesting/hunting and culling, especially when high densities initiate human-wildlife conflicts. The proceeding information and more details about Mule deer can be found in \cite{ferg05}.

\section{Eastern grey kangaroos}\label{egk}
\vspace{-0.3cm}
\setlength\intextsep{0pt}
\begin{wrapfigure}[5]{l}{0.29\textwidth}
\centering
\includegraphics[width=0.27\textwidth]{egk_list_sc.jpg}
\end{wrapfigure}
The Eastern Grey Kangaroo (\textit{Macropus giganteus}) is the second-largest native terrestrial mammal in Australia. Males can weigh up to eighty-five kilograms and have body lengths of over three metres (including tail). Despite the name, Eastern Greys may also have significant patches of light brown or cream colouring, especially on the underbelly. As with most other Australian mammals, they are marsupials and carry young in pouches. They travel in groups, called 'mobs', and have stable home ranges that vary between 30--1600 hectares. Eastern Greys are grazing animals and consume primarily grasses. Their distribution is quite broad, extending across a significant portion of eastern Australia from Cape York to north-east Tasmania. This area covers a wide range of habitat types including woodlands, shrub and heathland, and also human-modified landscapes such as golf courses and urban parklands. If conditions are right, Eastern Grey Kangaroos can increase to densities exceeding three individuals per hectare. Therefore, populations are often managed by government agencies through culling, sterilisation, and/or harvesting.

\section{Swamp wallabies}\label{bsw}
\vspace{-0.3cm}
\setlength\intextsep{0pt}
\begin{wrapfigure}{l}{0.29\textwidth}
\centering
\includegraphics[width=0.27\textwidth]{bsw_list_sc.jpg}
\end{wrapfigure}
The Swamp Wallaby (\textit{Wallabia bicolor}) is the only member in genus \textit{Wallabia} due to its unique physical and behavioural characteristics. It is a medium-sized marsupial, weighing between ten and twenty kilograms and well over a metre long (including tail). Swamp Wallabies also go by the name 'black wallabies', possibly due to appearing much darker than other wallabies, but their fur is normally brownish-gray or reddish-brown. As they are strictly solitary animals, it is quite rare to observe Swamp Wallabies in groups. Shrubs comprise the main source of food for Swamp Wallabies, both native and exotic, although they also occasionally forage on underground fungi. They are widely distributed along the eastern coastline of Australia as four sub-species. Swamp wallabies are occasionally culled to control populations, but the course fur and small body size make them unattractive for commercial harvesting.

\section{Wombats}\label{wom}
\vspace{-0.3cm}
\setlength\intextsep{0pt}
\begin{wrapfigure}{l}{0.29\textwidth}
\centering
\includegraphics[width=0.27\textwidth]{wom_list_sc.jpg}
\end{wrapfigure}
Common Wombats (\textit{Vombatus ursinus}) are robust, medium-sized marsupial mammals weighing between twenty and forty kilograms and having a body length of around a metre. Unlike most other ground or tree-dwelling marsupials, wombats live in underground burrows (sometimes up to twenty metres long). Their fur color is grayish-brown and does not vary much between individuals. Common Wombats are solitary and visit multiple burrows over several weeks. Fibrous grasses are the main food source of Common Wombats. Depending on the distribution of food resources, their home range will vary between two and eighty hectares. They occur predominantly in the south-east of Australia, beginning in the southern Queensland and continuing down through eastern Victoria. Although not officially managed wildlife, Common Wombats are often threatened by domesticated dogs and persecuted by humans.

\section{Koalas}\label{koa}
\vspace{-0.3cm}
\setlength\intextsep{0pt}
\begin{wrapfigure}{l}{0.29\textwidth}
\centering
\includegraphics[width=0.27\textwidth]{koa_list_sc.jpg}
\end{wrapfigure}
Koalas (\textit{Phascolarctos cinereus}) are perhaps one of Australia's most iconic animal. Often mistakenly referred to a 'koala bears', Koalas have no relationship to bears and are more closely related to wombats. They are the largest Australian arboreal marsupial, weighing between four and ten kilometres. Koalas in the north of the continent have short, light gray fur whilst their southern counterparts have longer grayish-brown fur. As they are solitary and territorial, it is uncommon to find several Koalas in a single tree. The Koala diet consists, almost exclusively, of eucalypt foliage -- nearly half of a kilogram per day. Home ranges of Koalas, which vary based on food availability, are between one and two hectares. They have a broad distribution along the eastern and south-eastern Australian coastlines extending from the southern part of Cape York to just over the state border between Victoria and South Australia. For many reasons, including threats to persistence and charismatic status, Koalas are a primary species involved in wildlife management operations in three states of Australia.

\section{Brushtail possums}\label{btp}
\vspace{-0.3cm}
\setlength\intextsep{0pt}
\begin{wrapfigure}{l}{0.29\textwidth}
\centering
\includegraphics[width=0.27\textwidth]{btp_list_sc.jpg}
\end{wrapfigure}
Common Brushtail Possums (\textit{Trichosurus vulpecula}) are small-medium marsupial mammals that are highly adapted to urban environments. Animals can weigh between one and five kilograms depending on geographic location, smaller sizes occurring in northern Australia and larger sizes in the south. Fur color also varies quite considerably by location but the most familiar is gray with a white underbelly and black brushy tail (hence the name). Possums are normally solitary and establish territories over relatively small home ranges. Their primary native diet is leaves, flowers, and fruits, although in highly modified landscapes, Brushtail Possums have resorted to omnivorous and opportunistic additions such as bird eggs, human-discarded food scraps, and even invertebrates. Brushtail Possums are widespread throughout Australia and occur in many habitats, albeit preferring dry eucalypt forests. In urban environments, their densities can exceed four individuals per hectare, thereby resulting in conflicts with human inhabitants.

\section{Ringtail possums}\label{rtp}
\vspace{-0.3cm}
\setlength\intextsep{0pt}
\begin{wrapfigure}{l}{0.29\textwidth}
\centering
\includegraphics[width=0.27\textwidth]{rtp_list_sc.jpg}
\end{wrapfigure}
Common Ringtail Possums (\textit{Pseudocheirus peregrinus}) are small marsupial mammals and often found in suburban gardens. They often weigh less than one kilogram and are no more than seven-hundred centimetres long (half of which is tail). Ringtail Possums use their distinctively curled tails as an effective 'fifth limb' similar to primates. They have relatively short fur that is brownish-grey in colour, with white on the underbelly and tip at the end of the tail. Ringtail Possums are strictly vegetarian and typically forage on leaves, flowers, and fruit. Although spanning several different habitat types, the species only occurs along the eastern seaboard of Australia, with limited distribution inland of the Great Dividing Range, and within the state of Tasmania. Ringtail Possums adapt well to urban environments and, whilst considered annoying to some residential gardeners, do not cause any significant issues.
\chapter{Species used in research}\label{apx:B}

\chapter{Data sources used for research}\label{apx:C}
\newpage

The following sections provide information on the sources of electronic data used in this thesis. Whilst most of the data is publicly accessible, some require additional permission for acquisition and use; these are identified in the descriptions. Any data modifications required to conform with project requirements are also described.

\section{Wildlife Victoria}

Wildlife Victoria is a not-for-profit organisation that provides emergency response service for sick, injured, or orphaned native wildlife. It has operated under the vision of living in a community that cares about the welfare of Australian wildlife for nearly thirty years. The organisation uses a professional database system to record details of all incidents reported across the State of Victoria. This database has more that 100,000 records, most of which have spatial and temporal data. This information is obtainable from Wildlife Victoria but remains their sole intellectual property and all uses of the data are subject to approval (approvals were obtained for the work in this thesis). More information on Wildlife Victoria can be found at \textit{https://www.wildlifevictoria.org.au}.

\section{California Roadkill Observation System}

The California Roadkill Observation System (CROS) is a web-based system used to catalogue observations of wildlife killed on roads throughout the state. The system is maintained by the Road Ecology Center based at the University of California, Davis and allows members of the public to enter observations from the field. At this time of writing, the system has 52,925 observations of 421 unique species originating from 1301 registered observers. More information on CROS can be found at \textit{http://www.wildlifecrossing.net/california}.

\section{VicRoads}

VicRoads is responsible for planning, developing, and maintaining more than 50,000 kilometres of major arterial roads in the State of Victoria. Further, the government organisation is also responsible for registering vehicles and licensing motorists to operate within the state. VicRoads hire independent cleaning contractors to remove debris from major sections of roads, including wildlife carcasses, and this information is recorded -- albeit scarcely. More information on VicRoads can be found at \textit{https://www.vicroads.vic.gov.au}.

\section{V/Line}

V/Line is Australia's largest regional public transport operator scheduling more than 1,700 train services between the urban centre of Melbourne and regional Victoria. V/line also maintains over 3,500 kilometres of railway used by passenger and freight transport throughout the state. Collisions with large species are recorded by train drivers as they have implications for train maintenance and safety. Although this information is obtainable from V/Line, it remains their sole intellectual property and all uses of the data are subject to approval (approvals were obtained for this thesis). More information on V/Line can be found at \textit{https://www.vline.com.au}.

\section{City of Bendigo}

The City of Bendigo is a large town located in regional Victoria and the third most populated urban area in the state (>100,000 persons). The council is responsible for litter abatement in public spaces (i.e. nature strips) and responds to reports from citizens. Kangaroo roadkill are commonly reported and the council maintains records of locations and times of these incidents. This information may be requested from the public works department. More information on Bendigo can be found at \textit{https://www.bendigo.vic.gov.au}.

\section{Crashstats}

Crashstats is a web-accessible, online repository of statistics on road crashes maintained by VicRoads. These data are based on report made to the Victorian Police and usually involve human injuries and/or deaths. More information on Crashstats, and access to data, can be found at \textit{https://www.vicroads.vic.gov.au/safety-and-road-rules/safety-statistics/crash-statistics}.

\section{Insurance Australia Group}

Insurance Australia Group is a large insurance underwriter that operates throughout Australia, New Zealand and Asia. They provide services for several motor-vehicle insurance agencies and therefore receive access to vast amounts of claim data resulting from collisions. One significant component of this data are records on collisions with wildlife, albeit spatial uncertainty is quite high and therefore aggregated to the town level. Although this information is obtainable from IAG, it remains their sole intellectual property and all uses of the data are subject to approval (approvals were obtained for this thesis). More information on IAG can be found at \textit{https://www.iag.com.au}.

\section{Victorian Biodiversity Atlas}

The Victorian Biodiversity Atlas (VBA) contains high-quality records of flora and fauna occurrences throughout the state and is maintained by the Arthur Rylah Institute, a scientific research branch of the Victorian State Government. More information is available at \textit{http://www.depi.vic.gov.au/environment-and-wildlife/biodiversity/victorian-biodiversity-atlas}.

\section{Global Biodiversity Information Facility}

The Global Biodiversity Information Facility (GBIF) is an international collaboration and government-funded open data infrastructure containing information about all types of life on Earth. It is the world's largest online repository of information, detailing 1.6 million species collected over three centuries. The system provides spatial and temporal records of species occurrence based on observations from citizen scientists, researchers and automated monitoring programmes, and museum specimen archives. More information and data are available at \textit{http://www.gbif.org}.

\section{WorldClim}

WorldClim is a collection of global climatic data. The data are represented as 1km$^2$ raster grids and include information on bioclimatic conditions for the past, present, and future. Interpolated temperature and rainfall data are also included in the collection. \cite{hijm05} details the methods used to construct the data. These data are made available at \textit{http://www.worldclim.org}. This thesis uses WorldClim data on isothermality, mean temperature of the wettest quarter, precipitation of the driest month ,seasonality of precipitation, and annual temperature range.

\section{Moderate Resolution Imaging Spectroradiometer}

The Moderate Resolution Imaging Spectroradiometer (MODIS) is a remote-sensing instrument carried on-board the two satellites, Terra and Aqua. These two satellites view the entire Earth's surface every 1 to 2 days and improve our understanding of global dynamics and processes. The work in this thesis uses one of the MODIS vegetation indices, the normalized difference vegetation index (NDVI), to represent relative quality of vegetation. More information and data are available at \textit{https://modis.gsfc.nasa.gov}. These data are provided in 16-day intervals and at multiple spatial resolutions, however, 1km$^2$ resolution (MOD13A2) is used in this thesis. To represent the seasonal change in quality of vegetation (greenness variable), I averaged the differences between January and July measures of NDVI over a ten-year period (2000--2010) that spanned drought and non-drought conditions in Australia.

\section{Defense Meteorological Satellite Program}

The Defense Meteorological Satellite Program (DMSP), operated by the National Aeronautics and Space Administration (NASA) and National Oceanic and Atmospheric Administration (NOAA) records global nighttime artificial lighting. The data for all available DMSP-OLS smooth resolution imagery are used to produce cloud-free composites. Sunlit and glare data are excluded based on solar elevation angles. More information and data are available at \textit{https://ngdc.noaa.gov/eog/dmsp.html}. In this thesis, I used data for the year 2013.

\section{Australian Grassland and Rangeland Assessment by Spatial Simulation}

The Australian Grassland and Rangeland Assessment by Spatial Simulation (Aussie GRASS) is used to monitor, at regional scales, pasture degradation and recovery using modelling that accounts for daily rainfall, evaporation, temperature, vapour pressure and solar radiation. It also accounts for soil and pasture types, tree and shrub cover, and abundance of domestic livestock and other key herbivores. The methods are detailed in \cite{cart03}. Although this information is obtainable from the Queensland Government, it remains their sole intellectual property and all uses of the data are subject to approval (approvals were obtained for this thesis). More information on Aussie GRASS can be found at \textit{https://www.longpaddock.qld.gov.au/about/researchprojects/aussiegrass}.

\section{Geoscience Australia}

Geoscience Australia provides information and scientific data on the geology and geography of Australia. Data can be downloaded from \textit{http://www.ga.gov.au/data-pubs}. I used a 9-second digital elevation model (DEM) to represent elevation for Australia in this thesis; methods are detailed in \cite{hutc91,hutc11}. Slope aspect values were derived from the DEM by using a geographic information system (GIS).

\section{United States Geologic Survey}

The United States Geologic Survey (USGS) provides information and scientific data on the geology and geography of the United States with select extended datasets describing global features. More information and data are available at \textit{https://www.usgs.gov}. I used data from the USGS to represent two variables in the thesis. Hydrological flow accumulation was obtained from HydroSHEDS (\textit{https://hydrosheds.cr.usgs.gov}), a mapping product that provides hydrographic information at regional and global scales. This data was available at 15 arc-second resolution and was projected and re-scaled to a 1km$^2$ resolution. I used SRTM (Shuttle Radar Topography Mission) 3 arc-second, void-filled elevation data (\textit{https://lta.cr.usgs.gov/SRTM1Arc}) to generate elevation variables for both California and Australia in this thesis.

\section{Global Land Cover Facility}

The Global Land Cover Facility (GLCF) provides data and products to describe global environmental systems and is based on remotely-sensed satellite data with a focus on land cover at different scales. More information and data are available at \textit{http://glcf.umd.edu}. I obtained a 30-metre resolution tree cover layer containing estimated percentage of horizontal ground in each pixel covered by woody vegetation greater than 5 meters in height; originally derived from Landsat Vegetation Continuous Fields (VCF) -- details in \cite{sext13}. For the tree cover variable used in this thesis, I spatially aggregated the data to a 1km$^2$ resolution.

\section{VicMap}

The VicMap collection of electronic spatial information is maintained by the Victorian Government and contains GIS data on many geographic features used in urban and regional planning and research. Information one these offerings is provided at \textit{http://www.depi.vic.gov.au/forestry-and-land-use/spatial-data-and-resources/vicmap}. In this thesis, I obtained VicMap spatial data for roads, water features, and administrative boundaries such as towns. Using these features, I calculated distances to towns, roads, and linear waterways with geoprocessing tools and represented the outputs as 1km$^2$ resolution spatial grids.

\section{SoilGrids}

SoilGrids is a system for global soil mapping based on predictions from machine learning and statistics, maintained by the International Soil Reference and Information Centre (ISRIC). The methods are described in \cite{heng14}, and data can be obtained at \textit{http://www.isric.org/content/soilgrids}. Soil predictions are made at a 1km$^2$ resolution scale which I use to represent variables pertaining to soil density, conductivity, and content (e.g. sand or clay) in this thesis.

\section{Australian Bureau of Statistics}

The Australian Bureau of Statistics (ABS) is a government agency that provides data and reports on economic, social, population and environmental topics. More information and data are available at \textit{http://www.abs.gov.au}. Data obtained from the ABS were joined to spatial data in this thesis to create a population density variable. I used pycnophylactic interpolation \citep{tobl79} to obtain pixel-based population and income estimates whilst accounting for irregular administrative boundaries (e.g. towns).

\section{United States Census Bureau}

The United States Census Bureau is the leading source for demographic and economic data about the United States which guides a significant portion of government spending ($\approx$ 400 billion USD). More information and data are available at \textit{https://www.census.gov}. For the California-based work this thesis, data obtained from the Census Bureau were joined to spatial data to create a population density variable. I used pycnophylactic interpolation to obtain pixel-based population and income estimates at 1km$^2$ resolution (see previous section on data from the Australian Bureau of Statistics).

%\chapter{Data sources used for research}\label{apx:C}

%\chapter{List of studies used for comparing modelling variables}\label{apx:D}
\newpage

\renewcommand\baselinestretch{0.9}\selectfont
\setlength{\extrarowheight}{.5em}
\scriptsize

\begin{longtable}[c]{@{}p{3.5cm}p{2.0cm}p{2.5cm}p{5.5cm}@{}}
\caption[Frequency of variables used in wildlife-vehicle collision studies.]{Frequency of variables used in wildlife-vehicle collision studies.}
\label{var_rep}\\
\toprule
Variable Name & Total Studies Using Variable & Percent Representation (71 Studies) & Study Reference Number in \Cref{coll_studies} \\ 
  \hline
  \endhead
Traffic Volume &  39 & 54.93 & 4, 5, 6, 7, 9, 10, 14, 17, 18, 19, 21, 25, 27, 28, 29, 33, 34, 35, 36, 37, 39, 40, 43, 47, 49, 50, 54, 55, 59, 61, 62, 63, 64, 65, 66, 68, 69, 70, 71 \\ 
  Topography &  26 & 36.62 & 2, 3, 8, 9, 10, 11, 13, 16, 20, 21, 23, 30, 32, 34, 36, 41, 44, 51, 53, 54, 56, 57, 61, 62, 66, 69 \\ 
  Land Use &  24 & 33.80 & 2, 3, 10, 11, 17, 19, 20, 21, 23, 24, 26, 32, 34, 35, 36, 44, 49, 61, 63, 66, 67, 68, 69, 70 \\ 
  Traffic Speed &  23 & 32.39 & 3, 6, 10, 12, 15, 19, 22, 29, 31, 32, 36, 37, 38, 41, 44, 47, 54, 58, 61, 63, 66, 68, 69 \\ 
  Vegetation Cover &  21 & 29.58 & 2, 8, 21, 23, 27, 30, 33, 34, 39, 41, 47, 48, 51, 54, 63, 65, 66, 67, 68, 70, 71 \\ 
  Road Structure &  17 & 23.94 & 2, 8, 9, 10, 11, 18, 20, 21, 23, 33, 36, 41, 42, 51, 61, 62, 63 \\ 
  Habitat Type &  16 & 22.54 & 3, 8, 9, 10, 15, 23, 24, 36, 37, 46, 48, 57, 64, 65, 66, 71 \\ 
  Road Sinuosity &  15 & 21.13 & 3, 8, 10, 13, 15, 23, 31, 33, 34, 37, 39, 41, 53, 63, 69 \\ 
  Distance to Water &  14 & 19.72 & 3, 8, 10, 15, 21, 23, 32, 41, 47, 53, 56, 57, 66, 71 \\ 
  Development Density &  13 & 18.31 & 2, 3, 11, 15, 20, 21, 34, 37, 41, 61, 63, 65, 68 \\ 
  Hydrology &  13 & 18.31 & 3, 11, 20, 23, 33, 34, 35, 39, 61, 62, 63, 68, 69 \\ 
  Tree Density &  11 & 15.49 & 20, 23, 34, 41, 44, 48, 53, 54, 61, 62, 66 \\ 
  Road Verge &  11 & 15.49 & 1, 8, 14, 19, 21, 36, 41, 54, 63, 65, 69 \\ 
  Road Width &  10 & 14.08 & 3, 8, 19, 20, 23, 28, 36, 37, 65, 69 \\ 
  Road Topology &  10 & 14.08 & 8, 9, 13, 20, 21, 23, 37, 41, 63, 71 \\ 
  Temperature &  10 & 14.08 & 12, 13, 15, 16, 22, 27, 33, 39, 56, 64 \\ 
  Rainfall &   9 & 12.68 & 13, 15, 27, 33, 39, 53, 54, 56, 64 \\ 
  Season &   9 & 12.68 & 17, 18, 38, 43, 44, 49, 58, 64, 70 \\ 
  Road Class &   8 & 11.27 & 21, 26, 38, 44, 46, 58, 63, 68 \\ 
  Species Presence &   7 & 9.86 & 2, 4, 11, 26, 55, 56, 60 \\ 
  Species Traits &   7 & 9.86 & 5, 25, 26, 28, 29, 38, 40 \\ 
  Distance to Vegetation &   7 & 9.86 & 3, 9, 41, 47, 57, 61, 71 \\ 
  Time of Day &   7 & 9.86 & 17, 22, 25, 38, 45, 46, 58 \\ 
  Species Density &   6 & 8.45 & 12, 33, 42, 50, 52, 66 \\ 
  Taxon &   6 & 8.45 & 18, 37, 38, 42, 54, 58 \\ 
  Distance to Town &   5 & 7.04 & 9, 10, 20, 21, 53 \\ 
  Road Segment Length &   5 & 7.04 & 19, 36, 49, 50, 59 \\ 
  Ecotone &   4 & 5.63 & 3, 10, 47, 66 \\ 
  Distance to Development &   4 & 5.63 & 8, 23, 66, 71 \\ 
  Year &   4 & 5.63 & 16, 26, 42, 43 \\ 
  Vegetation Condition &   3 & 4.23 & 47, 56, 57 \\ 
  Distance to Structure &   3 & 4.23 & 3, 23, 41 \\ 
  Distance to Infrastructure &   3 & 4.23 & 10, 21, 61 \\ 
  Road Density &   3 & 4.23 & 47, 61, 62 \\ 
  Sex Ratio &   2 & 2.82 & 37, 38 \\ 
  Harvesting &   2 & 2.82 & 61, 62 \\ 
  Habitat Connectivity &   2 & 2.82 & 8, 60 \\ 
  Wind &   2 & 2.82 & 12, 27 \\ 
  Climate &   2 & 2.82 & 18, 26 \\ 
  Air Pressure &   2 & 2.82 & 13, 27 \\ 
  Soil &   2 & 2.82 & 27, 56 \\ 
  Moon &   2 & 2.82 & 22, 45 \\ 
  Light &   2 & 2.82 & 12, 31 \\ 
  Road Risk &   1 & 1.41 & 60 \\ 
  Registered Vehicles &   1 & 1.41 & 66 \\ 
  Management &   1 & 1.41 & 1 \\ 
  Distance to Shelter &   1 & 1.41 & 57 \\ 
  Road Condition &   1 & 1.41 & 31 \\ 
  Visibility &   1 & 1.41 & 8 \\ 
  Humidity &   1 & 1.41 & 27 \\ 
  Cloud Cover &   1 & 1.41 & 45 \\ 
  Snow &   1 & 1.41 & 22 \\ 
  Month &   1 & 1.41 & 42 \\ 
   \hline
\end{longtable}

\begin{longtable}[c]{@{}p{.75cm}p{6cm}p{1.25cm}p{5.5cm}@{}}
\caption[Published studies using wildlife-vehicle collision modellling.]{Published studies using wildlife-vehicle collision modellling.}
\label{coll_studies}\\
\toprule
ID & Author(s) & Year & Publication \\ 
  \hline
  \endhead
    1 & Andreassen, H.P., Gundersen, H. \textit{et al.} & 2005 & Journal of Wildlife Management \\ 
    2 & Barnum, S., Rinehart, K. \textit{et al.} & 2007 & ICOET Proceedings \\ 
    3 & Barthelmess, E.L. & 2014 & Biodiversity and Conservation \\ 
    4 & Bauduin, S., Martin, J. \textit{et al.} & 2013 & Biological Conservation  \\ 
    5 & Beaudry, F., Demaynadier, P.G. \textit{et al.} & 2010 & Journal of Wildlife Management \\ 
    6 & Bissonette, J.A. and Adair, W. & 2008 & Biological Conservation \\ 
    7 & Brockie, R.E., Sadleir, R.M. \textit{et al.} & 2009 & New Zealand Journal of Zoology \\ 
    8 & Clevenger, A.P. \textit{et al.} & 2015 & Ecosphere \\ 
    9 & Clevenger, A.P., Chruszcz, B. \textit{et al.} & 2003 & Biological Conservation  \\ 
   10 & Colino-Rabanal, V., Lizana, M. \textit{et al.} & 2011 & European Journal of Wildlife Research \\ 
   11 & Cserkesz, T., Ottlecz, B. \textit{et al.} & 2013 & European Journal of Wildlife Research \\ 
   12 & DeVault, T.L., Blackwell, B.F. \textit{et al.} & 2014 & PloS one \\ 
   13 & Dussault, C., Poulin, M. \textit{et al.} & 2006 & Wildlife Biology \\ 
   14 & Eigenbrod, F., Hecnar, S.J. \textit{et al.} & 2008 & Biological conservation \\ 
   15 & Farmer, R.G. and Brooks, R.J. & 2012 & Journal of Wildlife Management \\ 
   16 & Fonnesbeck, C. \textit{et al.} & 2008 & Endangered Species Research \\ 
   17 & Gagnon, J.W., Theimer, T.C. \textit{et al.} & 2007 & Journal of Wildlife Management \\ 
   18 & Garriga, N., Santos, X. \textit{et al.} & 2012 & Biodiversity and Conservation \\ 
   19 & Gkritza, K. \textit{et al.} & 2013 & Journal of Transportation Engineering \\ 
   20 & Gomes, L., Grilo, C. \textit{et al.} & 2008 & Ecological Research \\ 
   21 & Grilo, C., Bissonette, J.A. \textit{et al.} & 2009 & Biological Conservation \\ 
   22 & Gundersen, H. and Andreassen, H.P. & 1998 & Wildlife Biology \\ 
   23 & Gunson, K.E., Clevenger, A.P. \textit{et al.} & 2009 & Environmental Management \\ 
   24 & Gunson, K.E., Ireland, D. \textit{et al.} & 2012 & North-Western Journal of Zoology \\ 
   25 & Hels, T. and Buchwald, E. & 2001 & Biological Conservation \\ 
   26 & Hothorn, T., Brandl, R. \textit{et al.} & 2012 & PloS one \\ 
   27 & Inbar, M. and Mayer, R.T. & 1999 & Wildlife Society Bulletin \\ 
   28 & Jaarsma, C., Van Langevelde, F. \textit{et al.} & 2007 & Ecological Informatics \\ 
   29 & Jaarsma, C.F., van Langevelde, F. \textit{et al.} & 2006 & Transportation Research Part D: Transport and Environment  \\ 
   30 & Jensen, R.R., Gonser, R.A. \textit{et al.} & 2014 & Applied Geography \\ 
   31 & Joyce, T.L. and Mahoney, S.P. & 2001 & Wildlife Society Bulletin \\ 
   32 & Kanda, L.L., Fuller, T.K. \textit{et al.} & 2006 & The American Midland Naturalist \\ 
   33 & Klöcker, U., Croft, D.B. \textit{et al.} & 2006 & Wildlife Research \\ 
   34 & Kolowski, J.M. and Nielsen, C.K. & 2008 & Biological Conservation  \\ 
   35 & Langen, T.A., Ogden, K.M. \textit{et al.} & 2009 & Journal of Wildlife Management \\ 
   36 & Lao, Y. \textit{et al.} & 2011 & Accident Analysis and Prevention \\ 
   37 & Lao, Y., Zhang, G. \textit{et al.} & 2011 & Accident Analysis and Prevention \\ 
   38 & Lee, E., Croft, D.B. \textit{et al.} & 2010 & Book Chapter \\ 
   39 & Lee, E., Klöcker, U., Croft, D.B. \textit{et al.} & 2004 & Australian Mammalogy \\ 
   40 & Litvaitis, J.A. and Tash, J.P. & 2008 & Environmental Management \\ 
   41 & Malo, J.E., Suárez, F. \textit{et al.} & 2004 & Journal of Applied Ecology \\ 
   42 & Markolt, F. and Szemethy, L. \textit{et al.} & 2012 & North-Western Journal of Zoology \\ 
   43 & Mazerolle, M.J. & 2004 & Herpetologica \\ 
   44 & Meisingset, E.L., Loe, L.E. \textit{et al.} & 2014 & Journal of Wildlife Management \\ 
   45 & Mizuta, T. & 2014 & The Wilson Journal of Ornithology \\ 
   46 & Neumann, W., Ericsson, G. \textit{et al.} & 2012 & Biological Conservation  \\ 
   47 & Ng, J. \textit{et al.} & 2008 & Human-Wildlife Interactions \\ 
   48 & Nielsen, C.K., Anderson, R.G. \textit{et al.} & 2003 & Journal of Wildlife Management \\ 
   49 & Orłowski, G. and Nowak, L. & 2006 & Polish Journal of Ecology \\ 
   50 & Polak, T., Rhodes, J.R. \textit{et al.} & 2014 & Journal of Applied Ecology \\ 
   51 & Puglisi, M.J., Lindzey, J.S. \textit{et al.} & 1974 & Journal of Wildlife Management \\ 
   52 & Ramp, D. and Ben-Ami, D. & 2006 & Journal of Wildlife Management \\ 
   53 & Ramp, D., Caldwell, J. \textit{et al.} & 2005 & Biological Conservation \\ 
   54 & Ramp, D., Wilson, V.K. \textit{et al.} & 2006 & Biological conservation \\ 
   55 & Rhodes, J., Lunney, D. \textit{et al.} & 2014 & PLoS One \\ 
   56 & Roger, E., Bino, G. \textit{et al.} & 2012 & Landscape ecology \\ 
   57 & Roger, E. and Ramp, D. & 2009 & Diversity and Distributions \\ 
   58 & Rowden, P., Steinhardt, D. \textit{et al.} & 2008 & Accident Analysis and Prevention  \\ 
   59 & Saeki, M. and Macdonald, D. & 2004 & Biological Conservation  \\ 
   60 & Santos, S.M. and Louren{\c{c}}o, R. \textit{et al.} & 2013 & PloS One \\ 
   61 & Seiler, A. & 2005 & Journal of Applied Ecology \\ 
   62 & Seiler, A. & 2004 & Wildlife Biology \\ 
   63 & Seo, C., Thorne, J. \textit{et al.} & 2015 & Landscape and Ecological Engineering \\ 
   64 & Shepard, D.B., Dreslik, M.J. \textit{et al.} & 2008 & Copeia \\ 
   65 & Skorka, P., Lenda, M. \textit{et al.} & 2013 & Biological Conservation  \\ 
   66 & Snow, N.P., Porter, W.F. \textit{et al.} & 2015 & Biological Conservation  \\ 
   67 & Snow, N.P., Williams, D.M. \textit{et al.} & 2014 & Landscape Ecology \\ 
   68 & Sudharsan, K., Riley, S.J. \textit{et al.} & 2009 & Human Dimensions of Wildlife \\ 
   69 & Taylor, B.D. and Goldingay, R.L. & 2004 & Wildlife Research \\ 
   70 & Thurfjell, H., Spong, G. \textit{et al.} & 2015 & Landscape and Urban Planning  \\ 
   71 & Waller, J.S. and Servheen, C. & 2005 & Journal of Wildlife Management \\
   \hline
\end{longtable}

%\section{Ecology and Evolution}

%\Cref{sec:egk} was published during PhD candidature and the content in the chapter is nearly identical to the journal article with the following exceptions: (a) supporting information listed in the publication has been included within the thesis chapter for completeness (resulting in shifting figure numbers), and (b) R scripts used for analysis have been included in \Cref{apx:A} for the reader's information.
%\includepdf[pages={1-13}]{Visintin_2016_r.pdf}


%---------------------------------------------------------- 
% BACK
% list of symbols / references / index etc
%---------------------------------------------------------- 
%\backmatter
%
%\chapter{List of Symbols}
%
%
%The following list is neither exhaustive nor exclusive, but may be helpful.
%\begin{list}{}{%
%\setlength{\labelwidth}{24mm}
%\setlength{\leftmargin}{35mm}}
%\item[$a$, $b$, $c$, $d$\dotfill] annihilation operators
%\item[$a^\dagger$, $b^\dagger$, $c^\dagger$, $d^\dagger$\dotfill] creation
%operators
%\end{list}

\end{document}
