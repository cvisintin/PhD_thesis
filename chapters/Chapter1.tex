\chapter{General Introduction}\label{sec:intro}
\newpage

\section{Wildlife-human interactions}

\subsection{Impacts of wildlife on humans}

Positive, negative, and neutral...

\subsection{Effects of the built environment on wildlife}

Positive, negative, and neutral...

\subsection{Impacts of linear infrastructure on wildlife}

Linear infrastructure is most commonly thought of as roads, however, can also include railways, utility easements, and other notable continuous environmental interruptions in the landscape.

Positive, negative, and neutral...
road reserves led to retention of habitat - no roads or low traffic - positive

\section{Road ecology}

The field of road ecology addresses a wide range of environmental problems resulting from the construction and use of roads across the natural landscape.  The term was originally introduced by Richard T.T. Forman who is often also referred to as the 'father of landscape ecology'; a similar, yet broader discipline in the environmental sciences. Roads pose several ecological threats including destruction of habitat through fragmentation and pollution of various sorts (light, chemical, acoustic).

\subsection{Wildlife mortality on roads}

Perhaps one of the most visually apparent effects of the use of roads (and other vehicular networks) is the direct mortality of wildlife species; often referred to as 'roadkill' in the common vernacular. 

\subsection{Factors influencing collision risk}

Add literature review and predictors

\subsection{Management approaches to wildlife-vehicle collisions}

It is in the best interest of authorities to manage wildlife-vehicle collisions for numerous economic, social and environmental reasons.

\section{Risk management and decision-making}

\subsection{Distribution of risk in space and time}

\subsection{Spatial statistics and modelling}

More recently, computational techniques involving statistics and geographical information systems (GIS) have aided management authorities in identifying and analysing high-risk areas for collisions with animals.

When statistical modelling of events is undertaken, it is common to analyse either binary data (occurred/did not occur) or count data (how many occurred).  As it is not plausible to have negative events, these data are always constrained to be positive.  Moreover, unless one is interested in rates, such as number of events per unit (often spatial or temporal), data for events consists of positive integers.  Two distributions satisfy these requirements, Bernoulli and Poisson, and thus, are often chosen for modelling collision events.

Spatial statistical modelling is subject to several issues that must be considered when completing analyses.  These considerations have origins in statistical, physical and ecological theory.

Modelling events on linear networks is a special case of one-dimensional analysis as it would be impossible for collisions to occur outside the network.  More specifically, collisions will always occur coincident with the movement vector of a vehicle.

\subsubsection{Bias and uncertainty in data}

Data has inherent uncertainties and biases that must be addressed.

\subsubsection{Temporal variation in data}

Patterns in both natural and anthropogenic systems are influenced by time.

Moreover, ecological systems are subject to stochasticity.

%\subsubsection{Spatial and temporal autocorrelation}
%
%Although natural phenomenon are highly stochastic, patterns of influence are observed when species or events are closely located in both space and time.

\section{Aim and scope of the thesis}

The broad aim of this work is to predict where and when wildlife will be at the highest risk across the landscape due to collisions with moving vehicles.

Chapter two introduces a general conceptual framework to model where collisions with wildlife are likely to occur on a large road network.

Chapter three extends the work of chapter two by applying the model framework to seven mammal species commonly involved in wildlife-vehicle collisions.

In chapter four, the conceptual framework is applied to a different geographic area sharing similar features.

Modelling collision risk with train networks in both space and time is introduced in chapter five.

The predictions from the conceptual framework are compared to additional datasets collected with various methods in chapter six.