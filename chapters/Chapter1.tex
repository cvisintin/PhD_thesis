\chapter{General Introduction}\label{sec:intro}
\newpage

\section{The wildlife and road conundrum}

\subsection{Human-wildlife interactions}

We are now in what some refer to as the \textit{Anthropocene}, where \textit{Homo sapiens} exert strong influences on climatic, geologic and biotic systems \citep{crut06}. Studies estimate that humans have transformed more than seventy-five percent of the earth's ice-free terrestrial surface \citep[e.g.][]{elli08}. These modifications result from large-scale activities; perhaps the most influential being intensified agricultural production, provision of linear infrastructure (e.g. roads, rail, and utility easements), and urban development \citep{vito97,sand02,fole05}. Human population is forecast to grow in the future, and this will increase pressure on unmodified landscapes.

As permanent human settlements, industry, and related infrastructure spread over the Earth’s surface, bringing people closer to animals, the dominant anthopocentricism of western culture emphasises a conceptual distance between humans and the natural world.  Emblematic of this distance, and disjunction, is the field of research referred to as `human-wildlife interactions'; which has a dedicated journal (Human-Wildlife Interactions: ISSN 2155-3858) and monographs \citep[e.g.][]{manf08}. In this field, all manners of interactions are considered; both positive \citep{dall12} or negative \citep{jone99} -- see Figure XX for possible human-wildlife interactions. 

\begin{center}
--Insert diagram of human-wildlife interactions here--
\end{center}

For stationary configurations of human settlement, increased exposure is due to tolerant species remaining or re-establishing in human-dominated landscapes \citep{soul16}. Although the suite of organisms that humans have contact with is quite varied -- the most frequent contact possibly being with invertebrates -- medium to large species (i.e., $>$10 kg) tend to get more attention, especially when issues arise \citep{seor16}. Human beings generally appreciate the existence of wildlife within and around their places of settlement \citep{dick86,kell93,raad07}. Bird watching is a popular recreational activity practised worldwide and many communities rely on wildlife tourism as an economic resource \citep{news05}. In urban settings, humans may also consider species that surround them throughout their everyday lives as fellow-creatures \citep{zinn08}.

Certain kinds of interactions, usually where wildlife and humans compete for resources, or intersect in a way that causes nuisance or harm to human activities are referred to as `human-wildlife conflicts'. This, however, may be misleading as most studies using the term `wildlife-human conflicts' actually reference conflicts amongst humans in regards to wildlife \citep{pete10}. Nevertheless, studies have demonstrated notable impacts on society (e.g. human injuries, illness, and economic losses) from wildlife \citep{cono95}. Larger species are more likely to cause significant harm to person or damage to property \citep{cono01} and are less easily managed by individuals that lack proper training. Therefore, the public expectation is of government intervention to manage human-wildlife conflicts \citep{reit99}. Further, the general public often views the management of human-wildlife conflicts to reside in the scope of wildlife-focussed disciplines, however, issues arising from human-wildlife interactions are complex and may require knowledge from several non-ecological fields \citep{deck97,madd04}. For example, specialist knowledge on both animal behaviour (zoology) and human behaviour (sociology) may be combined to arrive at effective solutions \citep{dicka10}.

%From the other perspective, large charismatic species are often persecuted when residing in close proximity to human settlements and therefore targets of conservation programs \citep{trev03}.

%\subsection{Effects of wildlife on humans}
%
%\begin{figure*}[htp]
%  \centering
%  \includegraphics[width=0.8\textwidth]{microbats.jpg}
%  \caption[Public involvement with wildlife]{Public involvement in scientific research to learn about microbats in the Royal Botanic Gardens, Victoria, Australia. Photograph by Meg Bauer.}
%  \label{microbats}
%\end{figure*}
%
%Human beings generally appreciate the existence of wildlife within and around their places of settlement \citep{dick86,kell93,news05,raad07}. Bird watching is a popular recreational activity practised worldwide and many communities rely of wildlife tourism as an economic resource. Further, citizens are excited to learn about some of the more cryptic species that inhabit their immediate environment (Visintin, personal observation -- see \Cref{microbats}). In urban settings, human-wildlife interactions may be considered benign due to human's consideration of species that surround them throughout their everyday lives as fellow-creatures \citep{zinn08}.

%In many cases, wildlife aggression towards humans is due to protection of offspring; for example, swooping Australian magpies \citep{jone99} as shown in \Cref{magpie}.

%\begin{figure*}[htp]
%  \centering
%  \includegraphics[width=0.8\textwidth]{magpie.jpg}
%  \caption[Wildlife aggression towards humans]{Australian magpie swooping humans during breeding season. Photograph by Adam McLean.}
%  \label{magpie}
%\end{figure*}

%\subsection{Effects of the built environment on wildlife}
%
%In contrast to the natural environment, the built environment refers to human-manufactured structures, infrastructure, and recreational landscapes (e.g. urban parks) where people reside and conduct daily activities.  The built environment has mixed effects on wildlife persistence. Some species thrive in urban landscapes (termed `urban adapters') whilst others perish \citep{shoc06}. `Urban adapter' species benefit from increased availability of resources. For example, gulls and other urban birds rely on overflowing rubbish bins for food \citep{ditc06} and bats utilise small cavities in buildings and overpasses for shelter \citep{kunz82}. Many of the urban-adapted species, however, are invasive and displace native wildlife \citep{mcki06}. Built environments can also negatively effect sensitive species through direct displacement from habitat \citep{czec97}, or disturbance due to light and noise pollution \citep{parr16}. Further, wildlife mortality may result from static (e.g. bird strikes to glass) and non-static (e.g. car strikes to wildlife) elements of the built environment. 

\subsection{Impacts of linear infrastructure on wildlife}

Linear infrastructure can traverse vast areas between human settlements, and penetrate remaining natural and semi-natural areas. In some cases, these easements may be the only remaining semi-natural refuges for wildlife \citep{benn91}. Linear infrastructure is most commonly thought of as roads, however, can also include railways, utility easements, and other continuous human-engineered resources in the landscape. Impacts of linear infrastructure on wildlife can be positive, neutral, or negative. For example, disused road reserves lead to retention of habitat that may benefit wildlife as they often contain old and diverse remnant vegetation \citep{lent11}. Further, linear infrastructure without moving vehicles may provide connection corridors and assist migration for some species \citep{rvdr15}. Effects may be neutral when suitable habitat is provided on road reserves with very little traffic volume or occupied by species with low mobility. Nonetheless, negative effects of linear infrastructure may certainly outweigh the benefits, especially with respect to roads and wildlife persistence \citep{fahr09}. There are few places left in the terrestrial environment that are not accessible by some form of road (\Cref{roads}) and, consequently, the vast sphere of ecological influence makes roads a global concern \citep{laur14}. The field of `road ecology' was thus developed to study problems and solutions associated with the construction and use of roads across the natural landscape \citep{form03,rvdr15}. 

\begin{figure*}[!t]
  \centering
  \includegraphics[width=0.9\textwidth]{world_roads.png}
  \caption[Global distribution of known roads]{Global distribution of known roads. Actual distribution may be under-represented in some developing countries due to data deficiencies. Data used to create the map was sourced from the Center for International Earth Science Information Network (CIESIN).}
  \label{roads}
\end{figure*}

\subsection{Wildlife mortality on roads}

Human-wildlife conflicts occur where moving vehicles and animals co-occur in space and time, and this is facilitated by transportation networks. Private transport is increasingly common; people travel further and faster all the time, particularly in developing nations. This will continue to increase as developing nations become more affluent \citep{rvdr15}. What will also increase is wildlife mortality due to collisions with moving vehicles; commonly referred to as `roadkill'. It is estimated that billions of fauna expire annually due to collisions with transportation networks \citep{seil06}. Animals which are struck and expire on roadsides or railway corridors attract predators and may result in secondary collisions which further increase mortality rates \citep{spel98}. From a conservation perspective, these mortalities can result in population effects and disruption to food-chain networks \citep{ramp06b,cham10,pola14}.  Although collisions with wildlife bring about matters of animal welfare, biodiversity, and wildlife persistence, perhaps the largest concern is human safety. 

From an anthropocentric perspective, animal-vehicle collisions with larger species cause injuries, property damage, and occasional loss of human life which compels managers to find solutions. The visibility of the problem results in negative public perception and has caused some road authorities to regularly patrol problematic roads and collect carcasses; in some cases the data is recorded \citep{huij07a}. This has also initiated partnerships between academics and road authorities to conduct scientific research on animal-vehicle collisions.  These endeavours have created a wealth of research and literature on larger common species \citep{biss08b,clev02,romi96,sudh09}, but proportionally lower amounts on smaller (and perhaps more threatened) fauna \citep{clev03}.

\subsection{Factors influencing collision risk}

The factors influencing collision risk are quite broad and \cite{form03} groups these into four major categories: animal behaviour, human behaviour, the road environment, and the landscape environment. For example, road ecology theory suggests that increases in collisions may be due to the following: animal behaviours, such as migration, flight response, or feeding patterns, thereby increasing species presence on roads; the operation of motor vehicles by humans at high frequencies, high speeds, and at dangerous times; landscape features attracting animals to, and influencing movement across, roads and; road features hindering safe vehicle operation. In seventy-one studies that analysed wildlife-vehicle collision risk, the ten most common variables selected for use in predictive models were classified in three of the four categories proposed by road ecology theory (\Cref{wvc_studies}). See \Cref{apx:D} for all variables and studies used in the analysis. Traffic related variables (volume and speed) -- which may be considered human behaviour -- featured in the top four most commonly used predictor variables.

\begin{figure*}[!t]
  \centering
  \includegraphics[width=0.7\textwidth]{wvc_studies.png}
  \caption[Variables used in wildlife-vehicle collision studies]{Common variables used in wildlife-vehicle collision studies. The dots show the frequency each modelling variable was used in a total of seventy-one studies on wildlife-vehicle collisions.}
  \label{wvc_studies}
\end{figure*}

\subsection{Management of wildlife-vehicle collisions}

%Worldwide, many studies have been undertaken to better understand wildlife-vehicle collisions in an effort to supplement management.  Some studies related collisions to animal biology, behaviour and characteristics. In Australia, \cite{coul97} showed a male bias across several species of macropod roadkills, whilst \cite{kloc06} determined a female bias in red kangaroos.  \cite{lee04} concluded that kangaroo density near roads increased during periods of drought due to relative greenness of vegetation at road verges. \cite{ramp06b} used collision events in a population viability analysis model to evaluate effects on swamp wallabies.  In North America, animal travel speed was related to collisions \citep{litv08}. Research has also demonstrated that collision occurrence is related to many other combinations of environmental and anthropogenic variables \citep{barn07}.

Research has demonstrated that collision occurrence is related to many combinations of environmental and anthropogenic variables \citep{barn07}. Some of these factors are more easily managed than others and that presents an optimisation problem for managers; where to have the largest effect depends on knowing the relative influence of each factor on collision risk. In some cases the conflicts can be mitigated with engineering solutions (e.g. tunnels and land bridges \citep{bond08}, or rope ladders \citep{soan13}), but these are more suited to known species movements or behaviours. Where mobile wildlife are in contact diffusely over a road network throughout the year, at all hours, it is a more difficult problem to manage.  Mitigation strategies to reduce wildlife-vehicle collisions have varying costs and effectiveness \citep{huij09} and costs can be compounded by additional collisions occurring from ineffective mitigation. Therefore, an important objective in managing wildlife-vehicle collisions is knowing where, when, and how to intervene.

%may include (in order of increasing costs), road signage \citep{bond13}, discouragement and/or detection devices \citep{huij03,huij06}, fencing installations \citep{clev02,jaeg04}, road crossing structures (overpasses and underpasses) , or a combination of treatments \citep{dani98,pola14,romi96}. 

%One difficulty is the implementation and/or translation of scientific research and problem-solving into management actions \citep{roux06}.  The disconnects between environmental research, conservation policy and management/implementation are extensively referred to in the ecological literature.  Technology, most notably electronic media and computers, has been used to overcome some of these problems and functions as a tool to bridge the information divide.  Software applications to inform management decisions have been developed in areas of broad landscape and seascape conservation \citep{ball09,watt09}, species distributions \citep{phil06}, climate change \citep{east09}, water resources \citep{liu08}, and many others.  Moreover, the translation of scientific research into accessible public knowledge and education is also a formidable problem \citep{lubc98}.  This is an area that deserves more attention as human behaviour change can be a powerful upstream solution to complement collision management efforts.

\section{Improving management and decision-making}

\subsection{Incorporating risk theory}

Wildlife-vehicle collisions are a risk analysis problem. Risks are chances of adverse events, that have defined consequences, happening in specified time frames \citep{burg05}. Managing risks involves assessing potential hazards, exposure to hazards, and probabilities of hazards occurring, simulating outcomes under different scenarios, and making decisions based on acceptable thresholds. To help make decisions about risk, it is useful to employ quantitative methods either independently or in combination with qualitative methods. Quantitative analysis is used to understand, communicate, and manage uncertainty about risk in many fields such as engineering \citep{apos04}, biological sciences \citep{sute16}, and finance \citep{mcne15}.  All of these fields define risk a bit differently and the parameterisations used in their analyses vary accordingly. Among these disciplines, there are some important ideas about general risk theory that can be useful to help frame wildlife-vehicle collision risk at a conceptual level.

\subsection{Spatial statistics and modelling}

Many studies have utilised environmental modelling to analyse and predict risks imposed on species (fauna and flora), biotic systems, and human beings resulting from anthropocentric activity. The outputs of these models have many benefits including informing management operations, supplementing additional models, facilitating scientific research, and optimising mitigation strategies. The use of modelling in conservation efforts is well established \citep{star86}, however, predicting risks to wildlife by vehicles is a relatively new practice. The most common use of modelling has been to identify collision hotspots in geographic space in an effort to direct management efforts. For example, \cite{malo04} predicted collision risk of multiple species at two spatial scales using logistic regression. Similar methods have been used by \cite{roge09} to predict collision risk with wombats, and \cite{sudh09} for deer. Whilst the predictions from such models may be useful to managers, a comparative framework that describes simple relationships between variables, with tunable parts, and allows simulation of outcomes is more useful. \cite{clev15} alludes to this when calling for a general framework in road ecology to combine the disparate factors relating to wildlife-vehicle collisions.

When undertaking statistical modelling of events, such as collisions, it is common to analyse either binary data (occurred/did not occur) or count data (how many occurred).  As it is not plausible to have negative or partial events, these data are always constrained to be zero or positive integers. Bernoulli and Poisson distributions both satisfy these requirements and, thus, are often chosen for modelling collision events. A common statistical model used for collisions is based on generalised linear regression \citep{mccu89}; a robust and widely-used statistical method in which non-continuous (discrete) responses are regressed on a linear predictor using a link function. Spatial statistics is an extension of these methods and studies geographic properties of objects or events; in two-dimensional space, the choice of link function must be deliberate \citep{badd10}. Related to this concept, patterns of events can be analysed using point process models which consider events occurring across space with a single rate parameter. Modelling events on linear networks is a special case of one-dimensional point process analysis that assumes collisions occurring outside the network are impossible \citep{okab09} -- collisions will always occur coincident with the movement vectors of vehicles. One advantage of using Poisson point processes is that they assume collision events occur over a domain bound by time and are not influenced by the choice of sampling area or spatial scale. Yet, these sorts of models have limited use in road ecology studies.

\subsection{Important considerations}

The contemporary modeller has considerable access to publicly-available data. Anthropogenic data, such as demography, exist for most developed parts of the world. Environmental data, particularly satellite-based and remotely-sensed, have global coverage. Collecting data can be quite expensive, and this information provides low-cost sources of information and has the potential to significantly expand the scope of modelling wildlife-vehicle collisions. However, data has inherent uncertainties and biases \citep{glym97}; uncertainty may result from measurement error, under-reporting, and general environmental stochasticity whilst bias is often due to selective sampling (either intentional or unintentional). Considering these shortcomings, data should always include metadata that reports these deficiencies so modellers may adequately address them in analyses. Many online datasets, derived by scientific organisations and often post-processed, include metadata about uncertainties.

Citizen-science data can supplement data-deficient ecological studies, and whilst shown to be effective in road ecology \citep{paul14}, is rarely used in the discipline. The use of such data, however, should be subject to rigorous quality control and verification which may, or may not, be supplemented by adequate metadata. Records may not be useful due to incomplete/incorrect data or geographical ambiguity. Although under-reporting of data has limited effects on model robustness, spatial biases can adversely affect model performance \citep{snow15}. In particular, survey efforts are seldom reported in collision data and vary considerably amongst and within jurisdictions \citep{huij07a}.

It is important to identify and address spatial biases when developing predictive models. Models trained on observations that are spatially biased (e.g. collected at particular places or dependent on characteristics of reporting individuals) may cause model predictions to reflect patterns of collection rather than actual distributions of organisms or events. Methods have been developed to address bias in models fit to both presence-absence \citep{hijm12} and presence-only data \citep{kram13}. For both types, implicit biases in data may be incorporated into modelling if they are adequately described in metadata \citep{wart13}. For example, many of the records in fauna atlases are subject to geographical bias (e.g. close to roads and towns), and this is also prevalent in collision or carcass data.
%\subsection{Temporal variation in data}

Temporal variation occurs in both natural and anthropogenic systems. Wildlife species have different patterns of activities (e.g. diurnal, nocturnal, crepuscular) that make them more or less susceptible to wildlife-vehicle collisions during the day. Seasonal patterns are particularly important for migratory species \citep{hick85,bern92}, where seasonal variation has a large influence on collision risk \citep{shep08}. These patterns are more easily incorporated into modelling as there is less uncertainty about observations -- the temporal resolution is large and thus discrepancies in time between a collision event and carcass discovery are negligible. In contrast, daily patterns in collision data are subject to measurement errors and require more careful scrutiny, however, are highly informative in determining when collision risks are high. To address some of this uncertainty, studies may represent temporal data as a categorical variable \citep{mizu14} with discrete ranges \citep{rowd08}. As a continuous variable, hour may be represented by cyclic functions in predictive models \citep{neum12,thur15}. When predictions of collision risk are made using temporal information, they are referred to as `hot moments'.

\subsection{Reducing wildlife-vehicle collisions}

It is clear that wildlife-vehicle collisions are a complex problem that has global implications. This problem will continue into the future as more developing nations expand their linear transportation networks. Effective management will require knowing where, when, and how to mitigate collision risks, but this will require a clear understanding of the factors, and their interactions, relating to wildlife-vehicle collisions. Previous work on modelling and predicting collision risk have been contextually useful but their generality or relationship to management objectives were not explored. There is a critical need for research that develops a conceptual risk framework for modelling and predicting wildlife-vehicle collisions. This framework should be flexible to different parameterisations, informative to management objectives, generalisable to different spatial scales and transportation networks, and responsive to temporal information. These features will support global efforts to reduce wildlife-vehicle collisions.

\section{Thesis aims and scope}

The broad aim of this thesis is to analyse and predict where and when wildlife will be at high risk of collisions with moving vehicles, across large transportation networks. As I intend the work to supplement managers working in the wildlife and transportation fields, I have developed analyses that are generalisable, transferable, reproducible, and based on open-source data and methods. Therefore, each chapter tests different model applications and demonstrates management value. \Cref{sec:egk} introduces a general conceptual framework to model wildlife-vehicle collision risk using kangaroos in south-east Australia as a case study. Using the same geographic area, \Cref{sec:6sp} extends the work of \Cref{sec:egk} by applying the model framework to six mammal species commonly involved in wildlife-vehicle collisions. In \Cref{sec:cal}, the conceptual framework is applied to a different geographic area (California, USA) and a different species (Mule deer). To analyse risk on a different transportation network, I extend the conceptual model to examine wildlife-train collisions in both space and time in \Cref{sec:train}. To assess the performance of the model introduced in \Cref{sec:egk}, and demonstrate the importance of unified and standardised data repositories of wildlife-vehicle collisions, model predictions are compared to additional datasets collected by various organisations in \Cref{sec:val}. \Cref{sec:conc} summarises a selection of key findings within the chapters whilst suggesting novel areas for further research and potential applications of the work.