\chapter{General Introduction}\label{sec:intro}
\newpage

\section{Human-wildlife interactions}

We are now in what some refer to as the \textit{Anthropocene}, where \textit{Homo sapiens} exert considerable influence on climatic, geologic and biotic systems \citep{crut06}. Recent studies have estimated that humans have transformed more than 75\% of the earth's ice-free terrestrial surface \citep{elli08}. These modifications result from activities such as intensified agricultural production, provision of linear infrastructure (e.g. roads, rail, and utility easements), and urban development \citep{vito97,sand02,fole05}.

In addition to generating other ecological effects, as humans modify and distribute across landscapes, they decrease their proximity, and increase exposure, to other biotic organisms \citep{}. For stationary configurations of human settlement, increased exposure is due to tolerant species remaining or re-establishing in human-dominated landscapes \citep{soul16}. Exposure is compounded by human mobility via transportation which extends interactions to biotic organisms outside areas of human settlement.

This close proximity results in what is generally referred to in the literature as 'human-wildlife interactions'. Although the suite of organisms that humans have contact with is quite varied, medium to large species ($>$10kg) tend to get more attention when conflicts arise \citep{seor16}. This is partly because larger species are more likely to cause - or perceived to cause - significant harm to person or damage to property. Further, larger species are less easily managed by individuals that lack proper training \citep{}. For example, rodent incursions into households do not draw the same attention as bear incursions. From the other perspective, large charismatic species are often persecuted when residing in close proximity to human settlements and therefore targets of conservation programs \citep{trev03}.

Managing human-wildlife interactions is often considered the work of environmentally-focussed disciplines, however, issues arising from human-wildlife interactions are complex and may require knowledge from several non-environmental fields \citep{}. For example, specialist knowledge on both animal behaviour (zoology) and human behaviour (sociology) may be used to arrive at effective solutions \citep{dick10}. Due to the broad reach of this topic, entire books and journals have been designated specifically to human-wildlife interactions (e.g. \cite{manf08}, \cite{}).

\subsection{Impacts of wildlife on humans}

When human-wildlife interactions are negative, they are referred to as 'wildlife-human conflicts' \citep{}.  Wildlife impacts on humans can have multiple be positive, negative, or neutral...

\subsection{Effects of the built environment on wildlife}

Impacts of the built environment on wildlife can be positive, negative, or neutral...

\subsection{Impacts of linear infrastructure on wildlife}

Linear infrastructure is most commonly thought of as roads, however, can also include railways, utility easements, and other continuous human-engineered resources in the landscape. 

Impacts of linear infrastructure on wildlife can be positive, negative, and neutral...(road reserves led to retention of habitat - no roads or low traffic - positive; Roads pose several ecological threats including destruction of habitat through fragmentation and pollution of various sorts (light, chemical, acoustic)).

\section{Road ecology}

The field of road ecology addresses a wide range of environmental problems resulting from the construction and use of roads across the natural landscape \citep{form03,rvdr15}. Road networks contribute to fragmentation, modification and loss of habitat, generate environmental pollution and facilitate human disturbance to natural landscapes (e.g. clearing, logging, mining, hunting). There are few places left in the terrestrial environment that are not accessible by some form of road and, consequently, the sphere of ecological influence is vast making roads a global concern \citep{}.

\subsection{Wildlife mortality on roads}

Perhaps the most notable (and visible) issue is wildlife mortality due to collisions with moving vehicles; often referred to as 'roadkill' in the common vernacular.  It is estimated that billions of fauna expire annually due to collisions with transportation networks \citep{seil06}.  Animals which are struck and expire on roadsides or railway corridors attract predators and may result in secondary collisions which further increase mortality rates \citep{spel98}.  From a conservation perspective, these mortalities can result in population effects and disruption to food-chain networks.  Although collisions with wildlife bring about matters of animal welfare, biodiversity, and wildlife persistence, perhaps the largest concern is human safety. 

From an anthropocentric perspective, animal-vehicle collisions with larger species cause injuries, property damage, and occasional loss of human life which compels managers to find solutions. The visibility of the problem results in negative public perception and has caused many road authorities to add carcass collection to road maintenance operations; in some cases the data is recorded \citep{huij07a}. This has also initiated a multitude of funding opportunities and partnerships between academics and road authorities to conduct scientific research on animal-vehicle collisions.  These endeavours have created a wealth of research and literature on larger common species \citep{biss08,clev01,romi96,sudh09}, but proportionally lower amounts on smaller (and perhaps more threatened) fauna \citep{clev03}.

\subsection{Factors influencing collision risk}

Worldwide, many studies have been undertaken to understand wildlife-vehicle collisions.  Some studies related collisions to animal biology, behaviour and characteristics.  In Australia, \cite{coul97} showed a male bias across several species of macropod roadkills, while \cite{kloc06} determined a female bias in red kangaroos.  \cite{lee04} concluded that kangaroo density near roads increased during periods of drought due to relative greenness of vegetation at road verges. \cite{ramp06} used collision events in a population viability analysis model to evaluate effects on swamp wallabies.  In North America, animal travel speed was related to collisions \citep{litv08}.

Research has also related both environmental and anthropogenic variables to collision occurrence \citep{barn07,lee04,ramp05}.

\cite{form03} groups factors into four major categories: animal behaviour, human behavior, the road environment and the landscape environment.

(add literature review and predictors)

\subsection{Management approaches to wildlife-vehicle collisions}

It is in the best interest of authorities to manage wildlife-vehicle collisions for numerous economic, social and environmental reasons...

Studies have assessed the effectiveness of mitigation strategies including fencing installations \citep{clev01,jaeg04}, road crossing structures (overpasses and underpasses) \citep{bond08}, discouragement and/or detection devices \citep{huij03,huij06}, road signage \citep{bond13}, or a combination of treatments \citep{dani98,pola14,romi96}.

Despite all of the research, management of animal-vehicle collisions still remains a daunting task and shares similar issues as broader ecological management.  One difficulty is the implementation and/or translation of scientific research and problem-solving into management actions \citep{roux06}.  The disconnects between environmental research, conservation policy and management/implementation are extensively referred to in the ecological literature.  Technology, most notably electronic media and computers, has been used to overcome some of these problems and functions as a tool to bridge the information divide.  Software applications to inform management decisions have been developed in areas of broad landscape and seascape conservation \citep{ball09,watt09}, species distributions \citep{phil06}, climate change \citep{east09}, water resources \citep{liu08}, and many others.  Moreover, the translation of scientific research into accessible public knowledge and education is also a formidable problem \citep{lubc98}.  This is an area that deserves more attention as human behaviour change can be a powerful upstream solution to complement collision management efforts.

\section{Risk management and decision-making}

Risk management seeks to reduce the chances of adverse outcomes...

To help make decisions about risk, is is useful to employ quantitative methods...

\subsection{Distribution of risk in space and time}

For many systems, risks may vary based on location and time...

\subsection{Spatial statistics and modelling}

More contemporary studies have utilised environmental modelling to analyse and predict risks imposed on species (fauna and flora), biotic systems, and human beings resulting from anthropocentric activity. It can be argued that the outputs of these models have a multitude of benefits including informing management operations, supplementing additional models, facilitating scientific research, and optimising mitigation strategies.  The use of modelling in conservation efforts is well established \citep{star86}, however, predicting risks to wildlife by vehicle networks is still relatively new.   Computer-based statistical modelling has been used to describe and predict wildlife-vehicle collisions for the past two decades \citep{guns11}.  The most common use of modelling has been to identify collision hotspots in geographic space in an effort to direct management efforts.  For example, \cite{malo04} predicted collision risk of multiple species at two spatial scales using logistic regression.  Similar methods have been used by \cite{roge09} to predict collision risk with wombats, and \cite{sudh09} for deer.

When statistical modelling of events is undertaken, it is common to analyse either binary data (occurred/did not occur) or count data (how many occurred).  As it is not plausible to have negative events, these data are always constrained to be positive.  Moreover, unless one is interested in rates, such as number of events over a domain (often in spatial or temporal units), data for events consists of positive integers.  Two distributions satisfy these requirements, Bernoulli and Poisson, and thus, are often chosen for modelling collision events...

Spatial statistical modelling is subject to several issues that must be considered when completing analyses.  These considerations have origins in statistical, physical and ecological theory...

Modelling events on linear networks is a special case of one-dimensional analysis as it would be impossible for collisions to occur outside the network.  More specifically, collisions will always occur coincident with the movement vector of a vehicle...

\subsubsection{Bias and uncertainty in data}

Data has inherent uncertainties and biases...

\subsubsection{Temporal variation in data}

Patterns in both natural and anthropogenic systems are influenced by time...

\section{Aim and scope of the thesis}

The broad aim of this thesis is to analyse where and when wildlife will be at the high risk, due to collisions with moving vehicles, across large transportation networks. As I intend the work to supplement managers working in the wildlife and transportation fields, I have developed methods that are generalisable, transferable, reproducible, and based on open-source data. Therefore, each chapter tests variations on model application and demonstrates management value. \Cref{sec:egk} introduces a general conceptual framework to model wildlife-vehicle collision risk and uses kangaroos in south-east Australia as a case study. Using the same geographic area, \cref{sec:6sp} extends the work of \cref{sec:egk} by applying the model framework to six mammal species commonly involved in wildlife-vehicle collisions. In \cref{sec:cal}, the conceptual framework is applied to a different geographic area (California, USA) and a different species (mule deer). To analyse risk on a different transportation network, I adapt the model to examine wildlife-train collisions in both space and time in \cref{sec:train}. To assess the performance of the model introduced in \cref{sec:egk}, and demonstrate the importance of unified and standardised data collections, model predictions are compared to additional datasets collected by various organisations in \cref{sec:val}. Each chapter is written as an independent piece of work and, therefore, includes more detailed introductions and discussions. As such, there will be some unavoidable repetition in the chapters.