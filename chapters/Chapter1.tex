\chapter{General Introduction}\label{sec:intro}
\newpage

\section{The wildlife and roads conundrum}

\subsection{Human-wildlife interactions}

We are now in what some refer to as the \textit{Anthropocene}, where \textit{Homo sapiens} exert strong influences on climatic, geologic and biotic systems \citep{crut06}. Studies estimate that humans have transformed more than seventy-five percent of the earth's ice-free terrestrial surface \citep[e.g.][]{elli08}. These modifications result from large-scale activities; perhaps the most influential being intensified agricultural production, provision of linear infrastructure (e.g. roads, rail, and utility easements), and urban development \citep{vito97,sand02,fole05}. The human population is forecast to grow in the future, intensifying existing activities in modified landscapes and increasing pressure on unmodified landscapes. For example, recent projections indicate a population between 9.6 and 12.3 billion in the year 2100 \citep{gerl14}. The largest rates of increase are projected for developing countries and these will have environmental consequences such as land transformation. Further, these issues are compounded by the high amount of biodiversity and low amount of conservation resources in developing countries. 

Human settlements, industry, and related infrastructure spread over the Earth's surface reducing the physical separation between people and wildlife. This results in what is termed `human-wildlife interactions' \citep[see][]{manf08}. In this field of research, all manners of interactions are considered; positive \citep[e.g. psychological well-being, see][]{dall12}, negative \citep[e.g. swooping birds, see][]{jone99} and neutral. Herein, I define wildlife as any non-domesticated organism in the biological kingdom of \textit{Animalia}. 

Human-wildlife interactions are multifaceted because the same organism may be associated with positive interactions for one person or part of society, and negative interactions for another. For example, some people may appreciate the existence of pigeons in urban environments whilst others may view the birds as a nuisance or invasive species. Cultural values have a large influence on these views; some groups celebrate wildlife species that are believed to have spiritual kinship (e.g. \textit{totemism}) or persecute wildlife due to religious beliefs (e.g. killing of Aye-ayes on sight in Madagascar). Further, the human dimension of wildlife interactions may involve humans in the capacity of actors, interacting directly with wildlife, or observers who witness how others do or have interacted with wildlife. In wildlife-vehicle collisions, for example, people are both actors as vehicle occupants and, subsequently, observers as motorists passing road-killed wildlife.

Interactions can be frequent or infrequent, fleeting or long-run, and have large or small impact either individually or in aggregate. These impacts can be considered from the perspective of humans, wildlife, or both. The outcomes or consequences for wildlife can be tied to vital rate parameters, such as mating and breeding success, birth and death rates, and be applicable to individual animals, populations, or even evolutionary significant units or species. Positive outcomes of interactions with wildlife for humans are diverse and can be grouped as consumptive and non-consumptive \citep{char02}. Negative consequences for humans include nuisance, damage to property, or risk to health or life and, like positive outcomes, can be contemplated at the level of individual actors/observers, or the level of society. These impacts can be conceptually diagrammed as a product of the frequency, or duration, and consequence of the interaction (\Cref{hmi_diagram}). Although many interactions usually have opposite effects on humans and wildlife (i.e. good for humans, bad for wildlife, and vice versa), some human-wildlife interactions may have benefits (i.e. `mutualism') or negative outcomes (e.g. wildlife-vehicle collisions) for both humans and wildlife. Each situation requires unique management strategies; for instance, the severity and nature of elephant raids on crops will warrant different mitigation than coyote attacks on domesticated pets. 

\begin{figure*}[!t]
  \centering
  \includegraphics[width=0.8\textwidth]{hwi_diagram2.png}
  \caption[Conceptual diagram for framing human-wildlife interactions]{Conceptual diagram for framing human-wildlife interactions from both a human perspective (upper panels) and wildlife perspective (lower panels). The consequence (perceived or actual) of an interaction can range from positive (blue shading in right-hand panels) to negative (red shading in left-hand panels), or be all negative or all positive.  For example, bird feeding by residents in backyards (\includegraphics[height=0.3cm]{icon_bf.png}) is all positive (ignoring possible spread of disease for illustration purposes) and a wildlife-vehicle collision (\includegraphics[height=0.3cm]{icon_wvc.png}) is only negative (hence only a single representation of those icons in each panel). Crop-raiding by elephants (\includegraphics[height=0.3cm]{icon_ec.png}) has only negative consequences for humans but the consequences for elephants may be negative due to dispersal or control measures to prevent crop raiding or positive if the elephant successfully obtains food. Impacts are also amplified by the frequency or duration of an interaction. Rarer events with higher negative consequences may be equivalent to frequent events with lower negative consequences such as raccoons feeding from rubbish bins (\includegraphics[height=0.3cm]{icon_rb.png}). The concentric zones and numbers are based on a relative Likert-scale and may be used to sum interaction scores between the two perspectives to determine priorities for mitigation.}
  \label{hmi_diagram}
\end{figure*}

In urban areas, exposure to wildlife is due to tolerant species remaining or re-establishing in human-dominated landscapes \citep{soul16}. Although the suite of organisms that humans have contact with is quite varied -- the most frequent contact possibly being with invertebrates -- medium to large species (i.e., $>$10 kg) tend to get more attention, especially when issues arise \citep{seor16}. Certain kinds of interactions, usually where wildlife and humans compete for resources, or intersect in a way that causes nuisance or harm to human activities are referred to as `human-wildlife conflicts' \citep[but see][]{pete10}. Studies have demonstrated notable impacts on society (e.g. human injuries, illness, and economic losses) from wildlife \citep[e.g.][]{cono95}. Larger species are more likely to cause significant harm to person or damage to property \citep{cono01} and are less easily managed by individuals that lack proper training. Therefore, the public expects government intervention to manage human-wildlife conflicts \citep{reit99}. Further, the general public often views the management of human-wildlife conflicts to reside in the scope of wildlife-focussed disciplines. However, issues arising from human-wildlife interactions are complex and may require knowledge from several non-ecological fields \citep{deck97,madd04}. For example, specialist knowledge on both animal behaviour (zoology) and human behaviour (sociology) may be combined to arrive at effective solutions \citep{dicka10}.

\subsection{Impacts of linear infrastructure on wildlife}

Linear infrastructure can traverse vast areas between human settlements, and penetrate remaining natural and semi-natural areas. In some cases, these easements may be the only remaining semi-natural refuges for wildlife \citep{benn91}. Linear infrastructure is most commonly thought of as roads, however, can also include railways, utility easements, and other continuous human-engineered resources in the landscape.

Impacts of linear infrastructure on wildlife can be positive, neutral, or negative. For example, disused road reserves lead to retention of habitat that may benefit wildlife as they often contain old and diverse remnant vegetation \citep{lent11}. Further, linear infrastructure without moving vehicles may provide connection corridors and assist migration for some species \citep{rvdr15}. Effects may be neutral when suitable habitat is provided on road reserves with very little traffic volume or occupied by species with low mobility. Nonetheless, negative effects of linear infrastructure on wildlife persistence may certainly outweigh the benefits.

There are few places left on Earth that are not accessible by some form of road (\Cref{roads}) and, consequently, the vast sphere of ecological influence makes roads a global concern \citep{laur14}. The field of `road ecology' evolved to study problems and solutions associated with the construction and use of roads across the natural landscape \citep[see][]{form03}. It is well documented that roads have negative impacts on wildlife \citep{form98,spel98,rvdr15}, both indirectly and directly. The construction and use of road networks contributes to fragmentation, modification and loss of habitat, generates environmental pollution and facilitates human disturbance to natural landscapes (e.g. clearing, logging, mining, hunting). Roads also have direct impacts on wildlife populations through mortality resulting from wildlife-vehicle collisions \citep{fahr09}.

\begin{figure*}[!t]
  \centering
  \includegraphics[width=\textwidth]{world_roads.png}
  \caption[Global distribution of known roads]{Global distribution of known roads. Actual distribution may be under-represented in some developing countries due to data deficiencies. Data used to create the map was sourced from the Center for International Earth Science Information Network (www.ciesin.org); accessed 3 February, 2017.}
  \label{roads}
\end{figure*}

\subsection{Wildlife mortality on roads}

Human-wildlife conflicts occur where moving vehicles and animals co-occur in space and time, and this is facilitated by transportation networks. Private transport is increasingly common; people travel further and faster. Growth of roads is particularly large in developing nations, which will increase as developing nations become more affluent \citep{rvdr15}. The rate of mortality of wildlife will also increase due to collisions with moving vehicles. It is estimated that billions of vertebrate fauna die annually due to collisions with vehicles on transportation networks \citep{seil06}. Animals that are struck and die on roadsides or railway corridors attract predators and may result in secondary collisions which further increase mortality rates \citep{spel98}. From a conservation perspective, these mortalities can result in population effects and disruption to food-chain networks \citep{ramp06b,cham10,pola14}.  Although collisions with wildlife bring about matters of animal welfare, biodiversity, and wildlife persistence, perhaps the largest concern to road managers is human safety. 

From an anthropocentric perspective, wildlife-vehicle collisions with larger species cause injuries, property damage, and occasional loss of human life which compels road and wildlife managers to find solutions. Both the visibility of the problem -- resulting in negative public reactions -- and potential threats to human safety have caused some road authorities to regularly patrol problematic roads and collect carcasses. In some cases the data are recorded \citep{huij07a}. This has also initiated partnerships between academics and road authorities to conduct scientific research on animal-vehicle collisions.  These endeavours have created a wealth of research and literature on larger common species \citep[e.g.][]{biss08b,clev02,romi96,sudh09}, but proportionally lower amounts on smaller (and perhaps more threatened) vertebrate fauna \citep[e.g.][]{clev03}.

\section{Improving management and decision-making}

\subsection{Understanding wildlife-vehicle collisions}

The factors influencing collision risk are quite broad and can be grouped into four major categories: animal behaviour, human behaviour, the road environment, and the landscape environment \citep{form03}. For example, increases in collisions may be due to the following: animal behaviours, such as migration, flight response, or feeding patterns, thereby increasing species presence on roads; the operation of motor vehicles by humans at high volumes, high speeds, and at dangerous times; landscape features attracting animals to, and influencing movement across, roads and; road features hindering safe vehicle operation. In seventy-one studies that analysed wildlife-vehicle collision risk, traffic related variables (volume and speed) -- which may be considered human behaviour -- featured in the top four most commonly used predictor variables (\Cref{wvc_studies}). See \Cref{apx:D} for all variables and studies used in the analysis.

\begin{figure*}[!t]
  \centering
  \includegraphics[width=0.7\textwidth]{wvc_studies.png}
  \caption[Variables used in wildlife-vehicle collision studies]{Common variables used in wildlife-vehicle collision studies. The dots show the frequency each modelling variable was used in seventy-one studies (\Cref{apx:D}) on wildlife-vehicle collisions.}
  \label{wvc_studies}
\end{figure*}

Collision occurrence is related to many combinations of environmental and anthropogenic variables \citep{barn07}. Some of these factors are more easily managed than others and that presents an optimisation problem for managers; where to have the largest effect for a given expenditure depends on knowing the relative influence of each factor on collision risk. In some cases the conflicts can be mitigated with engineering solutions, such as tunnels and land bridges \citep{bond08}, or rope ladders \citep{soan13}, but these are more suited to known species movements or behaviours. Where mobile wildlife are in contact diffusely over a road network it is a more difficult problem to manage. In these cases, mitigation can either focus on excluding or discouraging wildlife from roads or changing human behaviour.

Mitigation strategies to reduce wildlife-vehicle collisions have varying costs and effectiveness \citep{huij09} and costs can be compounded by additional collisions occurring from ineffective mitigation. Fencing is very effective at excluding animals from roads and reducing WVC, however, can result in other negative impacts to wildlife such as limitation of dispersal \citep[i.e. barrier effects, see][]{jaeg04}. Fences are often used in conjunction with crossing structures to facilitate connectivity whilst reducing collisions but these installations can be expensive. Discouragement devices to scare animals off of roads are much less expensive but have marginal effectiveness \citep{bend03,sche03}. Influencing human behaviour is perhaps the least expensive form of mitigation yet difficult to implement effectively. Road signage is the most common form, however, is predominantly ineffective as humans become easily conditioned to warning messages \citep{bond13}. For costly mitigation, it is therefore important to know exactly where to mitigate along transportation networks.

In addition to knowing \textit{how} and \textit{where} to mitigate, an important objective in managing wildlife-vehicle collisions is also knowing \textit{when} to mitigate. This applies to non-permanent forms of mitigation such as changes in vehicular movements (e.g. frequency and speed), seasonal road closures \citep[see][]{hilt12}, or warning systems \citep[but see][for review of automated signs]{huij03}. Temporal variation occurs in both natural and anthropogenic systems. Wildlife species have different patterns of activities (e.g. diurnal, nocturnal, crepuscular) that make them more or less susceptible to wildlife-vehicle collisions during the day. When predictions of collision risk are made using temporal information, they are referred to as `hot moments'. In the few studies that represent temporal variation at the hourly scale \citep[e.g.][]{neum12}, the models do not consider changing patterns of daylight throughout the year -- which often influences wildlife activity. For example, using hour as a predictor will have considerable error when species do not have consistent daily patterns of movement throughout the year. 

\subsection{Incorporating risk theory \& modelling}

Preventing wildlife-vehicle collision may be viewed as a risk analysis problem. Risks are chances of adverse events, that have defined consequences, happening in specified time frames \citep{burg05}. Managing risks involves a) assessing hazards (i.e. potentially harmful situations), exposure to hazards, and probabilities of hazards occurring, b) simulating outcomes under different scenarios, and c) making decisions based on acceptable thresholds. To assess risk, it is useful to employ quantitative methods either independently or in combination with qualitative methods. Quantitative analysis is used to understand, communicate, and manage uncertainty about risk in many fields such as engineering \citep[e.g.][]{apos04}, biological sciences \citep[e.g.][]{sute16}, and finance \citep[e.g.][]{mcne15}.  All of these fields define risk differently and the parameterisations used in their analyses vary accordingly.

Many studies have utilised environmental modelling to analyse and predict risks to species (fauna and flora) and biotic systems resulting from human activity. The outputs of these models have many benefits including informing management operations, supplementing additional models, facilitating scientific research, and optimising mitigation strategies. The use of modelling in conservation efforts is well established \citep{star86}, however, predicting risks to wildlife by vehicles is a relatively new practice. The most common use of modelling has been to identify collision `hotspots' in geographic space in an effort to direct management efforts. For example, \cite{malo04} predicted collision risk along a 4-lane highway in Spain for multiple species at two spatial scales using logistic regression. Similar methods have been used to predict collision risk with wombats \citep{roge09}, and deer \citep{sudh09}.

Whilst the predictions from such models may be useful to managers, a comparative framework that describes simple relationships between variables, with tunable parts, and allows simulation of outcomes is more useful. First, it can identify simple predictors for collision risk that are able to be mitigated, such as traffic volume. This also allows sensitivity analyses on proposed management scenarios to be conducted (i.e. varying predictor values and analysing resulting changes in collision occurrence). Further, although management implications are highlighted in many studies, it is not always clear how the results of the analyses could be easily applied due to unknown interactions between variables and confounding effects in the model predictors \citep{guns11}. Some studies explicitly suggest mitigation that relates directly to environmental variables based on model results \citep[e.g.][]{gril09}.  Although these recommendations are useful in the specific contexts of the studies, they may not generalise to varying spatial scales -- specifically large areas. In response to these shortcomings, road ecology needs a general framework to combine the disparate factors relating to wildlife-vehicle collisions \citep{clev15} and better support management objectives, globally.

\subsection{The challenges of disparate data}

The contemporary modeller has considerable access to publicly-available data. Anthropogenic data, such as demography, exist for most developed parts of the world. Environmental data, particularly satellite-based and remotely-sensed, have global coverage. Collecting data can be quite expensive, and this information provides low-cost sources of information and has the potential to significantly expand the scope of modelling wildlife-vehicle collisions. Further, citizen-science data can supplement data-deficient studies, although the use of this data in collision modelling is quite low relative to the amount in existence \citep[but see][]{paul14}. Worldwide, collision data is collected by non-government organisations, private-sector entities (e.g. insurance companies), academic institutions, government agencies (e.g. road authorities and environmental managers), community groups, and individual citizen scientists but these data are not always easily attainable due to privacy restrictions or data formats. 

Consequently, many studies that use predictive modelling to determine where and when WVC will occur are limited to single sources of data -- either field collected based on predetermined project requirements \citep[e.g.][]{lang09,roge09} or pre-existing from independent surveys or routine collections \citep[e.g.][]{hoth12,malo04}. Bias and error may arise from the data sources and collection methods, however, these are not often explicitly reported or accounted for in the modelling methods. Further, few studies utilise independently-obtained data to validate model performance. Although cataloguing and reporting systems are used to store data on WVC, few are open access or incorporate data from other institutions or previous surveying events \citep[but see][]{shil15b}. As disparate sources of collision data are combined, management will require an analytic modelling tool that is flexible to these data and can explicitly handle associated biases and errors.

\section{Thesis aims and scope}

It is clear that wildlife-vehicle collisions are a complex problem that have global implications. This problem will continue into the future as more developing nations expand their linear transportation networks thereby bringing humans and wildlife into closer proximity. Effective management will require knowing where, when, and how to mitigate collision risks, but this will require a clear understanding of the factors, and their interactions, relating to wildlife-vehicle collisions. Previous work on modelling and predicting collision risk have been contextually useful but their generality or relationship to management objectives were not explored. There is a critical need for a conceptual risk framework for modelling and predicting wildlife-vehicle collisions that is flexible to different parameterisations, informative for management objectives, generalisable to different spatial scales and transportation networks, and responsive to temporal information. These features will support global efforts to reduce wildlife-vehicle collisions.

My aim is to introduce a conceptual model framework to analyse and predict where and when wildlife will be at high risk of collisions with moving vehicles across large transportation networks, irrespective of species or geographic area. As I intend the work to support managers working in the wildlife and transportation fields, I develop analyses that are generalisable, transferable, reproducible, and based on open-source data and methods. In each chapter, I test different model applications and demonstrate management value. I introduce a general conceptual framework to model wildlife-vehicle collision risk using tunable sub-models and publicly-accessible data in \Cref{sec:egk}. Using the same geographic area, in \Cref{sec:6sp} I test the ability of the conceptual model framework to generalise to six mammal species commonly involved in wildlife-vehicle collisions. In \Cref{sec:cal}, I apply the conceptual framework to a different geographic area (California, USA) and a different species (Mule deer) to test its flexibility to varying locations and spatial scales. In addition to road vehicles, trains also hit wildlife but collision risk on railways is seldom studied. I test the conceptual model's ability to generalise to other transportation networks and analyse and predict wildlife-train collisions in both space and time in \Cref{sec:train}. To assess the performance of the model introduced in \Cref{sec:egk}, and demonstrate the importance of unified and standardised data repositories of wildlife-vehicle collisions, I compare model predictions to additional datasets collected by various organisations in \Cref{sec:val}. In \Cref{sec:conc}, I summarise a selection of key findings within the chapters whilst suggesting novel areas for further research and potential applications of the work.