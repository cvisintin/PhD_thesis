\chapter{General Introduction}\label{sec:intro}
\newpage

-People have spread themselves all over the world

-This means we keep running into wildlife.

-This is called a human-wildlfe interaction. It’s normally big animals and normally a conflict.

-A good example of this type of conflict is roadkill. Roads are everywhere (blah blah) and spreading.

-Collisions with large animals big problems in lots of the world. This is a really important problem. (all that stuff you wrote...)

-We have a few things that we try and do to mitigate and they often work well, but it’s hard to know where to put them. We really need to understand this properly, but to do that we need to understand risks better.

-Subheading: How can we solve the problem above?

-People do a lot of work to try and predict where roadkill occurs, but it’s got a few important gaps (all your stuff about 1D 2D aspects).

-Well, first we need to understand a little bit about risk and uncertainty (your current sections)

-To do this, we need a lot of data. It costs a lot to go and collect it all ourselves - too much! So another good option is to dig into publicly available data.

-There are a few issues with using this kind of data, but by being super clever about it, and considering these biases, we can draw together information from multiple sources that allows us to really try and explain the problem better than other people have in the past.

-Where the magic happens

-Subheading: Aims and scope

-Aims

\section{The wildlife and road conundrum}

\subsection{Human-wildlife interactions}

We are now in what some refer to as the \textit{Anthropocene}, where \textit{Homo sapiens} exert considerable influence on climatic, geologic and biotic systems \citep{crut06}. Recent studies have estimated that humans have transformed more than 75\% of the earth's ice-free terrestrial surface \citep{elli08}. These modifications result from activities such as intensified agricultural production, provision of linear infrastructure (e.g. roads, rail, and utility easements), and urban development \citep{vito97,sand02,fole05}.

In addition to generating other ecological effects, as humans modify and distribute across landscapes, they decrease their proximity, and increase exposure, to other biotic organisms. For stationary configurations of human settlement, increased exposure is due to tolerant species remaining or re-establishing in human-dominated landscapes \citep{soul16}. This close proximity results in what is generally referred to in the literature as 'human-wildlife interactions'. Although the suite of organisms that humans have contact with is quite varied -- the most frequent contact possibly being with invertebrates -- medium to large species (i.e., $>$10kg) tend to get more attention when conflicts arise \citep{seor16}. This is partly because larger species are more likely to cause -- or perceived to cause -- significant harm to person or damage to property \citep{cono01}. Further, larger species are less easily managed by individuals that lack proper training and the public holds an expectation of government intervention \citep{reit99}. For example, rodent incursions into households do not draw the same attention as bear incursions. From the other perspective, large charismatic species are often persecuted when residing in close proximity to human settlements and therefore targets of conservation programs \citep{trev03}. The proximity between humans and wildlife will inevitably reduce as human population continues to grow. In response, books \citep{manf08} and dedicated journals (Human-Wildlife Interactions - ISSN 2155-3858) specifically address the topic.

Public perception may view the management of human-wildlife interactions to reside in the scope of environmentally-focussed disciplines, however, issues arising from human-wildlife interactions are complex and may require knowledge from several non-environmental fields \citep{deck97,madd04}. For example, specialist knowledge on both animal behaviour (zoology) and human behaviour (sociology) may be used to arrive at effective solutions \citep{dicka10}. When human-wildlife interactions result in negative outcomes, they are referred to as 'wildlife-human conflicts'. This term, however, may be misleading as most studies using the terminology actually refer to conflicts amongst humans in regards to wildlife \citep{pete10}. Nevertheless, studies have demonstrated notable impacts on society (e.g. human injuries, illness, and economic losses) from wildlife \citep{cono95}. 

%\subsection{Effects of wildlife on humans}
%
%\begin{figure*}[htp]
%  \centering
%  \includegraphics[width=0.8\textwidth]{microbats.jpg}
%  \caption[Public involvement with wildlife]{Public involvement in scientific research to learn about microbats in the Royal Botanic Gardens, Victoria, Australia. Photograph by Meg Bauer.}
%  \label{microbats}
%\end{figure*}
%
%Human beings generally appreciate the existence of wildlife within and around their places of settlement \citep{dick86,kell93,news05,raad07}. Bird watching is a popular recreational activity practised worldwide and many communities rely of wildlife tourism as an economic resource. Further, citizens are excited to learn about some of the more cryptic species that inhabit their immediate environment (Visintin, personal observation - see \Cref{microbats}). In urban settings, human-wildlife interactions may be considered benign due to human's consideration of species that surround them throughout their everyday lives as fellow-creatures \citep{zinn08}.

%In many cases, wildlife aggression towards humans is due to protection of offspring; for example, swooping Australian magpies \citep{jone99} as shown in \Cref{magpie}.

%\begin{figure*}[htp]
%  \centering
%  \includegraphics[width=0.8\textwidth]{magpie.jpg}
%  \caption[Wildlife aggression towards humans]{Australian magpie swooping humans during breeding season. Photograph by Adam McLean.}
%  \label{magpie}
%\end{figure*}

%\subsection{Effects of the built environment on wildlife}
%
%In contrast to the natural environment, the built environment refers to human-manufactured structures, infrastructure, and recreational landscapes (e.g. urban parks) where people reside and conduct daily activities.  The built environment has mixed effects on wildlife persistence. Some species thrive in urban landscapes (termed 'urban adapters') whilst others perish \citep{shoc06}. 'Urban adapter' species benefit from increased availability of resources. For example, gulls and other urban birds rely on overflowing rubbish bins for food \citep{ditc06} and bats utilise small cavities in buildings and overpasses for shelter \citep{kunz82}. Many of the urban-adapted species, however, are invasive and displace native wildlife \citep{mcki06}. Built environments can also negatively effect sensitive species through direct displacement from habitat \citep{czec97}, or disturbance due to light and noise pollution \citep{parr16}. Further, wildlife mortality may result from static (e.g. bird strikes to glass) and non-static (e.g. car strikes to wildlife) elements of the built environment. 

\subsection{Impacts of linear infrastructure on wildlife}

Exposure to wildlife is compounded by human mobility via transportation which extends interactions to biotic organisms outside areas of human settlement. Linear infrastructure is most commonly thought of as roads, however, can also include railways, utility easements, and other continuous human-engineered resources in the landscape. Impacts of linear infrastructure on wildlife can be positive, neutral, or negative. For example, disused road reserves lead to retention of habitat that may benefit wildlife as they often contain old and diverse remnant vegetation \citep{benn91}. Further, linear infrastructure without moving vehicles may provide connection corridors and assist migration for some species \citep{rvdr15}. Effects may be neutralised when suitable habitat is provided on road reserves with very little traffic volume or occupied by species with low mobility. Nonetheless, it can be argued that the negative effects of linear infrastructure outweigh the benefits, especially with respect to roads and wildlife persistence \citep{fahr09}.
%\section{Road ecology}
The field of road ecology addresses a wide range of environmental problems resulting from the construction and use of roads across the natural landscape \citep{form03,rvdr15}. There are few places left in the terrestrial environment that are not accessible by some form of road (\Cref{roads}) and, consequently, the sphere of ecological influence is vast making roads a global concern \citep{laur14}.

\begin{figure*}[htp]
  \centering
  \includegraphics[width=0.9\textwidth]{world_roads.png}
  \caption[Global distribution of known roads]{Global distribution of known roads. Data used to create the map was sourced from the Center for International Earth Science Information Network (CIESIN).}
  \label{roads}
\end{figure*}

\subsection{Wildlife mortality on roads}

Perhaps the most directly visible effect of roads is wildlife mortality due to collisions with moving vehicles; often referred to as 'roadkill' in the common vernacular. It is estimated that billions of fauna expire annually due to collisions with transportation networks \citep{seil06}. Animals which are struck and expire on roadsides or railway corridors attract predators and may result in secondary collisions which further increase mortality rates \citep{spel98}. From a conservation perspective, these mortalities can result in population effects and disruption to food-chain networks.  Although collisions with wildlife bring about matters of animal welfare, biodiversity, and wildlife persistence, perhaps the largest concern is human safety. 

From an anthropocentric perspective, animal-vehicle collisions with larger species cause injuries, property damage, and occasional loss of human life which compels managers to find solutions. The visibility of the problem results in negative public perception and has caused many road authorities to add carcass collection to road maintenance operations; in some cases the data is recorded \citep{huij07a}. This has also initiated a multitude of partnerships between academics and road authorities to conduct scientific research on animal-vehicle collisions.  These endeavours have created a wealth of research and literature on larger common species \citep{biss08b,clev01,romi96,sudh09}, but proportionally lower amounts on smaller (and perhaps more threatened) fauna \citep{clev03}.

\subsection{Management of wildlife-vehicle collisions}

Mitigation strategies to reduce wildlife-vehicle collisions include fencing installations \citep{clev01,jaeg04}, road crossing structures (overpasses and underpasses) \citep{bond08}, discouragement and/or detection devices \citep{huij03,huij06}, road signage \citep{bond13}, or a combination of treatments \citep{dani98,pola14,romi96}. As mitigation can have significant costs \citep{huij09}, an important objective in managing wildlife-vehicle collisions is knowing where, when, and how to intervene. Further, these costs may be compounded by additional collisions occurring from ineffective mitigation.

%One difficulty is the implementation and/or translation of scientific research and problem-solving into management actions \citep{roux06}.  The disconnects between environmental research, conservation policy and management/implementation are extensively referred to in the ecological literature.  Technology, most notably electronic media and computers, has been used to overcome some of these problems and functions as a tool to bridge the information divide.  Software applications to inform management decisions have been developed in areas of broad landscape and seascape conservation \citep{ball09,watt09}, species distributions \citep{phil06}, climate change \citep{east09}, water resources \citep{liu08}, and many others.  Moreover, the translation of scientific research into accessible public knowledge and education is also a formidable problem \citep{lubc98}.  This is an area that deserves more attention as human behaviour change can be a powerful upstream solution to complement collision management efforts.

\subsection{Factors influencing collision risk}

Worldwide, many studies have been undertaken to understand wildlife-vehicle collisions.  Some studies related collisions to animal biology, behaviour and characteristics. In Australia, \cite{coul97} showed a male bias across several species of macropod roadkills, whilst \cite{kloc06} determined a female bias in red kangaroos.  \cite{lee04} concluded that kangaroo density near roads increased during periods of drought due to relative greenness of vegetation at road verges. \cite{ramp06b} used collision events in a population viability analysis model to evaluate effects on swamp wallabies.  In North America, animal travel speed was related to collisions \citep{litv08}. Research has also demonstrated that collision occurrence is related to combinations of environmental and anthropogenic variables \citep{barn07}.

The factors influencing collision risk are quite broad and \cite{form03} groups these into four major categories: animal behaviour, human behaviour, the road environment, and the landscape environment. For example, road ecology theory suggests that increases in collisions may be due to the following: animal behaviours, such as migration, flight response, or feeding patterns, thereby increasing species presence on roads; the operation of motor vehicles by humans at high frequencies, high speeds, and at dangerous times; landscape features attracting animals to, and influencing movement across, roads and; road features hindering safe vehicle operation. Based on a review of seventy-five published studies that modelled wildlife-vehicle collision risk (\Cref{wvc_studies}), the ten most commonly used variables were classified in three of the four categories proposed by road ecology theory. Traffic related variables (volume and speed) -- considered human behaviour -- featured in the top three most commonly used predictor variables.

\begin{figure*}[htp]
  \centering
  \includegraphics[width=0.9\textwidth]{WVC_Var_Hist.png}
  \caption[Variables used in wildlife-vehicle collision studies]{Common variables used in wildlife-vehicle collision studies. The histogram represents the percentage representation (rounded to nearest whole number) of each modelling variable in a total of 75 studies on wildlife-vehicle collisions.}
  \label{wvc_studies}
\end{figure*}

\section{Improving management and decision-making}

\subsection{Risk}

Risks are chances of adverse events, that have defined consequences, happening in specified time frames \citep{burg05}. Managing risks involves assessing potential hazards, exposure to hazards, and probabilities of hazards occurring, simulating outcomes under different scenarios, and making decisions based on acceptable thresholds. To help make decisions about risk, it is useful to employ quantitative methods either independently or in combination with qualitative methods. Quantitative analysis is used to reduce uncertainty about risk in many fields such as engineering \citep{apos04}, biological sciences \citep{sute16}, and finance \citep{mcne15}.  All disciplines define risk a bit differently and the parameterisations used in the analyses vary accordingly.

\subsection{Spatial statistics and modelling}

More contemporary studies have utilised environmental modelling to analyse and predict risks imposed on species (fauna and flora), biotic systems, and human beings resulting from anthropocentric activity. It can be argued that the outputs of these models have a multitude of benefits including informing management operations, supplementing additional models, facilitating scientific research, and optimising mitigation strategies.  The use of modelling in conservation efforts is well established \citep{star86}, however, predicting risks to wildlife by vehicle networks is still relatively new. Computer-based statistical modelling has been used to describe and predict wildlife-vehicle collisions for the past two decades \citep{guns11}.  The most common use of modelling has been to identify collision hotspots in geographic space in an effort to direct management efforts.  For example, \cite{malo04} predicted collision risk of multiple species at two spatial scales using logistic regression.  Similar methods have been used by \cite{roge09} to predict collision risk with wombats, and \cite{sudh09} for deer.

When statistical modelling of events, such as collisions, is undertaken, it is common to analyse either binary data (occurred/did not occur) or count data (how many occurred).  As it is not plausible to have negative events, these data are always constrained to be positive. Two distributions satisfy these requirements, Bernoulli and Poisson, and thus, are often chosen for modelling collision events. A common statistical model used for collisions is based on generalised linear regression \citep{mccu89}; a robust and widely-used statistical method in which non-continuous (discrete) responses are regressed on a linear predictor using a link function. Spatial statistics is an extension of these methods to study geographic properties of objects or events; in two-dimensional space, the choice of link function must be deliberate \citep{badd10}. Related to this concept, patterns of events can be analysed using point process models which consider events occurring across space with a single rate parameter. Modelling events on linear networks is a special case of one-dimensional point process analysis as it would be impossible for collisions to occur outside the network \citep{okab09}.  More specifically, collisions will always occur coincident with the movement vectors of vehicles.

\subsection{Data}

Data has inherent uncertainties and biases \citep{glym97}; uncertainty may result from measurement error, under-reporting, and general environmental stochasticity whilst bias is often due to selective sampling (either intentional or unintentional). Considering these shortcomings, data should always include metadata that reports these deficiencies so modellers may adequately address them in analyses.

The contemporary modeller has considerable access to publicly-available data. Anthropogenic data, such as demography, exist for most developed parts of the world. Environmental data, particularly satellite-based remotely-sensed, have global coverages. Many of these datasets, especially those subject to post-processing, include metadata about uncertainties. Citizen-science data can supplement data-deficient ecological studies, and are used in road ecology \citep{paul14}. However, the use of such data should be subject to rigorous quality control and verification which may, or may not, be supplemented by adequate metadata. Records may not be useful due to incomplete/incorrect data or geographical ambiguity. Although under-reporting of data has limited effects on model robustness, spatial biases can adversely affect model performance \citep{snow15}. In particular, survey efforts are seldom reported in collision data and vary considerably amongst and within jurisdictions \citep{huij07a}. Information on survey effort can improve models by generating parameters that account for detectability \citep{dora14}.

It is important to identify and address spatial biases when developing predictive models. Models trained on observations that are spatially biased (e.g. collected at particular places/times or dependent on characteristics of reporting individuals) may cause model predictions to reflect patterns of collection rather than actual distributions of organisms or events. Methods have been developed to address bias in models fit to both presence-absence \citep{hijm12} and presence-only data \citep{kram13}. For both types, implicit biases in data may be incorporated into modelling if they are adequately described in metadata \citep{wart13}. Many of the records in fauna atlases are subject to geographical bias (e.g. close to roads and towns); \cite{grah04} elaborates on the use of such data in scientific studies.
%\subsection{Temporal variation in data}
Temporal variation occurs in both natural and anthropogenic systems. Wildlife species have different patterns of activities (e.g. diurnal, nocturnal, crepuscular) that make them more or less susceptible to wildlife-vehicle collisions during the day. Seasonal patterns are particularly important for migratory species \citep{hick85,bern92}, where seasonal variation has a large influence on collision risk \citep{shep08}. These patterns are more easily incorporated into modelling as there is less uncertainty about observations - the temporal resolution is large and thus discrepancies in time between a collision event and carcass discovery are negligible. In contrast, daily patterns in collision data are subject to measurement errors and require more careful scrutiny, however, are highly informative in determining when collision risks are high. To address some of this uncertainty, studies may represent temporal data as a categorical variable \citep{mizu14} with discrete ranges \citep{rowd08}. As a continuous variable, hour may be represented by cyclic functions in predictive models \citep{neum12,thur15}. In the literature, predictions of collision risk made using temporal information are referred to as 'hot moments'.

So it's clear that [insert problem here] is a huge and far-reaching problem. With important consequences for [XX] and [YY]. Previous attempts to solve [problem] have been ok, but are missing [crucial gap] and this means that [highlight consequence of crucial gap]. There is a critical need for research that does [whatever you are about to say your aims are] that uses [whatever methods you're going to use] to address this gap

\section{Thesis aims and scope}

The broad aim of this thesis is to analyse and predict where and when wildlife will be at high risk, due to collisions with moving vehicles, across large transportation networks. As I intend the work to supplement managers working in the wildlife and transportation fields, I have developed methods that are generalisable, transferable, reproducible, and based on open-source data. Therefore, each chapter tests different model applications and demonstrates management value. \Cref{sec:egk} introduces a general conceptual framework to model wildlife-vehicle collision risk and uses kangaroos in south-east Australia as a case study. Using the same geographic area, \cref{sec:6sp} extends the work of \cref{sec:egk} by applying the model framework to six mammal species commonly involved in wildlife-vehicle collisions. In \cref{sec:cal}, the conceptual framework is applied to a different geographic area (California, USA) and a different species (Mule deer). To analyse risk on a different transportation network, I extend the conceptual model to examine wildlife-train collisions in both space and time in \cref{sec:train}. To assess the performance of the model introduced in \cref{sec:egk}, and demonstrate the importance of unified and standardised data repositories of wildlife-vehicle collisions, model predictions are compared to additional datasets collected by various organisations in \cref{sec:val}. \Cref{sec:conc} summarises a selection of key findings within the chapters whilst suggesting novel areas for further research and potential applications of the work.