\chapter{Application of wildlife-vehicle collision model framework on different continents: Victoria and California}\label{sec:cal}
\newpage

\begin{localsize}{10}
\section*{\centering Abstract}

Our study is the first to apply a generalised model framework to analyse and predict WVC in two continents. Despite being different species occurring in different geographic areas, we examine whether collision risk correlates similarly with animal behaviour (occurrence of the species) and human behaviour (traffic volume and speed).

We applied a conceptual risk framework in the State of Victoria in south-east Australia and a section of central California in the USA. Our study species were eastern grey kangaroos in Victoria and mule deer in California. We trained our collision models with citizen-science collected, collision/carcass records for each species and predicted collision risk for all road segments in each study area.  We used a very different set of data, police records of road accidents, to evaluate the predictions.

All predictor variables were highly significant in the collision models and the shape of the response to each predictor were similar for both species.  Relative collision risk increased with increased species occurrence, traffic volume, and traffic speed. Speed had the highest relative importance followed by species occurrence. Predicted relative collision risk on all road segments agreed with the distribution of known collision hotspots throughout most of the two study areas. For both species, collision model predictions using the validation datasets were highly correlated with independent observations, indicating well calibrated models.

Our study demonstrates that a conceptual risk framework generalises well to species in different geographical areas. The framework enables managers to conduct sensitivity analyses and calculate overall reductions in expected collisions based on applying mitigation strategies on different road segments. Our analysis also suggests appropriate mitigation for both species may include reduced speeds or fencing on road segments where species occurrence is predicted to be high.

\end{localsize}

\newpage
\section{Introduction}

Wildlife-vehicle collisions (WVC) result from human development and activity. Approximately one-million vertebrates are killed per day in the United States (Forman and Alexander, 1998) and up to 27-millions birds are killed annually in select European countries (Erritzoe et al., 2003).  Costs of WVC, including vehicle repair, human medical costs and the value of a human life, are estimated to exceed eight billion dollars annually in the United States (Huijser et al., 2007) with deer collisions comprising the largest share (see Bissonette et al., 2008).  These costs and impacts will continue to worsen globally as new roads are constructed and existing roads are upgraded in developing countries; with worldwide traffic volume doubling by 2050 and increasing fivefold in developing countries (van der Ree et al., 2015).

Mitigation strategies aim to reduce the rate and severity of WVC by influencing either human or animal behaviour (see Huijser and McGowen, 2010).  Traffic planning (e.g. the magnitude and routing of vehicles), speed control (e.g. signage, traffic calming mechanisms and penalty enforcement), and education/training can influence the operation of vehicles on roads.  Exclusion (e.g. fencing) and re-routing, either over or under the road, can influence the presence and movement of wildlife on the roads.  Other strategies involve visual or auditory discouragement such as flashing lights and ultra-sonic whistles, however, these are largely ineffective (Reeve and Anderson, 1993; Bender, 2003; Scheifele et al., 2003; Ramp and Croft, 2006). Choices to implement mitigation strategies are influenced by an understanding of the factors contributing to WVC in a particular area.

Decisions about where and when to mitigate are important because applying ineffective strategies incurs costs.  To address this issue, many studies examine patterns of collisions and related variables to inform management, often involving spatial modelling and prediction (see Gunson et al., 2011).  Most research is specific to an area (i.e. section of road) or problem and is not easily transferable due to variation in scale, geographic location and/or species traits, and confounding effects of explanatory variables.  Although these issues cannot be totally eliminated - partly due to the uncertainty present in ecological systems - it is possible to develop analytical frameworks to assist managers to deal with uncertainty and make inferences regarding wildlife-vehicle collision risk (see Visintin et al., 2016).

We tested the general applicability of a previously developed conceptual model framework for predicting collision risk in California and Victoria.  We had two aims; first we were interested in the applicability of our methods to planning for WVC mitigation irrespective of geographical location or scale (see van der Ree et al., 2011).  We propose that our analytical methods will supplement management efforts by allowing clearer inferences about the factors related to collision risk, identifying potential biases and uncertainties in the analysis (thereby suggesting further research or data collection), and suggesting appropriate mitigation for WVC.  Second, collisions of vehicles occur at a similar frequency for both kangaroos and deer, have similar impacts (Langley and Mathison, 2008), and are subject to similar management practices (McShea and Underwood, 1997; Croft, 2004).  So while models of collision risk should differ between species because of the different contexts, we expect that our previous framework should reveal similar patterns of collision risk for the two species as a function of traffic attributes and species occurrence.  To our knowledge, this work represents the first study to test a broadly applicable model for two analogous species on different continents for the purpose of satisfying contrasting needs of management.

\section{Materials \& Methods}

\subsection{Overview of Workflow}

Our analysis is comprised of the following overall workflow for each of the two study areas:
1)  Define the study area and divide it into a regular spatial grid.
2)  Split all roads in the study area into segments using the spatial grid.
3)  Develop traffic models and predict relative traffic volume and speed for all road segments in the study area.
4)  Develop a species distribution model and predict relative occurrence for each road segment based on the grid cell in which it occurs.
5)  Classify road segments based on occurrence of collision ('1' for one or more collisions; '0' otherwise) for the training data.
6)  Develop a regression model of the collisions (training data) with species occurrence, traffic volume and traffic speed as explanatory variables.
7)  Use the regression model to predict relative rate of collisions for each road segment.
8)  Compare the predicted rate of collision to a new set of collision data (validation data) to evaluate the regression model.
Each of these steps is explained in more detail in the following sections.

\subsection{Study Areas}

We use two unique geographic locations to test our framework; the 227,819 km2 State of Victoria in south-east Australia and a 146,478 km2 section of central California in North America (Figure 1). The case study regions were chosen because of availability of data on animal occurrence, collision records, and the road networks. Each locale has a single agency responsible for the major roads (e.g. high-speed, high-capacity, or both) while all other roads are managed by municipal districts. We predict collision risk for all sealed roads within the study areas. To organise our spatial data and modelling, we overlay spatial grids of one km2 resolution on the study area. Each grid cell is used as the modelling unit for species occurrence. All roads in the study area are bisected by the grid resulting in road segments that are approximately one kilometre or less in length. We use 655,348 road segments for Victoria and 646,705 road segments for California as our modelling units for each respective collision model.

\subsection{Data}

We selected one species from each study area that are frequently involved in WVC. Eastern grey kangaroos (Macropus giganteus, Shaw) are the second largest mammal in Australia - up to 85 kilograms for males (Van Dyck and Strahan, 2008) - and share management issues with ungulates found in North America and Europe (Croft, 2004; Coulson and Eldridge, 2010).  Mule deer (Odocoileus hemionus, Rafinesque) are common across western North America and adults range in size up to 120 kilograms (Kays and Wilson, 2009).

To train our collision models, we obtained spatially unique collision/carcass records for each species from two citizen-science databases; the Wildlife Victoria Database, or WVD, (Wildlife Victoria, 2016) for kangaroos, and the California Roadkill Observation System, or CROS, (see Shilling and Waetjen, 2015) for deer.  The collision/carcass records spanned a six year period between 1 January, 2010 and 1 January, 2016 for kangaroos and a ten year period between 1 June, 2006 and 1 June, 2016 for deer. We limited our datasets to records with associated global positioning system (GPS) coordinates. Kangaroo records were initially inspected for spatial accuracy and we only retained observations with less than three-hundred metre error. Deer records from CROS have greater than ninety-nine percent species identification accuracy and less than one-hundred metre spatial error (Shilling and Waetjen, unpublished observations).

For each target species, we selected road segments that intersected with reported species’ collision records and coded them with ones. We coded all other road segments in each study area with zeros, to represent background data, and combined them with the collision record segments.  After removing spatial duplicates, there were 4,245 presence and 640,470 background points in the kangaroo collision modelling dataset and 933 presence and 644,296 background points in the deer collision modelling dataset.

Traffic volume and speed values for all road segments in both Victoria and California were predicted following methods of Visintin et al. (2016).  We regressed annual average daily traffic (AADT) and speed on explanatory variables of distance to anthropogenic development (derived from remotely-sensed land use), distance to highway/freeway, road class, road density within one km of each road segment, and population density (from the Australian Bureau of Statistics and the United States Census Bureau, respectively) in random forest models (Brieman, 2001). We developed species distribution models (SDMs) to predict occurrence for each species across their respective geographic areas (see Appendix S1 in Supporting Information) and used the mid-points of the road segments to sample each species occurrence prediction grids. This resulted in two modelling datasets, each with a binary dependent variable of collision (1) or background (0) and three continuous predictors of species occurrence, traffic volume, and traffic speed.  

We used police records of road accidents to evaluate our model fits.  For each species, we extracted spatially-unique records where the incidents were collisions with our target species.  For kangaroos, we used data reported between 1 January, 2010 and 1 October, 2016. For deer, the reporting period was 1 February 2015 to 1 December 2016.  Again, we used all road segments as background and coded the segments with collisions as ones and the remainder with zeros. The final datasets had 481 presence and 644,234 background points for kangaroo collision model validation and 1,795 presence and 643,434 background points for deer. As with the training data, each road segment had values for species occurrence, traffic volume and traffic speed.

\subsection{Collision Modelling \& Validation}

We used a quantitative risk model (Visintin et al., 2016) to fit and compare the relationship of species presence and road threat to collision likelihood and the open-source software package ’R’ version 3.3.0 (R Development Core Team, 2004) to perform all statistical analyses. We regressed collisions on the predictor variables with an added quadratic term for traffic volume. Our model is expressed as:

cloglog(pi) = β0 + β1 log(Oi) + β2 log(Vi) + β3 (log(Vi))2 + β4 log(Si)			(1)

where (pi = Pr(Yi=1)) is the relative likelihood of a collision occurring on a road segment i, Oi is species occurrence, Vi is traffic volume, Si is traffic speed. We chose the complementary log-log link on the linear predictor due to the mathematical theory underpinning our model - risk being measured by the rate of collisions (see Visintin et al., 2016).

To test for spatial autocorrelation, we calculated normally-distributed, randomised quantile residuals (Dunn and Smyth, 1996) using 5000 simulations from each model fit. We then calculated Moran's I on the residuals at 20 spatial lags at intervals of both one kilometre and 250 metres.  For each spatial resolution, we repeated the Moran's I calculation twenty times for 5000 randomly sampled road segments (due to computational limitations) and plotted the twenty trend lines for visual inspection at both spatial scales (Figure 3).

The two models were used to predict relative collision risk of all road segments in each respective study area (Figure 4).  We tested the calibration strength and discrimination ability of our models with their respective independent validation datasets. To assess calibration strength, we regressed the collision observations in the validation sets on predictions made using the validation data in the trained models. An intercept coefficient of 'zero' and slope coefficient of 'one' indicates a perfectly calibrated model (see Miller et al., 1991). To measure ability to discriminate between true positive and false positives we used a receiver operating characteristic (ROC) score - a score of one indicating perfect discrimination ability while 0.5 suggesting a performance no better than random (see Metz, 1978).

\section{Results}

All of the predictor variables demonstrated plausible relationships to collision likelihood in the partial dependency plots. The probability of collision for both species changed similarly with occurrence of the species (Figure 2a and b), traffic volume (Figure 2c and d), and traffic speed (Figure 2e and f).

Traffic speed and species occurrence were highly significant variables for kangaroos and deer (Table 2). Speed had the highest relative importance for both kangaroos and deer based on an analysis of variance (ANOVA) for each predictor. Increasing speed from 80 to 100 km hr-1 approximately doubled the relative collision risk for both species.  Collision risk with deer was less than kangaroos at lower speeds and increased at a faster rate at higher speeds. Species occurrence influenced collision risk the second most for both species.  The response shape of collision risk against likelihood of occurrence was similar for both species, however, deer had larger confidence intervals around the marginal response (Figure 2a and b).

Relative collision risk  for kangaroos increased with traffic volume until approximately 5,000 vehicles day-1 and then declined (Figure 2c); the relationship was similar for deer although the peak was closer to 10,000 vehicles day-1 (Figure 2d). Collision likelihood of kangaroos decreased more quickly than deer at higher traffic volumes, but deer had larger confidence intervals around the marginal response curve indicating the possibility of a shape more consistent with kangaroos (Figure 2d). The actual magnitude of the relative risk is not comparable between species because sizes of the collision and background datasets were not standardised across the two species.

Spatial correlation in the randomised quantile residuals was low and showed no patterns across distances at the two spatial scales analysed (Figure 3), suggesting assumptions of statistical independence were reasonable. All of the predictor variables were highly significant and the signs of the coefficients indicated predictor-response relationships that were plausible (Table 2). The deer model reduced deviance compared to the null model by nearly twice that of the kangaroo model and also had higher discrimination ability.

For each species, the models had similarly high calibration statistics when independent data (police reports) were used for validation. The kangaroo model had a slope coefficient of .99 and the deer model had a slope coefficient of .94 (Table 1). Each model had a different intercept coefficient due to varying numbers of collisions used for model fitting.

The predictions for relative collision risk across all road segments were plausible in most regions of the two study areas.  In Victoria, predicted collision risk was highest in the northern suburbs of Melbourne where collisions frequently occur. In contrast, predictions in California were highest on road segments in both the San Francisco Bay Area (SFBA), where deer collisions are common, and along the Interstate 5 and 99 corridors through the Central Valley region, where deer collisions are less commonly observed/reported.

\section{Discussion}

Our results suggest that the proposed conceptual framework may have utility irrespective of locality, spatial scale, or species.  Species occurrence, traffic volume and speed, and  predictor variable data are publicly available for many jurisdictions, therefore, an analyst only requires data on vehicle collisions with wildlife species to train the models in the conceptual framework, make inferences and predict risk.  Moreover, the collection of data on collisions with vehicles by road authorities (via carcass collection operations) or citizen scientists such as wildlife groups is increasing due to human safety or animal welfare concerns and technological innovations (see Olson 2014; Shilling et al., 2015).  Analysis from our framework may guide mitigation actions in several problematic areas to reduce collisions.

Forman (2003) identifies both animal and human behaviour as major drivers for WVC.  Our framework operates on this hypothesised relationship and indicates maximum relative risk where animal presence and vehicles moving at speed co-occur in space.  Road authorities mitigate collision risk by managing human activity (traffic volume and speed) or animal behaviour (occurrence on or movement across the road) and these have different costs.  Our results demonstrated that speed is an important predictor of collision risk for both kangaroos and deer and therefore mitigation that either addresses animal presence near high speed road segments (e.g. fencing and crossing structures) or traffic speeds (e.g. controls and enforcement) in collision hotspot areas should be considered.  This pattern is consistent with other studies on deer collisions (Gkritza, 2013; Meisingset, 2014; Sudharsan, 2009), kangaroo collisions (Rowden, 2008), and other taxa (Gunson, 2011).  Moreover, the traffic models demonstrated plausible fits to the posted speed training data (Appendix S2: Table S2) therefore suggesting greater reliability in their predictions.  However, when assessing where to place mitigation for speed, the analyst and managers should also consider the uncertainty around predictions at each road segment for both speed and animal presence. If there is higher confidence around speed predictions, for example, more emphasis (and funds) may be placed on controlling vehicle speeds, rather than excluding animals.

The conceptual framework with sub-models allows us to more clearly identify bias and uncertainty in the analysis.  For example, the two occurrence models are produced using presence-only data that explicitly assumes perfect detection and includes potential sampling bias.  If these assumptions are deemed unsatisfactory, an analyst may choose to use alternative statistical methods that allow incorporation of uncertainty into the model (see Dorazio, 2014) or improve the training data (e.g. increase accuracy standards or collect additional records that include true absences).  The deer occurrence predictions may also be less reliable than the kangaroo predictions due to a smaller number of training observations.  Likewise, the lower statistical significance and wide confidence intervals in the marginal effects of traffic volume on collision likelihood for deer may warrant further analysis. The wide confidence intervals mean that reducing vehicles on problematic road segments may not have the desired effect.  Moreover, reported or observed collision data are subject to the same limitations as species occurrence data (e.g. reporting biases or data deficiencies such as under-reporting of actual presences) and should be taken into account.

As expected, collision risk for kangaroos and deer were similar in response to species occurrence, traffic volume and traffic speed.  Collision risk increased monotonically with increasing species occurrence for both kangaroos and deer, however, there was an order of magnitude greater change for kangaroos due to differences in ratios of collision to background observations.  Assuming species occurrence is of equal importance as the other collision model predictors, the smaller increase in collision risk for deer may be attributed to lower predictions of species occurrence arising from a mis-specified species distribution model (Guillera-Arroita, 2015).  Collision risk response to traffic volume was uni-modal for both species as expected, however, the change in collision risk was an order of magnitude greater per unit of traffic volume for kangaroos.  This response shape can arise from different mechanisms.  First, there may be a potential observer bias meaning people are less likely to stop or report collisions on high-volume roads out of concern for their own safety or simply not seeing the dead animal.  Second, there may be road avoidance effects as some species are repelled by roads when traffic volume increases beyond a threshold value due to high noise loads or small crossing intervals (see van Langevelde and Jaarsma, 2004; Seiler, 2005; Gagnon et al., 2007).  Our results are consistent with other studies on behavioural responses to roads that include kangaroos and deer (see Jacobson et al., 2016).  At higher values of traffic speed, collision risk increased monotonically with increasing speed with a similar magnitude for both species (Figures 2e and 2f).  Kangaroos were exposed to higher collision risk at lower traffic speeds than deer.  This may be due to Melbourne’s suburbs expanding into natural kangaroo habitat, thereby increasing wildlife-vehicle collisions.

We did not explore temporal effects in this study, however, this may be incorporated into our framework.  One such method is to add a function-based term to the model that allows collision risk to vary as a function of time and season (Visintin et al., manuscript in prep). The functional form may be expressed in relation to the known activity periods of the target species. Kangaroos are crepuscular, with peaks of activity at dawn and dusk, and deer are either crepuscular or nocturnal depending on the season.  Hour of day and time of year are useful predictors of collisions with ungulates (Meisingset et al., 2014; Mountrakis and Gunson, 2009), however, not yet fully tested for kangaroos.

The inferences made from the conceptual framework will be more robust as more observations are added to train the models.  We currently use a mixture of professional and citizen-science collected data - each with unique advantages, disadvantages, and implications for analysis.  Citizen-science data can supplement data-deficient ecological studies, and are effective in road ecology (Paul et al., 2014).  Moreover, citizen-science data is relatively low-cost and can cover large spatial scales.  For example, the collision records in both the Wildlife Victoria database and the California Roadkill Observation System are systematically collected and stored in databases and employ mechanisms to elicit data from unrestrained spatial distances.  As innovative and standardised data collection techniques are implemented (see Aanensen et al., 2009; Donaldson and Lafon, 2010; Shilling et al., 2015), inferences and predictions from the model framework will improve.

Our analysis suggests appropriate mitigation for both species may include reduced speeds or fencing on road segments where species occurrence is predicted to be high.  Although omitted for this work, our methods enable managers to conduct sensitivity analyses and calculate overall reductions in expected collisions based on mixed applications of mitigation on different road segments (assuming the mitigation is 100\% effective, but see Huijser et al., 2009).  As costs are often known for specific mitigation measures, simulations can determine optimal uses of resources to maximise reductions of collisions and minimise costs. We are excited to apply these methods to other species and geographical areas.