\chapter{A conceptual risk framework for modelling and predicting wildlife-vehicle collisions}\label{sec:egk}
\newpage

\begin{localsize}{11}
\section*{\centering Abstract}

Collisions of vehicles with wildlife kill and injure animals, and are also a risk to vehicle occupants, but preventing these collisions is challenging. Surveys to identify problem areas are expensive and logistically difficult. Computer modelling has identified correlates of collisions yet these can be difficult for managers to interpret in a way that will help them reduce collision risk.

We introduce a novel method to predict collision risk by modelling hazard (presence and movement of vehicles) and exposure (animal presence) across geographic space. To estimate the hazard, we predict relative traffic volume and speed along road segments across south-east Australia using regression models based on human demographic variables. We model exposure by predicting suitable habitat for our case study species (Eastern Grey Kangaroo \textit{Macropus giganteus}) based on existing fauna survey records and geographic and climatic variables. Records of reported kangaroo-vehicle collisions are used to investigate how these factors collectively contribute to collision risk.

The species occurrence (exposure) model generated plausible predictions across the study area, reducing the null deviance by 30.4\%. The vehicle (hazard) models explained 54.7\% variance in the traffic volume data and 58.7\% in the traffic speed data. Using these as predictors of collision risk explained 23.7\% of the deviance in incidence of collisions. Discrimination ability of the model was good when predicting to an independent dataset.

The research demonstrates that collision risks can be modelled across geographic space with a conceptual analytical framework using existing sources of data, reducing the need for expensive or time-consuming field data collection. The framework is novel because it disentangles natural and anthropogenic effects on the likelihood of wildlife-vehicle collisions by representing hazard and exposure with separate, tunable sub-models.

\end{localsize}

\newpage
\section{Introduction}

Roads have well-documented negative ecological impacts \citep{form98,spel98,rvdr15} including effects on terrestrial fauna. Road construction and use fragments and destroys habitat, causes pollution (e.g. noise, light and chemical run-off), and kills and injures animals. Perhaps the most visible impact is direct mortality through wildlife-vehicle collisions (WVC) -- billions of vertebrate fauna are killed annually around the world \citep{seil06}. Such an issue has prompted many road management authorities to routinely collect animal carcasses struck and killed by moving vehicles to reduce visual impacts for road travellers \citep{huij07b} and avoid secondary collisions with scavenging wildlife species. In addition, many governments around the world incur significant costs installing wildlife-proof fencing and under- and over-passes to reduce the rate of wildlife-vehicle collisions and improve landscape connectivity \citep{rvdr15}.

Worldwide, the frequency, magnitude and distribution of WVC have been widely studied. Many such studies relate rate of collisions to environmental conditions, anthropogenic variables, and animal biology, behaviour and characteristics. While many studies have used statistical modelling to determine hotspots for WVC, most are limited to a single stretch of road or small collections of roads \citep[e.g.][]{gund98,clev02,clev03,ramp05,ramp06b,gome08,lang09,hurl09,roge09,hoth12,mark12,sant13,seo15}. These models perform relatively well at local scales, and some have been employed to communicate areas of high-risk to road managers, but many cannot extrapolate to other sections of road or entire networks. We extend this work by developing and testing a framework that may be applied at much larger scales; consistent with the boundaries of road authority jurisdictions (i.e. state/provincial). \cite{clev15} asserts that the variability of significant predictors among regions and geographic scales highlights a critical need for a useful broad-scale conceptual framework to analyse/predict WVC.

Managers often have limited time and budgets to survey WVC across large areas or road networks. Many existing studies utilise data collected at fine spatial scales to model WVC, which is only feasible for limited areas because it is prohibitively expensive to collect data at the required regional scale. Methods to incorporate publicly-accessible, existing sources of data or model data-deficient parameters are useful to reduce costs. For example, remotely-sensed data is useful to determine environmental influences, while GIS-based census data may be used to characterise road conditions. Our framework is a desktop-based exercise for managers to determine risk across a large network with the ability to fine-tune based on available data. Moreover, it can be adapted with additional or modified data without changing the underlying methodology.

When modelling is used to determine hotspots for WVC, predictions are made using single models that combine both environmental and anthropogenic variables \citep[e.g.][]{malo04,ramp05,gome08,roge09,hoth12,bart14,meis14,snow14}. These studies are valuable for managers to identify locations  for mitigation or further study, however, they are often limited in their ability to extrapolate beyond the study area and do not clearly indicate potential confounding effects or suggest what mitigation to use (e.g. on the road environment or on the species). For example, vegetation on the road verge is commonly used to model collisions, and has been shown to be an effective predictor in previous work.  But as a single covariate among others in a regression model, it is difficult to determine if vegetation is related to collisions based on its affect on visibility for drivers, or its attraction for animals. We extend the utility of this research by disentangling the effects of human activity and wildlife behaviour, which has not been achieved before. We aim to improve the accessibility of collision analysis and predictions by relating risk to two components, exposure and hazard. This hierarchical classification and structure of the predictors enables a more straight-forward calibration and interpretation of the models -- a useful feature for managers -- without compromising predictive performance.

\section{Materials and Methods}

\subsection{Conceptual Model Framework}
Risk (as a rate) of animal collisions can be expressed as a function of exposure and hazard: 
\begin{equation} \label{eq:21}
R_i=aE_iH_i
\end{equation}

where $R_i$ is the risk, $E_i$ is the exposure, $H_i$ is the hazard, $a$ is a constant of proportionality and $i$ represents a modelling unit (e.g. site, road, road segment). This equation is multiplicative and to enable linear analysis, we logarithmically transform the variables: 
\begin{equation} \label{eq:22}
\text{log}(R_i)=\text{log}(a)+\text{log}(E_i)+\text{log}(H_i)
\end{equation}

This suggests risk is perfectly related to both exposure and hazard and we allow modified responses by expressing the constant of proportionality as an intercept and introducing regression coefficients such that risk is modelled as: 
\begin{equation} \label{eq:23}
\text{log}(R_i)=\beta_0+\beta_1\text{log}(E_i)+\beta_2\text{log}(H_i)
\end{equation}
\begin{center}
--or equivalently as--
\end{center}
\begin{equation} \label{eq:24}
R_i=e^{\beta_0+\beta_1\text{log}(E_i)+\beta_2\text{log}(H_i)}
\end{equation}

Here, $\beta_1$ and $\beta_2$ indicate the relative influence of each predictor on risk. This makes inferences on the model fit more tractable and clearly identifies areas for management focus (e.g. the exposure or the hazard). If risk is exactly related to exposure and hazard as in equation \ref{eq:22}, then both regression coefficients $\beta_1$ and $\beta_2$ will equal one.

For this study, we represent exposure with animal presence, and hazard with both traffic volume and speed as neither one in isolation would be a realistic threat. We envisage risk as being measured by the rate of collisions. Using this configuration, the rate of collisions ($C_i$) is modelled as a function of three predictors of (1) species occurrence, (2) traffic volume and (3) traffic speed: 
\begin{equation} \label{eq:25}
C_i=e^{\beta_0+\beta_1\text{log}(O_i)+\beta_2\text{log}(V_i)+\beta_3\text{log}(S_i)}
\end{equation}

where $O_i$ is species occurrence, $V_i$ is traffic volume, $S_i$ is traffic speed, in a given place $i$.

While we consider risk as being measured by the rate of collisions (e.g. collisions per month), our data consist of binary events (observations of individual collisions and no-collisions, coded as ones or zeroes, respectively). If collisions are treated as events in a Poisson encounter model, then the probability of no collisions occurring equals $e^{-C_i}$. Let $Y_i=1$ indicate a collision occurred at site $i$, and $Y_i=0$ indicate no collision, with $p_i=\text{Pr}(Y_i=1)$. Since the probability of no-collision is also equal to one minus the probability of collision:
\begin{equation} \label{eq:26}
\text{log}(1-p_i)=-e^{\beta_0+\beta_1\text{log}(O_i)+\beta_2\text{log}(V_i)+\beta_3\text{log}(S_i)}
\end{equation}
\begin{center}
--or equivalently as-- 
\end{center}
\begin{equation} \label{eq:27}
\text{log}(-\text{log}(1-p_i))=\beta_0+\beta_1\text{log}(O_i)+\beta_2\text{log}(V_i)+\beta_3\text{log}(S_i)
\end{equation}

The left-hand side of equation \ref{eq:27} is the complementary log-log link function, which has similar properties to the more common logit function of logistic regression. However, we use the complementary log-log function here, because it relates more clearly to the rate of collisions (\ref{eq:25}), and to our conceptual risk model (\ref{eq:21}). Thus: 
\begin{equation} \label{eq:28}
\text{cloglog}(p_i)=\beta_0+\beta_1\text{log}(O_i)+\beta_2\text{log}(V_i)+\beta_3\text{log}(S_i)
\end{equation}

Information on species occurrence, traffic volume and traffic speed are unlikely to be available for every place $i$ and we propose that this can be modelled. Several transportation modelling methods exist for estimating traffic patterns: generalised linear modelling \citep[e.g.][]{seav00,zhao01}, neural network analysis \citep[e.g.][]{dudd13}, empirical Bayes estimation \citep[e.g.][]{yang02}, universal kriging \citep[e.g.][]{eom06,selb13}, and support vector machines \citep[e.g.][]{cast09}. Modelling approaches to estimate species occurrence include correlative \citep[e.g.][]{guis05,elit09} and mechanistic \citep[e.g.][]{kear09} species distribution modelling (SDM) and extensions of population viability analysis (PVA) \citep[e.g.][]{gilp86,shaf90}.

We apply our framework to study vehicle collisions with kangaroos in Australia using species distribution modelling to estimate kangaroo occurrence and linear regression to estimate traffic volume and speed (\Cref{egk_framework}).

\begin{figure*}[!h]
  \centering
  \includegraphics[scale=.9]{egk_fig01.png}
  \caption[Diagram of the modelling framework]{Diagram of the modelling framework. Three sub-models are used to generate covariates used in the collision model per the ``risk equals exposure multiplied by hazard" analytical framework.}
  \label{egk_framework}
\end{figure*}
 
\subsection{Study Area \& Species}

We selected the State of Victoria in south-east Australia as a study area as its diversity across its area of 227,819 square kilometres provides a good platform to illustrate the framework. Our study combines all sealed roads within the state ($\approx$150,000 kilometres) and predicts collision risk across six motorway class types. To organise our spatial data, we overlaid a spatial grid of one square kilometre resolution on the study area. To produce modelling units for the collision model, we further segmented the roads by intersecting all roads and the spatial grid. The open-source software package \textit{R} \citep{rdct16} was used to perform all spatial and statistical analyses.

We used the native species Eastern Grey Kangaroo (\textit{Macropus giganteus}, Shaw, hereafter referred to as ``grey kangaroo'') as the case study species (see \Cref{apx:B} for more information). In Victoria, the grey kangaroo is the most abundant of the macropod family and frequently involved in wildlife-vehicle collisions. Between 2005 and 2013, over 600 incidents were reported to the Victorian Police and documented in the VicRoads Crashstats database (see \Cref{apx:C}). Actual incident rates, however, are much higher as the largest Victorian wildlife organisation, Wildlife Victoria (see \Cref{apx:C}), received over 5000 reports of grey kangaroo-vehicle collisions over the same period. Grey kangaroos are the second largest native terrestrial mammal in Australia and share many similar characteristics and management issues with ungulates found in North America and Europe \citep{crof04,coul10}. Moreover, there is a research gap involving collision modelling of single macropod species in Australia \citep{bond14}.

\subsection{Species Occurrence Sub-Model}

We downloaded survey records of grey kangaroos from the Victorian Biodiversity Atlas (VBA, see \Cref{apx:C}) occurring in the period 2000--2014, and spatial accuracy within 500 metres. The presences were both systematic targeted surveys and incidental sightings from data maintained by the Arthur Rylah Institute, a division of the Victorian Department of Environment, Land, Water and Planning.  Kangaroos are generalists and widely distributed; it was therefore assumed that although abundance may fluctuate based on drought conditions, distribution would not change significantly from the year 2000.  We used data from this period because it improved the sample size for occurrence modelling.  We derived background data by randomly sampling 10,000 points across the entire study area.  After sampling values from the covariate rasters and eliminating null values, the final dataset included 901 species presence and 9957 background observations.

To estimate occurrence of grey kangaroos, we used Boosted Regression Trees \citep[BRT, see][]{frie02}. BRT modelling offers advantages of handling different types of predictor variables, accommodating missing data and outliers, fitting complex non-linear relationships, and incorporating interaction effects between predictors \citep{elit08}. We selected a tree complexity of five (limit on number of terminal nodes per tree used to include potential interactions) and a learning rate of .005 (contribution of each tree to the model). Classification methods have an established use in studies of species distributions \citep[see][]{walk90,skid96}, however, our framework is not limited to any particular modelling method. We selected seven predictors (\Cref{egk_variables}) based on the biology and behaviour of grey kangaroos. All of the species occurrence sub-model predictor variables were below a pairwise correlation threshold of 0.75 to reduce potential effects of multicollinearity. We predicted relative likelihood of grey kangaroo occurrence from the model fit at a one square kilometre resolution across Victoria (\Cref{egk_occmap}).

\begin{table}[!t]
\caption[Variables used in statistical models]{Variables used in statistical models.}
\centering
\begin{tabularx}{0.9\textwidth}{llY} \toprule
Model &Variable    &Definition\\ \midrule 
\multirow{8}{*}{\rotatebox[origin=c]{90}{Species Occurrence}}   &KANG        &presences and pseudo-absences of Eastern Grey Kangaroos\\
           		 &ELEV        &elevation of terrain in meters above sea level\\
           		 &GREEN       &remote-sensed mean seasonal change in greenness (2003--2013) in vegetation\\
           		 &LIGHT       &remote-sensed relative artificial light intensity\\
           		 &MNTEMPWQ    &mean temperature of wettest quarter in $^{\circ}C$\\
           		 &PRECDM      &precipitation of driest month in millimetres\\
           		 &SLOPE       &slope of terrain in decimal percent rise\\
           		 &TREEDENS    &tree canopy coverage within 1 square kilometre in decimal percentage\\
&&\\
\multirow{6}{*}{\rotatebox[origin=c]{90}{Traffic Volume}}   &AADT        &average annual daily traffic counts per road segment\\
           		 &KMTODEV     &distance in kilometres to commercial planning zones\\
           		 &KMTOHWY     &distance in kilometres to major road segments (freeways and highways)\\
           		 &POPDENS     &2011 population divided by area in square kilometres\\
           		 &RDCLASS     &road class category (`freeway', `highway', `arterial', `sub-arterial', `collector' or `local')\\
           		 &RDDENS      &total lengths in kilometres of road segments within 1 square kilometre\\
&&\\
\multirow{3}{*}{\rotatebox[origin=c]{90}{Traffic Speed}} &&\\
    			 &SPEEDLMT    &posted speed limit per road segment	\\
           		 &RDCLASS     &road class category (see above)\\
           		 &RDDENS      &total lengths in kilometres of road segments within 1 square kilometre\\
&&\\
&&\\
\multirow{4}{*}{\rotatebox[origin=c]{90}{Collision}}  		 &COLL        &presences and pseudo-absences of grey kangaroo-vehicle collisions\\
           		 &EGK         &predicted relative likelihood of kangaroo presence\\
           		 &TVOL        &predicted traffic volume (number of vehicles per day) per road segment\\
           		 &TSPD        &predicted posted traffic speed (kilometres per hour) per road segment\\
\bottomrule
\end{tabularx}
\label{egk_variables}
\end{table}

\begin{figure*}[!t]
  \centering
  \includegraphics[scale=1]{egk_fig02b.png}
  \caption[Predicted relative likelihood of grey kangaroo occurrence in Victoria]{Predicted relative likelihood of grey kangaroo presence in study area. Darker shades indicate higher relative probabilities of occurrence (mean: 0.057; range: 0.002--0.986).}
  \label{egk_occmap}
\end{figure*}

\subsection{Traffic Volume \& Speed Sub-Models}

Average Annual Daily Traffic (AADT) represents the sum of traffic travelling in both directions which pass a roadside observation point during a full year divided by 365 days for a given road segment. AADT volume is usually only available for major road segments and we did not have data for most local, collector and sub-arterial roads under municipal district control. We predicted volume estimates for all road segments in the study area with Random Forest regression \citep[see][]{brei01}. The dependent variable was 2013 AADT recorded by VicRoads on 3174 road segments. We included five predictor variables (\Cref{egk_variables}) that related to processes in traditional four-step traffic demand modelling (trip generation, trip distribution, mode choice and route assignment). All of the traffic model predictor variables were below a pairwise correlation threshold of 0.7 to reduce potential effects of multicollinearity. The traffic volume sub-model used the log-link function on the dependent variable (\Cref{egk_models}) due to the approximate log-normal distribution of AADT. We predicted AADT to all road segments using the model fit (\Cref{egk_tvolmap}).

\begin{figure*}[!h]
  \centering
  \includegraphics[scale=0.9]{egk_fig03b.png}
  \caption[Predicted relative traffic volume in Victoria]{Predicted relative traffic volume in number of vehicles day$^{-1}$ per road segment in study area. Darker shades indicate higher predicted traffic volumes (mean: 4481; range: 274--60850).}
  \label{egk_tvolmap}
\end{figure*}

Traffic speed was modelled and predicted using a similar methodology to the traffic volume model. As with the traffic volume data, we did not have access to municipal records, however, we required speed values for all road segments across Victoria. We obtained posted speed limit data, for the year 2014, for all major road segments (n=42,439) and used road density and road class predictors (\Cref{egk_variables}) in a linear regression model (\Cref{egk_models}) to predict speed for all road segments (\Cref{egk_tspdmap}). 

\begin{figure*}[!h]
  \centering
  \includegraphics[scale=0.9]{egk_fig04b.png}
  \caption[Predicted relative traffic speed in Victoria]{Predicted relative traffic speed in kilometres hour$^{-1}$ per road segment in study area. Darker shades indicate higher predicted traffic speeds (mean: 62; range: 42--106).}
  \label{egk_tspdmap}
\end{figure*}

\subsection{Collision Modelling}

To produce a kangaroo-vehicle collision dataset, we obtained records from the Wildlife Victoria database for a three year period between 1 January, 2010 and 1 January, 2013. The data for collisions were selected from a period where consistent techniques were used to collect and record the data.  Pre-2010 records exist, however, are sparse and more prone to error.  We first verified all automatically geocoded coordinates for accuracy using an online latitude/longitude mapping system (www.gpsvisualizer.com) and excluded all records with a geographical accuracy exceeding 300 metres. We selected road segments that were closest to the collision records in space and coded them with ones. To produce background points, we randomly selected approximately twice the number of collision-coded road segments and coded them with zeros. We combined the collision and background segments to produce a final dataset of 2264 collision and 4489 background segments. Using the same methodology, we developed an additional dataset of road segments for a period of one year between 1 January, 2013 and 1 January, 2014 for model validation (2125 collision and 4212 background records). Each segment contained predicted values for both traffic speed and volume from the previously described sub-models. As species occurrence predictions were expressed across a one square kilometre raster grid, we used the mid-point of each road segment for sampling the species occurrence sub-model predictions.

Our collision model fitted predicted values from the sub-models to collision or background occurrences at each road segment (\Cref{egk_models}). We used the fitted collision model to predict relative probabilities of collision on all road segments in the study area and, by using symbol classification (colour and line thickness) in \textit{QGIS} (www.qgis.org), produced a map (\Cref{egk_collmap}).

\begin{figure*}[!t]
  \centering
  \includegraphics[scale=0.9]{egk_fig05b.png}
  \caption[Map of collision risk per road segment in Victoria]{Map of collision risk per road segment. Darker shades indicate higher relative risk of collisions with kangaroos (mean: 0.24; range: 0.01--0.99).}
  \label{egk_collmap}
\end{figure*}

To determine the effectiveness of partitioning the variables into logical sub-models, we also modelled collision risk with a combined set of all variables (\Cref{egk_models}). That is, we analysed a logistic regression model that related collisions directly to the original variables used to model kangaroo occurrence, traffic volume and traffic speed. This emulated past work for purposes of comparison. We also compiled a list of variables used in seventy-one collision modelling papers (see \Cref{apx:D}) and verified that all of our predictors (with the exception of artificial light) were used in at least three or more published studies. The remaining highly-used variables were either irrelevant to our scope or difficult to obtain.

\section{Results}

The species occurrence model predicted the relative likelihood of kangaroo occurrence to vary (.002 to .986) across the study area using 5400 regression trees. The deviance explained by the model was approximately 30.4\% (null deviance = 0.572, estimated cross-validation (CV) deviance = 0.398 $\pm$0.008, CV AUC = 0.878 $\pm$0.006; see \Cref{egk_models}). The three most influential variables were artificial light (19.6\% contribution to model), elevation (18.4\%), and precipitation of the driest month (14.8\%). Partial dependence plots demonstrated plausible relationships between the predictors and grey kangaroo presence (\Cref{egk_sdm_vars}). We extended our model predictions to continental Australia (including areas well beyond Victoria) which aligned well with known grey kangaroo range (\Cref{egk_map_aus}). Approximately fifty percent of the grey kangaroo collision records were within areas where the predicted probability of grey kangaroo occurrence was above 0.2. Collision records in areas of lower predicted occurrence areas may be due to misclassification errors (i.e. misidentification of the species of kangaroo), reporting bias/spatial error in the collision data, or sampling bias in the data used to train the kangaroo occurrence model.

\begin{table}[!t]
\caption[Statistical models used in the conceptual framework]{Statistical models used in the conceptual framework. Deviance explained is the percentage of variance in the data explained by the model. The receiver operator characteristic (ROC) is the ability of the model to correctly identify collisions in independent datasets (1 being perfect, and 0.5 no better than random).}
\centering
\begin{tabularx}{0.9\textwidth}{lYll} \toprule
Model Type		    &Model &Deviance Explained	&ROC\\ \midrule 
Species Occurrence  &\scriptsize{$\Pr(\text{KANG}=1) \approx \mathrm{logit}^{-1}(\beta_0 + \beta_1\text{ELEV} + \beta_2\text{GREEN} + \beta_3\text{LIGHT} + \beta_4\text{MNTEMPWQ} + \beta_5\text{PRECDM} + \beta_6\text{SLOPE} + \beta_7\text{TREEDENS})$}        &30.4\%		&0.88\\
           &                                                                                &&\\
Traffic Volume    &\scriptsize{$\log(\text{AADT}) \approx \beta_0 + \beta_1\text{KMTODEV} + \beta_2\text{KMTOHWY} + \beta_3\text{POPDENS} + \beta_4\text{RDDENS} + \beta_5\text{RDCLASS}$} &54.4\% 	&---\\
           &                                                                                &&\\
Traffic Speed    &\scriptsize{$\text{SPEEDLMT} \approx \beta_0 + \beta_1\text{RDCLASS} + \beta_2\text{RDDENS}$}          &58.7\%	&---\\
           &                                                                                &&\\
Collision  &\scriptsize{$\mathrm{cloglog}(\Pr(\text{COLL}=1)) \approx \beta_0 + \beta_1\ln (\text{EGK}) + \beta_2\ln (\text{TVOL}) + \beta_3\ln (\text{TSPD})$}        &23.7\%	&0.81\\
           &                                                                                &&\\
Alternative Collision  &\scriptsize{$\mathrm{cloglog}(\Pr(\text{COLL}=1)) \approx \beta_0 + \beta_1\text{ELEV} + \beta_2\text{GREEN} + \beta_3\text{KMTODEV} + \beta_4\text{KMTOHWY} + \beta_5\text{LIGHT} + \beta_6\text{MNTEMPWQ} + \beta_7\text{POPDENS} + \beta_{8}\text{PRECDM} + \beta_{9}\text{RDCLASS} + \beta_{10}\text{RDDENS} + \beta_{11}\text{SLOPE} + \beta_{12}\text{TREEDENS}$}        &24.9\%	&0.84\\
\bottomrule
\end{tabularx}
\label{egk_models}
\end{table}

\begin{figure*}[!h]%
  \centering%
  \subfloat[]{\includegraphics[width=0.3\textwidth]{egk_light.png}}%
  \subfloat[]{\includegraphics[width=0.3\textwidth]{egk_elev.png}}%
  \subfloat[]{\includegraphics[width=0.3\textwidth]{egk_precdm.png}}%
  \caption[Effects of predictor variables on relative likelihood of grey kangaroo occurrence]{Effects of three most influential predictors on relative likelihood of grey kangaroo occurrence. Grey kangaroo occurrence is expressed on a relative probability scale. Artificial Light (LIGHT) is a surrogate for urban development where higher positive values represent intensely urbanised and populated areas. Elevation (ELEV) is measured in metres above sea level. Precipitation of the driest month (PRECDM) is the mean amount of rainfall in the summer expressed in millimetres.}%
  \label{egk_sdm_vars}
\end{figure*}

\begin{figure*}[!t]
  \centering
  \includegraphics[scale=.5]{egk_MAP_AUS.png}
  \caption[Predicted relative likelihood of grey kangaroo occurrence across Australia]{Predicted relative likelihood of grey kangaroo occurrence across Australia. Darker shades indicate higher relative likelihood of occurrence (mean: 0.023; range: 0.001--0.986). The dashed line represents the existing known range of grey kangaroos as reported by the International Union for the Conservation of Nature. Victoria is located in the south-east corner of the mainland.}
  \label{egk_map_aus}
\end{figure*}

The traffic model volume explained 54.4\% of the variation in the AADT data and the traffic speed model explained 58.7\% of the variation in the posted speed limit data (\Cref{egk_models}).  Each model used the default value of 500 trees to fit the data.  All predictor variables used in the traffic models demonstrated plausible relationships to both AADT and speed (\Cref{egk_trans_vars}). Traffic volume showed a generally decreasing trend with increasing distance to development and major thoroughfares and a generally increasing trend with increasing population and road density. Traffic speed showed a generally decreasing trend with increasing road density. Both traffic speed and volume decreased as road class changed from `freeway' to `local'. The most influential variable for traffic volume was road class (33.8\% relative contribution to reduction of variance), followed by distance to urban development (19.7\%) and distance to freeways and highways (16.4\%). For traffic speed, the most influential variable was road class, however, road density was nearly as influential.

\begin{figure*}[!t]
  \centering
  \subfloat[]{\includegraphics[width=0.3\textwidth]{egk_popdens_volume.png}\label{egk_trans_vars:a}}
  \subfloat[]{\includegraphics[width=0.3\textwidth]{egk_rddens_volume.png}\label{egk_trans_vars:b}}
  \subfloat[]{\includegraphics[width=0.3\textwidth]{egk_rdclass_volume.png}\label{egk_trans_vars:c}}\\
  \subfloat[]{\includegraphics[width=0.3\textwidth]{egk_kmtodev_volume.png}\label{egk_trans_vars:d}}
  \subfloat[]{\includegraphics[width=0.3\textwidth]{egk_kmtohwy_volume.png}\label{egk_trans_vars:e}}\\
  \subfloat[]{\includegraphics[width=0.3\textwidth]{egk_rddens_speed.png}\label{egk_trans_vars:f}}
  \subfloat[]{\includegraphics[width=0.3\textwidth]{egk_rdclass_speed.png}\label{egk_trans_vars:g}}
  \caption[Effects of predictor variables on traffic volume and speed]{Effects of predictor variables on traffic volume and speed. AADT is the predicted mean average annual daily traffic in vehicles day$^{-1}$. SPEED is the predicted traffic speed in km hour$^{-1}$. POPDENS is population density in persons square kilometre$^{-1}$. RDDENS is road density per road segment in kilometres square kilometre$^{-1}$. KMTODEV is road segment distance to urban land use in kilometres. KMTOHWY is road segment distance to freeway/highway in kilometres.  RDCLASS is designated road segment functional classification.}
  \label{egk_trans_vars}
\end{figure*}

\begin{table}[!t]
\caption[Summary of collision and alternative collision models]{Summary of collision and alternative collision model fits. Coefficients and significance of variables are shown with relative contribution to model fit. Highly significant variables are marked with an asterisk. Contribution of variables to reduction in deviance (via analysis of variance) are expressed as decimal percentages (ANOVA).}
\centering
\begin{tabularx}{0.9\textwidth}{lllllll} \toprule
Model Type &Variable         &Coefficient &Std. Error			&Wald-$\chi^2$ &$\PRZ$ &ANOVA\\ \midrule 
Collision  				&\emph{Intercept}	& $-$12.82			& 0.6635   	& $-$19.33	&\TL 2e$^{-16}$*	& --- \\
           				& EGK				& 0.6583			& 0.0206   	& 32.01		&\TL 2e$^{-16}$*	& 0.7268 \\
           				& TVOL				& 0.2715			& 0.0252   	& 10.77		&\TL 2e$^{-16}$*	& 0.0005 \\
           				& TSPD				& 2.694				& 0.1308   	& 20.59		&\TL 2e$^{-16}$*	& 0.2726 \\
           				&                  	&                 	&         	&         	&       	& \\
Alternative Collision	& \emph{Intercept} 	& $-$2.567 			& 0.2642	& $-$9.716 	&\TL 2e$^{-16}$*	& --- \\ 
   						& ELEV				& 0.003177			& 0.0002233	& 14.23 	&\TL 2e$^{-16}$*	& 0.1729 \\ 
   						& GREEN				& 1.405				& 0.344 	& 4.085		& 4.42e$^{-5}$*	& 0.0011 \\ 
   						& KMTODEV			& $-$0.02784			& 0.002097 	& $-$13.28 	&\TL 2e$^{-16}$*	& 0.2079 \\ 
   						& KMTOHWY			& 0.006001			& 0.004206 	& 1.427 	& 0.1537 	& 0.0004 \\ 
   						& LIGHT				& 0.004142			& 0.002046 	& 2.025 	& 0.043 	& 0.0119 \\ 
   						& MNTEMPWQ			& 0.1519			& 0.01673 	& 9.077 	&\TL 2e$^{-16}$*	& 0.0398 \\ 
   						& POPDENS			& $-$0.0006986		& 0.0000535	& $-$13.06 	&\TL 2e$^{-16}$*	& 0.0922 \\ 
   						& PRECDM			& 0.02573			& 0.003404 	& 7.559 	& 4.05e$^{-14}$*	& 0.0483 \\ 
   						& RDCLASS			& $-$0.4166			& 0.01467 	& $-$28.4 	&\TL 2e$^{-16}$*	& 0.4205 \\ 
   						& RDDENS			& 0.02353			& 0.008762 	& 2.686 	& 0.0072 	& 0.0038 \\ 
   						& SLOPE				& 0.008252			& 0.00739 	& 1.117 	& 0.2641	& 0.0004 \\ 
   						& TREEDENS			& $-$0.1761			& 0.1387 	& $-$1.27 	& 0.2041	& 0.0008 \\ 
\bottomrule
\end{tabularx}
\label{egk_sum_coll}
\end{table}

The collision model explained 23.7\% of the deviance (\Cref{egk_models}). Using the independent dataset to verify the predictive accuracy of the model resulted in an ROC score of 0.81.  All three variables were highly significant ($p<$.0001) with traffic speed and grey kangaroo occurrence contributing the most to overall reduction in deviance; 27.3\% and 72.7\%, respectively.  The Akaike Information Criterion (AIC) score -- measuring the quality of the model given the data and the parameters -- was 6,579.  All predictor variables demonstrated logical relationships to collision likelihood in the partial dependency plots (\Cref{egk_effects}). The rapid ascent and gradual levelling off of collision risk to increasing traffic volume suggests a threshold of around 2,000 vehicles day$^{-1}$.

\begin{figure*}[!t]
  \centering
  \subfloat[]{\label{egk_effects:a}\includegraphics[width=0.3\textwidth]{egk_fig06a.png}}
  \subfloat[]{\label{egk_effects:b}\includegraphics[width=0.3\textwidth]{egk_fig06b.png}}
  \subfloat[]{\label{egk_effects:c}\includegraphics[width=0.3\textwidth]{egk_fig06c.png}}
  \caption[Effects of predictor variables on relative likelihood of collision]{Effects of predictor variables on relative likelihood of collision. EGK is the relative likelihood of kangaroo occurrence. TVOL is the predicted daily traffic volume in vehicles day$^{-1}$. TSPD is the predicted traffic speed in kilometres hour$^{-1}$.}
  \label{egk_effects}
\end{figure*}

Our alternative collision model (all sub-model predictor variables combined in a single model) explained 24.9\% of the deviance; less than 2\% more than the four-model framework.  The model AIC was 6,496.2. The most influential variables were road class (42.1\% relative contribution to reduction of deviance), distance to urban development (20.8\%), elevation (17.3\%) and population density (9.2\%). We used the same independent dataset to verify the predictive accuracy of the model, resulting in an ROC score of 0.84.

\section{Discussion}

This research introduces a new conceptual framework for predicting risk of wildlife-vehicle collisions. It is distinct from other collision models in its treatment of the analytical modelling framework. Where other research has related multiple environmental and anthropogenic variables to collision occurrence in a single statistical model \citep[e.g.][]{lee04,ramp05,kloc06,litv08,guns09,roge09} and predicted risk from model fits \citep[e.g.][]{malo04,sudh09,guns12}, we separate and develop the predictors in sub-models that would most suitably inform management. Each sub-model may be independently scrutinised for bias, uncertainty, and spatial autocorrelation and tuned accordingly. In this case study, our approach identifies relationships between, and effects of, species presence and traffic volume/speed on collision risk to grey kangaroos. In a similar manner, \cite{baud13} developed an index of co-occurrence to assess collision risk between manatees and recreational watercraft, however, the management implications were not extensively discussed. Particular to our study, road managers and environmental managers may be interested in whether to reduce collisions by focusing on the road environment \citep[e.g.][]{clev02,jaeg04,bond08,bond13}, traffic conditions, or species \citep[e.g.][]{huij03,huij06}. Both the collision and alternative collision models had similar fits and made similar predictions. However, we argue that cause and effect is easier to interpret when using our proposed framework. For example, the variable of population density (human) is shown to contribute to collisions, however, it is unclear to what degree it is correlated with grey kangaroo occurrence, traffic conditions, or both, in the alternative collision model.

The models used in the study were all correlative and assumed static equilibrium in the environment. Temporal patterns of wildlife-vehicle collisions exist and would be useful to incorporate into the collision model, but the variable of time was not easily integrated into this study. The original collision dataset indicated that the lowest number of incidents occurred in summer (December--February) and the highest were reported in August. The highest reporting times were consistent with the crepuscular (active between dusk and dawn) nature of kangaroo movements \citep{mccu00} and peaked at approximately 7:00 and 17:00 hours. Comparing the model performance on data accounting for time of day and of year is an area for future research. Other modelling methods that explicitly address interactions between space and time exist, such as multi-dimensional Poisson process models, and would be useful to incorporate into the framework.

This study demonstrates the usefulness of existing sources of data for scientific research and environmental management. Data provided by non-government organisations such as Wildlife Victoria, are not only inexpensive to collate, but are also valuable ecological indicators and sources of information, in particular, on species distributions. However, it is cautioned that use of such data should be subject to rigorous quality control and verification. A large amount of entered records were not useful for the scale of this study due to incomplete data or geographical ambiguity. Fauna atlases have the same potential issues. Many of the records in the Victorian Biodiversity Atlas were subject to geographical bias (e.g. close to roads and towns) and some potential inaccuracies (e.g. misidentification of species or approximate location). \cite{grah04} elaborates on the use of such data in scientific studies. Moreover, testing the model with alternative sources of collision reports (e.g. insurance records) would help account for biases present in the data. Although under-reporting of collision data has limited effects on model robustness, spatial biases can adversely affect model performance \citep{snow15}.

Our statistical modelling methods were chosen to match the data and purpose \citep[see][]{wint05,guil15}, and are based on well-established analyses in the modelling literature. However, our framework is not restricted to the particular models that we built; any appropriate statistical methods may be used for each of the sub-models. For example, boosted regression trees are a well developed method for species distribution modelling \citep{elit08} but may also be suited to predict traffic volume and speed. Further exploration of modelling methods, including those based on machine learning, may result in better calibrated models with less uncertainty and more robust inferences. The use of mechanistic models to explain population dynamics (requiring collection of additional information such as age, sex, and size of species involved in wildlife-vehicle collisions) may be beneficial as the integration of Population Viability Analysis into species distribution and collision models can be informative \citep[see][]{tyre01,elit10,pola14}.

All of the models with binary dependent variables were subject to spatial autocorrelation producing potentially biased standard errors and predictions of grey kangaroo presence and collision risk. Techniques to incorporate autocovariates in the models similar to \cite{cras12} might improve the predictive performance but may only be logical for particular species (e.g. ranging or nomadic species). Further, we chose to test the model framework using a binomial dependent variable, however, it can be adapted to analyse other data types such as non-negative integers (e.g. number of collisions) or categorical information (e.g. groupings of number of collisions). These alternative methods might usefully identify hotspots (i.e. sites with multiple collisions) at smaller scales, such as road segments, or simulate effects of planned road expansion projects. For example, population effects on target species can be quantified and estimated based on predicting the magnitudes of collision hotspots.
 
\subsection{Applied Management Implications}
To realise the full potential use by managers, a model framework must be conceptually simple, flexible and adaptable. All of the input data must be accessible and the framework must allow inferences which are relevant and draw conclusions which are tractable. Our study uses vehicle collisions with grey kangaroos in Australia to demonstrate this analytic framework, however, we envision extending the model to (i) explore and predict risk to other species (e.g. wombats or deer) arising from vehicle collisions by identifying and quantifying probabilities of occurrence using alternative species distribution models and (ii) quantify risks to wildlife arising from other anthropogenic threats such as linear infrastructure (e.g. electrocutions), pollution (e.g. entanglements) or introduction of domestic predators (e.g. dog and cat attacks). Managers can use our framework across several disciplines, and it would benefit from additional testing across several spatial and temporal scales to determine its full potential flexibility and generality.