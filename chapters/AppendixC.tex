\chapter{Data sources used for research}\label{apx:C}
\newpage

The following sections provide information on the sources of electronic data used in this thesis. Whilst most of the data is publicly accessible, some require additional permission for acquisition and use; these are identified in the descriptions. Any data modifications required to conform with project requirements are also described.

\section{Wildlife Victoria}

Wildlife Victoria is a not-for-profit organisation that provides emergency response service for sick, injured, or orphaned native wildlife. It has operated under the vision of living in a community that cares about the welfare of Australian wildlife for nearly thirty years. The organisation uses a professional database system to record details of all incidents reported across the State of Victoria. This database has more that 100,000 records, most of which have spatial and temporal data. This information is obtainable from Wildlife Victoria but remains their sole intellectual property and all uses of the data are subject to approval (approvals were obtained for the work in this thesis). More information on Wildlife Victoria can be found at \textit{https://www.wildlifevictoria.org.au}.

\section{California Roadkill Observation System}

The California Roadkill Observation System (CROS) is a web-based system used to catalogue observations of wildlife killed on roads throughout the state. The system is maintained by the Road Ecology Center based at the University of California, Davis and allows members of the public to enter observations from the field. At this time of writing, the system has 52,925 observations of 421 unique species originating from 1301 registered observers. More information on CROS can be found at \textit{http://www.wildlifecrossing.net/california}.

\section{VicRoads}

VicRoads is responsible for planning, developing, and maintaining more than 50,000 kilometres of major arterial roads in the State of Victoria. Further, the government organisation is also responsible for registering vehicles and licensing motorists to operate within the state. VicRoads hire independent cleaning contractors to remove debris from major sections of roads, including wildlife carcasses, and this information is recorded - albeit scarcely. More information on VicRoads can be found at \textit{https://www.vicroads.vic.gov.au}.

\section{V/Line}

V/Line is Australia's largest regional public transport operator scheduling more than 1,700 train services between the urban centre of Melbourne and regional Victoria. V/line also maintains over 3,500 kilometres of railway used by passenger and freight transport throughout the state. Collisions with large species are recorded by train drivers as they have implications for train maintenance and safety. Although this information is obtainable from V/Line, it remains their sole intellectual property and all uses of the data are subject to approval (approvals were obtained for this thesis). More information on V/Line can be found at \textit{https://www.vline.com.au}.

\section{City of Bendigo}

The City of Bendigo is a large town located in regional Victoria and the third most populated urban area in the state (>100,000 persons). The council is responsible for litter abatement in public spaces (i.e. nature strips) and responds to reports from citizens. Kangaroo roadkill are commonly reported and the council maintains records of locations and times of these incidents. This information may be requested from the public works department. More information on Bendigo can be found at \textit{https://www.bendigo.vic.gov.au}.

\section{Crashstats}

Crashstats is a web-accessible, online repository of statistics on road crashes maintained by VicRoads. These data are based on report made to the Victorian Police and usually involve human injuries and/or deaths. More information on Crashstats, and access to data, can be found at \textit{https://www.vicroads.vic.gov.au/safety-and-road-rules/safety-statistics/crash-statistics}.

\section{Insurance Australia Group}

Insurance Australia Group is a large insurance underwriter that operates throughout Australia, New Zealand and Asia. They provide services for several motor-vehicle insurance agencies and therefore receive access to vast amounts of claim data resulting from collisions. One significant component of this data are records on collisions with wildlife, albeit spatial uncertainty is quite high and therefore aggregated to the town level. Although this information is obtainable from IAG, it remains their sole intellectual property and all uses of the data are subject to approval (approvals were obtained for this thesis). More information on IAG can be found at \textit{https://www.iag.com.au}.

\section{Worldclim}

WorldClim

http://www.worldclim.org

\section{MODIS}

https://modis.gsfc.nasa.gov

\section{LandSat}

https://landsat.gsfc.nasa.gov

\section{NASA OLS DMSP}

https://ngdc.noaa.gov/eog/dmsp.html

\section{Aussie Grass}

https://www.longpaddock.qld.gov.au/about/researchprojects/aussiegrass

\section{Geoscience Australia}

http://www.ga.gov.au/data-pubs

\section{Victorian Biodiversity Atlas (VBA)}

http://www.depi.vic.gov.au/environment-and-wildlife/biodiversity/victorian-biodiversity-atlas

\section{Global Biodiversity Information Facility (GBIF)}

http://www.gbif.org

\section{United States Geologic Survey (USGS)}

https://www.usgs.gov

HydroSHEDS

SRTM

\section{Global Land Cover Facility (GLCF)}

http://glcf.umd.edu/data/

\section{VicMap}

http://www.depi.vic.gov.au/forestry-and-land-use/spatial-data-and-resources/vicmap

\section{Australian Bureau of Statistics}

http://www.abs.gov.au

\section{SoilGrids}

http://www.isric.org/content/soilgrids
