\chapter{Data sources used for research}\label{apx:C}
\newpage

The following sections provide information on the sources of electronic data used in this thesis. Whilst most of the data is publicly accessible, some require additional permission for acquisition and use; these are identified in the descriptions. Any data modifications required to conform with project requirements are also described.

\section{Wildlife Victoria}

Wildlife Victoria is a not-for-profit organisation that provides emergency response service for sick, injured, or orphaned native wildlife. It has operated under the vision of living in a community that cares about the welfare of Australian wildlife for nearly thirty years. The organisation uses a professional database system to record details of all incidents reported across the State of Victoria. This database has more that 100,000 records, most of which have spatial and temporal data. This information is obtainable from Wildlife Victoria but remains their sole intellectual property and all uses of the data are subject to approval (approvals were obtained for the work in this thesis). More information on Wildlife Victoria can be found at \textit{https://www.wildlifevictoria.org.au}.

\section{California Roadkill Observation System}

The California Roadkill Observation System (CROS) is a web-based system used to catalogue observations of wildlife killed on roads throughout the state. The system is maintained by the Road Ecology Center based at the University of California, Davis and allows members of the public to enter observations from the field. At this time of writing, the system has 52,925 observations of 421 unique species originating from 1301 registered observers. More information on CROS can be found at \textit{http://www.wildlifecrossing.net/california}.

\section{VicRoads}

VicRoads is responsible for planning, developing, and maintaining more than 50,000 kilometres of major arterial roads in the State of Victoria. Further, the government organisation is also responsible for registering vehicles and licensing motorists to operate within the state. VicRoads hire independent cleaning contractors to remove debris from major sections of roads, including wildlife carcasses, and this information is recorded -- albeit scarcely. More information on VicRoads can be found at \textit{https://www.vicroads.vic.gov.au}.

\section{V/Line}

V/Line is Australia's largest regional public transport operator scheduling more than 1,700 train services between the urban centre of Melbourne and regional Victoria. V/line also maintains over 3,500 kilometres of railway used by passenger and freight transport throughout the state. Collisions with large species are recorded by train drivers as they have implications for train maintenance and safety. Although this information is obtainable from V/Line, it remains their sole intellectual property and all uses of the data are subject to approval (approvals were obtained for this thesis). More information on V/Line can be found at \textit{https://www.vline.com.au}.

\section{City of Bendigo}

The City of Bendigo is a large town located in regional Victoria and the third most populated urban area in the state (>100,000 persons). The council is responsible for litter abatement in public spaces (i.e. nature strips) and responds to reports from citizens. Kangaroo roadkill are commonly reported and the council maintains records of locations and times of these incidents. This information may be requested from the public works department. More information on Bendigo can be found at \textit{https://www.bendigo.vic.gov.au}.

\section{Crashstats}

Crashstats is a web-accessible, online repository of statistics on road crashes maintained by VicRoads. These data are based on report made to the Victorian Police and usually involve human injuries and/or deaths. More information on Crashstats, and access to data, can be found at \textit{https://www.vicroads.vic.gov.au/safety-and-road-rules/safety-statistics/crash-statistics}.

\section{Insurance Australia Group}

Insurance Australia Group is a large insurance underwriter that operates throughout Australia, New Zealand and Asia. They provide services for several motor-vehicle insurance agencies and therefore receive access to vast amounts of claim data resulting from collisions. One significant component of this data are records on collisions with wildlife, albeit spatial uncertainty is quite high and therefore aggregated to the town level. Although this information is obtainable from IAG, it remains their sole intellectual property and all uses of the data are subject to approval (approvals were obtained for this thesis). More information on IAG can be found at \textit{https://www.iag.com.au}.

\section{Victorian Biodiversity Atlas}

The Victorian Biodiversity Atlas (VBA) contains high-quality records of flora and fauna occurrences throughout the state and is maintained by the Arthur Rylah Institute, a scientific research branch of the Victorian State Government. More information is available at \textit{http://www.depi.vic.gov.au/environment-and-wildlife/biodiversity/victorian-biodiversity-atlas}.

\section{Global Biodiversity Information Facility}

The Global Biodiversity Information Facility (GBIF) is an international collaboration and government-funded open data infrastructure containing information about all types of life on Earth. It is the world's largest online repository of information, detailing 1.6 million species collected over three centuries. The system provides spatial and temporal records of species occurrence based on observations from citizen scientists, researchers and automated monitoring programmes, and museum specimen archives. More information and data are available at \textit{http://www.gbif.org}.

\section{WorldClim}

WorldClim is a collection of global climatic data. The data are represented as 1km$^2$ raster grids and include information on bioclimatic conditions for the past, present, and future. Interpolated temperature and rainfall data are also included in the collection. \cite{hijm05} details the methods used to construct the data. These data are made available at \textit{http://www.worldclim.org}. This thesis uses WorldClim data on isothermality, mean temperature of the wettest quarter, precipitation of the driest month ,seasonality of precipitation, and annual temperature range.

\section{Moderate Resolution Imaging Spectroradiometer}

The Moderate Resolution Imaging Spectroradiometer (MODIS) is a remote-sensing instrument carried on-board the two satellites, Terra and Aqua. These two satellites view the entire Earth's surface every 1 to 2 days and improve our understanding of global dynamics and processes. The work in this thesis uses one of the MODIS vegetation indices, the normalized difference vegetation index (NDVI), to represent relative quality of vegetation. More information and data are available at \textit{https://modis.gsfc.nasa.gov}. These data are provided in 16-day intervals and at multiple spatial resolutions, however, 1km$^2$ resolution (MOD13A2) is used in this thesis. To represent the seasonal change in quality of vegetation (greenness variable), I averaged the differences between January and July measures of NDVI over a ten-year period (2000--2010) that spanned drought and non-drought conditions in Australia.

\section{Defense Meteorological Satellite Program}

The Defense Meteorological Satellite Program (DMSP), operated by the National Aeronautics and Space Administration (NASA) and National Oceanic and Atmospheric Administration (NOAA) records global nighttime artificial lighting. The data for all available DMSP-OLS smooth resolution imagery are used to produce cloud-free composites. Sunlit and glare data are excluded based on solar elevation angles. More information and data are available at \textit{https://ngdc.noaa.gov/eog/dmsp.html}. In this thesis, I used data for the year 2013.

\section{Australian Grassland and Rangeland Assessment by Spatial Simulation}

The Australian Grassland and Rangeland Assessment by Spatial Simulation (Aussie GRASS) is used to monitor, at regional scales, pasture degradation and recovery using modelling that accounts for daily rainfall, evaporation, temperature, vapour pressure and solar radiation. It also accounts for soil and pasture types, tree and shrub cover, and abundance of domestic livestock and other key herbivores. The methods are detailed in \cite{cart03}. Although this information is obtainable from the Queensland Government, it remains their sole intellectual property and all uses of the data are subject to approval (approvals were obtained for this thesis). More information on Aussie GRASS can be found at \textit{https://www.longpaddock.qld.gov.au/about/researchprojects/aussiegrass}.

\section{Geoscience Australia}

Geoscience Australia provides information and scientific data on the geology and geography of Australia. Data can be downloaded from \textit{http://www.ga.gov.au/data-pubs}. I used a 9-second digital elevation model (DEM) to represent elevation for Australia in this thesis; methods are detailed in \cite{hutc91,hutc11}. Slope aspect values were derived from the DEM by using a geographic information system (GIS).

\section{United States Geologic Survey}

The United States Geologic Survey (USGS) provides information and scientific data on the geology and geography of the United States with select extended datasets describing global features. More information and data are available at \textit{https://www.usgs.gov}. I used data from the USGS to represent two variables in the thesis. Hydrological flow accumulation was obtained from HydroSHEDS (\textit{https://hydrosheds.cr.usgs.gov}), a mapping product that provides hydrographic information at regional and global scales. This data was available at 15 arc-second resolution and was projected and re-scaled to a 1km$^2$ resolution. I used SRTM (Shuttle Radar Topography Mission) 3 arc-second, void-filled elevation data (\textit{https://lta.cr.usgs.gov/SRTM1Arc}) to generate elevation variables for both California and Australia in this thesis.

\section{Global Land Cover Facility}

The Global Land Cover Facility (GLCF) provides data and products to describe global environmental systems and is based on remotely-sensed satellite data with a focus on land cover at different scales. More information and data are available at \textit{http://glcf.umd.edu}. I obtained a 30-metre resolution tree cover layer containing estimated percentage of horizontal ground in each pixel covered by woody vegetation greater than 5 meters in height; originally derived from Landsat Vegetation Continuous Fields (VCF) -- details in \cite{sext13}. For the tree cover variable used in this thesis, I spatially aggregated the data to a 1km$^2$ resolution.

\section{VicMap}

The VicMap collection of electronic spatial information is maintained by the Victorian Government and contains GIS data on many geographic features used in urban and regional planning and research. Information one these offerings is provided at \textit{http://www.depi.vic.gov.au/forestry-and-land-use/spatial-data-and-resources/vicmap}. In this thesis, I obtained VicMap spatial data for roads, water features, and administrative boundaries such as towns. Using these features, I calculated distances to towns, roads, and linear waterways with geoprocessing tools and represented the outputs as 1km$^2$ resolution spatial grids.

\section{SoilGrids}

SoilGrids is a system for global soil mapping based on predictions from machine learning and statistics, maintained by the International Soil Reference and Information Centre (ISRIC). The methods are described in \cite{heng14}, and data can be obtained at \textit{http://www.isric.org/content/soilgrids}. Soil predictions are made at a 1km$^2$ resolution scale which I use to represent variables pertaining to soil density, conductivity, and content (e.g. sand or clay) in this thesis.

\section{Australian Bureau of Statistics}

The Australian Bureau of Statistics (ABS) is a government agency that provides data and reports on economic, social, population and environmental topics. More information and data are available at \textit{http://www.abs.gov.au}. Data obtained from the ABS were joined to spatial data in this thesis to create a population density variable. I used pycnophylactic interpolation \citep{tobl79} to obtain pixel-based population and income estimates whilst accounting for irregular administrative boundaries (e.g. towns).

\section{United States Census Bureau}

The United States Census Bureau is the leading source for demographic and economic data about the United States which guides a significant portion of government spending ($\approx$ 400 billion USD). More information and data are available at \textit{https://www.census.gov}. For the California-based work this thesis, data obtained from the Census Bureau were joined to spatial data to create a population density variable. I used pycnophylactic interpolation to obtain pixel-based population and income estimates at 1km$^2$ resolution (see previous section on data from the Australian Bureau of Statistics).
