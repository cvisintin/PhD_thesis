\chapter{Species used in research}\label{apx:B}
\newpage

The following sections provide information on the species used in this research. Unless otherwise cited, all of the descriptions provided were synthesised from information found in \cite{vand08}. All of the images are used under the Creative Commons License, version 4.0.

\section{Mule deer}\label{deer}
\vspace{-0.3cm}
\setlength\intextsep{0pt}
\begin{wrapfigure}{l}{0.29\textwidth}
\centering
\includegraphics[width=0.27\textwidth]{deer_list_sc.jpg}
\end{wrapfigure}
Mule deer (\textit{Odocoileus hemionus}) are placental mammals that belong to a family of hoofed species that annually shed their antlers. They are of considerable size, males averaging between seventy-five and one-hundred kilograms (females are approximately seventy-five percent the weight). The fur of mule deer changes colour seasonally, from blue-gray in winter to reddish-brown in summer. Mule deer travel in herds during migration, however, males are generally solitary. Their home ranges is influenced by habitat quality and season and may cover between 100--5000 hectares. Due to limited ability to digest fibrous roughage, Mule deer require easily digestible forage that includes leaves, needles, stems, fruits/nuts, shrubs and grasses -- both living and dead. Occurring across most of western North America, except in tundra, sub-tropic, and extreme desert regions, Mule deer are quite adaptable to most habitats, however, migrate seasonally. Mule deer are subject to population management through harvesting/hunting and culling, especially when high densities initiate human-wildlife conflicts. The proceeding information and more details about Mule deer can be found in \cite{ferg05}.

\section{Eastern grey kangaroos}\label{egk}
\vspace{-0.3cm}
\setlength\intextsep{0pt}
\begin{wrapfigure}[5]{l}{0.29\textwidth}
\centering
\includegraphics[width=0.27\textwidth]{egk_list_sc.jpg}
\end{wrapfigure}
The Eastern Grey Kangaroo (\textit{Macropus giganteus}) is the second-largest native terrestrial mammal in Australia. Males can weigh up to eighty-five kilograms and have body lengths of over three metres (including tail). Despite the name, Eastern Greys may also have significant patches of light brown or cream colouring, especially on the underbelly. As with most other Australian mammals, they are marsupials and carry young in pouches. They travel in groups, called 'mobs', and have stable home ranges that vary between 30--1600 hectares. Eastern Greys are grazing animals and consume primarily grasses. Their distribution is quite broad, extending across a significant portion of eastern Australia from Cape York to north-east Tasmania. This area covers a wide range of habitat types including woodlands, shrub and heathland, and also human-modified landscapes such as golf courses and urban parklands. If conditions are right, Eastern Grey Kangaroos can increase to densities exceeding three individuals per hectare. Therefore, populations are often managed by government agencies through culling, sterilisation, and/or harvesting.

\section{Swamp wallabies}\label{bsw}
\vspace{-0.3cm}
\setlength\intextsep{0pt}
\begin{wrapfigure}{l}{0.29\textwidth}
\centering
\includegraphics[width=0.27\textwidth]{bsw_list_sc.jpg}
\end{wrapfigure}
The Swamp Wallaby (\textit{Wallabia bicolor}) is the only member in genus \textit{Wallabia} due to its unique physical and behavioural characteristics. It is a medium-sized marsupial, weighing between ten and twenty kilograms and well over a metre long (including tail). Swamp Wallabies also go by the name 'black wallabies', possibly due to appearing much darker than other wallabies, but their fur is normally brownish-gray or reddish-brown. As they are strictly solitary animals, it is quite rare to observe Swamp Wallabies in groups. Shrubs comprise the main source of food for Swamp Wallabies, both native and exotic, although they also occasionally forage on underground fungi. They are widely distributed along the eastern coastline of Australia as four sub-species. Swamp wallabies are occasionally culled to control populations, but the course fur and small body size make them unattractive for commercial harvesting.

\newpage
\section{Wombats}\label{wom}
\vspace{-0.3cm}
\setlength\intextsep{0pt}
\begin{wrapfigure}{l}{0.29\textwidth}
\centering
\includegraphics[width=0.27\textwidth]{wom_list_sc.jpg}
\end{wrapfigure}
Common Wombats (\textit{Vombatus ursinus}) are robust, medium-sized marsupial mammals weighing between twenty and forty kilograms and having a body length of around a metre. Unlike most other ground or tree-dwelling marsupials, wombats live in underground burrows (sometimes up to twenty metres long). Their fur color is grayish-brown and does not vary much between individuals. Common Wombats are solitary and visit multiple burrows over several weeks. Fibrous grasses are the main food source of Common Wombats. Depending on the distribution of food resources, their home range will vary between two and eighty hectares. They occur predominantly in the south-east of Australia, beginning in the southern Queensland and continuing down through eastern Victoria. Although not officially managed wildlife, Common Wombats are often threatened by domesticated dogs and persecuted by humans.

\section{Koalas}\label{koa}
\vspace{-0.3cm}
\setlength\intextsep{0pt}
\begin{wrapfigure}{l}{0.29\textwidth}
\centering
\includegraphics[width=0.27\textwidth]{koa_list_sc.jpg}
\end{wrapfigure}
Koalas (\textit{Phascolarctos cinereus}) are perhaps one of Australia's most iconic animal. Often mistakenly referred to a 'koala bears', Koalas have no relationship to bears and are more closely related to wombats. They are the largest Australian arboreal marsupial, weighing between four and ten kilometres. Koalas in the north of the continent have short, light gray fur whilst their southern counterparts have longer grayish-brown fur. As they are solitary and territorial, it is uncommon to find several Koalas in a single tree. The Koala diet consists, almost exclusively, of eucalypt foliage -- nearly half of a kilogram per day. Home ranges of Koalas, which vary based on food availability, are between one and two hectares. They have a broad distribution along the eastern and south-eastern Australian coastlines extending from the southern part of Cape York to just over the state border between Victoria and South Australia. For many reasons, including threats to persistence and charismatic status, Koalas are a primary species involved in wildlife management operations in three states of Australia.

\section{Brushtail possums}\label{btp}
\vspace{-0.3cm}
\setlength\intextsep{0pt}
\begin{wrapfigure}{l}{0.29\textwidth}
\centering
\includegraphics[width=0.27\textwidth]{btp_list_sc.jpg}
\end{wrapfigure}
Common Brushtail Possums (\textit{Trichosurus vulpecula}) are small-medium marsupial mammals that are highly adapted to urban environments. Animals can weigh between one and five kilograms depending on geographic location, smaller sizes occurring in northern Australia and larger sizes in the south. Fur color also varies quite considerably by location but the most familiar is gray with a white underbelly and black brushy tail (hence the name). Possums are normally solitary and establish territories over relatively small home ranges. Their primary native diet is leaves, flowers, and fruits, although in highly modified landscapes, Brushtail Possums have resorted to omnivorous and opportunistic additions such as bird eggs, human-discarded food scraps, and even invertebrates. Brushtail Possums are widespread throughout Australia and occur in many habitats, albeit preferring dry eucalypt forests. In urban environments, their densities can exceed four individuals per hectare, thereby resulting in conflicts with human inhabitants.

\section{Ringtail possums}\label{rtp}
\vspace{-0.3cm}
\setlength\intextsep{0pt}
\begin{wrapfigure}{l}{0.29\textwidth}
\centering
\includegraphics[width=0.27\textwidth]{rtp_list_sc.jpg}
\end{wrapfigure}
Common Ringtail Possums (\textit{Pseudocheirus peregrinus}) are small marsupial mammals and often found in suburban gardens. They often weigh less than one kilogram and are no more than seven-hundred centimetres long (half of which is tail). Ringtail Possums use their distinctively curled tails as an effective 'fifth limb' similar to primates. They have relatively short fur that is brownish-grey in colour, with white on the underbelly and tip at the end of the tail. Ringtail Possums are strictly vegetarian and typically forage on leaves, flowers, and fruit. Although spanning several different habitat types, the species only occurs along the eastern seaboard of Australia, with limited distribution inland of the Great Dividing Range, and within the state of Tasmania. Ringtail Possums adapt well to urban environments and, whilst considered annoying to some residential gardeners, do not cause any significant issues.