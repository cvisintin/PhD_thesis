\chapter{General Discussion}\label{sec:conc}
\newpage

The work featured in this thesis presents a modelling framework for determining wildlife-vehicle collision risk that may be used without taxonomic or geographic restriction by wildlife and road managers.

Three main results can be drawn from the chapters.
First, speed is a considerable hazard for wildlife.
Second, temporal variation is important to incorporate into models that predict risk.
Third, models and predictions improve data quantity and quality increases. 

\section{Generalising the methods to other wildlife threats}

The conceptual framework developed in this thesis may be used to analyse risk for other ecological problems. Its modular nature allows the analysis of other risks to wildlife (\Cref{gen_framework}) such as attacks from domestic predators or human poaching. Further, the exposure of species to hazards may be represented by data generated by alternative models; for example connectivity or population viability.

\begin{figure*}[htp]
  \centering
  	\begin{minipage}[t][][b]{.13\textwidth}
    	\centering
    	\includegraphics[width=\textwidth]{MODELS_connectivity.png}\\
    	\includegraphics[width=\textwidth]{MODELS_population.png}   	
    \end{minipage}
    \hspace{.05\textwidth}
  	\begin{minipage}{.38\textwidth}
    	\centering
    	\includegraphics[width=\textwidth]{MODELS.png}
    \end{minipage}
    \hspace{.05\textwidth}
  	\begin{minipage}[t][][b]{.13\textwidth}
    	\centering
    	\includegraphics[width=\textwidth]{MODELS_predators.png}\\
    	\includegraphics[width=\textwidth]{MODELS_infrastructure.png}
    \end{minipage}    	   	   
  \caption[Conceptual risk model framework]{Conceptual risk model framework.}
  \label{gen_framework}
\end{figure*}


\section{Benefits of spatial modelling}

Most, if not all, problems resulting from the distribution of humans across the landscape have inherent spatial properties. 

\subsection{Access to spatial data}

As technology advances, the amount and quality of publicly-accessible spatial will inevitable increase.

\subsection{Management decision support}

The use of geographic information systems to catalogue and view spatial features is widespread in management. Contemporary versions of these software applications incorporate functionality to internally or externally perform spatial analyses, including modelling.

\subsection{Communication and public outreach}

Conveying information via spatial methods is an effective communication tool. Most of the general public is familiar with geographic information.

\section{Limitations of models and caveats}

Models are subject to uncertainty.
Bayesian versus frequentist.

Predictions in models are relative as opposed to absolute.
The data used are presence-only.

Temporal effects were not easily integrated into collision risk modelling on roads.

\section{Novel future research required}

-Traits of species that can explain collisions? Investigate effects of body size on collisions - may have impact on reporting rate. Temporal implications?

-Address population effects to species.  Does this matter for common species and can it be applied to more rare species?

-Size of vehicle is more informative than speed as momentum and stopping times may influence collisions. Investigate literature regarding this. Create new predictions for vehicles.

-Seasonality and mobility should be incorporated into the model - perhaps as weightings to exposure?

-Investigate literature regarding modelling pedestrian-vehicle collisions and integrate with work or fill gaps in literature.

-Spatio-temporal effects will impact model inference; drop out chunks of spatial and temporal data and cross-validate models.

-Results are influenced by choice of background road segments - address this using sensitivity analysis.

-Incorporation of detectability in model framework.

\section{Potential applications and extensions of research}

In an effort to maximise applicability of this work to reducing wildlife-vehicle collisions, I have utilised publicly-accessible data and open-source software tools, and developed reproducible computer code \Cref{apx:A}. This allows the methods to be used for developing additional resources to supplement management and engage the public. Given these qualities, there are several foreseeable extensions of this work. 

\subsection{Development of centralised data collection and reporting systems}

\Cref{sec:val} demonstrated improvements in modelling wildlife-vehicle collision risk when combining disparate sources of data. Worldwide, data is collected by governments, private-sector corporations, non-profit organisations, community groups, and individual citizen scientists. This information is stored in several different formats (e.g. electronic or hard-copy) and rarely made publicly-accessible. Data viewable through world-wide web portals usually does not allow access to detailed spatial information. 

\begin{figure*}[htp]
  \centering
  \includegraphics[width=0.8\textwidth]{interfacing.pdf}
  \caption[Centralised data collection and reporting system]{Schematic diagram of centralised data collection and reporting system for wildlife-vehicle collisions. Arrows indicate directions of information flow. Additional collection of collisions data (in blue) is by both citizen scientists (top) and professionals (bottom).}
  \label{wvc_server}
\end{figure*}

\subsubsection{Publicly accessible wildlife-vehicle collision atlas}

Beyond offering incredible global interconnectedness, the information age has been marked by a massive amount of publicly-accessible data via world-wide web portals. Atlases of species occurrence data have existed for nearly two decades \citep{}. The modelling and prediction of collision risk would benefit tremendously from the creation of an open-access, single-source repository of wildlife-vehicle collision records. 

\subsubsection{Standardised collection techniques and metadata}

For data in a central repository to be useful, they will need to be collected using broadly standardised techniques and include metadata.  

\subsection{Integration of modelling with existing and developing technology}

In developed nations, technology pervades every aspect of human daily lives. Every office desk is adorned by a computer and nearly all persons carry a mobile device (mostly smartphones). We use these tools to elicit information and to assist us with decision-making. There are models continuously running and recalibrating on servers throughout the world. Some help to optimise our search results on a webpage whilst others determine efficient ways to manage electricity grids. This approach to information with real-time modelling can be applied to road safety, specifically wildlife-vehicle collisions, to provide decision support for managers or warnings to motorists.

\subsubsection{Desktop-based decision support tool for managers}

Decision support tools are used by managers in a wide range of fields.
\Cref{wvc_tool} shows a web-based interface to the model framework presented in this thesis. Developed as a proof of concept, I wrapped the modelling methods with graphical user interface (GUI) that allows managers to change values of species occurrence, traffic volume, and traffic speed on road segments and simulate expected collision risk across a network. This is quite rudimentary, but illustrates the potential features that a decision-support tool may provide to direct efforts on reducing wildlife-vehicle collisions.

\begin{figure*}[htp]
  \centering
  \includegraphics[width=0.8\textwidth]{shinyapp.png}
  \caption[Decision support tool for wildlife-vehicle collisions]{Screen capture of web-based interface for a wildlife-vehicle collision risk modelling tool. The tool was originally developed for Bendigo, a medium-sized town in south-east Australia that experiences high numbers of kangaroo-vehicle collisions.}
  \label{wvc_tool}
\end{figure*}

\subsubsection{Real-time warning app for mobile devices}

Human behaviour change is a difficult task and a vast amount of sociological literature is devoted to studying it \citep{}.
Two key factors assist with changing human behaviour.
One is the use of non-static media to prevent habituation.
Another is convenience and proximity to information.
