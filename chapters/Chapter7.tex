\chapter{General Discussion}\label{sec:conc}
\newpage

\section{Synthesis}

In this thesis, I present a modelling framework for determining wildlife-vehicle collision risk that may be used without taxonomic or geographic restriction by wildlife, road, and environmental managers. Drawing on well-established concepts of road ecology and risk theory, this conceptual framework is the first of its kind to address the disparate factors that contribute to wildlife mortality on roads. \cite{clev15} has recently identified this critical need in the discipline of road ecology. In many respects, my work is quite similar to predictive modelling studies that have come before it. I use regression modelling to relate collisions to environmental and anthropogenic variables and make predictions of risk to segments on linear transportation networks. However, I extend this previous work in several notable ways. Perhaps the most useful deviation from other studies is the grouping of predictors that influence wildlife-collision risk into two categories that more directly relate to management activities. Among the suite of mitigation used by transport managers, most of it corresponds to controlling either animal presence on a transportation network (e.g. fencing) or the operation of vehicles (e.g. speed) \citep{}.  Using species occurrence to represent exposure and traffic volume/speed to represent hazard creates a more direct interpretation by managers. Further, all of the other predictors used in past wildlife-vehicle collision modelling may be incorporated into the sub-models in each respective group thereby reducing uncertainty about potentially confounding effects. From this simplified conceptual framework, managers are able to compare the relative influences between exposure and hazard, as well as the uncertainties of each one. If costs of mitigation are known for each category, the conceptual framework is useful for performing cost-benefit analyses - this is explored in \Cref{sec:train}.

Another useful departure from previous work that uses generalised linear regression to model wildlife-vehicle collisions is the choice of link function. This thesis is the first study to employ the use of a complementary log-log (cloglog) link function for modelling wildlife-vehicle collision events. This is important because the choice of a link function has theoretical underpinnings which are normally based on assumptions about how events arise in a system. I use the cloglog link as it assumes that collisions arise from a Poisson process which is reasonable for collision events occurring over a domain bound by time. Another attractive feature of the cloglog is that it is not influenced by the choice of sampling area. This makes it possible to compare parameter estimates for models using different spatial scales.  In contrast to several other studies that use a logit link (i.e. logistic regression) which imply strong assumptions about spatial scale and data generating processes (e.g. \cite{}), this thesis demonstrates a robust and flexible modelling method that will enhance new and existing road ecology work.

Beyond demonstrating the broad utility of the conceptual framework, my thesis also highlights three key results. First, the speed of moving vehicles is a considerable hazard for wildlife when they have access to transportation networks. \Cref{sec:6sp,sec:cal} demonstrated similar patterns of collision risk with respect to traffic speed for seven mammals. Nearly all species had exponential increases in risk with increasing speed. Collision risk also increased exponentially with increasing train speed (\Cref{sec:train}), suggesting this phenomenon is not isolated to road networks. These results are consistent with the broader literature on wildlife-vehicle collision analysis, but also other types of collision analysis such as pedestrian-vehicle \citep{} and vehicle-vehicle \citep{}. As mitigation to reduce animal proximity to, or exclude entirely from, transportation networks is costly \citep{}, these results suggest that speed reductions may be a more viable option for reducing collision risk. The conceptual framework I have developed in this thesis is able to assist managers to target specific areas for speed reductions and also simulate the effects from these proposed mitigations. Further, the predictions of risk from the conceptual framework may also be used to influence motorist behaviour (i.e. speed) in areas (or times) of high risk. Both of these potential uses are discussed further in subsequent sections.

Second, temporal variation is important to incorporate into models that predict collision risk. Both animal and human activities have distinct patterns that correspond to time. In \Cref{sec:train}, I introduced a function into the model framework that accounted for cyclical patterns of animal activity at the hour scale. This was a significant variable and useful for extending predictions from the spatial domain to the spatio-temporal domain. Temporal variation is rarely accounted for in predictive modelling of wildlife-vehicle collisions (but see \cite{}). Extending collision modelling work that use cyclical functions to characterise expected patterns of activity for a single species, the function used in \Cref{sec:train} generalises to any species. Further, the function also uses sunrise and sunset times (relative to geographic location) to consider time of day with respect to ambient light - a factor known to influence animal activity. Both of these are useful contributions to improve future collision analysis. One reason that collision modellers may be reluctant to utilise fine resolution temporal data (e.g. hourly scale) is uncertainty. Indeed, all of the collision data used in \Cref{sec:egk,sec:6sp,sec:cal,sec:val} had temporal information at an hourly scale, however, differences in actual collision times and reporting or carcass collection times were difficult to determine. Thus, I excluded the use of this data for modelling and prediction on the road networks. Patterns are evident from the histograms of reported collisions for all six species (\Cref{temporal_all}) and suggests temporal components may be useful for modelling and prediction collisions if the uncertainty can be reduced or accounted for. In contrast, the data used in \Cref{sec:train} had a high degree of certainty around the collision times reported by the train drivers. This was likely due to the systematic and procedural nature of train operations. Road collision events are not subject to the same stringencies thereby making it more necessary to document uncertainty in the data and represent it in the modelling. Uncertainty is elaborated upon in subsequent discussion. 

\begin{figure*}[htp]
  \centering
	\begin{minipage}[t]{.9\textwidth}
    	\centering
    	\includegraphics[width=.45\textwidth]{egk_coll_hour.png}\hspace{.05\textwidth}
    	\includegraphics[width=.45\textwidth]{btp_coll_hour.png}\\ 
    	\includegraphics[width=.45\textwidth]{rtp_coll_hour.png}\hspace{.05\textwidth}
    	\includegraphics[width=.45\textwidth]{bsw_coll_hour.png}\\
    	\includegraphics[width=.45\textwidth]{wom_coll_hour.png}\hspace{.05\textwidth}
    	\includegraphics[width=.45\textwidth]{koa_coll_hour.png}
    \end{minipage}
  \caption[Total collisions by hour for six mammal species]{Histograms showing the distributions of total collisions by hour for six mammal species. Note, Records indicate the time that wildlife-vehicle collision events were reported and may not accurately reflect actual times due to reporting lags.}
  \label{temporal_all}
\end{figure*}

Lastly, collision models and predictions may benefit from increases in data quantity and quality. \Cref{sec:val} introduced the first known study in the discipline of road ecology to compare the effects of combining data from different data sources on collision risk modelling and predictions. This compliments work that evaluates the use of citizen science to collect collision data \citep{paul14, dwye16} and improvements in technology to collect data \cite{olso14}. All of these studies call for the development of a centralised data storage system to improve road ecologists' understanding where and when collisions occur. In \Cref{sec:val}, as more disparate collision datasets were combined together, model performance and calibration improved. Prospects for centralising collision data are discussed in a subsequent section. 

\section{Benefits of spatial modelling}

Most, if not all, problems resulting from the distribution of humans across the landscape have inherent spatial properties. Following on from this, research involving ecological phenomenon and issues often involves spatial analyses and outputs (e.g. species distribution modelling, see \cite{}). The use of spatial modelling in this thesis was fundamental to its intended purpose - to represent of collision risk across large transportation networks. Further, the use of spatial modelling in this work also supports its future application as a decision support tool for management or a public outreach mechanism.

\subsection{Access to spatial data}

Currently, there is a remarkable amount of publicly-accessible spatial data available on the world-wide web. Both the amount and quality of data are expected to increase in future years \citep{}. Data derived by scientific or government organisations often includes additional metadata that details reliability and data-generating procedures. In this thesis, nearly all of the predictor data used in modelling were derived from information that existing in the public domain. This enables the methods to be reproducible and transferable; for example, I developed several of the predictors in \Cref{sec:cal} using the same source of satellite-derived data but for different geographic locations.

\subsection{Spatial analysis in management}

The use of geographic information systems (GIS) to catalogue and view spatial features is widespread in management. Contemporary versions of these software applications incorporate functionality to internally or externally perform spatial analyses, including modelling. As managers are familiar with spatial data, I employed spatial methods in this thesis that would easily integrate into GIS software. For example, all computer code was written in \textit{R} \citep{}, statistical software that can be called to perform operations on spatial data from within a program such \textit{QGIS} \citep{}. Further, I developed computer code to convey collision risk in ways that managers would be able to most effectively utilise (e.g. maps).

\subsection{Communication and public outreach}

Conveying information created with spatial methods is an effective communication tool. Most of the general public is familiar with geographic information through daily exposure to maps and other spatial data \citep{}. Spatial data and analysis are natural candidates for data visualisation, a technique used to communicate information through the use of visual objects. Data visualisation may increase public engagement through interactive media. Although the spatial outputs included in this thesis are static, the methods used to create them are easily adaptable to an interface that enables users to perform analysis on collision risk by varying model parameters.

\section{Limitations of models and caveats}

Models are simplified abstractions of reality and therefore subject to uncertainty \citep{burg05}. Models are useful as conceptual tools for organising complex problems and supplementing decision-making, however, must be used cautiously. Further, \cite{levi66} asserts that models can only be built to maximise two of three characteristics, generality, realism, and precision. The conceptual model framework introduced in this thesis is meant to be general and realistic. By general, I mean applicable to any ecological issue involving threats to wildlife. Although not as strong of a characteristic, our model exhibits realism in its theoretical treatment of collisions resulting from the coincidence of animals and moving vehicles in space and time. Much of the literature in road ecology on collision modelling operates on this assumption \citep{}, albeit does make this explicit in the modelling methods. Nevertheless, assumptions about process (how collisions occur across the landscape) and inadvertent omissions of pertinent data both add uncertainty to the model framework.

One strategy to address model uncertainty is to perform the analyses in this thesis using Bayesian inference. This method allows uncertainty to be accounted for in the model through the use of prior information. All model parameters are estimated and represented with distributions thereby allowing scrutiny of the 

Yet another source of uncertainty is in the data used to train collision model. This information is derived from models that have their own inherent uncertainties.   

The data used to train the collision model in this thesis are what is referred to as 'presence-only' which require special assumptions and treatment \citep{}. Because the models lack true absences, the predictions are relative to the as opposed to absolute

\section{Novel future research required}

-Traits of species that can explain collisions? Investigate effects of body size on collisions - may have impact on reporting rate. Temporal implications?

-Address population effects to species.  Does this matter for common species and can it be applied to more rare species?

-Size of vehicle is more informative than speed as momentum and stopping times may influence collisions. Investigate literature regarding this. Create new predictions for vehicles.

-Seasonality and mobility should be incorporated into the model - perhaps as weightings to exposure?

-Investigate literature regarding modelling pedestrian-vehicle collisions and integrate with work or fill gaps in literature.

-Spatio-temporal effects will impact model inference; drop out chunks of spatial and temporal data and cross-validate models.

-Results are influenced by choice of background road segments - address this using sensitivity analysis.

-Incorporation of detectability in model framework.

\section{Generalising the methods to other wildlife threats}

The conceptual framework developed in this thesis may be used to analyse risk for other ecological problems, thereby increasing its potential contribution other environmental research. Its modular nature allows analysis of other adverse risks to wildlife (\Cref{gen_framework}) by replacing the hazard component with an alternative threatening process of interest, such as attacks from domesticated predators or human poaching. Further, the exposure of species to hazards may be represented by data generated by alternative models; for example movement patterns from connectivity analysis \citep{} or local abundances from population viability analysis \citep{}. \cite{baud13} used a similar approach to determine collision risk between seafaring vessels and manatees, although did not emphasise the generality in their approach. The modularity in the conceptual framework allows unlimited possibilities for future research projects.

\begin{figure*}[htp]
  \centering
  	\begin{minipage}[t][][b]{.13\textwidth}
    	\centering
    	\includegraphics[width=\textwidth]{MODELS_connectivity.png}\\
    	\includegraphics[width=\textwidth]{MODELS_population.png}   	
    \end{minipage}
    \hspace{.05\textwidth}
  	\begin{minipage}{.38\textwidth}
    	\centering
    	\includegraphics[width=\textwidth]{MODELS.png}
    \end{minipage}
    \hspace{.05\textwidth}
  	\begin{minipage}[t][][b]{.13\textwidth}
    	\centering
    	\includegraphics[width=\textwidth]{MODELS_predators.png}\\
    	\includegraphics[width=\textwidth]{MODELS_infrastructure.png}
    \end{minipage}    	   	   
  \caption[Conceptual risk model framework]{Conceptual risk model framework.}
  \label{gen_framework}
\end{figure*}

\section{Potential applications of research}

In an effort to maximise potential applications of this work to reducing wildlife-vehicle collisions, I have utilised publicly-accessible data and open-source software tools, and developed reproducible computer code (\Cref{apx:A}). This allows the methods to be used for developing additional resources to supplement management and engage the public. Given these qualities, there are several foreseeable future extensions of this work. 

\subsection{Development of centralised data collection and reporting systems}

Beyond offering incredible global interconnectedness, the information age has been marked by a massive amount of data generation, publicly-accessible via world-wide web portals. One example is atlases of species occurrence data which have existed for nearly two decades \citep{}. The modelling and prediction of collision risk would benefit tremendously from the creation of an open-access, single-source repository of wildlife-vehicle collision records. Worldwide, collision data is collected by governments, private-sector corporations, non-profit organisations, community groups, and individual citizen scientists. This information is stored in several different formats (e.g. electronic or hard-copy) and rarely made publicly-accessible. Collision data that is made available through world-wide web portals is often read-only and does not provide access to detailed spatial information. I propose a centralised data system that not only catalogues collision records and metadata, but also provide other important functions as shown in \cref{wvc_server}. Most importantly, the system should be able to communicate with other data systems and external applications. Such systems exist for data storage and sharing \citep{}, however, do not yet employ fully automated interfacing with external modelling tools.

\begin{figure*}[htp]
  \centering
  \includegraphics[width=0.8\textwidth]{interfacing.pdf}
  \caption[Centralised data collection and reporting system]{Schematic diagram of centralised data collection and reporting system for wildlife-vehicle collisions. Arrows indicate directions of information flow. Additional collection of collisions data (in blue) is by both citizen scientists (top) and professionals (bottom).}
  \label{wvc_server}
\end{figure*}

%\subsubsection{Publicly accessible wildlife-vehicle collision atlas}
%\subsubsection{Standardised collection techniques and metadata}

\subsection{Integration of modelling with existing and developing technology}

In developed nations, technology pervades every aspect of human lives \citep{}. Every office desk is adorned by a computer and nearly all persons carry a mobile device (mostly smartphones). We use these tools to elicit information and to assist us with decision-making. Unbeknownst to a significant portion of the public, models are continuously running and recalibrating on servers throughout the world to deliver information to these devices \citep{}. Some help to optimise our search results on a web page whilst others determine efficient transportation routes. This approach to disseminating real-time, model-generated information can be applied to road safety, specifically wildlife-vehicle collisions, to provide decision support for managers or warnings to motorists. I envision the work in this thesis to be easily extended to either application.

\subsubsection{Desktop-based decision support tool for managers}

Decision support tools are used by managers in a wide range of fields \citep{}. \Cref{wvc_tool} shows a web-based interface to the model framework presented in this thesis. Developed as a proof of concept, I wrapped the modelling methods with a graphical user interface (GUI) that allows managers to change values of species occurrence, traffic volume, and traffic speed on road segments and simulate expected collision risk across a network. Although this is quite rudimentary, it illustrates the potential features that a decision-support tool may provide to direct efforts on reducing wildlife-vehicle collisions.

\begin{figure*}[htp]
  \centering
  \includegraphics[width=0.8\textwidth]{shinyapp.png}
  \caption[Decision support tool for wildlife-vehicle collisions]{Screen capture of web-based interface for a wildlife-vehicle collision risk modelling tool. The tool was originally developed for Bendigo, a medium-sized town in south-east Australia that experiences high numbers of kangaroo-vehicle collisions.}
  \label{wvc_tool}
\end{figure*}

\subsubsection{Real-time warning app for mobile devices}

Human behaviour change is a difficult task and a vast amount of sociological literature is devoted to studying it. Studies suggest two key factors that assist with changing human behaviour; 1) use of non-static media to prevent habituation \citep{}, and 2) convenience and proximity to information \citep{}. Both of these factors suggest that mobile device based applications may be effective at influencing human behaviour. As such, mobile devices and global-positioning units are ideal candidates for receiving relative collision risk (based on geographic location and time of day) from model predictions and disseminating the information to motorists. The dynamic nature of a warning application may exceed the utility of traditional warning devices, such as signage, which are considered largely ineffective \citep{}.  Smartphone applications that report locations of collisions (and also accept observations) exist and are in use today \citep{aane09}, however, do not utilise modelling mechanisms. In this respect, these applications use the technology to report collisions and, thus, may be only slightly more effective than road signs.  

\section{Concluding remarks}

Human-wildlife interactions, and wildlife-vehicle collisions, are a global phenomenon. Issues such as animal mortality on roads will be exasperated by future human development as we reduce our proximity to wildlife. To address these issues, we need to be strategic about where and when we mitigate. This thesis provides analytical tools for making better decisions about environmental management. It is my hope that the contents of this work will assist others in solving contemporary ecological issues.