\chapter{General Discussion}\label{sec:conc}
\newpage

\section{Generalising the methods to other wildlife threats}

%\section{Contribution of research}

\section{Benefits of spatial modelling}

\subsection{Access to spatial data}

\subsection{Management decision support}

\subsection{Communication and public outreach}

\section{Limitations of models and caveats}

The validation sets vary in detection levels; determine what these are and discuss.

%\subsection{Disparate sources of information}

%\subsection{Detectability issues}

\section{Novel future research required}

-Traits of species that can explain collisions? Investigate effects of body size on collisions - may have impact on reporting rate. Temporal implications?

-Address population effects to species.  Does this matter for common species and can it be applied to more rare species?

-Size of vehicle is more informative than speed as momentum and stopping times may influence collisions. Investigate literature regarding this. Create new predictions for vehicles.

-Seasonality and mobility should be incorporated into the model - perhaps as weightings to exposure?

-Investigate literature regarding modelling pedestrian-vehicle collisions and integrate with work or fill gaps in literature.

-Spatio-temporal effects will impact model inference; drop out chunks of spatial and temporal data and cross-validate models.

-Results are influenced by choice of background road segments - address this using sensitivity analysis.

\section{Potential applications of research}

\subsection{Development of central data collection and reporting system}

\subsubsection{Publicly accessible wildlife-vehicle collision atlas}

\subsubsection{Standardised collection techniques and metadata}

\subsection{Integration of modelling with existing and developing technology}

\subsubsection{Desktop-based decision support tool for managers}

\subsubsection{Real-time warning app for mobile devices}
