\chapter{Validating collision model predictions with disparate datasets: a case for a unified system}\label{sec:val}
\newpage

\begin{localsize}{10}
\section*{\centering Abstract}

Wildlife-vehicle collisions are often costly due to property damage and injury and have been estimated to kill billions of fauna each year. Many studies use predictive modelling to determine where and when WVC will occur. These models are often trained on collision data that is sourced from non-government organisations, private-sector entities, academic institutions, government agencies, community groups, and individual citizen scientists. These data are not easily attainable and often have implicit biases and errors.

We analysed change in model performance by combining four disparate datasets of collisions in the State of Victoria in south-east Australia. For each combination of training datasets, we fitted a model, made predictions, and validated the predictions with an independent dataset not used to train the model. We also used a collision dataset of total insurance claims per town to validate the predictions of models using different combinations of data. 

Among the four datasets, the discrimination ability of the model did not increase as data were added to models. Using the town-level independent data to validate models resulted in an increase in model calibration as more combinations of training data were used. Likewise, the variation in the data explained by the model also increased as more combinations of data were used.

We demonstrate that increasing the amount of collision data used for modelling, by combining disparate datasets, has the potential to improve model performance and predictive strength. These results highlight the importance of uniting the currently disparate sources of collision data with a comprehensive and systematic archive. We also suggest qualities that are integral to the success of such a system.
\end{localsize}

\newpage
\section{Introduction}

Roads are critical infrastructure, but their development and use has ecological impacts \citep{form03}. Perhaps one of the most directly visible and confronting issues is vehicle strikes to animals, especially large terrestrial vertebrates. These collision events are often costly due to property damage, injury, and death. Globally, wildlife-vehicle collisions (WVC) on transportation networks have been estimated to kill billions of fauna each year \citep{seil06}. In North America, millions of animals are estimated to be killed by vehicle strikes each day \citep{form98} and annual economic costs due to collisions exceed eight billion dollars \citep{huij07b}. However, these adverse effects are not unique to North America and occur at all locations where roads dissect habitat for wildlife. Due to such large impacts, governments, scientists, private-sector companies and citizens throughout many nations have undertaken data collection and analysis of WVC (see \cite{rvdr15}.

A large amount of research seeks to determine the frequencies and magnitudes of wildlife-vehicle collisions. This often involves examining several environmental (e.g. landscape, climate, road characteristics), anthropogenic (e.g. human behaviour, vehicle movements) and biotic (e.g. animal traits) factors at different spatial and temporal scales \citep{litv08}. More recently, computer modelling has been employed to predict where and when collisions are most likely to occur based on a large range of predictor variables \citep{guns11}. These predictions aim to help environmental managers and road authorities to identify and mitigate problematic areas and better construct new roads. Effective decisions made from these predictions rely on robust, calibrated models as incorrect inferences may result in costly outcomes such as ineffective mitigation (see \cite{huij09}) or continuing effects of unmitigated collisions (see \cite{biss08b}). Data deficiencies generate uncertainty in models and result from data collected with specific procedures or spatial/temporal trends (e.g. bias), measurement error or inappropriate sample size \citep{bean12}. Although other factors, such as model misspecification, may also result in biased and uncertain predictions, we argue that the quality and quantity of data used to train WVC predictive models are of paramount importance.

Many studies that use predictive modelling to determine where and when WVC will occur are limited to single sources of data - either field collected based on predetermined project requirements (e.g. \cite{lang09}; \cite{roge09}) or pre-existing from independent surveys or routine collections (e.g. \cite{hoth12}; \cite{malo04}). Bias and error may arise from the data sources and collection methods, however, these are not often explicitly reported or accounted for in the modelling methods. Further, few studies utilise independently-obtained data to validate model performance; often this is due to limitations in availability. In response to the importance of consistent data collection, technology has been proposed to augment field activities. For example, mobile phones have been proposed to assist with data collection of wildlife-vehicle collisions \citep{aane09,olso14}. Tablets or personal digital assistants (PDA) have also been proposed to assist road authorities to record collisions \citep{amen07}. Cataloguing and reporting systems are used to store data on WVC, however, few are open access or incorporate data from other institutions or previous surveying events. Perhaps one of the largest systems in current use is the California Roadkill Observation System (CROS), maintained by the Road Ecology Center at the University of California, Davis \citep{shil15b}. This system offers many features such as public-accessibility, an intuitive user interface, and collection of important metadata on reliability and precision of reporting, but has yet to achieve significant uptake by agencies outside the academic and citizen-science networks it supports.

%\begin{table}[htp]
%\caption[Potential biases present in collision data]{Potential biases present in collision data collected under different example scenarios. Levels of bias are relative and coded as: high (H), medium (M), low (L)}
%\centering
%\begin{tabularx}{0.9\textwidth}{P{7cm}ccc} \toprule
%Scenario						&Spatial Bias		&Temporal Bias		&Species Bias \\ \midrule
%Contract for carcass collection along a specific road section	&H	&H	&L \\
%	&					&					& \\
%Well-designed survey to identify factors influencing WVC	&L	&L	&M \\
%	&					&					& \\
%Citizen recording carcasses of large animal along daily commute	&H	&	&H \\
%	&					&					& \\
%Retired citizen recording carcasses of all animals in regional area	&	&	& \\
%	&					&					& \\
%Citizen only recording carcasses of interesting animals	&	&	& \\
%	&					&					& \\
%Member reporting collision with animal to insurance company	&	&	& \\
%	&					&					& \\
%Road authority collecting carcasses of large animals	&	&	& \\
%	&					&					& \\
%City council collecting carcasses of reported animals	&	&	& \\
%	&					&					& \\
%Citizen reporting collision with animal to wildlife organisation	&	&	& \\
%	&					&					& \\
%Police responding to collision with animal	&					&					& \\
%\bottomrule
%\end{tabularx}
%\label{biases}
%\end{table}

Worldwide, collision data is collected by non-government organisations, private-sector entities (e.g. insurance companies), academic institutions, government agencies (e.g. road authorities and environmental managers), community groups, and individual citizen scientists but these data are not always easily attainable due to privacy restrictions or data formats. We assert that modelling and prediction of WVC will benefit from extensive, open-access data repositories that catalogue collision records from these entities. These mechanisms exist in other areas of research. For example, contemporary species distribution modelling draws from worldwide atlases of species occurrence; the Global Biodiversity Information Facility (www.gbif.org) contains more than 700 million records from over 800 data publishers and the Atlas of Living Australia (http://www.ala.org.au) provides over 50 million records from a wide range of data providers.

In our study, we analyse the predictions of a wildlife-vehicle collision risk model using a citizen-science dataset joined with four external independent datasets in different combinations. As the sample size and variation in collection techniques (location, expertise, redundancy) of our training data increases, we analyse the performance of our models. Our overall aim is to demonstrate the value of a unified, publicly-accessible data storage system to support analysis and prediction of WVC and determine characteristics of data that are important to ensure robust analyses.

\section{Materials \& Methods}

Our study analyses risk of WVC over approximately 150,000 kilometres of sealed roadway (i.e. only bitumen surfaces) in the State of Victoria in south-east Australia (\Cref{val_study_area}). The Victorian road authority (VicRoads) manages 25,256 kilometres of major roadway and seventy-nine municipal districts manage the remaining roads. We split the sealed road network into 644,715 road segments by intersecting the roads with a 1km$^2$ spatial grid. We predicted traffic volume and speed to each road segment using linear regression and predicted species occurrence to the spatial grid using species distribution models. Species occurrence was sampled using the midpoints of the road segments resulting in each having predicted values of kangaroo occurrence, traffic speed and traffic volume in the final dataset - more details are provided in \Cref{sec:egk}. 

\begin{figure*}[htp]
  \centering
  \includegraphics[scale=.5]{06_study_area.png}
  \caption[Study area for model validation]{Study area (state of Victoria, south-east Australia) showing all sealed road segments as light gray lines. The inset shows the geographic location of Victoria in Australia. The darker shaded region is the City of Bendigo and the lighter shaded region is the VicRoads Western District. Major towns ($>$100,000 residents) are starred and labelled accordingly.}
  \label{val_study_area}
\end{figure*}

Eastern Grey kangaroos (\textit{Macropus giganteus}, Shaw) are the second largest native mammal (up to 85 kilograms for males) in Australia \citep{coul10}, and the most commonly reported species struck by motor vehicles in Victoria \citep{rowd08}, prompting several organisations to record carcass collection and collision locations. We obtained four datasets of spatially-explicit collisions with Eastern Grey kangaroos in the state of Victoria in south-east Australia for our study (\Cref{val_data}). The data were sourced from: the City of Bendigo, a large regional center located in north-east Victoria, the Western District, a large VicRoads administrative boundary located in the center of the western portion of Victoria, Crashstats, an online repository of police-reported WVC, and Insurance Australia Group, an insurance company that records WVC reported by their members. Data from the City of Bendigo was based on carcasses reported to the 3,001 km$^2$ municipality by members of the public. The Western District data was comprised of reported carcasses, along 1,417 km of major roads within an area of 24,373 km$^2$, made by a VicRoads-authorised litter-abatement contractor. Crashstats data covered the entire state and was based on reports made by the state police regarding injuries from WVC. Similarly, the Insurance Australia Group data was also based on statewide reports of WVC, however, originating from insurance claims made by motorists. Each of the road segment-based datasets (City of Bendigo, Western District, and Crashstats) were used to augment a dataset obtained from Wildlife Victoria for modelling and to validate predictions resulting from these models. To develop each modelling dataset for our study, we selected road segments that intersected with reported kangaroo collision records to represent presences. All other road segments without collisions represented background data. When combining datasets, multiple collisions on the same road segments were reduced to single presences. Thus, each modelling dataset was comprised of 644,715 road segments having either a collision or no collision. Eight permutations were developed by combining independent data with Wildlife Victoria (baseline) data (codes shown in brackets):

\begin{enumerate}
	\item Wildlife Victoria (o)
	\item Wildlife Victoria + City of Bendigo (ob)
	\item Wildlife Victoria + Western District (ow)
	\item Wildlife Victoria + Crashstats (oc)
	\item Wildlife Victoria + City of Bendigo + Western District (obw)
	\item Wildlife Victoria + Western District + Crashstats (owc)
	\item Wildlife Victoria + Crashstats + City of Bendigo (ocb)
	\item Wildlife Victoria + City of Bendigo + Western District + Crashstats (obwc)
\end{enumerate}

\begin{table}[htp]
\caption[Datasets used to fit collision models and validate predictions]{Datasets used to fit models and validate predictions.}
\centering
\begin{tabularx}{0.9\textwidth}{YYYYYYYY} \toprule
Source						&Data Code	&Type	&Records	&Reporting Method	&Spatial Unit	&Spatial Coverage	&Temporal Coverage \\ \midrule

City of Bendigo		&b		&Public municipality			&395		&Public reported carcasses			&Road segment		&City (3,001 km$^2$)	&8 years : 2007-2014 \\

&&&&&&\\

VicRoads Western District	&w		&State government organisation	&815		&Independent contractor recorded  carcasses			&Road segment		&District (24,373 km$^2$)	&2 years : 2014-2015 \\

&&&&&&\\

Crashstats		&c			&State government organisation	&487		&Police reported wildlife strikes			&Road segment		&State of Victoria (227,819 km$^2$)	&7 years : 2010-2016 \\

&&&&&&\\

Wildlife Victoria	&o		&Non-government organisation	&4245		&Public reported observations of wildlife strikes			&Road segment		&State of Victoria (227,819 km$^2$)	&6 years : 2010-2015 \\

&&&&&&\\

Insurance Australia Group	&iag	&Private corporation			&1344		&Public reported claims of wildlife strikes			&Town		&State of Victoria (227,819 km$^2$)	&3 years : 2011-2013 \\

\bottomrule
\end{tabularx}
\label{val_data}
\end{table}

To test the effects of combining independent datasets, we fitted a collision model (\Cref{eq:28}) to all permutations of data.  We predicted collision risk from a model that regressed reported collisions on species occurrence, traffic volume and traffic speed:

\begin{equation} \label{eq:61}
cloglog(p_i) = \beta_0 + \beta_1*\text{log}(O_i) + \beta_2*\text{log}(V_i) + \beta_3*\text{log}(S_i)
\end{equation}

\noindent where $p_i=\text{Pr}(Y_i=1)$ is the relative likelihood of a collision occurring on a road segment $i$, $O_i$ is species occurrence, $V_i$ is traffic volume, $S_i$ is traffic speed.

To test model sensitivity to different combinations of data, we validated the predictions from our models using held-out datasets. For each combination of training datasets, we fitted a model, made predictions, and validated the predictions with an independent dataset not used to train the model. To assess model calibration, we regressed the observed collisions in each independent dataset on values predicted from models using the independent data (see \cite{mill91}). A perfectly calibrated model would report an intercept coefficient of zero and a slope coefficient of one. To measure the ability of each model to discriminate between false positive and false negative values, we also calculated receiver operator characteristic scores (see \cite{metz78}. A score of one indicates perfect discrimination while 0.5 indicates no better than random selection; 0.75 or above is often considered acceptable performance.

In addition to the four datasets with data on individual collisions, we also had access to dataset in which the number of collisions that involved insurance claims was recorded for towns. The Insurance Australia Group (``iag" hereafter) collision data detailed counts of collisions within each town boundary in Victoria (2917 total). From the predictions made to all road segments by the models using different combinations of data, we calculated the mean collision risk within each town boundary. Using each town as a datapoint, we regressed the mean collision risk prediction (log-transformed) on the total collisions reported. Due to the response data being positive integers, we used the Poisson link:

\begin{equation} \label{eq:62}
\text{log}(C_j) = \beta_0 + \beta_1\text{log}(P_j)
\end{equation}

\noindent where $C_j$ is the count of reported collisions and $P_j$ is the mean predicted collision risk, in a town $j$. As with the road segment-based data analysis, calibration was assessed using the intercept and slope of the model fits. We also calculated the reduction in deviance (unexplained variation in the data) for each model.

To further investigate how each independent dataset contributed to the models utilising combined data, we fitted the model described in \Cref{eq:61} to each road segment-based independent dataset (City of Bendigo, Western District, and Crashstats). We added an offset term of road length multiplied by the data period in years to standardise the model outputs for comparison. For each dataset, we restricted road segments to a domain that corresponded to the respective survey or reporting area. The City of Bendigo data were restricted to road segments that fell within the city boundary as carcass reports outside of the jurisdiction were not catalogued. Similarly, only road segments that were surveyed in the Western District were included for analysis. Crashstats covers the entire state and so all 644,715 road segments were used. 

\section{Results}

\begin{table}[htp]
\caption[Summary of model fits using original and validation data]{Summary of model fits using original and validation data in solidarity (i.e. no combinations of data are used). Highly significant variables (p$<$.0001) are marked with asterisks.}
\centering
\begin{tabularx}{0.9\textwidth}{lllllll} \toprule
Dataset & Variable & Coefficient & Standard Error & $Z\text{-value}$ & $\PRZ$ & Deviance Explained \\ 
  \midrule
Wildlife Victoria & Intercept & -42.30 & 1.07 & -39.66 & $<$.0001* & 11.83 \\ 
   & EGK & 0.70 & 0.01 & 47.77 & $<$.0001* &  \\ 
   & TVOL & 5.55 & 0.24 & 22.68 & $<$.0001* &  \\ 
   & TVOL2 & -0.33 & 0.02 & -21.76 & $<$.0001* &  \\ 
   & TSPD & 3.93 & 0.07 & 54.49 & $<$.0001* &  \\ 
  City of Bendigo & Intercept & -53.81 & 4.87 & -11.05 & $<$.0001* & 5.16 \\ 
   & EGK & 0.49 & 0.09 & 5.28 & $<$.0001* &  \\ 
   & TVOL & 11.71 & 1.20 & 9.74 & $<$.0001* &  \\ 
   & TVOL2 & -0.74 & 0.08 & -9.60 & $<$.0001* &  \\ 
   & TSPD & 1.19 & 0.39 & 3.08 & 0.0021 &  \\ 
  Western District & Intercept & -10.19 & 3.47 & -2.93 & 0.0034 & 5.15 \\ 
   & EGK & 0.02 & 0.05 & 0.36 & 0.7207 &  \\ 
   & TVOL & 0.37 & 0.80 & 0.47 & 0.6408 &  \\ 
   & TVOL2 & -0.07 & 0.05 & -1.32 & 0.1885 &  \\ 
   & TSPD & 2.03 & 0.45 & 4.48 & $<$.0001* &  \\ 
  Crashstats & Intercept & -51.74 & 2.89 & -17.88 & $<$.0001* & 11.75 \\ 
   & EGK & 0.55 & 0.04 & 12.89 & $<$.0001* &  \\ 
   & TVOL & 4.89 & 0.69 & 7.13 & $<$.0001* &  \\ 
   & TVOL2 & -0.32 & 0.04 & -7.19 & $<$.0001* &  \\ 
   & TSPD & 6.52 & 0.27 & 24.43 & $<$.0001* &  \\ 
\bottomrule
\end{tabularx}
\label{val_glm_perf}
\end{table}

All three models fitted to independent road segment-based datasets (City of Bendigo, Western District, and Crashstats) had mixed results when compared to the model fitted to the original dataset (\Cref{val_glm_perf}). Both of the models using statewide data (Wildlife Victoria and Crashstats) performed better than the two using localised data (City of Bendigo and Western District). The worst performing model was fit to data from the Western District of Victoria - only traffic speed was a significant predictor. In contrast, the only non-significant predictor was traffic speed in the model fit to data from the City of Bendigo. All other predictors in every model were significant. The reduction of unexplained variation in the data were remarkably similar for the two statewide data models ($\approx$11.8\%) and the localised data models ($\approx$5.2\%).

\begin{figure*}[htp]
  \centering
  \includegraphics[scale=.7]{06_occ.png}
  \includegraphics[scale=.7]{06_tvol.png}
  \includegraphics[scale=.7]{06_tspd.png}
  \caption[Marginal effects of predictor variables on relative likelihood of collision using independent datasets to train models]{Marginal effects of each predictor on relative likelihood of collisions. Codes for data combinations are: 'o' - Original (Wildlife Victoria); 'b' - City of Bendigo; 'w' - Western District; 'c' - Crashstats. Note likelihoods of collision have been rescaled for comparison as all datasets have different numbers of data points and thus variations in the range of predicted values.}
  \label{val_effects}
\end{figure*}

Most of the marginal effects of the predictors on collision risk had similar shapes for all of the model fits. Collision risk rose with increasing species occurrence for all models, however, the Western District data model reached nearly 75\% of the maximum collision likelihood at low values of species occurrence ($\approx$0.2). Collision risk had a quadratic relationship to traffic volume, and peaked between 2,000 and 7,000 vehicles day$^{-1}$ for all models except the Western District data model. The Western District data model showed a monotonically decreasing collision risk with rising traffic volume beginning at a value of $\approx$0.95 at zero vehicles day$^{-1}$. Collision risk increased with traffic speed for all models. Collision risk increased nearly linearly between 40 and 110 km hr$^{-1}$ in the model fit to the City of Bendigo data whilst collision risk rose slowly with the maximum increase of risk per unit speed occurring above 95 km hr$^{-1}$ in the model fit to the Crashstats data.

\begin{table}[htp]
\caption[Discrimination ability of models using all combinations of independent data]{Discrimination ability of models expressed as receiver operator characteristic scores. Data combinations used to model and make predictions are shown as column headings. Data used to validate model predictions are shown as row headings.}
\centering
\begin{tabularx}{0.9\textwidth}{YYYYYYYY} \toprule
	& Wildlife Victoria
(original) & original + City of Bendigo & original + Western District & original + Crashstats & original + City of Bendigo + Western District & original + Western District + Crashstats & original + Crashstats + City of Bendigo \\ 
  \midrule
City of Bendigo & 0.87 & - & 0.85 & 0.87 & - & 0.85 & - \\ 
Western District & 0.76 & 0.75 & - & 0.78 & - & - & 0.77 \\ 
Crashstats & 0.8 & 0.8 & 0.83 & - & 0.82 & - & - \\
\bottomrule
\end{tabularx}
\label{val_glm_roc}
\end{table}

The ability of models to discriminate between false-positive and true-positive rates of collisions were stable using different training data and validation data (\Cref{val_glm_roc}); discrimination ability did not increase as data were added to models. Among the different validation datasets, variation in the receiver operator characteristic (ROC) scores were less than 4\% between different combinations of data used to fit models. All of the models discriminated best when validated with the City of Bendigo dataset (ROC scores of 0.85-0.87) and worst with the Western District dataset (ROC scores of 0.75-0.78). 

\begin{figure*}[htp]
  \centering
  \includegraphics[scale=1]{06_calib.png}
  \caption[Collision model calibration for all combinations of original and independent data]{Model performance for all combinations of data. Codes for data combinations are: 'o' - Original (Wildlife Victoria); 'b' - City of Bendigo; 'w' - Western District; 'c' - Crashstats. Characters before the hyphen represent the datasets used for training the model and making predictions; characters after the hyphen indicate the data used for validation. Estimated calibration coefficients are shown as dots with bars representing standard errors.}
  \label{val_calib}
\end{figure*}

There was no observable trend in calibration when validating model predictions made with increasing amounts of data (\Cref{val_calib}). The lowest calibration scores were for models validated with the Western District (w) collision data. The model fit on the Wildlife Victoria data (o) and validated with data from Crashstats (c) had the highest calibration score which was close to one. Moreover, there was no apparent improvement in the relationship between observed and predicted values when increasing the amounts of data used for modelling. All of the models demonstrated similar trajectories for observed-versus-predicted rates at low values and then diverged at high values (\Cref{val_calib2}).

\begin{figure*}[htp]
  \centering
  \includegraphics[scale=1]{06_calib2.png}
  \caption[Comparisons of observations versus model predictions for all combinations of original and independent data]{Comparisons of observations versus model predictions for all combinations of data. Codes for data combinations are: 'o' - Original (Wildlife Victoria); 'b' - City of Bendigo; 'w' - Western District; 'c' - Crashstats. Characters before the hyphen represent the datasets used for training the model and making predictions; characters after the hyphen indicate the data used for validation.}
  \label{val_calib2}
\end{figure*}

Using the aggregated independent data (iag) to validate models resulted in a slight increase in model calibration as more combinations of training data were used (\Cref{val_calib_iag}).  The lowest calibration coefficient (0.943) occurred using only the original (o) data to fit the model and the highest (0.945) occurred using a combination of the original and all independent (obwc) data. Higher reductions of unexplained variation in the data also occurred with more combinations of training data (\Cref{val_calib_dev}). Deviance explained by the model fit on the original (o) data was 33.2\% whilst the model fit on a combination of the original and all independent (obwc) data explained 33.3\% of the deviance.

\begin{figure*}[htp]
  \centering
  \includegraphics[scale=1]{06_calib_iag.png}
  \caption[Model calibration for all combinations of training data using the aggregated independent data for validation]{Model performance for all combinations of data using the aggregated independent data (iag) for validation. Codes for data combinations are: 'o' - Original (Wildlife Victoria); 'b' - City of Bendigo; 'w' - Western District; 'c' - Crashstats. Characters before the hyphen represent the datasets used for training the model and making predictions; the same data aggregated data (iag) were used for all validation. Estimated calibration coefficients are shown as dots with bars representing standard errors.}
  \label{val_calib_iag}
\end{figure*}

\begin{figure*}[htp]
  \centering
  \includegraphics[scale=1]{06_dev_iag.png}
  \caption[Model discrimination ability for all combinations of training data using the aggregated independent data for validation]{Model performance for all combinations of data using the aggregated independent data (iag) for validation. Codes for data combinations are: 'o' - Original (Wildlife Victoria); 'b' - City of Bendigo; 'w' - Western District; 'c' - Crashstats. The percent of variation in the training data explained by the model (deviance) are shown as dots.}
  \label{val_calib_dev}
\end{figure*}

\section{Discussion}

This study demonstrated that increasing the amount of collision data used for modelling, by combining disparate datasets, improved model performance and predictive strength. The improvements were marginal, most likely due to the limited amount and significant variation of the data used in this study. However, even small, incremental improvements can be quite valuable when comparing to the costs WVC. For example, if the model trained on more data can predict and prevent a single additional WVC from occurring, it has tremendous value when considering the high costs of building roads, a human life, or other unquantifiable effects such as species extinctions. From this perspective, the results of this study suggest that the development of a unified data system has high utility given its relatively inexpensive costs to implement. Further, such a system would be able systematically catalogue disparate sources of data and enable more robust analyses of wildlife-vehicle collisions, however, there are several important factors to consider. Large quantities of data may help to overcome issues of underreporting \citep{snow15}, and limited temporal/spatial coverage that affect generality \citep{clev15}, but equally important is the quality of the data.

It is important to identify and address spatial biases when developing predictive models using presence-only data \citep{kram13}. Models trained on observations that are spatially biased (e.g. collected at particular places/times or dependent on characteristics of reporting individuals) may cause model predictions to reflect patterns of collection rather than distributions of collision risk. Each of our datasets had varying levels of spatial bias but these were difficult to identify. For example, the original Wildlife Victoria dataset appears to have a reporting bias that is spatially correlated with proximity to the urban centre of Melbourne. Other datasets can be expected to exhibit less spatial bias due to the nature of reporting - Crashstats is comprised of statewide police records of kangaroo-vehicle collisions. However, Crashstats data is subject to under-reporting because it only includes records involving human injuries or deaths resulting from WVC.  Regardless of differences, implicit biases in data may be considered in modelling efforts if they are adequately described in metadata \citep{wart13} but generally, collision data does not include this information.

Metadata should also include the survey effort involved in collecting collision data as this will have modelling implications and an effect on predictions of road risk. Despite its importance, this information is seldom reported and varies considerably amongst and within jurisdictions \citep{huij07a}. For example, the Western District data was collected using a combination of systematic surveys and opportunistic methods. Road cleaning contractors regularly patrolled major segments of roads in the district (Freeways, Highways, Major Arterials) and recorded carcasses collected but were also irregularly dispatched based on carcass observations reported by the public. It should be noted that contractors only collected carcasses that were in close proximity to the road, and deemed a risk to motorists, which leads to under-reporting. It is reasonable to assume our four datasets had different surveying efforts, however, because these were not explicitly defined, we did not incorporate them into the modelling. Information on survey effort can improve models by generating parameters that account for detectability \citep{dora14}.

Citizen-science collected data has the potential to contribute large amounts of observational data on wildlife-vehicle collisions \citep{cose14,dwye16,paul14}, however, controlling the quality of the data is critical to ensure sound statistical analyses. To augment collision or carcass observation data, some systems store information on reliability and accuracy (e.g. \cite{shil15a}). This is often reported by the user and requires additional review by a system administrator to determine its merit but is a good foundation for assessing and comparing the effects of the data on model outputs. In the Victorian data, we did not have access to such metadata and assumed all datasets had the same reliability and accuracy. Yet, we suspect there is some variation between datasets as some data, such as insurance or police reports, may be more reliable than observations reported by citizens. Moreover, our data only included records on collisions with kangaroos, which are quite identifiable; smaller species may be harder to identify and introduce additional uncertainty. Nonetheless, reliability may be incorporated into modelling if properly reported \citep{guns09}.

Changes in relative risk over different temporal scales is important to reflect in collision model predictions (see \Cref{sec:train}). Many species exhibit different patterns of activity, as do humans, and thus collision risk will ultimately vary hourly \citep{joyc01}, seasonally \citep{alve12,beau10,gril09}, or both \citep{mizu14,more13}. Most data collected on carcasses or injured animals incorporates this information but does not report the uncertainty around reported times. This is problematic due to potential temporal lags between actual collision events and subsequent observations. Similar to issues resulting from spatial bias, models trained on data with inaccurate times will result in predictions that correspond to collection times rather than collision times. Smaller species are subject to more reporting error, as compared to larger species, as they often become unrecognisable after a few hours or days. Although kangaroos are large species, we did not incorporate hourly effects due to a lack of information about the reported times and associated uncertainty. Seasonal effects are less prone to this shortcoming due to larger temporal scales, however, we excluded seasonal effects from this analysis as kangaroos do not exhibit seasonal patterns. 

Our road segment-based datasets were represented by truncated counts of collisions to `zeros' or `ones' based on the mathematical framework used for modelling. This was appropriate considering that more than one collision on a road segment rarely occurred in the data. More importantly, we were concerned about potential duplicate records arising from different data sources (i.e. the same collision reported twice). But it is beneficial to maximise the use of available collision data and some studies take advantage of count data to model and predict wildlife-vehicle collision risk \citep{cser13}; and explicitly reconcile disparate data sources within the model \citep{lao11a}. However, duplication of recorded collision events or carcass observations may affect model predictions and should be scrutinised. If duplicates could be identified reliably, then those records might help to examine biases in the reported collisions in a way that is similar to using multiple observers to estimate detection probabilities \citep{nich00}.

At time of writing, we are not aware of a unified data storage system that is accessible to and used by governments, organisations, companies and individual citizens to catalogue wildlife-vehicle collisions anywhere in the world. Although this study is limited to the State of Victoria in south-east Australia, the results and general discussion suggest qualities that a centralised storage system should have and we recommend the development and implementation of a WVC database to support future modelling and prediction efforts. Although systems maintained at the global scale, such as the Global Biodiversity Information Facility (GBIF), may reduce potential redundancies and ambiguities, national systems are also useful to achieve this objective. Of paramount importance is uniting the currently disparate sources of data with an archive that is both comprehensive and systematic. Such a repository will enable informed analyses of WVC, with improved model calibration, to ultimately make more reliable predictions of wildlife-vehicle collision risk. 
